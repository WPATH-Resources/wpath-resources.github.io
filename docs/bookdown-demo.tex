% Options for packages loaded elsewhere
\PassOptionsToPackage{unicode}{hyperref}
\PassOptionsToPackage{hyphens}{url}
%
\documentclass[
]{book}
\usepackage{amsmath,amssymb}
\usepackage{iftex}
\ifPDFTeX
  \usepackage[T1]{fontenc}
  \usepackage[utf8]{inputenc}
  \usepackage{textcomp} % provide euro and other symbols
\else % if luatex or xetex
  \usepackage{unicode-math} % this also loads fontspec
  \defaultfontfeatures{Scale=MatchLowercase}
  \defaultfontfeatures[\rmfamily]{Ligatures=TeX,Scale=1}
\fi
\usepackage{lmodern}
\ifPDFTeX\else
  % xetex/luatex font selection
\fi
% Use upquote if available, for straight quotes in verbatim environments
\IfFileExists{upquote.sty}{\usepackage{upquote}}{}
\IfFileExists{microtype.sty}{% use microtype if available
  \usepackage[]{microtype}
  \UseMicrotypeSet[protrusion]{basicmath} % disable protrusion for tt fonts
}{}
\makeatletter
\@ifundefined{KOMAClassName}{% if non-KOMA class
  \IfFileExists{parskip.sty}{%
    \usepackage{parskip}
  }{% else
    \setlength{\parindent}{0pt}
    \setlength{\parskip}{6pt plus 2pt minus 1pt}}
}{% if KOMA class
  \KOMAoptions{parskip=half}}
\makeatother
\usepackage{xcolor}
\usepackage{longtable,booktabs,array}
\usepackage{calc} % for calculating minipage widths
% Correct order of tables after \paragraph or \subparagraph
\usepackage{etoolbox}
\makeatletter
\patchcmd\longtable{\par}{\if@noskipsec\mbox{}\fi\par}{}{}
\makeatother
% Allow footnotes in longtable head/foot
\IfFileExists{footnotehyper.sty}{\usepackage{footnotehyper}}{\usepackage{footnote}}
\makesavenoteenv{longtable}
\usepackage{graphicx}
\makeatletter
\def\maxwidth{\ifdim\Gin@nat@width>\linewidth\linewidth\else\Gin@nat@width\fi}
\def\maxheight{\ifdim\Gin@nat@height>\textheight\textheight\else\Gin@nat@height\fi}
\makeatother
% Scale images if necessary, so that they will not overflow the page
% margins by default, and it is still possible to overwrite the defaults
% using explicit options in \includegraphics[width, height, ...]{}
\setkeys{Gin}{width=\maxwidth,height=\maxheight,keepaspectratio}
% Set default figure placement to htbp
\makeatletter
\def\fps@figure{htbp}
\makeatother
\setlength{\emergencystretch}{3em} % prevent overfull lines
\providecommand{\tightlist}{%
  \setlength{\itemsep}{0pt}\setlength{\parskip}{0pt}}
\setcounter{secnumdepth}{5}
\usepackage{booktabs}
\usepackage{amsthm}
\makeatletter
\def\thm@space@setup{%
  \thm@preskip=8pt plus 2pt minus 4pt
  \thm@postskip=\thm@preskip
}
\makeatother
\ifLuaTeX
  \usepackage{selnolig}  % disable illegal ligatures
\fi
\usepackage[]{natbib}
\bibliographystyle{apalike}
\IfFileExists{bookmark.sty}{\usepackage{bookmark}}{\usepackage{hyperref}}
\IfFileExists{xurl.sty}{\usepackage{xurl}}{} % add URL line breaks if available
\urlstyle{same}
\hypersetup{
  pdftitle={WPATH's Standards Of Care Version 8},
  pdfauthor={E. Coleman, et al.},
  hidelinks,
  pdfcreator={LaTeX via pandoc}}

\title{WPATH's Standards Of Care Version 8}
\author{E. Coleman, et al.}
\date{2023-08-14}

\begin{document}
\maketitle

{
\setcounter{tocdepth}{1}
\tableofcontents
}
\hypertarget{abstract}{%
\chapter*{Abstract}\label{abstract}}
\addcontentsline{toc}{chapter}{Abstract}

\textbf{Background:} Transgender healthcare is a rapidly evolving interdisciplinary field. In the last
decade, there has been an unprecedented increase in the number and visibility of transgender
and gender diverse (TGD) people seeking support and gender-affirming medical treatment
in parallel with a significant rise in the scientific literature in this area. The World Professional
Association for Transgender Health (WPATH) is an international, multidisciplinary, professional
association whose mission is to promote evidence-based care, education, research, public
policy, and respect in transgender health. One of the main functions of WPATH is to promote
the highest standards of health care for TGD people through the Standards of Care (SOC).
The SOC was initially developed in 1979 and the last version (SOC-7) was published in 2012.
In view of the increasing scientific evidence, WPATH commissioned a new version of the
Standards of Care, the SOC-8.

\textbf{Aim:} The overall goal of SOC-8 is to provide health care professionals (HCPs) with clinical
guidance to assist TGD people in accessing safe and effective pathways to achieving lasting
personal comfort with their gendered selves with the aim of optimizing their overall physical
health, psychological well-being, and self-fulfillment.

\textbf{Methods:} The SOC-8 is based on the best available science and expert professional consensus
in transgender health. International professionals and stakeholders were selected to serve
on the SOC-8 committee. Recommendation statements were developed based on data
derived from independent systematic literature reviews, where available, background reviews
and expert opinions. Grading of recommendations was based on the available evidence
supporting interventions, a discussion of risks and harms, as well as the feasibility and
acceptability within different contexts and country settings.

\textbf{Results:} A total of 18 chapters were developed as part of the SOC-8. They contain
recommendations for health care professionals who provide care and treatment for TGD
people. Each of the recommendations is followed by explanatory text with relevant references.
General areas related to transgender health are covered in the chapters Terminology, Global
Applicability, Population Estimates, and Education. The chapters developed for the diverse
population of TGD people include Assessment of Adults, Adolescents, Children, Nonbinary,
Eunuchs, and Intersex Individuals, and people living in Institutional Environments. Finally,
the chapters related to gender-affirming treatment are Hormone Therapy, Surgery and
Postoperative Care, Voice and Communication, Primary Care, Reproductive Health, Sexual
Health, and Mental Health.

\textbf{Conclusions:} The SOC-8 guidelines are intended to be flexible to meet the diverse health
care needs of TGD people globally. While adaptable, they offer standards for promoting
optimal health care and guidance for the treatment of people experiencing gender
incongruence. As in all previous versions of the SOC, the criteria set forth in this document
for gender-affirming medical interventions are clinical guidelines; individual health care
professionals and programs may modify these in consultation with the TGD person.

\hypertarget{citation}{%
\section*{Citation}\label{citation}}
\addcontentsline{toc}{section}{Citation}

To cite this article: E. Coleman, A. E. Radix, W. P. Bouman, G. R. Brown, A. L. C. de Vries, M. B.
Deutsch, R. Ettner, L. Fraser, M. Goodman, J. Green, A. B. Hancock, T. W. Johnson, D. H. Karasic,
G. A. Knudson, S. F. Leibowitz, H. F. L. Meyer-Bahlburg, S. J. Monstrey, J. Motmans, L. Nahata,
T. O. Nieder, S. L. Reisner, C. Richards, L. S. Schechter, V. Tangpricha, A. C. Tishelman, M. A.
A. Van Trotsenburg, S. Winter, K. Ducheny, N. J. Adams, T. M. Adrián, L. R. Allen, D. Azul, H.
Bagga, K. Başar, D. S. Bathory, J. J. Belinky, D. R. Berg, J. U. Berli, R. O. Bluebond-Langner, M.-
B. Bouman, M. L. Bowers, P. J. Brassard, J. Byrne, L. Capitán, C. J. Cargill, J. M. Carswell, S. C.
Chang, G. Chelvakumar, T. Corneil, K. B. Dalke, G. De Cuypere, E. de Vries, M. Den Heijer, A.
H. Devor, C. Dhejne, A. D'Marco, E. K. Edmiston, L. Edwards-Leeper, R. Ehrbar, D. Ehrensaft,
J. Eisfeld, E. Elaut, L. Erickson-Schroth, J. L. Feldman, A. D. Fisher, M. M. Garcia, L. Gijs, S. E.
Green, B. P. Hall, T. L. D. Hardy, M. S. Irwig, L. A. Jacobs, A. C. Janssen, K. Johnson, D. T. Klink,
B. P. C. Kreukels, L. E. Kuper, E. J. Kvach, M. A. Malouf, R. Massey, T. Mazur, C. McLachlan, S.
D. Morrison, S. W. Mosser, P. M. Neira, U. Nygren, J. M. Oates, J. Obedin-Maliver, G. Pagkalos,
J. Patton, N. Phanuphak, K. Rachlin, T. Reed, G. N. Rider, J. Ristori, S. Robbins-Cherry, S. A.
Roberts, K. A. Rodriguez-Wallberg, S. M. Rosenthal, K. Sabir, J. D. Safer, A. I. Scheim, L. J. Seal,
T. J. Sehoole, K. Spencer, C. St.~Amand, T. D. Steensma, J. F. Strang, G. B. Taylor, K. Tilleman,
G. G. T'Sjoen, L. N. Vala, N. M. Van Mello, J. F. Veale, J. A. Vencill, B. Vincent, L. M. Wesp, M.
A. West \& J. Arcelus (2022) Standards of Care for the Health of Transgender and Gender
Diverse People, Version 8, International Journal of Transgender Health, 23:sup1, S1-S259, DOI:
10.1080/26895269.2022.2100644

\hypertarget{original-article}{%
\section*{Original article}\label{original-article}}
\addcontentsline{toc}{section}{Original article}

To find original article, visit this link: \url{https://www.tandfonline.com/doi/pdf/10.1080/26895269.2022.2100644}

\hypertarget{introduction}{%
\chapter*{Introduction}\label{introduction}}
\addcontentsline{toc}{chapter}{Introduction}

\hypertarget{purpose-and-use-of-the-standards-of-care}{%
\section*{Purpose and use of the Standards of Care}\label{purpose-and-use-of-the-standards-of-care}}
\addcontentsline{toc}{section}{Purpose and use of the Standards of Care}

The overall goal of the World Professional
Association for Transgender Health's (WPATH)
Standards of Care---Eighth Edition (SOC-8) is to
provide clinical guidance to health care professionals to assist transgender and gender diverse
(TGD) people in accessing safe and effective
pathways to achieving lasting personal comfort
with their gendered selves with the aim of optimizing their overall physical health, psychological
well-being, and self-fulfillment. This assistance
may include but is not limited to hormonal and
surgical treatments, voice and communication
therapy, primary care, hair removal, reproductive
and sexual health, and mental health care.
Healthcare systems should provide medically necessary gender-affirming health care for TGD
people: See Chapter 2---Global Applicability,
Statement 2.1.

WPATH is an international, multidisciplinary,
professional association whose mission is to promote evidence-based care, education, research,
public policy, and respect in transgender health.
Founded in 1979, the organization currently has
over 3,000 health care professionals, social scientists, and legal professionals, all of whom are
engaged in clinical practice, research, education
and advocacy that affects the lives of TGD people. WPATH envisions a world wherein people
of all gender identities and gender expressions
have access to evidence-based health care, social
services, justice, and equality.

One of the main functions of WPATH is to
promote the highest standards of health care for
individuals through the Standards of Care (SOC)
for the health of TGD people. The SOC-8 is
based on the best available science and expert
professional consensus. The SOC was initially
developed in 1979, and the last version was published in 2012.

Most of the research and experience in this
field comes from a North American and Western
European perspective; thus, adaptations of the
SOC-8 to other parts of the world are necessary.
Suggestions for approaches to cultural relativity
and cultural competence are included in this version of the SOC

WPATH recognizes that health is not only
dependent upon high-quality clinical care but
also relies on social and political climates that
ensure social tolerance, equality, and the full
rights of citizenship. Health is promoted through
public policies and legal reforms that advance
tolerance and equity for gender diversity and that
eliminate prejudice, discrimination, and stigma.
WPATH is committed to advocacy for these policy and legal changes. Thus, health care professionals who provide care to TGD people are
called upon to advocate for improved access to
safe and licensed gender-affirming care while
respecting the autonomy of individuals.

While this is primarily a document for health
care professionals, individuals, their families, and
social institutions may also use the SOC-8 to understand how it can assist with promoting optimal
health for members of this diverse population.

The SOC-8 has 18 chapters containing recommendations for health care professionals working
with TGD people. Each of the recommendations
is followed by explanatory text with relevant references. The recommendations for the initiation
of gender-affirming medical and/or surgical treatments (GAMSTs) for adults and adolescents are
contained in their respective chapters (see
Assessment for Adults and Adolescent chapters).
A summary of the recommendations and criteria
for GAMST can be found in Appendix D.

\hypertarget{populations-included-in-the-soc-8}{%
\section*{Populations included in the SOC-8}\label{populations-included-in-the-soc-8}}
\addcontentsline{toc}{section}{Populations included in the SOC-8}

In this document, we use the phrase transgender
and gender diverse (TGD) to be as broad and
comprehensive as possible in describing members
of the many varied communities that exist globally of people with gender identities or expressions that differ from the gender socially
attributed to the sex assigned to them at birth.
This includes people who have culturally specific
and/or language-specific experiences, identities or
expressions, which may or may not be based on
or encompassed by Western conceptualizations
of gender or the language used to describe it.

WPATH SOC-8 expands who is included under
the TGD umbrella, and the settings in which
these guidelines should be applied to promote
equity and human rights.

Globally, TGD people encompass a diverse
array of gender identities and expressions and
have differing needs for gender-affirming care
across their lifespan that is related to individual
goals and characteristics, available health care
resources, and sociocultural and political contexts.
When standards of care are absent for certain
groups this vacuum can result in a multiplicity
of therapeutic approaches, including those that
may be counterproductive or harmful. The SOC-8
includes recommendations to promote health and
well-being for gender diverse groups that have
often been neglected and/or marginalized, including nonbinary people, eunuch, and intersex
individuals.

The SOC-8 continues to outline the appropriate
care of TGD youth, which includes, when indicated, the use of puberty suppression and, when
indicated, the use of gender-affirming hormones.

Worldwide, TGD people commonly experience
transphobia, stigmatization, ignorance, and refusal
of care when seeking health care services, which
contributes to significant health disparities. TGD
people often report having to teach their medical
providers how to care for them due to the latter's
insufficient knowledge and training. Intersectional
forms of discrimination, social marginalization,
and hate crimes against TGD people lead to
minority stress. Minority stress is associated with
mental health disparities exemplified by increased
rates of depression, suicidality, and non-suicidal
self-injuries than rates in cisgender populations.
Professionals from every discipline should consider the marked vulnerability of many TGD
people. WPATH urges health care authorities,
policymakers, and medical societies to discourage
and combat transphobia among health care professionals and ensure every effort is made to
refer TGD people to professionals with experience
and willingness to provide
gender-affirming care.

Flexibility in the SOC
The SOC-8 guidelines are intended to be flexible
to meet the diverse health care needs of TGD
people globally. While adaptable, they offer standards for promoting optimal health care and for
guiding treatment of people experiencing gender
incongruence. As in all previous versions of the
SOC, the criteria put forth in this document for
gender-affirming interventions are clinical guidelines; individual health care professionals and
programs may modify them in consultation with
the TGD person. Clinical departures from the
SOC may come about because of a patient's
unique anatomic, social, or psychological situation; an experienced health care professional's
evolving method of handling a common situation;
a research protocol; lack of resources in various
parts of the world; or the need for specific
harm-reduction strategies. These departures
should be recognized as such, explained to the
patient, and documented for quality patient care
and legal protection. This documentation is also
valuable for the accumulation of new data, which
can be retrospectively examined to allow for
health care---and the SOC---to evolve.

The SOC-8 supports the role of informed
decision-making and the value of harm reduction
approaches. In addition, this version of the SOC
recognizes and validates various expressions of
gender that may not necessitate psychological,
hormonal, or surgical treatments. Health care
professionals can use the SOC to help patients
consider the full range of health services open
to them in accordance with their clinical needs
for gender expression.

\hypertarget{diversity-versus-diagnosis}{%
\section*{Diversity versus Diagnosis}\label{diversity-versus-diagnosis}}
\addcontentsline{toc}{section}{Diversity versus Diagnosis}

The expression of gender characteristics, including identities, that are not stereotypically associated with one's sex assigned at birth is a common
and a culturally diverse human phenomenon that
should not be seen as inherently negative or
pathological. Unfortunately, gender nonconformity and diversity in gender identity and expression is stigmatized in many societies around the
world. Such stigma can lead to prejudice and
discrimination, resulting in ``minority stress.''
Minority stress is unique (additive to general
stressors experienced by all people), socially
based, and chronic, and may make TGD individuals more vulnerable to developing mental health
concerns such as anxiety and depression. In addition to prejudice and discrimination in society
at large, stigma can contribute to abuse and
neglect in one's interpersonal relationships, which
in turn can lead to psychological distress.
However, these symptoms are socially induced
and are not inherent to being TGD.

While Gender Dysphoria (GD) is still considered a mental health condition in the Diagnostic
and Statistical Manual of Mental Disorders,
(DSM-5-TR) of the American Psychiatric
Association. Gender incongruence is no longer
seen as pathological or a mental disorder in the
world health community. Gender Incongruence
is recognized as a condition in the International
Classification of Diseases and Related Health
Problems, 11th Version of the World Health
Organization (ICD-11). Because of historical and
current stigma, TGD people can experience distress or dysphoria that may be addressed with
various gender-affirming treatment options. While
nomenclature is subject to change and new terminology and classifications may be adopted by
various health organizations or administrative
bodies, the medical necessity of treatment and
care is clearly recognized for the many people
who experience dissonance between their sex
assigned at birth and their gender identity.

Not all societies, countries, or health care systems require a diagnosis for treatment. However,
in some countries these diagnoses may facilitate
access to medically necessary health care and can
guide further research into effective treatments.

\hypertarget{health-care-services}{%
\section*{Health care services}\label{health-care-services}}
\addcontentsline{toc}{section}{Health care services}

The goal of gender-affirming care is to partner
with TGD people to holistically address their
social, mental, and medical health needs and
well-being while respectfully affirming their gender identity. Gender-affirming care supports TGD
people across the lifespan---from the very first
signs of gender incongruence in childhood
through adulthood and into older age---as well
as people with concerns and uncertainty about
their gender identity, either prior to or after
transition.

Transgender health care is greater than the
sum of its parts, involving holistic inter- and
multidisciplinary care between endocrinology,
surgery, voice and communication, primary care,
reproductive health, sexual health and mental
health disciplines to support gender-affirming
interventions as well as preventive care and
chronic disease management. Gender-affirming
interventions include puberty suppression, hormone therapy, and gender-affirming surgeries
among others. It should be emphasized there is
no `one-size-fits-all' approach and TGD people
may need to undergo all, some, or none of these
interventions to support their gender affirmation.
These guidelines encourage the use of a
patient-centered care model for initiation of gender- affirming interventions and update many
previous requirements to reduce barriers to care.

Ideally, communication and coordination of care
should occur between providers to optimize outcomes and the timing of gender-affirming interventions centered on the patient's needs and desires and
to minimize harm. In well-resourced settings, multidisciplinary consultation and care coordination is
often routine, but many regions worldwide lack facilities dedicated to transgender care. For these regions,
if possible, it is strongly recommended that individual care providers create a network to facilitate transgender health care that is not available locally.

Worldwide, TGD people are sometime forced
by family members or religious communities to
undergo conversion therapy. WPATH strongly
recommends against any use of reparative or conversion therapy (see statements 6.5 and 18.10).

\hypertarget{health-care-settings}{%
\section*{Health care settings}\label{health-care-settings}}
\addcontentsline{toc}{section}{Health care settings}

The SOC-8 are guidelines rooted in the fundamental rights of TGD people that apply to all
settings in which health care is provided regardless
of an individual's social or medical circumstances.
This includes a recommendation to apply the standards of care for TGD people who are incarcerated or living in other institutional settings.

Due to a lack of knowledgeable providers,
untimely access, cost barriers and/or previous stigmatizing health care experiences, many TGD people take non-prescribed hormone therapy. This
poses health risks associated with the use of
unmonitored therapy in potentially supratherapeutic doses and the potential exposure to blood-borne
illnesses if needles are shared for administration.
However, for many individuals, it is the only
means of acquiring medically necessary
gender-affirming treatment that is otherwise inaccessible. Non-prescribed hormone use should be
approached with a harm-reduction lens to ensure
individuals are connected with providers who can
prescribe safe and monitored hormone therapy.

In some countries, the rights of TGD are
increasingly being recognized, and gender clinics
are being established that can serve as templates
for care. In other countries, however, such facilities are lacking and care may be more fragmented and under-resourced. Nonetheless,
different models of care are being pioneered,
including efforts to decentralize gender-affirming
care within primary care settings and establish
telehealth services to reduce barriers and improve
access. Regardless of the method of care delivery,
the principles of gender-affirming care as outlined
in the SOC-8 should be adapted to align with
local sociocultural, political, and medical contexts.

\hypertarget{methodology}{%
\section*{Methodology}\label{methodology}}
\addcontentsline{toc}{section}{Methodology}

This version of the Standards of Care (SOC-8)
is based upon a more rigorous and methodological evidence-based approach than previous versions. This evidence is not only based on the
published literature (direct as well as background
evidence) but also on consensus-based expert
opinion. Evidence-based guidelines include recommendations intended to optimize patient care
that are informed by a thorough review of evidence, an assessment of the benefits and harms,
values and preferences of providers and patients,
and resource use and feasibility.

While evidence-based research provides the
basis for sound clinical practice guidelines and
recommendations, it must be balanced by the
realities and feasibility of providing care in
diverse settings. The process for development of
the SOC-8 incorporated the recommendations on
clinical practice guideline development set forth
by the National Academies of Medicine and the
World Health Organization, which addressed
transparency, conflict-of-interest policy, committee composition, and group process.

The SOC-8 guidelines committee was multidisciplinary and consisted of subject matter experts,
health care professionals, researchers, and stakeholders with diverse perspectives and geographic
representation. A guideline methodologist assisted
with the planning and development of questions
and systematic reviews with additional input provided by an international advisory committee and
during the public comment period. All committee
members completed conflict of interest declarations. Recommendations in the SOC-8 are based
on available evidence supporting interventions, a
discussion of risks and harms, as well as feasibility
and acceptability within different contexts and
country settings. Consensus on the final recommendations was attained using the Delphi process
that included all members of the guidelines committee and required that recommendation statements were approved by at least 75\% of members.
A detailed overview of the SOC-8 Methodology
is included in Appendix A.

\hypertarget{soc-8-chapters-summary}{%
\section*{SOC-8 Chapters Summary}\label{soc-8-chapters-summary}}
\addcontentsline{toc}{section}{SOC-8 Chapters Summary}

The SOC-8 represents a significant advancement
from previous versions. Changes in this version
are based upon a fundamentally different methodology, significant cultural shifts, advances in
clinical knowledge, and appreciation of the many
health care issues that can arise for TGD people
beyond hormone therapy and surgery.

These updated guidelines continue the process
started with the SOC-7 in 2011 to broaden in
scope and move from a narrow focus on psychological requirements for ``diagnosing transgenderism'' and medical treatments for alleviation of
gender dysphoria to gender-affirming care for the
whole person. WPATH SOC-8 expands guidelines
specifying who is included under the TGD
umbrella, what should and should not be offered
with gender-affirming care, and the settings in
which these guidelines should be applied to promote equity and human rights.

The SOC-8 has several new chapters such as
the Assessment of Adults, Education, Eunuchs,
and a Nonbinary chapter. In addition, the chapter
for children and adolescents of the SOC-7 has
been divided into two different chapters. Overall,
the SOC-8 is considerably longer than previous
versions and provides a more in-depth introduction and recommendations for health care professionals. A summary of every chapter of the
SOC-8 can be found below:

\hypertarget{chapter-1terminology}{%
\subsection*{Chapter 1---Terminology}\label{chapter-1terminology}}
\addcontentsline{toc}{subsection}{Chapter 1---Terminology}

This new chapter lays the framework for language
used in the SOC-8 and offers consensually agreed
upon recommendations for the use of terminology. The chapter provides (1) terms and definitions, and (2) best practices for utilizing them.
This document is accompanied by a glossary (see
Appendix B) of common terms and language to
provide a framework for use and interpretation
of the SOC-8.

\hypertarget{chapter-2global-applicability}{%
\subsection*{Chapter 2---Global Applicability}\label{chapter-2global-applicability}}
\addcontentsline{toc}{subsection}{Chapter 2---Global Applicability}

This chapter references key literature related to
development and delivery of health care services,
broader advocacy care for TGD people from
beyond Western Europe and North America and
provides recommendations for adapting and
translating the SOC-8 to varied contexts.

\hypertarget{chapter-3population-estimates}{%
\subsection*{Chapter 3---Population Estimates}\label{chapter-3population-estimates}}
\addcontentsline{toc}{subsection}{Chapter 3---Population Estimates}

This chapter updates the population estimates of
TGD people in society. Based on the current
evidence, this proportion may range from a fraction of a percent to several percentage points
depending on the inclusion criteria, age group,
and geographic location.

\hypertarget{chapter-4education}{%
\subsection*{Chapter 4---Education}\label{chapter-4education}}
\addcontentsline{toc}{subsection}{Chapter 4---Education}

This new chapter provides a general review of
the literature related to education in TGD health
care. It offers recommendations at governmental,
nongovernmental, institutional and provider levels
to increase access to competent, compassionate
health care. The intent is to lay the groundwork
in the education area and invite a much broader
and deeper discussion among educators and
health care professionals.

\hypertarget{chapter-5assessment-of-adults}{%
\subsection*{Chapter 5---Assessment of Adults}\label{chapter-5assessment-of-adults}}
\addcontentsline{toc}{subsection}{Chapter 5---Assessment of Adults}

This new chapter provides guidance on the
assessment of TGD adults who are requesting
gender-affirming medical and surgical treatments
(GAMSTs). It describes and updates the assessment process as part of a patient-centered
approach and the criteria that health care professionals may follow in order to recommend
GAMSTs to TGD adults.

\hypertarget{chapter-6adolescents}{%
\subsection*{Chapter 6---Adolescents}\label{chapter-6adolescents}}
\addcontentsline{toc}{subsection}{Chapter 6---Adolescents}

This new chapter is dedicated to TGD adolescents,
is distinct from the child chapter, and has been
created for this 8th edition of the Standards of
Care given (1) the exponential growth in adolescent referral rates; (2) the increase in studies available specific to adolescent gender diversity-related
care; and (3) the unique developmental and genderaffirming care issues of this age group. This chapter
provides recommendations regarding the assessment process of adolescents requiring GAMSTs as
well as recommendations when working with TGD
youth and their families.

\hypertarget{chapter-7children}{%
\subsection*{Chapter 7---Children}\label{chapter-7children}}
\addcontentsline{toc}{subsection}{Chapter 7---Children}

This new chapter pertains to prepubescent gender
diverse children and focuses on developmentally
appropriate psychosocial practices and therapeutic
approaches.

\hypertarget{chapter-8nonbinary}{%
\subsection*{Chapter 8---Nonbinary}\label{chapter-8nonbinary}}
\addcontentsline{toc}{subsection}{Chapter 8---Nonbinary}

This new chapter in the SOC-8 consists of a
broad description of the term nonbinary and its
usage from a biopsychosocial, cultural, and intersectional perspective. The need for access to
gender-affirming care, specific gender-affirming
medical interventions, as well as an appropriate
level of support is discussed.

\hypertarget{chapter-9eunuchs}{%
\subsection*{Chapter 9---Eunuchs}\label{chapter-9eunuchs}}
\addcontentsline{toc}{subsection}{Chapter 9---Eunuchs}

This new chapter describes the unique needs of
eunuchs, and how the SOC can be applied to
this population.

\hypertarget{chapter-10intersex}{%
\subsection*{Chapter 10---Intersex}\label{chapter-10intersex}}
\addcontentsline{toc}{subsection}{Chapter 10---Intersex}

This chapter focuses on the clinical care of intersex individuals. It addresses the evolving terminology, prevalence, and diverse presentations of
such individuals and provides recommendations
for providing psychosocial and medical care with
their evidence-based explanations.

\hypertarget{chapter-11institutional-environments}{%
\subsection*{Chapter 11---Institutional Environments}\label{chapter-11institutional-environments}}
\addcontentsline{toc}{subsection}{Chapter 11---Institutional Environments}

This chapter has been expanded to include both
carceral and non-carceral settings and has been
built upon the last 3 versions of the SOC. This
chapter describes how the SOC-8 can be applied
to individuals living in these settings.

\hypertarget{chapter-12hormone-therapy}{%
\subsection*{Chapter 12---Hormone Therapy}\label{chapter-12hormone-therapy}}
\addcontentsline{toc}{subsection}{Chapter 12---Hormone Therapy}

This chapter describes the initiation of
gender-affirming hormone therapy, the recommended
regimens, screening for health concerns before and
during hormone therapy, and specific considerations
regarding hormone therapy prior to surgery. It
includes an expanded discussion about the safety of
gonadotropin releasing hormone (GnRH) agonists
in youth, various hormone regimens, monitoring to
include the development of potential therapy-related
health concerns, and guidance on how hormone
providers should collaborate with surgeons.

\hypertarget{chapter-13surgery-and-postoperative-care}{%
\subsection*{Chapter 13---Surgery and Postoperative Care}\label{chapter-13surgery-and-postoperative-care}}
\addcontentsline{toc}{subsection}{Chapter 13---Surgery and Postoperative Care}

This chapter describes a spectrum of
gender-affirming surgical procedures for the
diverse and heterogeneous community of individuals who identify as TGD. It provides a discussion about the optimal surgical training in
GAS procedures, post-surgical aftercare and
follow-up, access to surgery by adults and adolescents, and individually customized surgeries.

\hypertarget{chapter-14voice-and-communication}{%
\subsection*{Chapter 14---Voice and Communication}\label{chapter-14voice-and-communication}}
\addcontentsline{toc}{subsection}{Chapter 14---Voice and Communication}

This chapter describes professional voice and communication support and interventions that are inclusive of and attentive to all aspects of diversity and
no longer limited only to voice feminization and
masculinization. Recommendations are now framed
as affirming the roles and responsibilities of professionals involved in voice and communication support.

\hypertarget{chapter-15primary-care}{%
\subsection*{Chapter 15---Primary Care}\label{chapter-15primary-care}}
\addcontentsline{toc}{subsection}{Chapter 15---Primary Care}

This chapter discusses the importance of primary
care for TGD individuals, including topics of cardiovascular and metabolic health, cancer screening, and primary care systems.

\hypertarget{chapter-16reproductive-health}{%
\subsection*{Chapter 16---Reproductive Health}\label{chapter-16reproductive-health}}
\addcontentsline{toc}{subsection}{Chapter 16---Reproductive Health}

This chapter provides recent data on fertility perspectives and parenthood goals in gender diverse
youth and adults, advances in fertility preservation methods (including tissue cryopreservation),
guidance regarding preconception and pregnancy
care, prenatal counseling, and chest feeding.
Contraceptive methods and considerations for
TGD individuals are also reviewed.

\hypertarget{chapter-17sexual-health}{%
\subsection*{Chapter 17---Sexual Health}\label{chapter-17sexual-health}}
\addcontentsline{toc}{subsection}{Chapter 17---Sexual Health}

This new chapter acknowledges the profound
impact of sexual health on physical and psychological well-being for TGD people. The chapter
advocates for sexual functioning, pleasure, and
satisfaction to be included in TGD-related care.

\hypertarget{chapter-18mental-health}{%
\subsection*{Chapter 18---Mental Health}\label{chapter-18mental-health}}
\addcontentsline{toc}{subsection}{Chapter 18---Mental Health}

This chapter discusses principles of care for managing mental health conditions in TGD adults
and the nexus of mental health care and transition care. Psychotherapy may be beneficial but
should not be a requirement for gender-affirming
treatment, and conversion treatment should not
be offered.

\hypertarget{terminology}{%
\chapter{Terminology}\label{terminology}}

This chapter will lay the framework for language used in the SOC-8. It offers recommendations for use of terminology. It provides (1)
terms and definitions, and (2) best practices
for utilizing them. This document is accompanied by a glossary of common terms and language to provide a framework for use and
interpretation of the SOC-8. See Appendix B
for glossary

\hypertarget{terminology-1}{%
\section*{Terminology}\label{terminology-1}}
\addcontentsline{toc}{section}{Terminology}

In this document, we use the phrase transgender
and gender diverse (TGD) to be as broad and
comprehensive as possible in describing members
of the many varied communities globally of people with gender identities or expressions that
differ from the gender socially attributed to the
sex assigned to them at birth. This includes people who have culturally specific and/or
language-specific experiences, identities or expressions, and/or that are not based on or encompassed by Western conceptualizations of gender,
or the language used to describe it. TGD is used
for convenience as a shorthand for transgender
and gender diverse.

The decision to use transgender and gender
diverse resulted from an active process and was
not without controversy. Discussions centered on
avoiding over-emphasis on the term transgender,
integrating nonbinary gender identities and experiences, recognizing global variations in understandings of gender, avoiding the term gender
nonconforming, and recognizing the changing
nature of language because what is current now
may not be so in coming years. Thus, the term
transgender and gender diverse was chosen with
the intent to be most inclusive and to highlight
the many diverse gender identities, expressions,
experiences, and health care needs of TGD people. A Delphi process was used wherein SOC-8
chapter authors were anonymously and iteratively
surveyed over several rounds to obtain consensus
on terms. The SOC-8 presents standards of care
that strive to be applicable to TGD people globally, no matter how a person self-identifies or
expresses their gender.

\hypertarget{context}{%
\section*{Context}\label{context}}
\addcontentsline{toc}{section}{Context}

The language selected in this chapter may not
be (nor ever could be) comprehensive of every
culture and geographic region/locale. Differences
and debates over appropriate terms and specific
terminologies are common, and no single term
can be used without controversy. The goal of this
chapter is to be as inclusive as possible and offer
a shared vocabulary that is respectful and reflective of varied experiences of TGD people while
remaining accessible to health practitioners and
providers, and the public, for the purposes of
this document. Ultimately, access to
transition-related health care should be based on
providing adequate information and obtaining
informed consent from the individual, and not
on what words TGD people, or their service providers, use to describe their identities. Using language and terminology that is respectful and
culturally responsive is a basic foundation in the
provision of affirming care, as is reducing the
stigma and harm experienced by many TGD people seeking health care. It is vital for service providers to discuss with service users what language
is most comfortable for them and to use that
language whenever possible.

This chapter explains why current terms are
being used in preference to others. Rather than
use specific terms for medical, legal, and advocacy groups, the aim is to foster a shared language and understanding in the field of TGD
health, and the many related fields (e.g., epidemiology, law), in order to optimize the health of
transgender and gender diverse people.

Sex, gender, gender identity, and gender
expression are used in the English language as
descriptors that can apply to all people---those
who are TGD, and those who are not. There are
complex reasons why very specific language may
be the most respectful, most inclusive, or most
accepted by global TGD communities, including
the presence or absence of words to describe
these concepts in languages other than English;
the structural relationship between sex and gender; legal landscapes at the local, national, and
international levels; and the consequences of historical and present-day stigma that TGD people face.

Because at present, the field of TGD health is
heavily dominated by the English language, there
are two specific problems that constantly arise in
setting the context for terminology. The first
problem is that words exist in English that do
not exist in other languages (e.g., ``sex'' and ``gender'' are only represented by one word in Urdu
and many other languages). The second problem
is that there are words that exist outside of
English that do not have a direct translation into
English (e.g., \emph{travesti}, \emph{fa'afafine}, \emph{hijra}, \emph{selrata},
\emph{muxe}, \emph{kathoey}, \emph{transpinoy}, \emph{waria}, \emph{machi}).
Practically, this means the heavy influence of
English in this field impacts both what terms are
widely used and which people or identities are
most represented or validated by those terms.
The words used also shape the narratives that
contribute to beliefs and perceptions. While in
past versions of the Standards of Care, World
Professional Association for Transgender Health
(WPATH) has used only transgender as a broadly
defined umbrella term, version 8 broadens this
language to use TGD as the umbrella term
throughout the document (see Chapter 2---Global
Applicability).

Furthermore, the ever-evolving nature of language is impacted by external factors and the
social, structural, and personal pressures and violence enacted on TGD people and their bodies.
Many of the terms and phrases used historically
have been marred by how, when, and why they
were used in discussing TGD people, and have
thus fallen out of use or are hotly contested among
TGD people, with some individuals preferring
terms others find offensive. Some wish that these
Standards of Care could provide a coherent set of
universally accepted terms to describe TGD people, identities, and related health services. Such a
list, however, does not and cannot exist without
exclusion of some people and without reinforcing
structural oppressions, with regards to race,
national origin, Indigenous status, socioeconomic
status, religion, language(s) spoken, and ethnicity,
among other intersectionalities. It is very likely
that at least some of the terminology used in
SOC-8 will be outdated by the time version 9 is
developed. Some people will be frustrated by this
reality, but it is hoped it will be seen instead as
an opportunity for individuals and communities
to develop and refine their own lexicons and for
people to develop a still more nuanced understanding of the lives and needs of TGD people,
including TGD people's resilience and resistance
to oppression.

Finally, law and the work of legal professionals
are within the remit of these Standards of Care.
As such, language used most widely in international law is included here to help with the development of the functional definitions of these terms
and encourage their usage in legal contexts in lieu
of more antiquated and/or offensive terms. The
currently most thorough document in international
human rights law uses the term ``gender diverse.''

All the statements in this chapter have been
recommended based on a thorough review of
evidence, an assessment of the benefits and
harms, values and preferences of providers and
patients, and resource use and feasibility. In some
cases, we recognize evidence is limited and/or
services may not be accessible or desirable.

\hypertarget{statement-1.1-we-recommend-health-care-professionals-use-culturally-relevant-language-including-terms-to-describe-transgender-and-gender-diverse-people-when-applying-the-standards-of-care-in-different-global-settings.}{%
\section*{Statement 1.1: We recommend health care professionals use culturally relevant language (including terms to describe transgender and gender diverse people) when applying the Standards of Care in different global settings.}\label{statement-1.1-we-recommend-health-care-professionals-use-culturally-relevant-language-including-terms-to-describe-transgender-and-gender-diverse-people-when-applying-the-standards-of-care-in-different-global-settings.}}
\addcontentsline{toc}{section}{Statement 1.1: We recommend health care professionals use culturally relevant language (including terms to describe transgender and gender diverse people) when applying the Standards of Care in different global settings.}

Culturally relevant language is used to describe
TGD people in different global settings. For example, the concepts of sex, gender, and gender diversity
differ across contexts, as does the language used to
describe them. Thus, the language used when caring
for TGD people in Thailand is not going to be the
same as that used for TGD care in Nigeria. When
applying the Standards of Care globally, we recommend health care professionals (HCPs) utilize local
language and terms to deliver care in their specific
cultural and/or geographical locale.

Gender affirmation refers to the process of recognizing or affirming TGD people in their gender
identity---whether socially, medically, legally, behaviorally, or some combination of these (Reisner,
Poteat et al., 2016). Health care that is
gender-affirming or trans-competent utilizes culturally specific language in caring for TGD people.

Gender-affirming care is not synonymous with
transition-related care. Provision of transition-related
care, such as medical gender affirmation via hormones or surgery, does not alone ensure provision
of gender-affirming care, nor does it indicate the
quality or safety of the health care provided.

Consultation and partnerships with TGD communities can help to ensure relevancy and inclusivity of the language used in providing health
care locally in a particular context and setting.

\hypertarget{statement-1.2-we-recommend-health-care-professionals-use-language-in-health-care-settings-that-upholds-the-principles-of-safety-dignity-and-respect.}{%
\section*{Statement 1.2: We recommend health care professionals use language in health care settings that upholds the principles of safety, dignity, and respect.}\label{statement-1.2-we-recommend-health-care-professionals-use-language-in-health-care-settings-that-upholds-the-principles-of-safety-dignity-and-respect.}}
\addcontentsline{toc}{section}{Statement 1.2: We recommend health care professionals use language in health care settings that upholds the principles of safety, dignity, and respect.}

Safety, dignity, and respect are basic human
rights (International Commission of Jurists, 2007).
We recommend HCPs utilize language and terminology that uphold these human rights when providing care for TGD people. Many TGD people
have experienced stigma, discrimination, and mistreatment in health care settings, resulting in suboptimal care and poor health outcomes (Reisner,
Poteat et al., 2016; Safer et al., 2016; Winter, Settle
et al., 2016). Such experiences include misgendering, being refused care or denied services when
sick or injured and having to educate HCPs to be
able to receive adequate care (James et al., 2016).
Consequently, many TGD people feel unsafe
accessing health care. They may avoid health care
systems and seek other means of getting
health-related needs met, such as taking hormones
without a medical prescription or monitoring and
relying on peers for medical advice. Furthermore,
previous negative experiences in health care settings are associated with future avoidance of care
among TGD people.

Many TGD people have been treated unjustly,
with prejudice, and without dignity or respect by
HCPs, and lack of trust is often a barrier to care.
Using language grounded in the principles of
safety, dignity, and respect in health care settings
is paramount to ensure the health, well-being,
and rights of TGD people globally. Language is
a significant component of gender-affirming care,
but language alone does not resolve or mitigate
the systematic abuse and sometimes violence
TGD people face globally in care settings.
Language is but one important step toward
patient/client-centered and equitable health care
among TGD people. Other concrete actions HCPs
can take include obtaining informed consent and
refraining from making assumptions about a person's needs based on their gender or TGD status.

\hypertarget{statement-1.3-we-recommend-health-care-professionals-discuss-with-transgender-and-gender-diverse-people-what-language-or-terminology-they-prefer.}{%
\section*{Statement 1.3: We recommend health care professionals discuss with transgender and gender diverse people what language or terminology they prefer.}\label{statement-1.3-we-recommend-health-care-professionals-discuss-with-transgender-and-gender-diverse-people-what-language-or-terminology-they-prefer.}}
\addcontentsline{toc}{section}{Statement 1.3: We recommend health care professionals discuss with transgender and gender diverse people what language or terminology they prefer.}

In providing health care to TGD people, we
recommend HCPs discuss with their patients what
language or terminology they prefer be used when
referring to them. This discussion includes asking
TGD people how they would like to be addressed
in terms of name and pronouns, how they
self-identify their gender, and about the language
that should be used to describe their body parts.
Utilizing affirming language or terminology is a
key component of TGD-affirming care (Lightfoot
et al., 2021; Vermeir et al., 2018). Furthermore,
these discussions and communications can serve
to build rapport and reduce the mistrust many
TGD people feel toward HCPs and experience
within health care systems. Discussions and usage
of language or terminology can also facilitate
engagement and retention in care that is not specifically TGD-related, such as uptake of routine
preventive screenings and any necessary medical
follow-up of findings. In electronic health records,
organ/anatomical inventories can be standardly
used to inform appropriate clinical care, rather
than relying solely on assigned sex at birth and/
or gender identity designations.

HCPs and health care settings can implement
standardized procedures to facilitate these conversations such as: using intake forms that
include chosen pronouns and name, inviting
all staff (regardless of gender, i.e., cisgender,
TGD) to use pronouns in introductions, having
pronouns accompany names on a document for
all patients, and not using gendered honorifics
(e.g., Ms., Mr.). Policies for HCPs and health
care settings can be put in place to ensure a
TGD person's privacy and right to confidentiality, including when they disclose being a TGD
person, and if/how to appropriately document.
For example, a clinic policy may be to record
this information as private and confidential
between HCPs and patients/clients, and that it
should only be disclosed on a ``need to
know'' basis.

\hypertarget{global-applicability}{%
\chapter{Global Applicability}\label{global-applicability}}

People who defy cultural boundaries of sex and
gender have existed in cultures worldwide since
ancient times, sometimes acknowledged in local
language terms (Feinberg, 1996). In contrast to
the more recent pathologization of gender diversity
as an illness, some cultures traditionally celebrated
and welcomed this diversity (e.g., Nanda, 2014;
Peletz, 2009). Today, the English language umbrella
term transgender and gender diverse (TGD)
describes a huge variety of gender identities and
expressions, and therefore a population with
diverse health care experiences and needs. Together,
TGD people represent important aspects of human
diversity the World Professional Association for
Transgender Health (WPATH) asserts should be
valued and celebrated. TGD people continue to
make vital contributions to the societies in which
they live, although often these are unrecognized.

Disturbingly, many TGD people in the modern
world experience stigma, prejudice, discrimination, harassment, abuse and violence, resulting
in social, economic and legal marginalization,
poor mental and physical health, and even
death---a process that has been characterized as
a stigma-sickness slope (Winter, Diamond et al.,
2016). Experiences such as these (and the anticipation or fear of encountering such experiences)
leads to what Meyer has described as minority
stress (Meyer, 2003; see also Bockting et al., 2013
writing specifically about TGD people), and are
associated with poor physical (e.g.~Rich et al,
2020) and psychological (e.g., Bränström et al.,
2022; Scandurra et al., 2017; Shipherd et al., 2019,
Tan et al., 2021) health outcomes.

Violence against TGD people is a particular
problem. Seen from a global perspective, it is
widespread, diverse in nature (emotional, sexual
and physical, e.g., see Mujugira et al., 2021), and
involves a range of perpetrators (including State
actors). Statistics on murder, the form of violence
most extreme in its consequences, are alarming.
Worldwide, there were over 4,000 documented
killings between January 2008 and September
2021; a statistic widely regarded as flawed by
under-reporting (TGEU, 2020).

Since the publication of the Standards of Care
Version 7 (SOC-7), there have been dramatic
changes in perspectives on TGD people and their
health care. Mainstream global medicine no longer classifies TGD identities as a mental disorder.
In the Diagnostic and Statistical Manual Version
5 (DSM-5) from the American Psychiatric
Association (APA, 2013), the diagnosis of Gender
Dysphoria focuses on any distress and discomfort
that accompanies being TGD, rather than on the
gender identity itself. A text revision (DSM-5-TR)
was published in 2022. In the International
Classification of Diseases, Version 11 (ICD-11),
the diagnostic manual of the World Health
Organization (WHO, 2019b), the Gender
Incongruence diagnosis is placed in a chapter on
sexual health and focuses on the person's experienced identity and any need for gender-affirming
treatment that might stem from that identity.
Such developments, involving a depathologization
(or more precisely a de-psychopathologization)
of transgender identities, are fundamentally
important on a number of grounds. In the field
of health care, they may have helped support a
care model that emphasizes patients' active participation in decision-making about their own
health care, supported by primary health care
professionals (HCPs) (Baleige et al., 2021). It is
reasonable to suppose these developments may
also promote more socially inclusive policies such
as legislative reform regarding gender recognition
that facilitates a rights-based approach, without
imposing requirements for diagnosis, hormone
therapy and/or surgery. TGD people who have
changed gender markers on key documents enjoy
better mental health (e.g., Bauer et al., 2015;
Scheim et al., 2020). A more rights-based
approach in this area may contribute greatly to
the overall health and well-being of TGD people
(Arístegui et al., 2017).

Previous editions of the SOC have revealed
much of the recorded clinical experience and
knowledge in this area is derived from North
American and Western European sources. They
have focused on gender-affirming health care in
high income countries that enjoy relatively
well-resourced health care systems (including
those with trained mental health providers,
endocrinologists, surgeons and other specialists)
and where services are often funded publicly or
(at least for some patients) through private
insurance.

For many countries, health care provision for
TGD people is aspirational; with resourcing in this
area limited or non-existent, and services often
unavailable, inappropriate, difficult to access and/
or unaffordable. Few if any HCPs (primary or
specialist) may exist. Funding for gender-affirming
health care may be absent, with patients often
bearing the full costs of whatever health care they
access. Health care providers often lack clinical
and/or cultural competence in this area. Training
for work with these patients may be limited (e.g.,
Martins et al., 2020). For all these reasons and
because of mainstream ``Western'' medicine's historical view of TGD people as mentally disordered
(a perspective that has only recently changed),
TGD people have commonly found themselves
disempowered as health care consumers.

Health care providers have found the relevant
literature is largely North American and European,
which present particular challenges for persons
working in health care systems that are especially
poorly resourced. Recent initiatives that often
involve TGD stakeholders as partners are changing this situation somewhat by providing a body
of knowledge about good practice in other
regions, including how to provide effective,
culturally-competent TGD health care in low- and
middle-income countries outside the global north.

Within the field, a wide range of valuable
health care resources have been developed in
recent years. Dahlen et al (2021) review twelve
international clinical practice guidelines; over half
those reviewed originate from professional bodies
based in North America (e.g., Hembree et al.,
2017) or Europe (e.g., T'Sjoen et al., 2020). Three
are from WHO (the most recent being WHO,
2016). Nowadays, there are numerous other
resources, not on Dahlen et al.'s list, that explicitly
draw on expertise from regions outside North
America and Europe. Examples can be found in
Asia and the Pacific (APTN, 2022; Health Policy
Project et al., 2015), the Caribbean (PAHO, 2014),
Thailand, Australia (Telfer et al., 2020), Aotearoa
New Zealand (Oliphant et al., 2018), and South
Africa (Tomson et al., 2021) (see also TRANSIT
(UNDP et al., 2016)). These resources have commonly been created through the initiatives of or
in partnership with TGD communities locally or
internationally. This partnership approach,
focused on meeting local needs in culturally safe
and competent ways, can also have broad international relevance. Some of these publications
may be of particular value to those planning,
organizing and delivering services in low-income,
low-resource countries. There are likely to be
other resources published in languages other than
English of which we are unaware.

Globally, TGD identities may be associated
with differing conceptual frameworks of sex,
gender, and sexuality and exist in widely diverse
cultural (and sometimes spiritual) contexts and
histories. Considering the complex relationships
between social and cultural factors, the law, and
the demand for and provisions of gender-affirming
health care, the SOC-8 should be interpreted
through a lens that is appropriate for and within
the context of each HCP's individual practice
while maintaining alignment to the core principles that underscore it (APTN and UNDP,
2012; Health Policy Project et al., 2015;
PAHO, 2014).

It is within this context and by drawing broadly
on the experiences of TGD people and health
care providers internationally that we consider
the global applicability of SOC-8 within this
chapter. We set out key considerations for HCPs
and conclude by recommending core principles
and practices fundamental to contemporary
health care for TGD people, regardless of where
they live or whether there are resources available
to those who seek to provide such health care.

\hypertarget{statement-2.1-we-recommend-health-care-systems-should-provide-medically-necessary-gender-affirming-health-care-for-transgender-and-gender-diverse-people.}{%
\section*{Statement 2.1: We recommend health care systems should provide medically necessary gender-affirming health care for transgender and gender diverse people.}\label{statement-2.1-we-recommend-health-care-systems-should-provide-medically-necessary-gender-affirming-health-care-for-transgender-and-gender-diverse-people.}}
\addcontentsline{toc}{section}{Statement 2.1: We recommend health care systems should provide medically necessary gender-affirming health care for transgender and gender diverse people.}

Medical necessity is a term common to health
care coverage and insurance policies globally. A
common definition of medical necessity as used
by insurers or insurance companies is ``Health
care services that a physician and/or health care
professional, exercising prudent clinical judgment,
would provide to a patient for the purpose of
preventing, evaluating, diagnosing or treating an
illness, injury, disease or its symptoms, and that
are: (a) in accordance with generally accepted
standards of medical practice; (b) clinically
appropriate, in terms of type, frequency, extent,
site and duration, and considered effective for
the patient's illness, injury, or disease; and (c)
not primarily for the convenience of the patient,
physician, or other health care provider, and not
more costly than an alternative service or
sequence of services at least as likely to produce
equivalent therapeutic or diagnostic results as to
the diagnosis or treatment of that patient's illness,
injury or disease.'' The treating HCP asserts and
documents that a proposed treatment is medically
necessary for treatment of the condition
(American Medical Association, 2016).

Generally, ``accepted standards of medical practice'' means standards that are based on credible
scientific evidence published in peer-reviewed medical literature generally recognized by the relevant
medical community, designated Medical Specialty
Societies and/or legitimate Medical Colleges' recommendations, and the views of physicians and/
or HCPs practicing in relevant clinical areas.

Medical necessity is central to payment, subsidy, and/or reimbursement for health care in
parts of the world. The treating HCP may assert
and document that a given treatment is medically
necessary for the prevention or treatment of the
condition. If health policies and practices challenge the medical necessity of a treatment, there
may be an opportunity to appeal to a governmental agency or other entity for an independent
medical review.

It should be recognized gender diversity is
common to all human beings and is not pathological. However, gender incongruence that causes
clinically significant distress and impairment
often requires medically necessary clinical
interventions. In many countries, medically necessary gender-affirming care is documented by
the treating health professional as treatment for
Gender Incongruence (HA60 in ICD-11; WHO,
2019b) and/or as treatment for Gender Dysphoria
(F64.0 in DSM-5-TR; APA, 2022).

There is strong evidence demonstrating the benefits in quality of life and well-being of
gender-affirming treatments, including endocrine
and surgical procedures, properly indicated and
performed as outlined by the Standards of Care
(Version 8), in TGD people in need of these treatments (e.g., Ainsworth \& Spiegel, 2010; Aires
et al., 2020; Aldridge et al., 2020; Almazan \&
Keuroghlian, 2021; Al-Tamimi et al., 2019;
Balakrishnan et al., 2020; Baker et al., 2021;
Buncamper et al., 2016; Cardoso da Silva et al.,
2016; Eftekhar Ardebili, 2020; Javier et al., 2022;
Lindqvist et al., 2017; Mullins et al., 2021; Nobili
et al., 2018; Owen-Smith et al., 2018; Özkan et al.,
2018; T'Sjoen et al., 2019; van de Grift, Elaut
et al., 2018; White Hughto \& Reisner, Poteat et al.,
2016; Wierckx, van Caenegem et al., 2014; Yang,
Zhao et al., 2016). Gender-affirming interventions
may also include hair removal/transplant procedures, voice therapy/surgery, counseling, and other
medical procedures required to effectively affirm
an individual's gender identity and reduce gender
incongruence and dysphoria. Additionally, legal
name and sex or gender change on identity documents can also be beneficial and, in some jurisdictions, are contingent on medical documentation
that patients may call on practitioners to produce.

Gender-affirming interventions are based on
decades of clinical experience and research; therefore, they are not considered experimental, cosmetic, or for the mere convenience of a patient.
They are safe and effective at reducing gender
incongruence and gender dysphoria (e.g., Aires
et al., 2020; Aldridge et al., 2020; Al-Tamimi et al.,
2019; Balakrishnan et al., 2020; Baker et al., 2021;
Bertrand et al., 2017; Buncamper et al., 2016; Claes
et al., 2018; Eftekhar Ardebili, 2020; Esmonde et al.,
2019; Javier et al., 2022; Lindqvist et al., 2017; Lo
Russo et al., 2017; Marinkovic \& Newfield, 2017;
Mullins et al., 2021; Nobili et al., 2018;
Olson-Kennedy, Rosenthal et al., 2018; Özkan et al.,
2018; Poudrier et al., 2019; T'Sjoen et al., 2019; van
de Grift, Elaut et al., 2018; White Hughto \& Reisner,
Poteat et al., 2016; Wierckx, van Caenegem et al.,
2014; Wolter et al., 2015; Wolter et al., 2018).
Consequently, WPATH urges health care systems
to provide these medically necessary treatments and
eliminate any exclusions from their policy documents and medical guidelines that preclude coverage
for any medically necessary procedures or treatments for the health and well-being of TGD individuals. In other words, governments should ensure
health care services for TGD people are established,
extended or enhanced (as appropriate) as elements
in any Universal Health Care, public health, governmentsubsidized systems, or government-regulated private
systems that may exist. Health care systems should
ensure ongoing health care, both routine and specialized, is readily accessible and affordable to all
citizens on an equitable basis.

Medically necessary gender-affirming interventions are discussed in SOC-8. These include but
are not limited to hysterectomy +/- bilateral
salpingo-oophorectomy; bilateral mastectomy,
chest reconstruction or feminizing mammoplasty,
nipple resizing or placement of breast prostheses;
genital reconstruction, for example, phalloplasty
and metoidioplasty, scrotoplasty, and penile and
testicular prostheses, penectomy, orchiectomy,
vaginoplasty, and vulvoplasty; hair removal from
the face, body, and genital areas for gender affirmation or as part of a preoperative preparation
process; gender-affirming facial surgery and body
contouring; voice therapy and/or surgery; as well
as puberty blocking medication and
gender-affirming hormones; counseling or psychotherapeutic treatment as appropriate for the
patient and based on a review of the patient's
individual circumstances and needs.

\hypertarget{statement-2.2-we-recommend-health-care-professionals-and-other-users-of-the-standards-of-care-version-8-soc-8-apply-the-recommendations-in-ways-that-meet-the-needs-of-local-transgender-and-gender-diverse-communities-by-providing-culturally-sensitive-care-that-recognizes-the-realities-of-the-countries-they-are-practicing-in.}{%
\section*{Statement 2.2: We recommend health care professionals and other users of the Standards of Care, Version 8 (SOC-8) apply the recommendations in ways that meet the needs of local transgender and gender diverse communities, by providing culturally sensitive care that recognizes the realities of the countries they are practicing in.}\label{statement-2.2-we-recommend-health-care-professionals-and-other-users-of-the-standards-of-care-version-8-soc-8-apply-the-recommendations-in-ways-that-meet-the-needs-of-local-transgender-and-gender-diverse-communities-by-providing-culturally-sensitive-care-that-recognizes-the-realities-of-the-countries-they-are-practicing-in.}}
\addcontentsline{toc}{section}{Statement 2.2: We recommend health care professionals and other users of the Standards of Care, Version 8 (SOC-8) apply the recommendations in ways that meet the needs of local transgender and gender diverse communities, by providing culturally sensitive care that recognizes the realities of the countries they are practicing in.}

TGD people identify in many different ways
worldwide, and those identities exist within a
cultural context. In English speaking countries,
TGD people variously identify as \emph{transsexual},
\emph{trans}, \emph{gender nonconforming}, \emph{gender queer} or
\emph{diverse}, \emph{nonbinary}, or indeed \emph{transgender} and/or
\emph{gender diverse}, as well as by other identities;
including (for many identifying inside the gender
binary) \emph{male} or \emph{female}. (e.g., James et al., 2016;
Strauss et al., 2017; Veale et al., 2019).
Elsewhere, identities include but are not limited
to \emph{travesti} (across much of Latin America), \emph{hijra}
(across much of South Asia), \emph{khwaja sira} (in
Pakistan), \emph{achout} (in Myanmar), \emph{maknyah}, \emph{paknyah} (in Malaysia), \emph{waria} (Indonesia) \emph{kathoey},
\emph{phuying kham phet}, \emph{sao praphet song} (Thailand),
\emph{bakla}, \emph{transpinay}, \emph{transpinoy} (Philippines), \emph{fa'afafine}
(Samoa), \emph{mahu} (French Polynesia, Hawai'i), \emph{leiti}
(Tonga), \emph{fakafifine} (Niue), \emph{pinapinaaine} (Tuvalu
and Kiribati), \emph{vakasalewalewa} (Fiji), \emph{palopa} (Papua
Niugini), \emph{brotherboys} and \emph{sistergirls} (Aboriginal
and Torres Strait Islander people in Australia), and
\emph{akava'ine} (Cook Islands) (e.g., APTN and UNDP,
2012; Health Policy Project et al., 2015; Kerry,
2014). There are also a large number of two spirit
identities across North America (e.g., nadleehi in
Navajo (Diné) culture) (Sheppard \& Mayo, 2013).
The identities to which each of these terms refer
are often culturally complex and may exist in a
spiritual or religious context. Depending on the
cultures and the identities concerned, some may
be regarded as so-called ``third genders'' lying
beyond the gender binary (e.g., Graham, 2010;
Nanda, 2014; Peletz, 2009). Some TGD identities
are less firmly established than others. In many
places worldwide, the visibility of transgender men
and nonbinary trans masculine identities is relatively recent, with few or no applicable traditional
terms in local languages (Health Policy Project
et al., 2015). Regardless of where or with whom
HCPs work (including those working with ethnic
minority persons, migrants and refugees), they
need to be aware of the cultural context in which
people have grown up and live as well as the
consequences for health care.

Worldwide the availability, accessibility, acceptability and quality of health care vary greatly,
with resulting inequities within and across countries (OECD, 2019). In some countries, formal
health care systems exist alongside established
traditional and folk health care systems, with
indigenous models of health underpinning the
importance of holistic health care (WHO, 2019a).
HCPs should be aware of the traditions and realities within which health care is available and
provide support that is sensitive to the local
needs and identities of TGD people and provide
them with culturally competent and safe care.

\hypertarget{statement-2.3-we-recommend-health-care-providers-understand-the-impact-of-social-attitudes-laws-economic-circumstances-and-health-systems-on-the-lived-experiences-of-transgender-and-gender-diverse-people-worldwide.}{%
\section*{Statement 2.3: We recommend health care providers understand the impact of social attitudes, laws, economic circumstances, and health systems on the lived experiences of transgender and gender diverse people worldwide.}\label{statement-2.3-we-recommend-health-care-providers-understand-the-impact-of-social-attitudes-laws-economic-circumstances-and-health-systems-on-the-lived-experiences-of-transgender-and-gender-diverse-people-worldwide.}}
\addcontentsline{toc}{section}{Statement 2.3: We recommend health care providers understand the impact of social attitudes, laws, economic circumstances, and health systems on the lived experiences of transgender and gender diverse people worldwide.}

TGD people's lived experiences vary greatly,
depending on a range of factors, including social,
cultural (including spiritual), legal, economic and
geographic. When TGD people live in environments that affirm their gender and/or cultural
identities, then these experiences can be very
positive. Families are particularly important in
this regard (e.g., Pariseau et al., 2019; Yadegarfard
et al., 2014; Zhou et al., 2021). However, when
viewed from a global perspective, the circumstances in which TGD people live are often challenging. They are commonly denied widely
accepted rights in international human rights law.
These include rights to education, health and
protection from medical abuses, work and an
adequate standard of living, housing, freedom of
movement and expression, privacy, security, life,
family, freedom from arbitrary deprivation of liberty, fair trial, treatment with humanity while in
detention, and freedom from torture, inhuman
or degrading treatment or punishment
(International Commission of Jurists, 2007, 2017).

It is widely accepted that denial of rights can
impact sexual and gender minority health and
well-being (e.g., OHCHR et al., 2016; WHO,
2015). We therefore reaffirm here the importance
of the rights listed above for TGD people and
note WPATH's previous rights advocacy, including through numerous policy documents (e.g.,
WPATH, 2016, 2017, 2019). HCPs can play an
important role in rights advocacy, including the
right to quality gender-affirming health care that
is appropriate, affordable, and accessible.

Across the world, a large number of studies
detail the challenges TGD people face in their
lives, and the impact on their health and
well-being (e.g., Aurat Foundation, 2016;
Bhattacharya \& Ghosh, 2020; Chumakov et al.,
2021; Coleman et al., 2018; Heylens, Elaut et al.,
2014; Human Rights Watch, 2014; James et al,
2016; Lee, Operario et al., 2020; Luz et al., 2022;
McNeil et al., 2012, 2013; Motmans et al., 2017;
Muller et al., 2019; Scandurra et al., 2017; Strauss
et al., 2019; Suen et al., 2017; Valashany \&
Janghorbani, 2019; Veale et al., 2019; Wu et al.,
2017). The research shows TGD people often
experience stigma and prejudice as well as discrimination and harassment, abuse and violence,
or they live in anticipation and fear of such
actions. Social values and attitudes hostile to
TGD people, often communicated to young people in school curricula (e.g., Olivier \&
Thurasukam, 2018), are also expressed in family
rejection (e.g., Yadegarfard et al., 2014), and perpetuated in laws, policies and practices that limit
freedom to express one's gender identity and sexuality and hinder access to housing, public spaces,
education, employment and services (including
health care). The end result is TGD people are
commonly deprived of a wide range of opportunities available to their cisgender counterparts
and are pushed to the margins of society, without
family supports. To make matters worse, across
much of the world TGD people's access to legal
gender recognition is restricted or non-existent
(e.g., ILGA World, 2020a; TGEU, 2021; UNDP
and APTN, 2017). In some countries, such barriers nowadays draw on support from
``gender-critical theorists'' (as critiqued by e.g.,
Madrigal-Borloz, 2021; Zanghellini, 2020).

Inequities arise from a range of factors, including economic considerations and values underpinning the provision of health care systems,
particularly with regard to the emphasis placed on
public-, private- and self-funding of health care.
Lack of access to appropriate and affordable health
care can lead to a greater reliance on informal
knowledge systems. This includes information
about self-administration of hormones, which, in
many cases, is undertaken without necessary medical monitoring or supervision (e.g., Do et al.,
2018; Liu et al., 2020; Rashid et al., 2022; Reisner
et al., 2021; Winter \& Doussantousse, 2009).

In some parts of the world, large numbers of
transgender women employ silicone as a means
of modifying their bodies, drawing on the services
of silicone ``pumpers'' and/or attending pumping
``parties'', often within their communities. The
immediate results of silicone pumping contrast
with significant downstream health risks (e.g.,
Aguayo-Romero et al., 2015; Bertin et al., 2019;
Regmi et al., 2021), particularly where industrial
silicone or other injectable substances have been
used and where surgical removal may be difficult.

Finally, sexual health outcomes for TGD people
are poor. HIV prevalence for transgender women
reporting to clinical organizations in metropolitan
areas is approximately 19\% worldwide, which is
49 times higher than the background prevalence
rate in the general population (Baral et al., 2013).
Sexual health outcomes for transgender men are
also problematic (e.g., Mujugira et al., 2021).

\hypertarget{statement-2.4-we-recommend-translations-of-the-soc-focus-on-cross-cultural-conceptual-and-literal-equivalence-to-ensure-alignment-with-the-core-principles-that-underpin-the-soc-8.}{%
\section*{Statement 2.4: We recommend translations of the SOC focus on cross-cultural, conceptual and literal equivalence to ensure alignment with the core principles that underpin the SOC-8.}\label{statement-2.4-we-recommend-translations-of-the-soc-focus-on-cross-cultural-conceptual-and-literal-equivalence-to-ensure-alignment-with-the-core-principles-that-underpin-the-soc-8.}}
\addcontentsline{toc}{section}{Statement 2.4: We recommend translations of the SOC focus on cross-cultural, conceptual and literal equivalence to ensure alignment with the core principles that underpin the SOC-8.}

Much of the research literature on TGD people
is produced in high-income and English-speaking
countries. Global northern perspectives about
TGD people (including those related to health
care needs and provision) dominate this literature. A May 2021 Scopus database search undertaken by the current authors shows 99\% of the
literature on transgender health care comes out
of Europe, North America, Australia, or New
Zealand. Overall, 96\% of the literature is in the
English language. TGD people of the Global
South have received relatively little attention in
the English language literature, and the work of
those HCPs who interact with them has often
gone unrecognized and unpublished or has not
been translated into English. Applying resources
produced in the global north risks overlooking
the relevance and nuance of local knowledge,
cultural frameworks and practices, and missed
opportunities to learn from the work of others.

When translating the principles set out in the
SOC, we recommend following best practice
guidelines for language translation to ensure high
quality written resources are produced that are
culturally and linguistically appropriate to the local
situation. It is important translators have knowledge about TGD identities and cultures to check
that literal translations are culturally competent
and safe for local TGD people. It is also important
translation should follow established processes for
quality assurance (Centers for Medicare \& Medicaid
Services, 2010; Sprager \& Martinez, 2015)

\hypertarget{statement-2.5-we-recommend-health-care-professionals-and-policymakers-always-apply-the-soc-8-core-principles-to-their-work-with-transgender-and-gender-diverse-people-to-ensure-respect-for-human-rights-and-access-to-appropriate-and-competent-health-care-including}{%
\section*{Statement 2.5: We recommend health care professionals and policymakers always apply the SOC-8 core principles to their work with transgender and gender diverse people to ensure respect for human rights and access to appropriate and competent health care, including:}\label{statement-2.5-we-recommend-health-care-professionals-and-policymakers-always-apply-the-soc-8-core-principles-to-their-work-with-transgender-and-gender-diverse-people-to-ensure-respect-for-human-rights-and-access-to-appropriate-and-competent-health-care-including}}
\addcontentsline{toc}{section}{Statement 2.5: We recommend health care professionals and policymakers always apply the SOC-8 core principles to their work with transgender and gender diverse people to ensure respect for human rights and access to appropriate and competent health care, including:}

\hypertarget{general-principles}{%
\subsection*{General principles}\label{general-principles}}
\addcontentsline{toc}{subsection}{General principles}

\begin{itemize}
\tightlist
\item
  Be empowering and inclusive. Work to
  reduce stigma and facilitate access to
  appropriate health care, for all who seek it;
\item
  Respect diversity. Respect all clients and
  all gender identities. Do not pathologize
  differences in gender identity or expression;
\item
  Respect universal human rights, including
  the right to bodily and mental integrity,
  autonomy, and self-determination; freedom
  from discrimination and the right to the
  highest attainable standard of health.
\end{itemize}

\hypertarget{principles-around-developing-and-implementing-appropriate-services-and-accessible-health-care}{%
\subsection*{Principles around developing and implementing appropriate services and accessible health care}\label{principles-around-developing-and-implementing-appropriate-services-and-accessible-health-care}}
\addcontentsline{toc}{subsection}{Principles around developing and implementing appropriate services and accessible health care}

\begin{itemize}
\tightlist
\item
  Involve TGD people in the development
  and implementation of services;
\item
  Become aware of social, cultural, economic,
  and legal factors that might impact the
  health (and health care needs) of transgender and gender diverse people, as well as
  the willingness and capacity of the person
  to access services;
\item
  Provide health care (or refer to knowledgeable colleagues) that affirms gender identities and expressions, including health care
  that reduces the distress associated with
  gender dysphoria (if this is present);
\item
  Reject approaches that have the goal or
  effect of conversion, and avoid providing
  any direct or indirect support for such
  approaches or services
\end{itemize}

\hypertarget{principles-around-delivering-competent-services}{%
\subsection*{Principles around delivering competent services}\label{principles-around-delivering-competent-services}}
\addcontentsline{toc}{subsection}{Principles around delivering competent services}

\begin{itemize}
\tightlist
\item
  Become knowledgeable (get training, where
  possible) about the health care needs of
  transgender and gender diverse people, including the benefits and risks of
  gender-affirming care;
\item
  Match the treatment approach to the specific needs of clients, particularly their
  goals for gender identity and expression;
\item
  Focus on promoting health and well-being
  rather than solely the reduction of gender dysphoria, which may or may not be
  present;
\item
  Commit to harm reduction approaches
  where appropriate;
\item
  Enable the full and ongoing informed participation of transgender and gender diverse
  people in decisions about their health and
  well-being;
\item
  Improve experiences of health services,
  including those associated with administrative systems and continuity of care.
\end{itemize}

\hypertarget{principles-around-working-towards-improved-health-through-wider-community-approaches}{%
\subsection*{Principles around working towards improved health through wider community approaches}\label{principles-around-working-towards-improved-health-through-wider-community-approaches}}
\addcontentsline{toc}{subsection}{Principles around working towards improved health through wider community approaches}

\begin{itemize}
\tightlist
\item
  Put people in touch with communities and
  peer support networks;
\item
  Support and advocate for clients within
  their families and communities (schools,
  workplaces, and other settings) where
  appropriate.
\end{itemize}

We have already cited research detailing the broad
range of challenges TGD people may face; social
economic and legal obstacles, as well those related
to health care access. While overall health care services are diverse across the world (in terms of availability, accessibility, and quality), those services
available to TGD people are often inadequate.
Numerous reports from diverse regions worldwide
show, while TGD people may report positive health
care experiences, many others do not (e.g., Callander
et al., 2019; Costa, da Rosa Filho et al., 2018; Do
et al., 2018; Gourab et al., 2019; Health Policy
Project et al., 2015; Liu et al., 2020; Motmans et al.,
2017; Muller et al., 2019; PAHO, 2014; Reisner et al.,
2021; Strauss et al., 2017; TGEU, 2017). Mainstream
health care options often do not meet their needs
for general, sexual, or gender-affirming health care.
Standard patient management procedures at clinics
and hospitals often fail to recognize the gender identities of their TGD patients (including where outside
of the binary their patients identify). Patients may
be housed in wards that are gender inappropriate
for them, putting them at risk of sexual harassment.
TGD patients often encounter unsupportive or hostile attitudes from HCPs and ancillary staff and may
even be refused service. Of great concern, HCPs in
some parts of the world are involved in gender identity change efforts of the sort described earlier in
this chapter.

Throughout the world, there are many other
barriers to the provision of gender-affirming
health care. Health care professionals may often
be unwilling to provide the services TGD people
seek. In some countries, there may be laws or
regulations inhibiting or preventing them from
doing so. When general practitioners and other
health care providers do not have access to clear
guidelines in their own language, they may be
deterred from providing services. Even in situations where health care is available, patients may
find it is difficult to access because of distance,
gatekeeping practices, supply and demand issues
that result in long wait lists or cost increases.
Indeed, gender-affirming procedures may not be
incorporated into a universal health care provision or be covered by private insurance, even
though similar procedures may be covered for
cisgender patients.

For all these reasons, many TGD people avoid
formal health care services whenever they can.
Their own communities commonly fill the void,
acting as important resources for their members.
They provide social and emotional support, often
in an otherwise hostile environment. In addition,
they often act as reservoirs of shared information
about available options for health care, including
parallel and informal health care options outside
of (and more accessible and affordable than)
mainstream medicine. As we saw earlier in this
chapter, this often includes sharing of information
about silicone and other injectable substances for
bodily transformation and about hormones that
are self-administered without necessary medical
monitoring or supervision. WHO notes TGD
individuals who self-administer gender-affirming
hormones would benefit from access to
evidence-based information, quality products, and
sterile injection equipment (WHO, 2021). Access
to such information can form part of a broader
harm reduction approach (e.g., Idrus \&
Hyman, 2014).

Putting the important core principles outlined
above into practice can improve health care experiences and promote respect for TGD people in
all local contexts. This can occur regardless of
the realities of a health care system (including
the cultural, social, legal, economic context in
which health care is provided), the level of provision available, or the TGD people seeking such
services.

\hypertarget{population-estimates}{%
\chapter{Population Estimates}\label{population-estimates}}

\hypertarget{summary-of-reported-proportions-of-tgd-people-in-the-general-population}{%
\section*{Summary of reported proportions of TGD people in the general population}\label{summary-of-reported-proportions-of-tgd-people-in-the-general-population}}
\addcontentsline{toc}{section}{Summary of reported proportions of TGD people in the general population}

\begin{itemize}
\tightlist
\item
  Health systems-based studies: 0.03--0.1\%
\item
  Survey-based studies of adults: 0.3--0.5\% (transgender), 0.3--4.5\% (all TGD)
\item
  Survey-based studies of children and adolescents: 1.2--2.7\% (transgender), 2.5--8.4\% (all TGD)
\end{itemize}

In the previous edition of its Standards of Care,
Version 7, World Professional Association for
Transgender Health (WPATH) identified only a
small number of articles attempting to estimate
the size of the transgender and gender diverse
(TGD) population and characterized the
state-of-the-science as ``a starting point'' requiring
further systematic study (Coleman et al., 2012).
Since then, the literature on this topic has
expanded considerably as evidenced by a number
of recent reviews that have sought to synthesize
the available evidence (Arcelus et al., 2015; Collin
et al., 2016; Goodman et al., 2019; Meier \&
Labuski, 2013; Zhang et al., 2020).

In reviewing epidemiologic data pertaining to
the TGD population, it may be best to avoid the
terms ``incidence'' and ``prevalence.'' Avoiding
these and similar terms may preclude inappropriate pathologizing of TGD people (Adams
et al., 2017; Bouman et al., 2017). Moreover, the
term ``incidence'' may not be applicable in this
situation because it assumes TGD status has an
easily identifiable time of onset, a prerequisite
for calculating incidence estimates (Celentano \&
Szklo, 2019). For all the above reasons, we recommend using the terms ``number'' and ``proportion'' to signify the absolute and the relative size
of the TGD population.

Perhaps the most important consideration in
reviewing this literature is the variable definition
applied to the TGD population (Collin et al.,
2016; Meier \& Labuski, 2013). In clinic-based
studies, the data on TGD people are typically
limited to individuals who received
transgender-related diagnoses or counseling or
those who requested or underwent gender-affirming
therapy, whereas survey-based research typically
relies on a broader, more inclusive definition
based on self-reported gender identities.

Another methodological consideration in
assessing the size and distribution of the TGD
population is the need to understand what constitutes the sampling frame. As noted in recent
reviews (Goodman et al., 2019; Zhang et al.,
2020), many of the published studies, especially
those conducted more than a decade ago, first
assessed the number of patients seen at a particular clinical center and then divided that number
by an approximated population size. This was
unlikely to produce an accurate estimate because
the numerator in the calculations is not necessarily included in the denominator, and the true
size of the denominator often remains unknown.

With these considerations in mind, it is advisable to focus specifically on recent (published
within the last decade) peer-reviewed studies that
utilized sound methodology in identifying TGD
people within a well-defined sampling frame. For
all of the above reasons, the present chapter is
focused on studies that met the following inclusion criteria 1) appeared in press in 2009 or later;
2) used a clear definition of TGD status; 3) calculated proportions of TGD people based on a
well-defined population denominator; and 4)
were peer-reviewed. These types of studies can
provide more accurate contemporary estimates.

The available studies can be assigned into three
groups 1) those that reported proportions of TGD
people among individuals enrolled in large health
care systems; 2) those that presented results from
population surveys of predominantly adult participants; and 3) those that were based on surveys
of youth conducted in schools. Of these three
categories, the most informative and methodologically sound studies are summarized below.
Additional details about these and other similar
studies can be found in recent literature reviews
(Goodman et al., 2019; Zhang et al., 2020).

Among studies that estimated the size of the
TGD population enrolled in large health care
systems, all were conducted in the US, and all
relied on information obtained from electronic
health records. Four of those health system-based
studies relied exclusively on diagnostic codes to
ascertain the TGD population; two studies
(Blosnich et al., 2013; Kauth et al., 2014) used
data from the Veterans Health Affairs system,
which provides care to over 9 million people,
and two studies (Dragon et al., 2017; Ewald et al.,
2019) used claims data from Medicare, the federal
health insurance program that primarily covers
people 65 years of age or older. The proportions
of TGD people reported in these diagnostic
code-based studies ranged from approximately
0.02\% to 0.03\%. Another more recent publication
also used Medicare data along with commercial
insurance claims to identify TGD people and
applied expanded inclusion criteria to supplement
diagnostic codes with information on procedures
and hormone therapy (Jasuja et al., 2020). Using
this methodology, the proportion of TGD people
among all persons enrolled in the participating
health plans was 0.03\%. The sixth health
systems-based study (Quinn et al., 2017) was
conducted at Kaiser Permanente plans in the
states of Georgia and California; these plans provide care to approximately 8 million members
enrolled through employers, government programs, or individually. The TGD population in
the Kaiser Permanente study was ascertained
across all age groups using both diagnostic codes
and free-text clinical notes. The proportions of
TGD people identified at Kaiser Permanente were
higher than the corresponding proportions
reported in the Veterans Health Affairs and
Medicare studies with the most recent estimates
ranging from 0.04 to 0.08\%.

In contrast to results from the health
system-based studies, findings from surveys that
relied on self-reported TGD status produced
much higher estimates. Two US studies took
advantage of the Behavioral Risk Factor
Surveillance Study (BRFSS), which is an annual
telephone survey conducted in all 50 states and
US territories (Conron et al., 2012; Crissman
et al., 2017). The first study used data from the
2007--2009 BRFSS cycles in the state of
Massachusetts, and the second study used the
2014 BRFSS data from 19 states and the territory
of Guam. Both studies reported that approximately 0.5\% of adult participants (at least 18
years of age) responded \emph{``Yes''} to the question \emph{``Do
you consider yourself to be transgender?''}

An internet-based survey administered to a
sample of the Dutch population 15--70 years of
age (Kuyper \& Wijsen, 2014) asked participants
to score the following two questions using a
5-point Likert scale: \emph{``Could you indicate to which
degree you psychologically experience yourself as
a man?''} and \emph{``Could you indicate to which degree
you psychologically experience yourself as a
woman?''} The respondents were considered ``gender ambivalent'' if they gave the same score to
both statements and ``gender incongruent'' when
they reported a lower score for their sex assigned
at birth than for their gender identity. The proportions of participants reporting incongruent
and ambivalent gender identity were 1.1\% and
4.6\%, respectively, for persons who were assigned
male at birth (AMAB), and 0.8\% and 3.2\%,
respectively, for persons assigned female at
birth (AFAB).

A similarly designed study estimated the proportion of TGD residents in the Flanders region
of Belgium using a sample drawn from the country's National Register (Van Caenegem, Wierckx
et al., 2015). Participants were asked to score the
following statements: ``I feel like a woman'' and
``I feel like a man'' on a 5-point Likert scale. Using
the same definitions applied in the Dutch study
(Kuyper \& Wijsen, 2014), the proportion of gender incongruent individuals was 0.7\% for AMAB
people and 0.6\% for AFAB people. The corresponding estimates for gender ambivalence among
AMAB and AFAB people were 2.2\% and 1.9\%,
respectively.

A more recent population-based study evaluated the proportion of TGD people among
approximately 50,000 adult residents of Stockholm
County, Sweden (Åhs et al., 2018). The numerator
was determined by asking participants the following question: ``I would like hormones or surgery to be more like someone of a different sex.''
Two additional items were designed to identify
individuals experiencing gender incongruence: ``I
feel like someone of a different sex'' and ``I would
like to live as or be treated as someone of a different sex.'' The need for either hormone therapy
or gender-affirming surgery was reported by 0.5\%
of participants. Individuals who expressed feeling
like someone of a different sex and those who
wanted to live as or be treated as a person of
another sex constituted 2.3\% and 2.8\% of the
total sample, respectively.

Population-based data outside of North
America and Western Europe are less common.
One recent study offers valuable data from a large
representative survey of 6,000 adults in Brazil
(Spizzirri et al., 2021). Gender identity of participants was assessed based on the following three
questions 1) \emph{``Which of the following options best
describes how you currently feel?''} (Options: I feel
I am a man, I feel I am a woman, and I feel I
am neither a man nor a woman); 2) \emph{``What is
the sex on your birth certificate?''} (Options: male,
female, and undetermined); and 3) \emph{``Which of
these situations do you most closely relate to?''}
(Options: I was born male, but I have felt female
since childhood; I was born female, but I have
felt male since childhood; I was born male, and
I feel comfortable with my body; I was born
female, and I feel comfortable with my body).
Based on the responses to these three questions,
the authors determined 1.9\% of the survey
respondents were TGD (0.7\% defined as transgender, and 1.2\% defined as nonbinary).

The literature on the population proportions
of TGD youth (persons under 19 years of age)
includes several survey studies conducted in
schools. A 2012 national cross-sectional survey
in New Zealand collected information on TGD
identity among high school students (Clark et al.,
2014). Among over 8,000 survey participants,
1.2\% self-identified as TGD and 2.5\% reported
they were not sure. Another study of schoolchildren was based on a 2016 survey of 9th and 11th
grade students (ages 14--18 years) in the US state
of Minnesota (Eisenberg et al., 2017). Of the
nearly 81,000 survey respondents, 2.7\% reported
being TGD. A more recent study (Johns et al.,
2019) presented results of the Youth Risk Behavior
Survey (YRBS), which is conducted biennially
among local, state, and nationally representative
samples of US high school students in grades
9--12 (approximate age range 13--19 years). The
2017 YRBS cycle was carried out in 10 states and
9 large urban areas and included the following
sequence: \emph{``Some people describe themselves as
transgender when their sex at birth does not match
the way they think or feel about their gender. Are
you transgender?\emph{'' Among nearly 120,000 participants across the 19 sites, 1.8\% responded }``Yes,
I am transgender,''} and 1.6\% responded \emph{``I am not
sure if I am transgender.''}

Another recently published school-based study
in the US presented results of a 2015 survey
conducted in Florida and California with the aim
of identifying gender diverse children and adolescents in a sample of just over 6,000 students
in grades 9--12 (Lowry et al., 2018). ``High
gender-nonconforming'' was used to define
AMAB children who reported being very/mostly/
somewhat feminine or AFAB children who
reported being very/mostly/somewhat masculine.
Based on these definitions, the proportions of
TGD participants were reported to be 13\% among
AMAB students, 4\% among AFAB students, and
8.4\% overall.

Only one study examined the proportion of
self-identified TGD children in a younger age
group. Shields et al.~analyzed the data from a
2011 survey of 2,700 students in grades 6--8 (age
range 11--13 years) across 22 San Francisco public
middle schools (Shields et al., 2013). Thirty-three
children self-identified as TGD based on the
question ``What is your gender?'' where the possible responses were ``female, male, or transgender.'' The resulting proportion of transgender
survey respondents was 1.3\%. However, this definition would exclude TGD persons self-identifying
as nonbinary and those who do not explicitly
identify as transgender.

Taken together, these data indicate among
health system-based studies that relied on diagnostic codes or other evidence documented in
the medical records (Blosnich et al., 2013; Dragon
et al., 2017; Ewald et al., 2019; Kauth et al., 2014;
Quinn et al., 2017), the proportions of TGD people reported in recent years (2011--2016) ranged
from 0.02\% to 0.08\%. By contrast, when the TGD
status was ascertained based on self-report, the
corresponding proportions were orders of magnitude higher and reasonably consistent, if the
studies used similar definitions. When the surveys specifically inquired about ``transgender''
identity, the estimates ranged from 0.3\% to 0.5\%
among adults and from 1.2\% to 2.7\% in children
and adolescents. When the definition was
expanded to include broader manifestations of
gender diversity, such as gender incongruence or
gender ambivalence, the corresponding proportions were higher: 0.5\% to 4.5\% among adults
and 2.5\% to 8.4\% among children and
adolescents.

As reviewed elsewhere (Goodman et al., 2019),
another noteworthy observation is the continuous
increase in both the size and the composition of
the TGD population with upward trends in the
proportion of TGD people observed in health
care systems, through population-based surveys,
as well as in the data on legal gender recognition.
The higher estimates observed in more recent
literature support some of the previous publications indicating the size of TGD population was
likely underestimated in earlier studies (Olyslager
\& Conway, 2008).

The temporal trends in AMAB to AFAB ratio
have also been reported in studies analyzing
referrals to clinics as well as data from integrated
health systems; this ratio has changed from predominantly AMAB in previous decades to predominantly AFAB in recent years, especially
among TGD youth (Aitken et al., 2015; de Graaf,
Carmichael et al., 2018; de Graaf, Giovanardi
et al.~2018; Steensma et al., 2018; Zhang et al.,
2021). The trend towards a greater proportion of
TGD people in younger age groups and the
age-related differences in the AMAB to AFAB
ratio likely represent the ``cohort effect,'' which
reflects sociopolitical advances, changes in referral
patterns, increased access to health care and to
medical information, less pronounced cultural
stigma, and other changes that have a differential
impact across generations (Ashley 2019d; Pang
et al., 2020; Zhang et al., 2020).

Despite recent improvements in the quality of
published studies, an important limitation of the
existing literature is the relative paucity of
peer-reviewed publications from regions outside
of Western Europe or North America. Some of
the relevant information on global estimates can
be obtained from reports supported by the governments or non-governmental organizations
(Fisher et al., 2019; Kasianczuk \& Trofymenko,
2020), but these reports may be difficult to systematically identify and evaluate until they appear
in peer-reviewed literature. Other barriers to evaluating the global distribution of the TGD populations include inadequate access to demographic
data and over-representation of English-language
journals in the world literature.

These limitations notwithstanding, the available
highest-quality data clearly indicate TGD people
represent a sizable and growing proportion of the
general population. Based on the credible evidence available to date, this proportion may range
from a fraction of a percent to several percentage
points depending on the inclusion criteria, age
group, and geographic location. Accurate estimates of the proportion, distribution, and composition of the TGD population as well as a
projection of resources required to adequately
support the health needs of TGD people should
rely on systematically collected high-quality data,
which are now increasingly available. Continuous
and routine collection of these data is needed to
decrease variability and minimize over- and
under-estimation of the reported results. For
example, far more accurate and precise estimates
should become available when population censuses begin systematically collecting and reporting data on sex assigned at birth and gender
identity, including asexual and nonbinary categories, using the now well-validated two-step
method. The first such census-based estimate was
released by the national statistical office of
Canada. Based on the 2021 census data, 100,815
of 30.5 million Canadians self-identified as transgender or nonbinary; this accounted for 0.33\%
of the population 15 years of age or older
(Statistics Canada, 2022). Consistent with the
published literature, the proportions of transgender and nonbinary people were much higher for
Generation Z (born between 1997 and 2006,
0.79\%) and millennials (born between 1981 and
1996, 0.51\%) than for Generation X (born
between 1966 and 1980, 0.19\%), baby boomers
(born between 1946 and 1965, 0.15\%), and the
Interwar and Greatest Generations (born in 1945
or earlier, 0.12\%). While these results represent
the highest quality data available to date, it is
not clear how the population proportions reported
in Canada may compare with those in other
countries. The variability in the definitions of
what constitutes the TGD population and the
differences in data collection methods can be
reduced further by improving international
collaborations.

\hypertarget{education}{%
\chapter{Education}\label{education}}

This chapter will provide a general review of the
literature related to education in transgender and
gender diverse (TGD) health care.
Recommendations are offered at governmental,
nongovernmental, institutional, and provider levels with the goal of increasing access to competent, compassionate health care. In turn, this
increased access should improve health outcomes
in TGD populations. As this is a novel chapter
in the World Professional Association for
Transgender Health (WPATH) Standards of Care,
the intent is to lay the groundwork for the education area and invite a broader and deeper discussion among educators and health
professionals.

Health professionals involved in transgender
care encompass a broad range of disciplines.
Health professional education varies considerably
by country or region in terms of structure,
licensure, and policy. Published literature on
education in TGD health care is predominantly
from North America, Europe, Australia and New
Zealand. This chapter does not provide a review
of the education literature for each discipline,
the needs specific to each discipline (which can
be found in the relevant chapters), or the needs
specific to each country/region's health education system. Greater understanding and research
are needed on the intersection of health education systems, licensure, and transgender health
across the world.

On a global level, TGD health education is
imperative if national and international health disparities are to be addressed. Cultural competency
related to TGD communities continues to be lacking. The World Bank Group (2018) reports widespread discrimination, harassment, violence, and
abuse affecting TGD people. They also report
TGD people face the highest rates of violence and
discrimination (World Bank Group, 2018).
Although many higher income countries have
national antidiscrimination laws with gender identity as a protected characteristic, discrimination in
the workplace, in education, and in health care
remains problematic (World Bank Group, 2018).

Across disciplines, curricula at all levels---
undergraduate, graduate, residency, or continuing education---historically have ignored TGD
cultural or clinical education. The Joint
Commission (US) has recommended health care
organizations ``provide educational programs and
forums that support the unique needs of the
LGBT community'' and ``offer educational opportunities that address LGBT health issues'' (The
Joint Commission, 2011). However, this is not
enforced.

On an individual level, several questions need
answers. What type of education interventions
can most effectively address transphobia and
lead to long-standing changes in attitudes? What
interventions translate into increasing the number of care providers in this area as well as the
number of TGD people receiving care? Does
clinical exposure increase the confidence of providers over time? What educational interventions
lead to improved health outcomes in the TGD
population and, if so, when and how did these
interventions accomplish this? Although health
professions have begun to incorporate TGD
health into education using a variety of modalities and at varying levels of training, efforts
differ by health profession and are neither systemic nor systematic in nature (e.g., Brennan
et al., 2012; Chinn, 2013; Eliason et al., 2010;
Lim et al., 2015; Obedin-Maliver et al., 2011;
Rondahl, 2009).

Attaining cultural humility with the full appreciation of the intersectionality of humanity is an
ultimate educational goal. That said, this initial
call for education is focused on building the
foundation in cultural awareness and cultural
competency that is currently weak or non-existent
in much of the world.

All the statements in this chapter have been
recommended based on a thorough review of
evidence, an assessment of the benefits and
harms, values and preferences of providers and
patients, and resource use and feasibility. In
some cases, we recognize evidence is limited
and/or services may not be accessible or
desirable.

\hypertarget{recommendation-4.1-we-recommend-all-personnel-working-in-governmental-nongovernmental-and-private-agencies-receive-cultural-knowledge-training-focused-on-treating-transgender-and-gender-diverse-individuals-with-dignity-and-respect.}{%
\section*{Recommendation 4.1: We recommend all personnel working in governmental, nongovernmental, and private agencies receive cultural-knowledge training focused on treating transgender and gender diverse individuals with dignity and respect.}\label{recommendation-4.1-we-recommend-all-personnel-working-in-governmental-nongovernmental-and-private-agencies-receive-cultural-knowledge-training-focused-on-treating-transgender-and-gender-diverse-individuals-with-dignity-and-respect.}}
\addcontentsline{toc}{section}{Recommendation 4.1: We recommend all personnel working in governmental, nongovernmental, and private agencies receive cultural-knowledge training focused on treating transgender and gender diverse individuals with dignity and respect.}

Article 1 of the United Nations Universal
Declaration of Human Rights states, ``All human
beings are born free and equal in dignity and
rights'' (United Nations, 1948). Only recently has
this fundamental statement included the recognition that TGD rights are human rights (UNOCHR,
2018). Globally, training at all levels about TGD
communities continues to be lacking. As recently
as 2002, only 3\% of Fortune 500 companies had
antidiscrimination protection for TGD employees,
and none offered insurance coverage for
gender-affirming health care.
(Human Rights
Campaign Foundation, 2017). By 2022, 91\% of
Fortune 500 companies included gender identity
in US non-discrimination policies, and 66\% offered
TGD-inclusive insurance coverage. However, only
72\% provide any form of lesbian, gay, bisexual,
transgender and queer/questioning (LGBTQ) cultural knowledge training for their workforce
(Human Rights Campaign Foundation, 2022). This
lack of understanding fosters discrimination across
the board. Taken together, these inconsistencies
negatively affect the health of individuals and communities and exacerbate the health disparities and
inequities they face. In Britain, only 28\% of TGD
workers felt the senior leadership were committed
to TGD equality; only 21\% of TGD employees
would consider reporting transphobic harassment
in the workplace (Stonewall, 2018). For those who
are openly TGD, 34\% were excluded by their
co-workers, 35\% were abused by customers, 24\%
were denied promotion due to their gender identity, and 11\% were fired (Stonewall, 2018). In
southeastern Europe, the World Bank stated there
is widespread discrimination, harassment, violence,
and abuse, and TGD people in that region faced
the highest rates of violence and discrimination
(World Bank Group, 2018). Often the discrimination went unreported with 60\% of individuals not
filing a report because of a lack of faith the complaint would be addressed, a fear of further discrimination or ridicule, and a reluctance to be
outed (World Bank Group, 2018). Although many
countries in the region have national antidiscrimination laws with gender identity as a protected
characteristic, discrimination in the workplace, in
education, and in health care remains problematic
(World Bank Group, 2018). It is the responsibility
of the governmental, nongovernmental, and private
agencies in these countries with anti-discrimination
laws to ensure the rights of the TGD population.
They are, therefore, obligated to find ways in
which discrimination and stigma can be decreased.
One of these is through education. Local cultures
that foster anti-TGD attitudes are often a barrier
to this needed education. Although cultural competency trainings have led to equivocal results,
Shepherd (2019) recommends that providing cultural knowledge training that prioritizes local cultural issues and focuses on the values of openness,
non-judgment, and responsiveness may lead to the
desired results. Implementing cultural knowledge
training requires a leadership willing to prioritize
the training and to dedicate the time, money, and
human capital to delivering initial and ongoing
training.

\hypertarget{recommendation-4.2-we-recommend-all-members-of-the-health-care-workforce-receive-cultural-knowledge-training-focused-on-treating-transgender-and-gender-diverse-individuals-with-dignity-during-orientation-and-as-part-of-annual-or-continuing-education.}{%
\section*{Recommendation 4.2: We recommend all members of the health care workforce receive cultural-knowledge training focused on treating transgender and gender diverse individuals with dignity during orientation and as part of annual or continuing education.}\label{recommendation-4.2-we-recommend-all-members-of-the-health-care-workforce-receive-cultural-knowledge-training-focused-on-treating-transgender-and-gender-diverse-individuals-with-dignity-during-orientation-and-as-part-of-annual-or-continuing-education.}}
\addcontentsline{toc}{section}{Recommendation 4.2: We recommend all members of the health care workforce receive cultural-knowledge training focused on treating transgender and gender diverse individuals with dignity during orientation and as part of annual or continuing education.}

Across disciplines, curricula at all levels---
undergraduate, graduate, residency, or continuing
education---historically have ignored TGD cultural or clinical education. Factors contributing
to this lack of inclusion include lack of faculty
knowledge, experience, comfort with the subject
matter, faculty bias, limited space within the
existing curriculum, and lack of guidance on how
to integrate the topics (McDowell \& Bower, 2016).
Research into the lack of and the need for such
education does not specifically address TGD
health concerns. Rather, the existing literature
subsumes TGD health education within the
broader discussion of the lack of LGBTQ-focused
cultural and clinical-competency training. As an
example, nursing baccalaureate programs included
only an average of 2.12 hours of instruction on
LGBTQ health (Lim et al., 2015). A fair assumption is that the amount of time devoted to
TGD-specific health issues constituted only a
fraction of this time.

Within the broader context of LGBTQ competency, the lack of TGD cultural- and
clinical-competency training is a long-known
shortfall of health care education (Aldridge et al.,
2021). In the US, the Department of Health and
Human Services' Healthy People 2020, (United
States Department of Health and Human Services
(2013, April 10)), the National Academy of
Medicine (The Institute of Medicine, 2011), and
the Joint Commission (The Joint Commission,
2011) all recognized lack of education negatively
impacts the ability of LGBTQ people, including
TGD individuals, to obtain appropriate, medically
necessary care. The UK's House of Commons
Women and Equalities Committee found lack of
education contributed to TGD health disparities
in the National Health Service (House of
Commons Women and Equalities Committee,
2015, December 8). The lack of TGD health care
education has been identified in the US
(Obedin-Maliver et al., 2011), UK (Tollemache
et al., 2021), South Africa (de Vries et al., 2020;
Taylor et al., 2018; Wilson et al., 2014), Canada
(Bauer et al., 2014), Australia (Riggs \&
Bartholomaeus, 2016), Sweden, Spain, Serbia,
Poland (Burgwal et al., 2021), and Pakistan
(Martins et al., 2020) among other countries.

In addition to developing curriculum, Shepherd
(2022) states both clinical and organizational
components are necessary to improve clinical
encounters and consumer satisfaction. On an
organizational level, it must be feasible as well
as locally and practically oriented (Shepherd,
2022). On an individual level, in addition to
knowledge training, health care professionals are
better served employing generic traits that focus
on the values of openness, non-judgment, and
responsiveness (Shepherd, 2018).

\hypertarget{recommendation-4.3-we-recommend-institutions-involved-in-the-training-of-health-professionals-develop-competencies-and-learning-objectives-for-transgender-and-gender-diverse-health-within-each-of-the-competency-areas-for-their-specialty.}{%
\section*{Recommendation 4.3: We recommend institutions involved in the training of health professionals develop competencies and learning objectives for transgender and gender diverse health within each of the competency areas for their specialty.}\label{recommendation-4.3-we-recommend-institutions-involved-in-the-training-of-health-professionals-develop-competencies-and-learning-objectives-for-transgender-and-gender-diverse-health-within-each-of-the-competency-areas-for-their-specialty.}}
\addcontentsline{toc}{section}{Recommendation 4.3: We recommend institutions involved in the training of health professionals develop competencies and learning objectives for transgender and gender diverse health within each of the competency areas for their specialty.}

Each health profession has its own educational
institutions, administrative, and licensing bodies,
which vary by country and specialization within
the profession. No major health professional organizations, educational institutions, or licensing
bodies appear to require training in TGD health.
While these organizations increasingly recommend including LGBTQ intersex health, rarely
do they specify competencies, skills, or learning
objectives for working with TGD people within
their specialty. Published material on health professional education in TGD health is focused
primarily on nursing, medicine, and mental
health and is predominantly from North America,
Europe, Australia, and New Zealand. An increased
understanding of transgender health and medical/
health professional education systems and requirements globally is essential.

Despite the increasing visibility of TGD people,
access to knowledgeable and culturally- competent health professionals remain an overwhelming
need around the world (James et al., 2016; Lerner
et al., 2020; Müller, 2017). Lack of knowledgeable
providers is a major barrier to gender-affirming
care for transgender persons (Puckett et al., 2018;
Safer et al., 2016) and contributes to large health
disparities (Giffort \& Underman, 2016; Reisman
et al., 2019). The lack of adequate professional
education in TGD health is a global problem (Do
\& Nguyen, 2020; Martins et al., 2020;
Parameshwaran et al., 2017) that occurs at all
levels of training (Dubin et al., 2018) and traverses health disciplines (Glick et al., 2020;
Gunjawate et al., 2020; Johnson \& Federman,
2014) and medical specialties (Fung et al., 2020;
Korpaisarn and Safer, 2018).

Challenges remain as studies to date have
small sample sizes, involve one-time training,
include multiple disciplines at multiple career
levels, focus on short-term outcomes, and often
cover all LGBTQI topics rather than
TGD-specific ones that are usually acquired
post-licensure and are not the focus of most
currently studied educational interventions
(Dubin et al., 2018).

To successfully implement the recommendations, institutions may need to consider
developing 1) systemic and systematic approaches
to developing and implementing competencies
for each health discipline across the professional
lifespan; 2) standardized assessments for learners,
with input from the TGD community; and 3)
allotment of curricular resources, including
trained faculty, as well as time in accordance with
clear, consensual learning objectives (Dubin et al.,
2018; Pratt-Chapman, 2020). In addition, evaluations of these interventions should not only
focus on outcomes but also strive to understand
how, when, and why these outcomes are occurring (Allen et al., 2021).

\hypertarget{assessment-of-adults}{%
\chapter{Assessment of Adults}\label{assessment-of-adults}}

This chapter provides guidance for the assessment
of transgender and gender diverse (TGD) adults
who are requesting medically necessary
gender-affirming medical and/or surgical treatments
(GAMSTs) to better align their body with their
gender identity (see medically necessary statement
in Chapter 2---Global Applicability, Statement 2.1).

TGD adults are people at or above the age of
majority in their country, who have some form
of gender diversity. The developmental elements
of the adolescent chapter, including the importance of parental/caregiver involvement, may be
relevant for the care of young adults too, even if
they are above the age of majority.

This chapter includes all forms of gender identities and transitions including, but not limited
to, male, female, gender diverse, nonbinary, agender, and eunuch. The population of TGD adults
is heterogeneous and will vary according to their
clinical need, biological, psychological, and social
situations, as well as their access to health care.
As such, any assessment for GAMSTs will need
to be adapted to the scientific, clinical, and community knowledge base of the presenting gender
identity as well as local circumstances. This chapter recognizes individuals may experience different local levels of clinical or regulatory oversight
when the state or others are providing health care.

An individual's gender identity is an internal
identification and experience. The role of the
assessor is to assess for the presence of gender
incongruence and identify any co-existing mental
health concerns, to offer information about
GAMSTs, to support the TGD person in considering the effects/risks of GAMSTs, and to assess
if the TGD person has the capacity to understand
the treatment being offered and if the treatment
is likely to be of benefit. The assessor can also
assist a TGD person to consider choices that
could improve their GAMST outcomes. The
GAMST assessment approach described in this
chapter recognizes the lived experience and
self-knowledge of the TGD person and the clinical knowledge of the assessing health care professional (HCP). Consequently, with this approach,
the decision to move forward with GAMSTs is
shared between the TGD person and the
assessing HCP, with both playing a key part in
collaborative decision-making.

Some systems use a model of care for TGD
adults seeking GAMSTs that prioritizes the TGD
adult as the decision maker with the HCP acting
as an advisor, barring serious contraindications.
These models are used when considering hormone therapy rather than surgery and are often
called ``informed consent'' models (Deutsch, 2011,
2016a). Many such models utilize an abbreviated
assessment that focuses primarily on the ability
of a TGD person to grant informed consent and
to utilize information about GAMSTs to inform
their medical decision-making. There is significant variability in such models across jurisdictions, systems, and HCPs (Deutsch, 2011; Morenz
et al., 2020). Informed consent models have been
used for some time for hormone prescription in
many local settings.

This chapter is intended to offer flexible global
guidance that must be adapted to local circumstances. HCPs will need to determine which
assessment approaches best meet the needs in
their local settings. The evaluation of these
approaches is best undertaken in collaboration
with TGD people.

Since TGD people represent a diverse array of
gender identities and expressions and have differing needs for GAMSTs, no single assessment
process will fit every person or every situation.
Some TGD people may need a comparatively
brief assessment process for GAMSTs. For TGD
adults with a complex presentation or for those
who are requesting less common treatments or
treatments with limited research evidence, more
comprehensive assessments with different members of a multidisciplinary team will be required.
Assessments may be in person or through telehealth. While psychometric assessment tools have
been used in some instances, they are not a
required part of the assessment for GAMSTs.
Counseling or psychotherapy can be helpful when
requested by a TGD person. However, counseling
or psychotherapy specifically focused on their
TGD identity is not a requirement for the assessment or initiation of GAMSTs. Genital exams are
not a prerequisite for initiation of GAMTs and
should be performed only when clinically
indicated.

GAMSTs can be delivered in diverse settings.
Settings will depend on available health care systems within each country and may include nationalized/public health care, private sector settings,
community health care settings, and charitable
institutions. Local and regional circumstances may
therefore influence the availability of health care.
Regardless of the setting, health care offered to
TGD people should be of the highest possible
quality. World Professional Organization for
Transgender Health (WPATH) advocates for
assessment and treatment to be readily available.
Access to assessment and treatment for TGD
people seeking GAMSTs is critical given the clear
medical necessity of these interventions and the
profound benefits they offer to TGD people
(Aldridge et al., 2020; Byne et al., 2012). The guidance in this chapter will need to be adapted
according to local, as well as individual, clinical,
and social circumstances.

The statements below are based on significant
background literature, including literature demonstrating the strong positive impact of access to
GAMSTs; available empirical evidence; a favorable
risk-benefit ratio; and consensus of professional
best practice. The empirical evidence base for the
assessment of TGD adults is limited. It primarily
includes an assessment approach that uses specific criteria that are examined by an HCP in
close cooperation with a TGD adult and does
not include randomized controlled trials or
long-term longitudinal research (Olsen-Kennedy
et al., 2016). This is understandable given the
complexity and ethical considerations of allocating patients in need of care to different assessment groups and the lack of funding for research
and other resources to assess long-term outcomes
of assessment approaches.

The creation of this guidance has been a complex
undertaking. The criteria in this chapter have been
significantly revised from SOC-7 to reduce requirements and unnecessary barriers to care. It is hoped
that future research will explore the effectiveness
of this model as well as evolving assessment models
for hormone therapy and for surgery that will allow
continued improvements to be made.

All the statements in this chapter have been
recommended based on a thorough review of
evidence, an assessment of the benefits and
harms, values and preferences of providers and
patients, and resource use and feasibility. In some
cases, we recognize evidence is limited and/or
services may not be accessible or desirable.

\hypertarget{statement-5.1-we-recommend-health-care-professional-assessing-transgender-and-gender-diverse-adults-for-gender-affirming-treatments}{%
\section*{Statement 5.1: We recommend health care professional assessing transgender and gender diverse adults for gender-affirming treatments:}\label{statement-5.1-we-recommend-health-care-professional-assessing-transgender-and-gender-diverse-adults-for-gender-affirming-treatments}}
\addcontentsline{toc}{section}{Statement 5.1: We recommend health care professional assessing transgender and gender diverse adults for gender-affirming treatments:}

\hypertarget{statement-5.1.a-are-licensed-by-their-statutory-body-and-hold-at-a-minimum-a-masters-degree-or-equivalent-training-in-a-clinical-field-relevant-to-this-role-and-granted-by-a-nationally-accredited-statutory-institution.}{%
\subsection*{Statement 5.1.a: Are licensed by their statutory body and hold, at a minimum, a master's degree or equivalent training in a clinical field relevant to this role and granted by a nationally accredited statutory institution.}\label{statement-5.1.a-are-licensed-by-their-statutory-body-and-hold-at-a-minimum-a-masters-degree-or-equivalent-training-in-a-clinical-field-relevant-to-this-role-and-granted-by-a-nationally-accredited-statutory-institution.}}
\addcontentsline{toc}{subsection}{Statement 5.1.a: Are licensed by their statutory body and hold, at a minimum, a master's degree or equivalent training in a clinical field relevant to this role and granted by a nationally accredited statutory institution.}

TGD people, as with all other people seeking
health care, should have the highest quality of care
accessible that is commensurate with the quality
of care provided to all people utilizing health services (The Yogyakarta Principles, 2017). As this
will vary around the globe, the nature of the professional completing an assessment for GAMSTs
will vary according to the nature of health care in
the local setting as well as the regulatory requirements set by licensing and registration boards. It
is important the health care provided includes an
assessment conducted by a competent, statutorily
regulated HCP who has the competence to identify
gender incongruence and conditions that can be
mistaken for gender incongruence and who can
support the TGD person throughout the assessment process (RCGP, 2019). Assessors must be able
to refer to HCPs licensed to provide GAMSTs.

HCPs should have at a minimum a masters-level
qualification in a clinical field related to transgender health or equivalent further clinical training
and be statutorily regulated; examples include a
mental health professional (MHP), general medical
practitioner, nurse, or other qualified HCP. In some
settings, statutorily regulated HCPs with lower levels of qualification may practice under the clinical
supervision of a qualified HCP who takes ultimate
clinical responsibility for the quality and accuracy
of the completed GAMST assessment. For additional information see Chapter 4---Education.

Accessing a competent, statutorily regulated,
HCP with expertise in GAMST assessment can
sometimes be difficult. Consequently, ensuring
continuity of care and minimizing gaps in accessible care or significantly delayed care (e.g., a
long waiting list) may require that a statutorily
regulated HCP without expertise provide care and
support the assessment of a TGD person for
GAMSTs. Avoiding unnecessary delays in care is
critically important. However, TGD people should
be supported to access care with an experienced
HCP as soon as possible (RCGP, 2019).

Established practice requires the competence to
identify and diagnose gender incongruence
(Hembree et al., 2017; Reed et al., 2016; T'Sjoen
et al., 2020) and the ability to identify differentials
or conditions that may be mistaken as gender
incongruence (Byne et al., 2018; Dhejne et al.,
2016; Hembree et al., 2017). Established practice
also strongly emphasizes the need for ongoing continuing education in the assessment and provision
of care of TGD people (American Psychological
Association, 2015; T'Sjoen et al., 2020). For more
information see Chapter 4---Education.

\hypertarget{statement-5.1.b-for-countries-requiring-a-diagnosis-for-access-to-care-the-health-care-professional-should-be-competent-using-the-latest-edition-of-the-world-health-organizations-international-classification-of-diseases-icd-for-diagnosis.}{%
\subsection*{Statement 5.1.b: For countries requiring a diagnosis for access to care, the health care professional should be competent using the latest edition of the World Health Organization's International Classification of Diseases (ICD) for diagnosis.}\label{statement-5.1.b-for-countries-requiring-a-diagnosis-for-access-to-care-the-health-care-professional-should-be-competent-using-the-latest-edition-of-the-world-health-organizations-international-classification-of-diseases-icd-for-diagnosis.}}
\addcontentsline{toc}{subsection}{Statement 5.1.b: For countries requiring a diagnosis for access to care, the health care professional should be competent using the latest edition of the World Health Organization's International Classification of Diseases (ICD) for diagnosis.}

In countries that
have not implemented the latest ICD, other taxonomies may be used; efforts should be undertaken to utilize the latest ICD as soon as practicable.
In some countries, a diagnosis of gender incongruence may be necessary to access GAMSTs (as
described below). HCPs assessing TGD people in
those countries should be competent to diagnose
gender incongruence using the most current classification system necessary for TGD people to
access GAMSTs. The ICD-11 (WHO, 2019a) is a
classification system that focuses on the TGD
person's experienced identity and any need for
GAMSTs and does not consider a TGD identity
to be a mental illness.

\hypertarget{statement-5.1.c-are-able-to-identify-co-existing-mental-health-or-other-psychosocial-concerns-and-distinguish-these-from-gender-dysphoria-incongruence-and-diversity.}{%
\subsection*{Statement 5.1.c: Are able to identify co-existing mental health or other psychosocial concerns and distinguish these from gender dysphoria, incongruence, and diversity.}\label{statement-5.1.c-are-able-to-identify-co-existing-mental-health-or-other-psychosocial-concerns-and-distinguish-these-from-gender-dysphoria-incongruence-and-diversity.}}
\addcontentsline{toc}{subsection}{Statement 5.1.c: Are able to identify co-existing mental health or other psychosocial concerns and distinguish these from gender dysphoria, incongruence, and diversity.}

Gender diversity is a natural variation in people
and is not inherently pathological (American
Psychological Association, 2015). However, assessment is best provided by an HCP who possesses
some expertise in mental health in order to identify conditions that can be mistaken for gender
incongruence. Such conditions are rare and, when
present, are often psychological in nature (Byne
et al., 2012; Byne et al., 2018; Hembree et al., 2017).

The need to include an HCP with some expertise
in mental health does not require the inclusion of
a psychologist, psychiatrist, or social worker in each
assessment. Instead, a general medical practitioner,
nurse, or other qualified HCP could also fulfill this
requirement if they have sufficient expertise to identify gender incongruence, recognize mental health
concerns, distinguish between these concerns and
gender dysphoria, incongruence, and diversity, assist
a TGD person in care planning and preparation
for GAMSTs, and refer to a mental health professional (MHP), if needed. As discussed in greater
depth in the mental health chapter, MHPs have an
important role to play in the care of TGD people.
For example, the prejudice and discrimination experienced by some TGD people (Robles et al., 2016)
can lead to depression, anxiety, or worsening of
other mental health conditions. In such cases, an
MHP can diagnose, clarify, and treat mental health
conditions. MHPs and HCPs with expertise in mental health are well-placed to assess for GAMSTs, as
well as to support TGD people who require or
request mental health input or support during their
transition. For additional information see Chapter
18---Mental Health.

\hypertarget{statement-5.1.d-are-able-to-assess-capacity-to-consent-for-treatment.}{%
\subsection*{Statement 5.1.d: Are able to assess capacity to consent for treatment.}\label{statement-5.1.d-are-able-to-assess-capacity-to-consent-for-treatment.}}
\addcontentsline{toc}{subsection}{Statement 5.1.d: Are able to assess capacity to consent for treatment.}

An assessment for GAMSTs must include an
examination of the TGD person's ability to consent
to the proposed treatment. Consent requires the
cognitive capacity to understand the risks and
benefits of a treatment and the potential negative
and positive outcomes. It also requires the ability
to retain that information for the purposes of
making the decision (using aids as necessary) as
well as the cognitive ability to use that understanding to make an informed decision (American
Medical Association, 2021; Applebaum, 2007).

Some TGD individuals will have the capacity to
grant consent immediately during the assessment.
Some TGD individuals may need a longer process
to be able to consent through ongoing discussion
and the practice of medical decision-making skills.
The presence of psychiatric illness or mental health
symptoms do not pose a barrier to GAMSTs unless
the psychiatric illness or mental health symptoms
affect the TGD person's capacity to consent to the
specific treatment being requested or affect their
ability to receive treatment. This is especially important because GAMSTs have been found to reduce
mental health symptomatology for TGD people
(Aldridge et al., 2020).

Health care systems can consider GAMSTs for
individuals who may not be able to directly consent if an appropriate legal guardian or
regulator-approved independent decision maker
with the power to determine health care treatment grants consent and confirms the proposed
treatment is in alignment with the TGD individual's needs and wishes.

\hypertarget{statement-5.1.e-have-experience-or-be-qualified-to-assess-clinical-aspects-of-gender-dysphoria-incongruence-and-diversity.-for-supporting-text-see-statement-5.1.f.}{%
\subsection*{Statement 5.1.e: Have experience or be qualified to assess clinical aspects of gender dysphoria, incongruence, and diversity. For supporting text, see Statement 5.1.f.}\label{statement-5.1.e-have-experience-or-be-qualified-to-assess-clinical-aspects-of-gender-dysphoria-incongruence-and-diversity.-for-supporting-text-see-statement-5.1.f.}}
\addcontentsline{toc}{subsection}{Statement 5.1.e: Have experience or be qualified to assess clinical aspects of gender dysphoria, incongruence, and diversity. For supporting text, see Statement 5.1.f.}

\hypertarget{statement-5.1.f-undergo-continuing-education-in-health-care-relating-to-gender-dysphoria-incongruence-and-diversity.}{%
\subsection*{Statement 5.1.f: Undergo continuing education in health care relating to gender dysphoria, incongruence, and diversity.}\label{statement-5.1.f-undergo-continuing-education-in-health-care-relating-to-gender-dysphoria-incongruence-and-diversity.}}
\addcontentsline{toc}{subsection}{Statement 5.1.f: Undergo continuing education in health care relating to gender dysphoria, incongruence, and diversity.}

As in any other area of clinical practice, it is vital
HCPs who are providing assessment for the initiation
of GAMSTs are knowledgeable and experienced in
the health care of TGD people. If this is not possible
in the local context, the HCP providing the assessment should work closely with an HCP who is
knowledgeable and experienced. As part of their
clinical practice, HCPs should commit to ongoing
training in TGD health care, become a member of
relevant professional bodies, attend relevant professional meetings, workshops or seminars, consult with
an HCP with relevant experience, and/or engage with
the TGD community. This is particularly important
in TGD health care as it is a relatively new field,
and the knowledge and terminology are constantly
changing (American Psychological Association, 2015;
Thorne, Yip et al., 2019). Consequently, keeping up
to date in the areas of TGD health is vital for anyone
involved in an assessment for GAMSTs.

\hypertarget{statement-5.2-we-suggest-health-care-professionals-assessing-transgender-and-gender-diverse-adults-seeking-gender-affirming-treatment-liaise-with-professionals-from-different-disciplines-within-the-field-of-transgender-health-for-consultation-and-referral-if-required.}{%
\section*{Statement 5.2: We suggest health care professionals assessing transgender and gender diverse adults seeking gender-affirming treatment liaise with professionals from different disciplines within the field of transgender health for consultation and referral, if required.}\label{statement-5.2-we-suggest-health-care-professionals-assessing-transgender-and-gender-diverse-adults-seeking-gender-affirming-treatment-liaise-with-professionals-from-different-disciplines-within-the-field-of-transgender-health-for-consultation-and-referral-if-required.}}
\addcontentsline{toc}{section}{Statement 5.2: We suggest health care professionals assessing transgender and gender diverse adults seeking gender-affirming treatment liaise with professionals from different disciplines within the field of transgender health for consultation and referral, if required.}

If required and if possible, assessment for
GAMST should be conducted by a multidisciplinary team (Costa, Rosa-e-Silva et al., 2018;
Hembree et al., 2017; Karasic \& Fraser, 2018;
T'Sjoen et al., 2020) with team members who
have timely and adequate contact with one
another. This could include an MHP, an endocrinologist, a primary care provider, a surgeon,
a voice and communication specialist, TGD peer
navigator, and others. In some cases, a multidisciplinary team may not be required; however,
should a multidisciplinary team be needed, it is
critical HCPs be able to access colleagues from
different disciplines in a timely manner to complete the GAMST assessment and best support
the needs of the TGD person. It is also critical
TGD people be supported with follow-up
appointments with any HCP who was involved
during the assessment for GAMSTs, prior to,
during, and after the initiation of gender-affirming
treatments.

\textbf{The following recommendations are made
regarding the requirements for gender-affirming
medical and surgical treatment (all should
be met):}

\hypertarget{statement-5.3-we-recommend-health-care-professionals-assessing-transgender-and-gender-diverse-adults-for-gender-affirming-medical-and-surgical-treatment}{%
\section*{Statement 5.3: We recommend health care professionals assessing transgender and gender diverse adults for gender-affirming medical and surgical treatment:}\label{statement-5.3-we-recommend-health-care-professionals-assessing-transgender-and-gender-diverse-adults-for-gender-affirming-medical-and-surgical-treatment}}
\addcontentsline{toc}{section}{Statement 5.3: We recommend health care professionals assessing transgender and gender diverse adults for gender-affirming medical and surgical treatment:}

\hypertarget{statement-5.3.a-only-recommend-gender-affirming-medical-treatment-requested-by-a-tgd-person-when-the-experience-of-gender-incongruence-is-marked-and-sustained.}{%
\subsection*{Statement 5.3.a: Only recommend gender-affirming medical treatment requested by a TGD person when the experience of gender incongruence is marked and sustained.}\label{statement-5.3.a-only-recommend-gender-affirming-medical-treatment-requested-by-a-tgd-person-when-the-experience-of-gender-incongruence-is-marked-and-sustained.}}
\addcontentsline{toc}{subsection}{Statement 5.3.a: Only recommend gender-affirming medical treatment requested by a TGD person when the experience of gender incongruence is marked and sustained.}

To access GAMSTs, a TGD person's gender
incongruence must be marked and sustained.
This can include a need for GAMSTs and a
desire to be accepted as a person of the experienced gender. Consequently, a consideration
of the nature, length and consistency of gender
incongruence is important. This can include
such factors as a change of name and identity
documents, telling others about one's gender,
health care documentation, or changes in gender
expression. However, marked and sustained gender incongruence can exist in the absence of
disclosure to others by the TGD person
(Brumbaugh-Johnson \& Hull, 2019; Saeed et al.,
2018; Sequeira et al., 2020). An abrupt or superficial change in gender identity or lack of persistence is insufficient to initiate gender- affirming
treatments, and further assessment is recommended. In such circumstances, ongoing assessment is helpful to ensure the consistency and
persistence of gender incongruence before
GAMSTs are initiated.

While marked and sustained gender incongruence should be present, it is not necessary for TGD
people to experience severe levels of distress regarding their gender identity to access gender- affirming
treatments. In fact, access to gender-affirming treatment can act as a prophylactic measure to prevent
distress (Becker et al., 2018; Giovanardi et al., 2021;
Nieder et al., 2021; Nobili et al., 2018; Robles et al.,
2016). A TGD adult can have sustained gender
incongruence without significant distress and still
benefit from GAMSTs.

Established clinical practice examines the persistence of gender incongruence when considering
the initiation of GAMSTs (Chen \& Loshak, 2020).
In a review of 200 clinical notes, Jones, Brewin
et al.~(2017) identified the importance of the
``stability of gender identity'' when planning care.
Providing GAMSTs to TGD people with persistent gender incongruence has been associated
with low rates of patient regret and high rates of
patient satisfaction (Becker et al., 2018; El-Hadi
et al., 2018; Staples et al., 2020; Wiepjes et al.,
2018). However, while the ICD 11 (WHO, 2019a)
requires the presence of marked and persistent
gender incongruence for a diagnosis of gender
incongruence to be made, there is little specific
evidence concerning the length of persistence
required for treatment in adults. HCPs involved
in an assessment of a TGD person for GAMSTs
are encouraged to give due consideration to the
life stage, history, and current circumstances of
the adult being assessed.

\hypertarget{statement-5.3.b-ensure-fulfillment-of-diagnostic-criteria-prior-to-initiating-gender-affirming-treatments-in-regions-where-a-diagnosis-is-necessary-to-access-health-care.}{%
\subsection*{Statement 5.3.b: Ensure fulfillment of diagnostic criteria prior to initiating gender-affirming treatments in regions where a diagnosis is necessary to access health care.}\label{statement-5.3.b-ensure-fulfillment-of-diagnostic-criteria-prior-to-initiating-gender-affirming-treatments-in-regions-where-a-diagnosis-is-necessary-to-access-health-care.}}
\addcontentsline{toc}{subsection}{Statement 5.3.b: Ensure fulfillment of diagnostic criteria prior to initiating gender-affirming treatments in regions where a diagnosis is necessary to access health care.}

A diagnosis of gender incongruence may be necessary in some regions to access transition-related
care. When a diagnosis is necessary to access
GAMSTs, the assessment for GAMSTs will involve
determining and assigning a diagnosis. In these
instances, HCPs should have competence using the
latest International Classification of Diseases and
Related Health Problems (ICD) (WHO, 2019a). In
regions where a diagnosis is necessary to access
health care, a diagnosis of HA60 Gender Incongruence
of Adolescence or Adulthood should be determined
prior to gender-affirming interventions.
Gender-affirming interventions secondary to a diagnosis of HA6Z Gender Incongruence, Unspecified may
be considered in the context of a more comprehensive assessment by the multidisciplinary team.

There is evidence the use of rigid assessment
tools for ``transition readiness'' may reduce access
to care and are not always in the best interest of
the TGD person (MacKinnon et al., 2020).
Therefore, in situations where the assignment of
a diagnosis is mandatory to access care, the process should be approached with trust and
transparency between the HCP and the TGD
individual requesting GAMST, with the needs of
the TGD individual in mind. Indeed, high quality
relationships between TGD people and their
HCPs are associated with lower emotional distress
and better outcomes (Kattari et al., 2016). Because
many TGD people fear HCPs will erroneously
conflate transgender identity with mental illness
(Ellis et al., 2015), a diagnostic assessment should
be undertaken with sensitivity to facilitate the
best relationship between the provider and the
TGD individual.

\hypertarget{statement-5.3.c-identify-and-exclude-other-possible-causes-of-apparent-gender-incongruence-prior-to-the-initiation-of-gender-affirming-treatments.}{%
\subsection*{Statement 5.3.c: Identify and exclude other possible causes of apparent gender incongruence prior to the initiation of gender-affirming treatments.}\label{statement-5.3.c-identify-and-exclude-other-possible-causes-of-apparent-gender-incongruence-prior-to-the-initiation-of-gender-affirming-treatments.}}
\addcontentsline{toc}{subsection}{Statement 5.3.c: Identify and exclude other possible causes of apparent gender incongruence prior to the initiation of gender-affirming treatments.}

In rare cases, TGD individuals might have a condition that may be mistaken for gender incongruence or may have another reason for seeking
treatment aside from the alleviation of gender incongruence. In these cases, and when there is ambiguity
regarding the diagnosis of gender incongruence, a
more detailed and comprehensive assessment is
important. For example, further assessment might
be required to determine if gender incongruence
persists outside of an acute psychotic episode. If
gender incongruence persists after an acute psychotic
episode resolves, GAMSTs may be considered as
long as the TGD person has the capacity to consent
to and undergo the specific treatment. If gender
incongruence does not persist and only occurs
during such an episode, treatment should not be
considered. It is important such circumstances be
identified and excluded prior to the initiation of
GAMSTs (Byne et al., 2012, 2018; Hembree et al.,
2017). It is important to understand, however, TGD
people may present with gender incongruence and
with a mental health condition, autistic spectrum
disorder, or other neurodiversity (Glidden et al.,
2016). Indeed, some mental health conditions, such
as anxiety (Bouman et al., 2017), depression
(Heylens, Elaut et al., 2014; Witcomb et al., 2018),
and self-harm (Arcelus et al., 2016; Claes et al.,
2015) are more prevalent in TGD people who have
not accessed GAMSTs. Recent longitudinal studies
suggest mental health symptoms experienced by
TGD people tend to improve following GAMSTs
(Aldridge et al., 2020; Heylens, Verroken et al., 2014;
White Hughto \& Reisner, 2016). There is no evidence to suggest a benefit of withholding GAMSTs
from TGD people who have gender incongruence
simply on the basis that they have a mental health
or neurodevelopmental condition. For more information see Chapter 18---Mental Health.

\hypertarget{statement-5.3.d-ensure-any-mental-health-conditions-that-could-negatively-impact-the-outcome-of-genderaffirming-medical-treatments-are-assessed-with-risks-and-benefits-discussed-before-a-decision-is-made-regarding-treatment.}{%
\subsection*{Statement 5.3.d: Ensure any mental health conditions that could negatively impact the outcome of genderaffirming medical treatments are assessed, with risks and benefits discussed, before a decision is made regarding treatment.}\label{statement-5.3.d-ensure-any-mental-health-conditions-that-could-negatively-impact-the-outcome-of-genderaffirming-medical-treatments-are-assessed-with-risks-and-benefits-discussed-before-a-decision-is-made-regarding-treatment.}}
\addcontentsline{toc}{subsection}{Statement 5.3.d: Ensure any mental health conditions that could negatively impact the outcome of genderaffirming medical treatments are assessed, with risks and benefits discussed, before a decision is made regarding treatment.}

Like their cisgender counterparts, TGD people may have mental health problems. Treatment
for mental health problems can and should
occur in conjunction with GAMSTs when medical transition is needed. It is vital
gender-affirming care is not impeded unless, in
some extremely rare cases, there is robust evidence that doing so is necessary to prevent
significant decompensation with a risk of harm
to self or others. In those cases, it is also
important to consider the risks delaying
GAMSTs poses to a TGD person's mental and
physical health (Byne et al., 2018).

In general, social and medical transition of
TDG people are both associated with a reduction in mental health problems (Aldridge et al.,
2020; Bouman et al., 2017; Durwood et al., 2017;
Glynn et al., 2016; Hughto \& Reisner, 2016;
Wilson et al., 2015; Witcomb et al., 2018).
Unfortunately, the loss of social support and the
physical and financial stress that can be associated with the initiation of GAMSTs may exacerbate pre-existing mental health problems and
warrant additional support from the treating
HCP (Budge et al., 2013; Yang, Wang et al.,
2016). An assessment of mental health symptoms can improve transition outcomes, particularly when the assessment is used to facilitate
access to psychological and social support during
transition (Byne et al., 2012). A delay of transition in rare circumstances may be considered
if, for example, the TGD person is unable to
engage with the process of transition or would
be unable to manage aftercare following surgery,
even with support. Where a delay in GAMST
as a last resort has been found to be necessary,
the HCP should offer resources and support to
improve mental health and facilitate
re-engagement with the GAMST process as soon
as practicable. It should be noted access to medical transition for TGD people facilitates social
transition and improves safety in public (Rood
et al., 2017). In turn, the degree to which TGD
people's appearance conforms to their gender
identity is the best predictor of quality of life
and mental health outcomes following medical
transition (Austin \& Goodman, 2017). Delaying
access to GAMSTs due to the presence of mental
health problems may exacerbate symptoms
(Owen-Smith et al., 2018) and damage rapport;
consequently, this should be done only when all
other avenues have been exhausted.

\hypertarget{statement-5.3.e-ensure-any-physical-health-conditions-that-could-negatively-impact-the-outcome-of-gender-affirming-medical-treatments-are-assessed-with-risks-and-benefits-discussed-before-a-decision-is-made-regarding-treatment.}{%
\subsection*{Statement 5.3.e: Ensure any physical health conditions that could negatively impact the outcome of gender-affirming medical treatments are assessed, with risks and benefits discussed, before a decision is made regarding treatment.}\label{statement-5.3.e-ensure-any-physical-health-conditions-that-could-negatively-impact-the-outcome-of-gender-affirming-medical-treatments-are-assessed-with-risks-and-benefits-discussed-before-a-decision-is-made-regarding-treatment.}}
\addcontentsline{toc}{subsection}{Statement 5.3.e: Ensure any physical health conditions that could negatively impact the outcome of gender-affirming medical treatments are assessed, with risks and benefits discussed, before a decision is made regarding treatment.}

In rare cases, GAMSTs, such as hormonal and
surgical interventions, may have iatrogenic consequences or may exacerbate pre-existing physical
health conditions (Hembree et al., 2017). In these
instances, care should be taken, whenever possible, to manage pre-existing physical health conditions while initiating (if appropriate) or
continuing gender-affirming treatments. Any
interruptions in treatment should be as brief as
possible and with treatment re-initiated as soon
as practicable. Limited data and inconsistent findings suggest an association between cardiovascular and metabolic risks and hormone therapy in
TGD adults (Getahun, 2018; Iwamoto, Defreyne
et al., 2019; Iwamoto et al., 2021; Spanos et al.,
2020). Because of the possible harm related to
long-term treatment and the probable benefits
expected from the preventive measures applied
before and during hormone treatment, a careful
assessment of physical health conditions prior to
initiation of treatment is important. Some specific
conditions, such as a history of hormone-sensitive
cancer, may require further assessment and management that may preclude hormone treatment
(Center of Excellence for Transgender Health,
2016; Hembree et al., 2017).

Similar concerns may be present for TGD
adults who wish to access surgical interventions.
Each gender-affirming surgical intervention has
specific risks and potentially unfavorable consequences (Bryson \& Honig, 2019; Nassiri et al.,
2020; Remington et al., 2018). However,
intervention-specific risks associated with the
presence of specific physical conditions have not
been well researched. Thus, the kinds of medical
concerns raised by TGD people during the assessment are typically no different from those of any
other surgical candidate.

Taking into consideration the mental and physical health disparities (Brown \& Jones, 2016) and
barriers to health care (Safer et al., 2016) experienced by TGD people, the assessment of physical conditions by HCPs should not be limited
to a history of medical interventions. If the TGD
person has physical health conditions, it is
important these conditions are managed while
initiating or continuing GAMSTs whenever possible. Any interruption in treatment should be
made with a view toward re-initiating treatment
as soon as practicable. It is also important HCPs
develop a treatment strategy for managing physical conditions that facilitates health and promotes consistent adherence to a treatment plan.

\hypertarget{statement-5.3.f-assess-the-capacity-to-consent-for-the-specific-gender-affirming-treatments-prior-to-the-initiation-of-this-treatment.}{%
\subsection*{Statement 5.3.f: Assess the capacity to consent for the specific gender-affirming treatments prior to the initiation of this treatment.}\label{statement-5.3.f-assess-the-capacity-to-consent-for-the-specific-gender-affirming-treatments-prior-to-the-initiation-of-this-treatment.}}
\addcontentsline{toc}{subsection}{Statement 5.3.f: Assess the capacity to consent for the specific gender-affirming treatments prior to the initiation of this treatment.}

The practice of informed consent to treatment
is central to the provision of health care. Informed
consent is couched in the ethical principle that
recipients of health care should understand the
health care they receive and any potential consequences that could result. The importance of
informed consent is embedded in many legislative
and regulatory practices that guide HCPs around
the world (Jefford \& Moore, 2008). It is not possible to know all the potential consequences of
a health care treatment; instead, considering what
would be ``reasonable'' to expect is often used as
a minimum criterion for consent (Jefford \&
Moore, 2008; Spatz et al., 2016) and remains the
case with GAMSTs. Being able to consent to a
health care procedure or clinical intervention
requires several complex cognitive processes.
Consent requires the cognitive capacity to understand the risks and benefits of a treatment and
the potential negative and positive outcomes in
addition to the ability to retain that information
for the purposes of making the decision (using
aids as necessary) and the cognitive ability to use
that understanding to make an informed decision
(American Medical Association, 2021; Applebaum,
2007). It is vital the TGD person and the assessing HCP consider a priori the nature of the treatment sought and the potential positive and
negative effects it may have on the biological,
psychological, and social domains of the TGD
person's life.

It is important to recognize mental illness, in
particular symptoms of cognitive impairment or
psychosis, can impact a person's ability to grant
consent for GAMSTs (Hostiuc et al., 2018).
However, the presence of such symptoms does
not necessarily equate to an inability to give consent because many people with significant mental
health symptoms are able to understand the risks
and benefits of treatment enough to make an
informed decision (Carpenter et al., 2000).
Instead, it is important a careful assessment is
carried out that examines each TGD person's
ability to comprehend the nature of the specific
GAMST being considered, consider treatment
options, including risks and benefits, appreciate
the potential short- and long-term consequences
of the decision, and communicate their choice in
order to receive the treatment (Grootens-Wiegers
et al., 2017).

There may be instances in which an individual
lacks the capacity to consent to health care, such
as during an acute episode of psychosis or in
situations where an individual has long-term cognitive impairment. However, limits to capacity to
consent to treatment should not prevent individuals from receiving appropriate GAMSTs. For
some, understanding the risks and benefits may
require the use of repeated explanations in
jargon-free language over time or the use of diagrams to facilitate explanation and aid comprehension. A comprehensive and thorough
assessment undertaken by the multidisciplinary
health care team can further inform this process.
For others, an alternative decision maker, such
as a legal guardian or regulator-approved,
independent decision maker may need to be
appointed. These situations need to be considered
on a case-by-case basis with the aim of ensuring
the most affirmative and least restrictive health
care is provided to the individual. Also see
Chapter 11---Institutional Environments.

\hypertarget{statement-5.3.g-assess-the-capacity-of-the-gender-diverse-and-transgender-adult-to-understand-the-effect-of-gender-affirming-treatment-on-reproduction-and-explore-reproductive-options-with-the-individual-prior-to-the-initiation-of-gender-affirming-treatment.}{%
\subsection*{Statement 5.3.g: Assess the capacity of the gender diverse and transgender adult to understand the effect of gender-affirming treatment on reproduction and explore reproductive options with the individual prior to the initiation of gender-affirming treatment.}\label{statement-5.3.g-assess-the-capacity-of-the-gender-diverse-and-transgender-adult-to-understand-the-effect-of-gender-affirming-treatment-on-reproduction-and-explore-reproductive-options-with-the-individual-prior-to-the-initiation-of-gender-affirming-treatment.}}
\addcontentsline{toc}{subsection}{Statement 5.3.g: Assess the capacity of the gender diverse and transgender adult to understand the effect of gender-affirming treatment on reproduction and explore reproductive options with the individual prior to the initiation of gender-affirming treatment.}

As gender-affirming medical interventions
often affect reproductive capacity, HCPs should
ensure a TGD person is aware of the implications
for reproduction of the treatments and is familiar
with gamete storage and assistive reproductive
options. Gender-affirming hormone treatments
have been shown to impact reproductive functions and fertility, although the consequences are
heterogenous for people of all birth-assigned
sexes (Adeleye et al., 2019; Jindarak et al., 2018;
Taub et al., 2020). There may be individual differences and fluctuations in these effects on TGD
adults. It is therefore essential that HCPs inform
a TGD person about the possible impact of the
treatment on their reproductive potential during
the assessment and as part of the evaluation of
the person's capacity to consent for GAMSTs.
Reproductive options should be considered and
discussed prior to the initiation of gender-affirming
treatments. Because the literature is unclear about
the possibility of conception while on hormone
therapy, information about the necessity of using
contraception to avoid unwanted pregnancy and
the different methods of contraception available
may need to be provided (Light et al., 2014;
Schubert \& Carey,2020).

Cross-sectional studies in clinical and nonclinical samples from different populations consistently report TGD adults express parental desire
and wish to pursue fertility preservation with
varying rates that are related to age, gender, and
the duration of gender-affirming hormone treatment (Auer et al., 2018; De Sutter et al., 2002;
Defreyne, Van Schuvlenbergh et al., 2020;
Wierckx, Stuyver et al., 2012). In a small sample,
provision of fertility information was found to
have an influence on decision-making related to
the use of fertility preservation (Chen et al.,
2019). Although there was no comparison made
between groups who did and did not receive fertility counseling, high fertility preservation rates
occurred following comprehensive fertility counseling among transgender individuals (Amir
et al., 2020). Further, one study suggested consultation with a specialist reduced regret related
to the decision about whether to pursue fertility
preservation procedures (Vyas et al., 2021). For
more information see Chapter 16---
Reproductive Health.

\hypertarget{statement-5.4-we-suggest-as-part-of-the-assessment-for-gender-affirming-hormonal-or-surgical-treatment-professionals-who-have-competencies-in-the-assessment-of-transgender-and-gender-diverse-people-wishing-gender-related-medical-treatment-consider-the-role-of-social-transition-together-with-the-individual.}{%
\section*{Statement 5.4: We suggest, as part of the assessment for gender-affirming hormonal or surgical treatment, professionals who have competencies in the assessment of transgender and gender diverse people wishing gender-related medical treatment consider the role of social transition together with the individual.}\label{statement-5.4-we-suggest-as-part-of-the-assessment-for-gender-affirming-hormonal-or-surgical-treatment-professionals-who-have-competencies-in-the-assessment-of-transgender-and-gender-diverse-people-wishing-gender-related-medical-treatment-consider-the-role-of-social-transition-together-with-the-individual.}}
\addcontentsline{toc}{section}{Statement 5.4: We suggest, as part of the assessment for gender-affirming hormonal or surgical treatment, professionals who have competencies in the assessment of transgender and gender diverse people wishing gender-related medical treatment consider the role of social transition together with the individual.}

Social transition can be extremely beneficial
to many TGD people although not all TGD people are able to socially transition or wish to
socially transition (Bränström \& Pachankis, 2021;
Koehler et al., 2018; Nieder, Eyssel et al., 2020).
Consequently, some TGD people seek
gender-affirming interventions after social transition, some before, some during, and some in
the absence of social transition.

Social transition and gender identity disclosure can improve the mental health of a TGD
person seeking gender-affirming interventions
(Hughto et al., 2020; McDowell et al., 2019). In
addition, chest and facial surgeries prior to hormone therapy can facilitate social transition
(Altman, 2012; Davis \& Colton Meier, 2014;
Olson-Kennedy, Warus et al.~2018; Van Boerum
et al., 2019). As part of the assessment process,
HCPs should discuss which social role is most
comfortable for the TGD person, if a social transition is planned, and the timing for any planned
social transition (Barker \& Wylie, 2008). It is
imperative during the assessment process, HCPs
are respectful of the wide diversity of gendered
social roles, including nonbinary as well as
binary identities and presentations, which vary
according to cultural, local community, and individual understandings.

Not everyone who requests GAMSTs will wish
to or be able to socially transition. Little is known
about TGD people who do not socially transition
before, during, or after medical treatment, as this
has not been systematically studied. The most
frequent reasons that have been identified for
avoiding social transition are fear of being abandoned by family or friends, fearing economic loss
(Bradford et al., 2013), and being discriminated
against and stigmatized (Langenderfer-Magruder
et al., 2016; McDowell et al., 2019; White Hughto
et al., 2015). However, some people do not pursue
social transition because they feel hormonal or
surgical treatments offer enough subjective
improvement to reduce gender dysphoria.

If there is no clear plan for social transition
or if social transition is unwanted, additional
assessment is important to determine the specific
nature and advisability of the treatment request,
especially if surgical treatment is requested.
Additional assessment can offer the TGD person
an opportunity to consider the possible effects
of not socially transitioning while still obtaining
GAMSTs. Given the lack of data on health outcomes for TGD people who do not socially transition (Evans et al., 2021; Levine, 2009; Turban,
Loo et al., 2021), GAMSTs should be approached
cautiously in such circumstances.

\hypertarget{statement-5.5-we-recommend-transgender-and-gender-diverse-adults-who-fulfill-the-criteria-for-gender-affirming-medical-and-surgical-treatment-require-a-single-opinion-for-the-initiation-of-this-treatment-from-a-professional-who-has-competencies-in-the-assessment-of-transgender-and-gender-diverse-people-wishing-gender-related-medical-and-surgical-treatment.}{%
\section*{Statement 5.5: We recommend transgender and gender diverse adults who fulfill the criteria for gender-affirming medical and surgical treatment require a single opinion for the initiation of this treatment from a professional who has competencies in the assessment of transgender and gender diverse people wishing gender-related medical and surgical treatment.}\label{statement-5.5-we-recommend-transgender-and-gender-diverse-adults-who-fulfill-the-criteria-for-gender-affirming-medical-and-surgical-treatment-require-a-single-opinion-for-the-initiation-of-this-treatment-from-a-professional-who-has-competencies-in-the-assessment-of-transgender-and-gender-diverse-people-wishing-gender-related-medical-and-surgical-treatment.}}
\addcontentsline{toc}{section}{Statement 5.5: We recommend transgender and gender diverse adults who fulfill the criteria for gender-affirming medical and surgical treatment require a single opinion for the initiation of this treatment from a professional who has competencies in the assessment of transgender and gender diverse people wishing gender-related medical and surgical treatment.}

Previous versions of the SOC guidelines have
required TGD individuals to be assessed for
GAMSTs by two qualified HCPs. It was believed
having two independent opinions was best practice as it ensured safety for both TGD people
and HCPs. For example, it was assumed that
seeing two HCPs offered assuredness for both
TGD people and their assessing HCPs when pursuing irreversible medical interventions.

However, the limited research in the area indicates two opinions are largely unnecessary. For
example, Jones, Brewin et al.~(2017) reviewed the
case notes of experienced HCPs working within
a state-funded gender service and found there
was an overwhelming correlation between both
opinions---arguably making one of them redundant. Further, Bouman et al.~(2014) determined
the requirement for two independent assessors
reflected paternalism in health care services and
raised a potential breach of the autonomy of TGD
individuals. The authors posited when clients are
adequately prepared and assessed under the care
of a multidisciplinary team, a second independent
assessment is unnecessary.

Consequently, if written documentation or a
letter is required to recommend gender-affirming
medical and surgical treatment (GAMST), TGD
people seeking treatments including hormones, and
genital, chest, facial and other gender-affirming
surgeries require a single written opinion/signature
from an HCP competent to independently assess
and diagnose (Bouman et al., 2014; Yuan et al,
2021). Further written opinions/signatures may be
requested where there is a specific clinical need.

\hypertarget{statement-5.6-we-suggest-health-care-professionals-assessing-transgender-and-gender-diverse-people-seeking-gonadectomy-consider-a-minimum-of-6-months-of-hormone-therapy-as-appropriate-to-the-tgd-persons-gender-goals-before-the-tgd-person-undergoes-irreversible-surgical-intervention-unless-hormones-are-not-clinically-indicated-for-the-individual.}{%
\section*{Statement 5.6: We suggest health care professionals assessing transgender and gender diverse people seeking gonadectomy consider a minimum of 6 months of hormone therapy as appropriate to the TGD person's gender goals before the TGD person undergoes irreversible surgical intervention (unless hormones are not clinically indicated for the individual).}\label{statement-5.6-we-suggest-health-care-professionals-assessing-transgender-and-gender-diverse-people-seeking-gonadectomy-consider-a-minimum-of-6-months-of-hormone-therapy-as-appropriate-to-the-tgd-persons-gender-goals-before-the-tgd-person-undergoes-irreversible-surgical-intervention-unless-hormones-are-not-clinically-indicated-for-the-individual.}}
\addcontentsline{toc}{section}{Statement 5.6: We suggest health care professionals assessing transgender and gender diverse people seeking gonadectomy consider a minimum of 6 months of hormone therapy as appropriate to the TGD person's gender goals before the TGD person undergoes irreversible surgical intervention (unless hormones are not clinically indicated for the individual).}

The Endocrine Society Clinical Practice
Guidelines advise a period of consistent hormone
treatment prior to genital surgery (Hembree
et al., 2017). While there was limited supportive
research, this recommendation was considered to
be good clinical practice as it allows a more
reversible experience prior to the irreversible
experience of surgery. For example, there can be
changes in sexual desire after genital surgery that
removes the testicles (Lawrence, 2005; Wierckx,
Van de Peer et al., 2014). In this context, reversible testosterone suppression can offer a TGD
person a period of time to experience the absence
of testosterone and decide if this feels right for
them. It should be noted the effects of reduced
estrogen on a TGD person's sexual desire and
functioning following an oophorectomy is less
well documented.

Surgery that removes gonads is an irreversible
procedure that leads to loss of fertility and loss
of the effects of endogenous sex steroids. Both
effects must be discussed as a component of the
assessment process. For additional information
see Chapter 16---Reproductive Health. Of course,
hormones are not clinically indicated for TGD
adults who do not want them or in cases where
they are contraindicated due to health reasons.
For more information see Chapter 13---Surgery
and Postoperative Care.

\hypertarget{statement-5.7-we-recommend-health-care-professionals-assessing-adults-who-wish-to-detransition-and-seek-gender-related-hormone-intervention-surgical-intervention-or-both-utilize-a-comprehensive-multidisciplinary-assessment-that-will-include-additional-viewpoints-from-experienced-health-care-professionals-in-transgender-health-and-that-considers-together-with-the-individual-the-role-of-social-transition-as-part-of-the-assessment-process.}{%
\section*{Statement 5.7: We recommend health care professionals assessing adults who wish to detransition and seek gender-related hormone intervention, surgical intervention, or both, utilize a comprehensive multidisciplinary assessment that will include additional viewpoints from experienced health care professionals in transgender health and that considers, together with the individual, the role of social transition as part of the assessment process.}\label{statement-5.7-we-recommend-health-care-professionals-assessing-adults-who-wish-to-detransition-and-seek-gender-related-hormone-intervention-surgical-intervention-or-both-utilize-a-comprehensive-multidisciplinary-assessment-that-will-include-additional-viewpoints-from-experienced-health-care-professionals-in-transgender-health-and-that-considers-together-with-the-individual-the-role-of-social-transition-as-part-of-the-assessment-process.}}
\addcontentsline{toc}{section}{Statement 5.7: We recommend health care professionals assessing adults who wish to detransition and seek gender-related hormone intervention, surgical intervention, or both, utilize a comprehensive multidisciplinary assessment that will include additional viewpoints from experienced health care professionals in transgender health and that considers, together with the individual, the role of social transition as part of the assessment process.}

Many TGD adults may consider a range of
identities and elements of gender presentation
while they are exploring their gender identity and
are considering transition options. Accordingly,
people may spend some time in a gender identity
or presentation before they discover it does not
feel comfortable and later adapt it or shift to an
earlier identity or presentation (Turban, King
et al., 2021). Some TGD adults may also experience a change in gender identity over time so
that their needs for medical treatment evolve.
This is a healthy and reasonable process for determining the most comfortable and congruent way
of living, which is informed by the person's gender identity and the context of their life. This
process of identity exploration should not necessarily be equated with regret, confusion, or poor
decision-making because a TGD adult's gender
identity may change without devaluing previous
transition decisions (MacKinnon et al., 2021;
Turban, Loo et al., 2021). TGD adults should be
assisted in this exploration and any other changes
in their identity (Expósito-Campos, 2021). While
exploration continues, gender-affirming treatments
that are irreversible should be avoided until clarity
about long-term goals and outcomes is achieved.

The decision to detransition appears to be
rare (Defreyne, Motmans et al., 2017;
Hadje-Moussa et al., 2019; Wiepjes et al., 2018).
Estimates of the number of people who detransition due to a change in identity are likely to
be overinflated due to research blending different cohorts (Expósito-Campos, 2021). For example, detransition research cohorts often include
TGD adults who chose to detransition because
of a change in their identity as well as TGD
adults who chose to detransition without a
change in identity. While little research has been
conducted to systematically examine variables
that correlate with a TGD adult's decision to
halt a transition process or to detransition, a
recent study found the vast majority of TGD
people who opted to detransition did so due to
external factors, such as stigma and lack of
social support and not because of changes in
gender identity (Turban, King et al., 2021). TGD
adults who have not experienced a change in
identity may choose to halt transition or to
detransition because of oppression, violence, and
social/relational conflict, surgical complications,
health concerns, physical contraindications, a
lack of resources, or dissatisfaction with the
results (Expósito-Campos, 2021). In such cases,
MHPs are well placed to assist the TGD person
with these challenges.

While the choice to detransition is proportionally rare, it is expected an overall increase in the
number of adults who identify as TGD would
result in an increase in the absolute number of
people seeking to halt or reverse a transition.
However, while the absolute numbers may
increase, the percentage of people seeking to halt
or reverse permanent physical changes should
remain static and low. The existence of these rare
requests must not be used as a justification to
interrupt critical, medically necessary care, including hormone and surgical treatments, for the vast
majority of TGD adults.

Due to the limited research in this area, clinical guidance is based primarily on individual
case studies and the expert opinion of HCPs
working with TGD adults (Expósito-Campos,
2021; Richards \& Barrett, 2020). Accordingly, if
a TGD adult has undergone permanent physical
changes and seeks to undo them, the assessing
HCP should be a member of a comprehensive
multidisciplinary assessment team. A multidisciplinary team allows for the contribution of additional viewpoints from HCPs experienced in
transgender health. In collaboration with the
TGD adult, the multidisciplinary team is encouraged to thoroughly understand the motivations
for the original treatment and for the decision
to detransition. Any concerns with the previous
physical changes should be carefully explored and
a significant effort made to ensure similar concerns are not replicated by the reversal.

To ensure the greatest likelihood of satisfaction
and comfort with a reversal of permanent physical changes, the TGD adult and the multidisciplinary team should explore the role of social
transition in the assessment and in preparation
for the reversal. In such instances, it is highly
likely a prolonged period of living in role will
be necessary before further physical changes are
recommended. HCPs should support the TGD
adult through any social changes, as well as any
feelings of failure, shame, depression, or guilt in
deciding to make such a change. In addition,
people should be supported in coping with any
prejudice or social difficulties they may have
experienced that could have led to a decision to
detransition or that may have resulted from such
a decision. It is also important to help the person
remain engaged with health care throughout the
process (Narayan et al., 2021).

While available research shows consistent positive outcomes for the majority of TGD adults
who choose to transition (Aldridge et al., 2020;
Byne et al., 2012; Gorin-Lazard et al., 2012;
Owen-Smith et al., 2018; White Hughto \&
Reisner, 2016), some TGD adults may decompensate or experience a worsened condition following transition. Little research has been
conducted to systematically examine variables
that correlate with poor or worsened biological,
psychological, or social conditions following
transition (Hall et al., 2021; Littman, 2021);
however, this occurrence appears to be rare
(Hall et al., 2021; Wiepjes et al., 2018). In cases
where people decompensate after physical or
social transition and then remain in a poorer
biological, psychological, or social state than
they were in prior to transition, serious consideration should be given as to whether transition
is helpful at this time, for this person, or both.
In cases where treatment is no longer supported,
assistance should be arranged to support the
person to manage the process of stopping treatment and to manage any concomitant difficulties
(Narayan et al., 2021).

It is vital that people who detransition, for
any reason, be supported. It should be remembered, however, this is a rare occurrence and
the literature shows consistently positive outcomes for the vast majority of TGD adults who
transition to a gender that is comfortable for
them, including those who receive GAMSTs
(Byne et al., 2012; Green \& Fleming, 1990;
Lawrence, 2003; Motmans et al., 2012; Van de
Grift, Elaut et al., 2018).

\hypertarget{adolescents}{%
\chapter{Adolescents}\label{adolescents}}

\hypertarget{historical-context-and-changes-since-previous-standards-of-care}{%
\section*{Historical context and changes since previous Standards of Care}\label{historical-context-and-changes-since-previous-standards-of-care}}
\addcontentsline{toc}{section}{Historical context and changes since previous Standards of Care}

Specialized health care for transgender adolescents began in the 1980s when a few specialized
gender clinics for youth were developed around
the world that served relatively small numbers
of children and adolescents. In more recent years,
there has been a sharp increase in the number
of adolescents requesting gender care (Arnoldussen
et al., 2019; Kaltiala, Bergman et al., 2020). Since
then, new clinics have been founded, but clinical
services in many places have not kept pace with
the increasing number of youth seeking care.
Hence, there are often long waitlists for services,
and barriers to care exist for many transgender
youth around the world (Tollit et al., 2018).

Until recently, there was limited information
regarding the prevalence of gender diversity
among adolescents. Studies from high school
samples indicate much higher rates than earlier
thought, with reports of up to 1.2\% of participants identifying as transgender (Clark et al.,
2014) and up to 2.7\% or more (e.g., 7--9\%) experiencing some level of self-reported gender diversity (Eisenberg et al., 2017; Kidd et al., 2021;
Wang et al., 2020). These studies suggest gender
diversity in youth should no longer be viewed as
rare. Additionally, a pattern of uneven ratios by
assigned sex has been reported in gender clinics,
with adolescents assigned female at birth (AFAB)
initiating care 2.5--7.1 times more frequently as
compared to adolescents who are assigned male
at birth (AMAB) (Aitken et al., 2015; Arnoldussen
et al., 2019; Bauer et al., 2021; de Graaf,
Carmichael et al., 2018; Kaltiala et al., 2015;
Kaltiala, Bergman et al., 2020).

A specific World Professional Association for
Transgender Health's (WPATH) Standards of Care
section dedicated to the needs of children and
adolescents was first included in the 1998 WPATH
Standards of Care, 5th version (Levine et al.,
1998). Youth aged 16 or older were deemed
potentially eligible for gender-affirming medical
care, but only in select cases. The subsequent 6th
(Meyer et al., 2005) and 7th (Coleman et al.,
2012) versions divided medical-affirming treatment for adolescents into three categories and
presented eligibility criteria regarding age/puberty
stage---namely fully reversible puberty delaying
blockers as soon as puberty had started; partially
reversible hormone therapy (testosterone, estrogen) for adolescents at the age of majority, which
was age 16 in certain European countries; and
irreversible surgeries at age 18 or older, except
for chest ``masculinizing'' mastectomy, which had
an age minimum of 16 years. Additional eligibility criteria for gender-related medical care
included a persistent, long (childhood) history of
gender ``non-conformity''/dysphoria, emerging or
intensifying at the onset of puberty; absence or
management of psychological, medical, or social
problems that interfere with treatment; provision
of support for commencing the intervention by
the parents/caregivers; and provision of informed
consent. A chapter dedicated to transgender and
gender diverse (TGD) adolescents, distinct from
the child chapter, has been created for this 8th
edition of the Standards of Care given 1) the
exponential growth in adolescent referral rates;
2) the increased number of studies specific to
adolescent gender diversity-related care; and 3)
the unique developmental and gender-affirming
care issues of this age group.

Non-specific terms for gender-related care are
avoided (e.g., gender-affirming model, gender
exploratory model) as these terms do not represent unified practices, but instead heterogenous
care practices that are defined differently in various settings.

\hypertarget{adolescence-overview}{%
\section*{Adolescence overview}\label{adolescence-overview}}
\addcontentsline{toc}{section}{Adolescence overview}

Adolescence is a developmental period characterized by relatively rapid physical and psychological maturation, bridging childhood and
adulthood (Sanders, 2013). Multiple developmental processes occur simultaneously, including
pubertal-signaled changes. Cognitive, emotional,
and social systems mature, and physical changes
associated with puberty progress. These processes do not all begin and end at the same
time for a given individual, nor do they occur
at the same age for all persons. Therefore, the
lower and upper borders of adolescence are
imprecise and cannot be defined exclusively by
age. For example, physical pubertal changes may
begin in late childhood and executive control
neural systems continue to develop well into the
mid-20s (Ferguson et al., 2021). There is a lack
of uniformity in how countries and governments
define the age of majority (i.e., legal
decision-making status; Dick et al., 2014). While
many specify the age of majority as 18 years of
age, in some countries it is as young as 15 years
(e.g., Indonesia and Myanmar), and in others
as high as 21 years (e.g., the U.S. state of
Mississippi and Singapore).

\textbf{For clarity, this chapter applies to adolescents
from the start of puberty until the legal age of
majority (in most cases 18 years), however there
are developmental elements of this chapter,
including the importance of parental/caregiver
involvement, that are often relevant for the care
of transitional-aged young adults and should
be considered appropriately.}

Cognitive development in adolescence is often
characterized by gains in abstract thinking, complex reasoning, and metacognition (i.e., a young
person's ability to think about their own feelings
in relation to how others perceive them; Sanders,
2013). The ability to reason hypothetical situations enables a young person to conceptualize
implications regarding a particular decision.
However, adolescence is also often associated with
increased risk-taking behaviors. Along with these
notable changes, adolescence is often characterized by individuation from parents and the development of increased personal autonomy. There
is often a heightened focus on peer relationships,
which can be both positive and detrimental
(Gardner \& Steinberg, 2005). Adolescents often
experience a sense of urgency that stems from
hypersensitivity to reward, and their sense of
timing has been shown to be different from that
of older individuals (Van Leijenhorst et al., 2010).
Social-emotional development typically advances
during adolescence, although there is a great variability among young people in terms of the level
of maturity applied to inter- and intra-personal
communication and insight (Grootens-Wiegers
et al., 2017). For TGD adolescents making decisions about gender-affirming treatments---decisions that may have lifelong consequences---it is
critical to understand how all these aspects of
development may impact decision-making for a
given young person within their specific cultural
context.

\hypertarget{gender-identity-development-in-adolescence}{%
\section*{Gender identity development in adolescence}\label{gender-identity-development-in-adolescence}}
\addcontentsline{toc}{section}{Gender identity development in adolescence}

Our understanding of gender identity development in adolescence is continuing to evolve.
When providing clinical care to gender diverse
young people and their families, it is important
to know what is and is not known about gender
identity during development (Berenbaum, 2018).
When considering treatments, families may have
questions regarding the development of their
adolescent's gender identity, and whether or not
their adolescent's declared gender will remain
the same over time. For some adolescents, a
declared gender identity that differs from the
assigned sex at birth comes as no surprise to
their parents/caregivers as their history of gender diverse expression dates back to childhood
(Leibowitz \& de Vries, 2016). For others, the
declaration does not happen until the emergence
of pubertal changes or even well into adolescence (McCallion et al., 2021; Sorbara
et al., 2020).

Historically, social learning and cognitive
developmental research on gender development
was conducted primarily with youth who were
not gender diverse in identity or expression and
was carried out under the assumption that sex
correlated with a specific gender; therefore, little
attention was given to gender identity development. In addition to biological factors influencing
gender development, this research demonstrated
psychological and social factors also play a role
(Perry \& Pauletti, 2011). While there has been
less focus on gender identity development in
TGD youth, there is ample reason to suppose,
apart from biological factors, psychosocial factors
are also involved (Steensma, Kreukels et al.,
2013). For some youth, gender identity development appears fixed and is often expressed from
a young age, while for others there may be a
developmental process that contributes to gender
identity development over time.

Neuroimaging studies, genetic studies, and
other hormone studies in intersex individuals
demonstrate a biological contribution to the
development of gender identity for some
individuals whose gender identity does not match
their assigned sex at birth (Steensma, Kreukels
et al., 2013). As families often have questions
about this very issue, it is important to note it
is not possible to distinguish between those for
whom gender identity may seem fixed from birth
and those for whom gender identity development
appears to be a developmental process. Since it
is impossible to definitively delineate the contribution of various factors contributing to gender
identity development for any given young person,
a comprehensive clinical approach is important
and necessary (see Statement 3). Future research
would shed more light on gender identity development if conducted over long periods of time
with diverse cohort groups. Conceptualization of
gender identity by shifting from dichotomous
(e.g., binary) categorization of male and female
to a dimensional gender spectrum along a continuum (APA, 2013) would also be necessary.

Adolescence may be a critical period for the
development of gender identity for gender diverse
young people (Steensma, Kreukels et al., 2013).
Dutch longitudinal clinical follow-up studies of
adolescents with childhood gender dysphoria who
received puberty suppression, gender-affirming
hormones, or both, found that none of the youth
in adulthood regretted the decisions they had
taken in adolescence (Cohen-Kettenis \& van
Goozen, 1997; de Vries et al., 2014). These findings suggest adolescents who were comprehensively assessed and determined emotionally
mature enough to make treatment decisions
regarding gender- affirming medical care presented with stability of gender identity over the
time period when the studies were conducted.

When extrapolating findings from the
longer-term longitudinal Dutch cohort studies to
present-day gender diverse adolescents seeking care,
it is critical to consider the societal changes that
have occurred over time in relation to TGD people.
Given the increase in visibility of TGD identities,
it is important to understand how increased awareness may impact gender development in different
ways (Kornienko et al., 2016). One trend identified
is that more young people are presenting to gender
clinics with nonbinary identities (Twist \& de Graaf,
2019). Another phenomenon occurring in clinical
practice is the increased number of adolescents
seeking care who have not seemingly experienced,
expressed (or experienced and expressed) gender
diversity during their childhood years. One
researcher attempted to study and describe a specific form of later-presenting gender diversity experience (Littman, 2018). However, the findings of
the study must be considered within the context
of significant methodological challenges, including
1) the study surveyed parents and not youth perspectives; and 2) recruitment included parents from
community settings in which treatments for gender
dysphoria are viewed with scepticism and are criticized. However, these findings have not been replicated. For a select subgroup of young people,
susceptibility to social influence impacting gender
may be an important differential to consider
(Kornienko et al., 2016). However, caution must
be taken to avoid assuming these phenomena occur
prematurely in an individual adolescent while relying on information from datasets that may have
been ascertained with potential sampling bias
(Bauer et al., 2022; WPATH, 2018). It is important
to consider the benefits that social connectedness
may have for youth who are linked with supportive
people (Tuzun et al., 2022)(see Statement 4).

Given the emerging nature of knowledge
regarding adolescent gender identity development,
an individualized approach to clinical care is considered both ethical and necessary. As is the case
in all areas of medicine, each study has methodological limitations, and conclusions drawn from
research cannot and should not be universally
applied to all adolescents. This is also true when
grappling with common parental questions
regarding the stability versus instability of a particular young person's gender identity development. While future research will help advance
scientific understanding of gender identity development, there may always be some gaps.
Furthermore, given the ethics of self-determination
in care, these gaps should not leave the TGD
adolescent without important and necessary care.

\hypertarget{research-evidence-of-gender-affirming-medical-treatment-for-transgender-adolescents}{%
\section*{Research evidence of gender-affirming medical treatment for transgender adolescents}\label{research-evidence-of-gender-affirming-medical-treatment-for-transgender-adolescents}}
\addcontentsline{toc}{section}{Research evidence of gender-affirming medical treatment for transgender adolescents}

A key challenge in adolescent transgender care is
the quality of evidence evaluating the effectiveness
of medically necessary gender-affirming medical
and surgical treatments (GAMSTs) (see medically
necessary statement in the Global chapter,
Statement 2.1), over time. Given the lifelong implications of medical treatment and the young age
at which treatments may be started, adolescents,
their parents, and care providers should be
informed about the nature of the evidence base.
It seems reasonable that decisions to move forward
with medical and surgical treatments should be
made carefully. Despite the slowly growing body
of evidence supporting the effectiveness of early
medical intervention, the number of studies is still
low, and there are few outcome studies that follow
youth into adulthood. Therefore, a systematic
review regarding outcomes of treatment in adolescents is not possible. A short narrative review
is provided instead.

At the time of this chapter's writing, there were
several longer-term longitudinal cohort follow-up
studies reporting positive results of early (i.e.,
adolescent) medical treatment; for a significant
period of time, many of these studies were conducted through one Dutch clinic (e.g.,
Cohen-Kettenis \& van Goozen, 1997; de Vries,
Steensma et al., 2011; de Vries et al., 2014; Smith
et al., 2001, 2005). The findings demonstrated
the resolution of gender dysphoria is associated
with improved psychological functioning and
body image satisfaction. Most of these studies
followed a pre-post methodological design and
compared baseline psychological functioning with
outcomes after the provision of medical
gender-affirming treatments. Different studies
evaluated individual aspects or combinations of
treatment interventions and included 1)
gender-affirming hormones and surgeries
(Cohen-Kettenis \& van Goozen, 1997; Smith
et al., 2001, 2005); 2) puberty suppression (de
Vries, Steensma et al., 2011); and 3) puberty suppression, affirming hormones, and surgeries (de
Vries et al., 2014). The 2014 long-term follow-up
study is the only study that followed youth from
early adolescence (pretreatment, mean age of
13.6) through young adulthood (posttreatment,
mean age of 20.7). This was the first study to
show gender-affirming treatment enabled transgender adolescents to make age-appropriate
developmental transitions while living as their
affirmed gender with satisfactory objective and
subjective outcomes in adulthood (de Vries et al.,
2014). While the study employed a small (n =
55), select, and socially supported sample, the
results were convincing. Of note, the participants
were part of the Dutch clinic known for employing a multidisciplinary approach, including provision of comprehensive, ongoing assessment and
management of gender dysphoria, and support
aimed at emotional well-being.

Several more recently published longitudinal
studies followed and evaluated participants at
different stages of their gender-affirming treatments. In these studies, some participants may
not have started gender-affirming medical treatments, some had been treated with puberty suppression, while still others had started
gender-affirming hormones or had even undergone gender-affirming surgery (GAS) (Achille
et al., 2020; Allen et al., 2019; Becker-Hebly et al.,
2021; Carmichael et al., 2021; Costa et al., 2015;
Kuper et al., 2020, Tordoff et al., 2022). Given
the heterogeneity of treatments and methods, this
type of design makes interpreting outcomes more
challenging. Nonetheless, when compared with
baseline assessments, the data consistently demonstrate improved or stable psychological functioning, body image, and treatment satisfaction
varying from three months to up to two years
from the initiation of treatment.

Cross-sectional studies provide another design
for evaluating the effects of gender-affirming
treatments. One such study compared psychological functioning in transgender adolescents at
baseline and while undergoing puberty suppression with that of cisgender high school peers at
two different time points. At baseline, the transgender youth demonstrated lower psychological
functioning compared with cisgender peers,
whereas when undergoing puberty suppression,
they demonstrated better functioning than their
peers (van der Miesen et al., 2020). Grannis et al.
(2021) demonstrated transgender males who
started testosterone had lower internalizing mental health symptoms (depression and anxiety)
compared with those who had not started testosterone treatment.

Four additional studies followed different outcome designs. In a retrospective chart study,
Kaltiala, Heino et al.~(2020) reported transgender
adolescents with few or no mental health challenges prior to commencing gender-affirming
hormones generally did well during the treatment. However, adolescents with more mental
health challenges at baseline continued to experience the manifestations of those mental health
challenges over the course of gender-affirming
medical treatment. Nieder et al.~(2021) studied
satisfaction with care as an outcome measure and
demonstrated transgender adolescents were more
satisfied the further they progressed with the
treatments they initially started. Hisle-Gorman
et al.~(2021) compared health care utilization preand post-initiation of gender-affirming pharmaceuticals as indicators of the severity of mental
health conditions among 3,754 TGD adolescents
in a large health care data set. Somewhat contrary
to the authors' hypothesis of improved mental
health, mental health care use did not significantly change, and psychotropic medication prescriptions increased. In a large non-probability
sample of transgender-identified adults, Turban
et al.~(2022) found those who reported access to
gender-affirming hormones in adolescence had
lower odds of past-year suicidality compared with
transgender people accessing gender- affirming
hormones in adulthood.

Providers may consider the possibility an adolescent may regret gender-affirming decisions
made during adolescence, and a young person
will want to stop treatment and return to living
in the birth-assigned gender role in the future.
Two Dutch studies report low rates of adolescents (1.9\% and 3.5\%) choosing to stop puberty
suppression (Brik et al., 2019; Wiepjes et al.,
2018). Again, these studies were conducted in
clinics that follow a protocol that includes a
comprehensive assessment before the
gender-affirming medical treatment is started.
At present, no clinical cohort studies have
reported on profiles of adolescents who regret
their initial decision or detransition after irreversible affirming treatment. Recent research
indicate there are adolescents who detransition,
but do not regret initiating treatment as they
experienced the start of treatment as a part of
understanding their gender-related care needs
(Turban, 2018). However, this may not be the
predominant perspective of people who
detransition (Littman, 2021; Vandenbussche,
2021). Some adolescents may regret the steps
they have taken (Dyer, 2020). Therefore, it is
important to present the full range of possible
outcomes when assisting transgender adolescents. Providers may discuss this topic in a collaborative and trusting manner (i.e., as a
``potential future experience and consideration'')
with the adolescent and their parents/caregivers
before gender-affirming medical treatments are
started. Also, providers should be prepared to
support adolescents who detransition. In an
internet convenience sample survey of 237
self-identified detransitioners with a mean age
of 25.02 years, which consisted of over 90\% of
birth assigned females, 25\% had medically transitioned before age 18 and 14\% detransitioned
before age 18 (Vandenbussche, 2021). Although
an internet convenience sample is subject to
selection of respondents, this study suggests
detransitioning may occur in young transgender
adolescents and health care professionals should
be aware of this. Many of them expressed difficulties finding help during their detransition
process and reported their detransition was an
isolating experience during which they did not
receive either sufficient or appropriate support
(Vandenbussche, 2021).

To conclude, although the existing samples
reported on relatively small groups of youth (e.g.,
n = 22-101 per study) and the time to follow-up
varied across studies (6 months--7 years), this
emerging evidence base indicates a general
improvement in the lives of transgender adolescents who, following careful assessment, receive
medically necessary gender-affirming medical
treatment. Further, rates of reported regret during
the study monitoring periods are low. Taken as
a whole, the data show early medical intervention---as part of broader combined assessment
and treatment approaches focused on gender dysphoria and general well-being---can be effective
and helpful for many transgender adolescents
seeking these treatments.

\hypertarget{ethical-and-human-rights-perspectives}{%
\section*{Ethical and human rights perspectives}\label{ethical-and-human-rights-perspectives}}
\addcontentsline{toc}{section}{Ethical and human rights perspectives}

Medical ethics and human rights perspectives
were also considered while formulating the
adolescent SOC statements. For example, allowing irreversible puberty to progress in adolescents who experience gender incongruence is
not a neutral act given that it may have immediate and lifelong harmful effects for the transgender young person (Giordano, 2009; Giordano
\& Holm, 2020; Kreukels \& Cohen-Kettenis,
2011). From a human rights perspective, considering gender diversity as a normal and
expected variation within the broader diversity
of the human experience, it is an adolescent's
right to participate in their own decision-making
process about their health and lives, including
access to gender health services (Amnesty
International, 2020).

\hypertarget{short-summary-of-statements-and-unique-issues-in-adolescence}{%
\section*{Short summary of statements and unique issues in adolescence}\label{short-summary-of-statements-and-unique-issues-in-adolescence}}
\addcontentsline{toc}{section}{Short summary of statements and unique issues in adolescence}

These guidelines are designed to account for what
is known and what is not known about gender
identity development in adolescence, the evidence
for gender-affirming care in adolescence, and the
unique aspects that distinguish adolescence from
other developmental stages.

\emph{Identity exploration:} A defining feature of adolescence
is the solidifying of aspects of identity, including gender identity. Statement 6.2 addresses identity exploration in the context of gender identity development.
Statement 6.12.b accounts for the length of time
needed for a young person to experience a gender
diverse identity, express a gender diverse identity, or
both, so as to make a meaningful decision regarding
gender-affirming care.

\emph{Consent and decision-making:} In adolescence, consent
and decision-making require assessment of the individual's emotional, cognitive, and psychosocial development. Statement 6.12.c directly addresses emotional
and cognitive maturity and describes the necessary
components of the evaluation process used to assess
decision-making capacity.

\emph{Caregivers/parent involvement:} Adolescents are typically dependent on their caregivers/parents for
guidance in numerous ways. This is also true as
the young person navigates through the process of
deciding about treatment options. Statement 6.11
addresses the importance of involving caregivers/
parents and discusses the role they play in the
assessment and treatment. No set of guidelines can
account for every set of individual circumstances
on a global scale.

\hypertarget{statement-6.1-we-recommend-health-care-professionals-working-with-gender-diverse-adolescents}{%
\section*{Statement 6.1: We recommend health care professionals working with gender diverse adolescents:}\label{statement-6.1-we-recommend-health-care-professionals-working-with-gender-diverse-adolescents}}
\addcontentsline{toc}{section}{Statement 6.1: We recommend health care professionals working with gender diverse adolescents:}

\begin{enumerate}
\def\labelenumi{\alph{enumi}.}
\tightlist
\item
  Are licensed by their statutory body and hold a postgraduate degree or its equivalent in a clinical field relevant to this role granted by a nationally accredited statutory institution.
\item
  Receive theoretical and evidenced-based training and develop expertise in general child, adolescent, and family mental health across the developmental spectrum.
\item
  Receive training and have expertise in gender identity development, gender diversity in children and adolescents, have the ability to assess capacity to assent/consent, and possess general knowledge of gender diversity across the life span.
\item
  Receive training and develop expertise in autism spectrum disorders and other neurodevelopmental presentations or collaborate with a developmental disability expert when working with autistic/neurodivergent gender diverse adolescents.
\item
  Continue engaging in professional development in all areas relevant to gender diverse children, adolescents, and families.
\end{enumerate}

When assessing and supporting TGD adolescents and their families, care providers/health
care professionals (HCPs) need both general as
well as gender-specific knowledge and training.
Providers who are trained to work with adolescents and families play an important role in navigating aspects of adolescent development and
family dynamics when caring for youth and families (Adelson et al., 2012; American Psychological
Association, 2015; Hembree et al., 2017). Other
chapters in these standards of care describe these
criteria for professionals who provide gender care
in more detail (see Chapter 5---Assessment for
Adults; Chapter 7---Children; or Chapter 13---
Surgery and Postoperative Care). Professionals
working with adolescents should understand
what is and is not known regarding adolescent
gender identity development, and how this
knowledge base differs from what applies to
adults and prepubertal children. Among HCPs,
the mental health professional (MHP) has the
most appropriate training and dedicated clinical
time to conduct an assessment and elucidate
treatment priorities and goals when working with
transgender youth, including those seeking
gender-affirming medical/surgical care.
Understanding and managing the dynamics of
family members who may share differing perspectives regarding the history and needs of the
young person is an important competency that
MHPs are often most prepared to address.
When access to professionals trained in child
and adolescent development is not possible, HCPs
should make a commitment to obtain training in
the areas of family dynamics and adolescent development, including gender identity development.
Similarly, considering autistic/neurodivergent
transgender youth represent a substantial minority
subpopulation of youth served in gender clinics
globally, it is important HCPs seek additional
training in the field of autism and understand the
unique elements of care autistic gender diverse
youth may require (Strang, Meagher et al., 2018).
If these qualifications are not possible, then consultation and collaboration with a provider who
specializes in autism and neurodiversity is advised.

\hypertarget{statement-6.2-we-recommend-health-care-professionals-working-with-gender-diverse-adolescents-facilitate-the-exploration-and-expression-of-gender-openly-and-respectfully-so-that-no-one-particular-identity-is-favored.}{%
\section*{Statement 6.2: We recommend health care professionals working with gender diverse adolescents facilitate the exploration and expression of gender openly and respectfully so that no one particular identity is favored.}\label{statement-6.2-we-recommend-health-care-professionals-working-with-gender-diverse-adolescents-facilitate-the-exploration-and-expression-of-gender-openly-and-respectfully-so-that-no-one-particular-identity-is-favored.}}
\addcontentsline{toc}{section}{Statement 6.2: We recommend health care professionals working with gender diverse adolescents facilitate the exploration and expression of gender openly and respectfully so that no one particular identity is favored.}

Adolescence is a developmental period that
involves physical and psychological changes characterized by individuation and the transition to
independence from caregivers (Berenbaum et al.,
2015; Steinberg, 2009). It is a period during
which young people may explore different aspects
of identity, including gender identity.

Adolescents differ regarding the degree to
which they explore and commit to aspects of
their identity (Meeus et al., 2012). For some adolescents, the pace to achieving consolidation of
identity is fast, while for others it is slower. For
some adolescents, physical, emotional, and psychological development occur over the same general timeline, while for others, there are certain
gaps between these aspects of development.
Similarly, there is variation in the timeline for
gender identity development (Arnoldussen et al.,
2020; Katz-Wise et al., 2017). For some young
people, gender identity development is a clear
process that starts in early childhood, while for
others pubertal changes contribute to a person's
experience of themselves as a particular gender
(Steensma, Kreukels et al., 2013), and for many
others a process may begin well after pubertal
changes are completed. Given these variations,
there is no one particular pace, process, or outcome that can be predicted for an individual
adolescent seeking gender-affirming care.

Therefore, HCPs working with adolescents
should promote supportive environments that
simultaneously respect an adolescent's affirmed
gender identity and also allows the adolescent to
openly explore gender needs, including social,
medical, and physical gender-affirming interventions should they change or evolve over time.

\hypertarget{statement-6.3-we-recommend-health-care-professionals-working-with-gender-diverse-adolescents-undertake-a-comprehensive-biopsychosocial-assessment-of-adolescents-who-present-with-gender-identity-related-concerns-and-seek-medicalsurgical-transition-related-care-and-that-this-be-accomplished-in-a-collaborative-and-supportive-manner.}{%
\section*{Statement 6.3: We recommend health care professionals working with gender diverse adolescents undertake a comprehensive biopsychosocial assessment of adolescents who present with gender identity-related concerns and seek medical/surgical transition-related care, and that this be accomplished in a collaborative and supportive manner.}\label{statement-6.3-we-recommend-health-care-professionals-working-with-gender-diverse-adolescents-undertake-a-comprehensive-biopsychosocial-assessment-of-adolescents-who-present-with-gender-identity-related-concerns-and-seek-medicalsurgical-transition-related-care-and-that-this-be-accomplished-in-a-collaborative-and-supportive-manner.}}
\addcontentsline{toc}{section}{Statement 6.3: We recommend health care professionals working with gender diverse adolescents undertake a comprehensive biopsychosocial assessment of adolescents who present with gender identity-related concerns and seek medical/surgical transition-related care, and that this be accomplished in a collaborative and supportive manner.}

Given the many ways identity may unfold
during adolescence, we recommend using a comprehensive biopsychosocial assessment to guide
treatment decisions and optimize outcomes. This
assessment should aim to understand the adolescent's strengths, vulnerabilities, diagnostic profile,
and unique needs to individualize their care. As
mentioned in Statement 6.1, MHPs have the most
appropriate training, experience, and dedicated
clinical time required to obtain the information
discussed here. The assessment process should
be approached collaboratively with the adolescent
and their caregiver(s), both separately and
together, as described in more detail in Statement
6.11. An assessment should occur prior to any
medically necessary medical or surgical intervention under consideration (e.g., puberty blocking
medication, gender-affirming hormones, surgeries). See medically necessary statement in Chapter
2---Global Applicability, Statement 2.1; see also
Chapter 12---Hormone Therapy and Chapter 13---
Surgery and Postoperative Care.

Youth may experience many different gender
identity trajectories. Sociocultural definitions and
experiences of gender continue to evolve over
time, and youth are increasingly presenting with
a range of identities and ways of describing their
experiences and gender-related needs (Twist \& de
Graaf, 2019). For example, some youth will realize
they are transgender or more broadly gender
diverse and pursue steps to present accordingly.
For some youth, obtaining gender-affirming medical treatment is important while for others these
steps may not be necessary. For example, a process
of exploration over time might not result in the
young person self-affirming or embodying a different gender in relation to their assigned sex at
birth and would not involve the use of medical
interventions (Arnoldussen et al., 2019).

The most robust longitudinal evidence supporting the benefits of gender-affirming medical and
surgical treatments in adolescence was obtained
in a clinical setting that incorporated a detailed
comprehensive diagnostic assessment process over
time into its delivery of care protocol (de Vries \&
Cohen-Kettenis, 2012; de Vries et al., 2014). Given
this research and the ongoing evolution of gender
diverse experiences in society, a comprehensive
diagnostic biopsychosocial assessment during adolescence is both evidence-based and preserves the
integrity of the decision-making process. In the
absence of a full diagnostic profile, other mental
health entities that need to be prioritized and
treated may not be detected. There are no studies
of the long-term outcomes of gender-related medical treatments for youth who have not undergone
a comprehensive assessment. Treatment in this
context (e.g., with limited or no assessment) has
no empirical support and therefore carries the risk
that the decision to start gender-affirming medical
interventions may not be in the long-term best
interest of the young person at that time.

As delivery of health care and access to specialists varies globally, designing a particular
assessment process to adapt existing resources is
often necessary. In some cases, a more extended
assessment process may be useful, such as for
youth with more complex presentations (e.g., complicating mental health histories (Leibowitz \& de
Vries, 2016)), co-occurring autism spectrum characteristics (Strang, Powers et al., 2018), and/or an
absence of experienced childhood gender incongruence (Ristori \& Steensma, 2016). Given the
unique cultural, financial, and geographical factors
that exist for specific populations, providers should
design assessment models that are flexible and
allow for appropriately timed care for as many
young people as possible, so long as the assessment effectively obtains information about the
adolescent's strengths, vulnerabilities, diagnostic
profile, and individual needs. Psychometrically validated psychosocial and gender measures can also
be used to provide additional information.

The multidisciplinary assessment for youth
seeking gender-affirming medical/surgical interventions includes the following domains that correspond to the relevant statements:

\begin{itemize}
\tightlist
\item
  \textbf{Gender Identity Development:} Statements 6.12.a and 6.12.b elaborate on the factors associated with gender identity development within the specific cultural context when assessing TGD adolescents.
\item
  \textbf{Social Development and Support; Intersectionality:} Statements 6.4 and 6.11 elaborate on the importance of assessing gender minority stress, family dynamics, and other aspects contributing to social development and intersectionality.
\item
  \textbf{Diagnostic Assessment of Possible Co-Occurring Mental Health and/or Developmental Concerns:} Statement 6.12.d elaborates on the importance of understanding the relationship that exists, if at all, between any co-occurring mental health or developmental concerns and the young person's gender identity/gender diverse expression.
\item
  \textbf{Capacity for Decision-Making:} Statement 6.12.c elaborates on the assessment of a young person's emotional maturity and the relevance when an adolescent is considering gender affirming-medical/surgical treatments.
\end{itemize}

\hypertarget{statement-6.4-we-recommend-health-care-professionals-work-with-families-schools-and-other-relevant-settings-to-promote-acceptance-of-gender-diverse-expressions-of-behavior-and-identities-of-the-adolescent.}{%
\section*{Statement 6.4: We recommend health care professionals work with families, schools, and other relevant settings to promote acceptance of gender diverse expressions of behavior and identities of the adolescent.}\label{statement-6.4-we-recommend-health-care-professionals-work-with-families-schools-and-other-relevant-settings-to-promote-acceptance-of-gender-diverse-expressions-of-behavior-and-identities-of-the-adolescent.}}
\addcontentsline{toc}{section}{Statement 6.4: We recommend health care professionals work with families, schools, and other relevant settings to promote acceptance of gender diverse expressions of behavior and identities of the adolescent.}

Multiple studies and related expert consensus
support the implementation of approaches that
promote acceptance and affirmation of gender
diverse youth across all settings, including families, schools, health care facilities, and all other
organizations and communities with which they
interact (e.g., Pariseau et al., 2019; Russell et al.,
2018; Simons et al., 2013; Toomey et al., 2010;
Travers et al., 2012). Acceptance and affirmation
are accomplished through a range of approaches,
actions, and policies we recommend be enacted
across the various relationships and settings in
which a young person exists and functions. It is
important for the family members and community members involved in the adolescent's life to
work collaboratively in these efforts unless their
involvement is considered harmful to the adolescent. Examples proposed by Pariseau et al.~(2019)
and others of acceptance and affirmation of gender diversity and contemplation and expression
of identity that can be implemented by family,
staff, and organizations include:

\begin{enumerate}
\def\labelenumi{\arabic{enumi}.}
\tightlist
\item
  Actions that are supportive of youth drawn to engaging in gender-expansive (e.g., nonconforming) activities and interests;
\item
  Communications that are supportive when youth express their experiences about their gender and gender exploration;
\item
  Use of the youth's asserted name/pronouns;
\item
  Support for youth wearing clothing/uniforms, hairstyles, and items (e.g., jewelry, makeup) they feel affirm their gender;
\item
  Positive and supportive communication with youth about their gender and gender concerns;
\item
  Education about gender diversity issues for people in the young person's life (e.g., family members, health care providers, social support networks), as needed, including information about how to advocate for gender diverse youth in community, school, health care, and other settings;
\item
  Support for gender diverse youth to connect with communities of support (e.g., LGBTQ groups, events, friends);
\item
  Provision of opportunities to discuss, consider, and explore medical treatment options when indicated;
\item
  Antibullying policies that are enforced;
\item
  Inclusion of nonbinary experiences in daily life, reading materials, and curricula (e.g., books, health, and sex education classes, assigned essay topics that move beyond the binary, LGBTQ, and ally groups);
\item
  Gender inclusive facilities that the youth can readily access without segregation from nongender diverse peers (e.g., bathrooms, locker rooms).
\end{enumerate}

We recommend HCPs work with parents,
schools, and other organizations/groups to promote acceptance and affirmation of TGD identities
and expressions, whether social or medical interventions are implemented or not as acceptance
and affirmation are associated with fewer negative
mental health and behavioral symptoms and more
positive mental health and behavioral functioning
(Day et al., 2015; de Vries et al., 2016; Greytak
et al., 2013; Pariseau et al., 2019; Peng et al., 2019;
Russell et al., 2018; Simons et al., 2013; Taliaferro
et al., 2019; Toomey et al., 2010; Travers et al.,
2012). Russell et al.~(2018) found mental health
improvement increases with more acceptance and
affirmation across more settings (e.g., home,
school, work, and friends). Rejection by family,
peers, and school staff (e.g., intentionally using
the name and pronoun the youth does not identify
with, not acknowledging affirmed gender identity,
bullying, harassment, verbal and physical abuse,
poor relationships, rejection for being TGD, eviction) was strongly linked to negative outcomes,
such as anxiety, depression, suicidal ideation, suicide attempts, and substance use (Grossman et al.,
2005; Klein \& Golub; 2016; Pariseau et al., 2019;
Peng et al., 2019; Reisner, Greytak et al., 2015;
Roberts et al., 2013). It is important to be aware
that negative symptoms increase with increased
levels of rejection and continue into adulthood
(Roberts et al., 2013).

Neutral or indifferent responses to a youth's
gender diversity and exploration (e.g., letting a
child tell others their chosen name but not using
the name, not telling family or friends when the
youth wants them to disclose, not advocating
for the child about rejecting behavior from
school staff or peers, not engaging or participating in other support mechanisms (e.g., with
psychotherapists and support groups) have also
been found to have negative consequences, such
as increased depressive symptoms (Pariseau
et al., 2019). For these reasons, it is important
not to ignore a youth's gender questioning or
delay consideration of the youth's gender-related
care needs. There is particular value in professionals recognizing youth need individualized
approaches, support, and consideration of needs
around gender expression, identity, and embodiment over time and across domains and relationships. Youth may need help coping with the
tension of tolerating others' processing/adjusting
to an adolescent's identity exploration and
changes (e.g., Kuper, Lindley et al., 2019). It is
important professionals collaborate with parents
and others as they process their concerns and
feelings and educate themselves about gender
diversity because such processes may not necessarily reflect rejection or neutrality but may
rather represent efforts to develop attitudes and
gather information that foster acceptance (e.g.,
Katz-Wise et al., 2017).

\hypertarget{statement-6.5-we-recommend-against-offering-reparative-and-conversion-therapy-aimed-at-trying-to-change-a-persons-gender-and-lived-gender-expression-to-become-more-congruent-with-the-sex-assignedat-birth.}{%
\section*{Statement 6.5: We recommend against offering reparative and conversion therapy aimed at trying to change a person's gender and lived gender expression to become more congruent with the sex assignedat birth.}\label{statement-6.5-we-recommend-against-offering-reparative-and-conversion-therapy-aimed-at-trying-to-change-a-persons-gender-and-lived-gender-expression-to-become-more-congruent-with-the-sex-assignedat-birth.}}
\addcontentsline{toc}{section}{Statement 6.5: We recommend against offering reparative and conversion therapy aimed at trying to change a person's gender and lived gender expression to become more congruent with the sex assignedat birth.}

Some health care providers, secular or religious organizations, and rejecting families may
undertake efforts to thwart an adolescent's
expression of gender diversity or assertion of a
gender identity other than the expression and
behavior that conforms to the sex assigned at
birth. Such efforts at blocking reversible social
expression or transition may include choosing
not to use the youth's identified name and pronouns or restricting self-expression in clothing
and hairstyles (Craig et al., 2017; Green et al.,
2020). These disaffirming behaviors typically
aim to reinforce views that a young person's
gender identity/expression must match the gender associated with the sex assigned at birth or
expectations based on the sex assigned at birth.
Activities and approaches (sometimes referred
to as ``treatments'') aimed at trying to change a
person's gender identity and expression to
become more congruent with the sex assigned
at birth have been attempted, but these
approaches have not resulted in changes in gender identity (Craig et al., 2017; Green et al.,
2020). We recommend against such efforts
because they have been found to be ineffective
and are associated with increases in mental illness and poorer psychological functioning (Craig
et al., 2017; Green et al., 2020; Turban, Beckwith
et al., 2020).

Much of the research evaluating ``conversion
therapy'' and ``reparative therapy'' has investigated
the impact of efforts to change gender expression
(masculinity or femininity) and has conflated
sexual orientation with gender identity (APA,
2009; Burnes et al., 2016; Craig et al., 2017).
Some of these efforts have targeted both gender
identity and expression (AACAP, 2018).
Conversion/reparative therapy has been linked to
increased anxiety, depression, suicidal ideation,
suicide attempts, and health care avoidance (Craig
et al., 2017; Green et al., 2020; Turban, Beckwith
et al., 2020). Although some of these studies have
been criticized for their methodologies and conclusions (e.g., D'Angelo et al., 2020), this should
not detract from the importance of emphasizing
efforts undertaken a priori to change a person's
identity are clinically and ethically unsound. We
recommend against any type of conversion or
attempts to change a person's gender identity
because 1) both secular and religion-based efforts
to change gender identity/expression have been
associated with negative psychological functioning
that endures into adulthood (Turban, Beckwith
et al., 2020); and 2) larger ethical reasons exist
that should underscore respect for gender diverse
identities.

It is important to note potential factors driving
a young person's gender-related experience and
report of gender incongruence, when carried out
in the context of supporting an adolescent with
self-discovery, is not considered reparative therapy as long as there is no a priori goal to change
or promote one particular gender identity or
expression (AACAP, 2018; see Statement 6.2). To
ensure these explorations are therapeutic, we recommend employing affirmative consideration and
supportive tone in discussing what steps have
been tried, considered, and planned for a youth's
gender expression. These discussion topics may
include what felt helpful or affirming, what felt
unhelpful or distressing and why. We recommend
employing affirmative responses to these steps
and discussions, such as those identified in
SOC-8 Statement 6.4.

\hypertarget{statement-6.6-we-suggest-health-care-professionals-provide-transgender-and-gender-diverse-adolescents-with-health-education-on-chest-binding-and-genital-tucking-including-review-of-the-benefits-and-risks.}{%
\section*{Statement 6.6: We suggest health care professionals provide transgender and gender diverse adolescents with health education on chest binding and genital tucking, including review of the benefits and risks.}\label{statement-6.6-we-suggest-health-care-professionals-provide-transgender-and-gender-diverse-adolescents-with-health-education-on-chest-binding-and-genital-tucking-including-review-of-the-benefits-and-risks.}}
\addcontentsline{toc}{section}{Statement 6.6: We suggest health care professionals provide transgender and gender diverse adolescents with health education on chest binding and genital tucking, including review of the benefits and risks.}

TGD youth may experience distress related to
chest and genital anatomy. Practices such as chest
binding, chest padding, genital tucking, and genital packing are reversible, nonmedical interventions that may help alleviate this distress
(Callen-Lorde, 2020a, 2020b; Deutsch, 2016a;
Olson-Kennedy, Rosenthal et al., 2018; Transcare
BC, 2020). It is important to assess the degree
of distress related to physical development or
anatomy, educate youth about potential nonmedical interventions to address this distress, and
discuss the safe use of these interventions.

Chest binding involves compression of the
breast tissue to create a flatter appearance of the
chest. Studies suggest that up to 87\% of trans
masculine patients report a history of binding
(Jones, 2015; Peitzmeier, 2017). Binding methods
may include the use of commercial binders,
sports bras, layering of shirts, layering of sports
bras, or the use of elastics or other bandages
(Peitzmeier, 2017). Currently, most youth report
learning about binding practices from online
communities composed of peers (Julian,
2019). Providers can play an important role in
ensuring youth receive accurate and reliable
information about the potential benefits and risks
of chest binding. Additionally, providers can
counsel patients about safe binding practices and
monitor for potential negative health effects.
While there are potential negative physical
impacts of binding, youth who bind report many
benefits, including increased comfort, improved
safety, and lower rates of misgendering (Julian,
2019). Common negative health impacts of chest
binding in youth include back/chest pain, shortness of breath, and overheating (Julian, 2019).
More serious negative health impacts such as skin
infections, respiratory infections, and rib fractures
are uncommon and have been associated with
chest binding in adults (Peitzmeier, 2017). If
binding is employed, youth should be advised to
use only those methods considered safe for binding---such as binders specifically designed for the
gender diverse population---to reduce the risk of
serious negative health effects. Methods that are
considered unsafe for binding include the use of
duct tape, ace wraps, and plastic wrap as these
can restrict blood flow, damage skin, and restrict
breathing. If youth report negative health
impacts from chest binding, these should ideally
be addressed by a gender-affirming medical provider with experience working with TGD youth.

Genital tucking is the practice of positioning
the penis and testes to reduce the outward
appearance of a genital bulge. Methods of tucking
include tucking the penis and testes between the
legs or tucking the testes inside the inguinal canal
and pulling the penis back between the legs.
Typically, genitals are held in place by underwear
or a gaff, a garment that can be made or purchased. Limited studies are available on the specific risks and benefits of tucking in adults, and
none have been carried out in youth. Previous
studies have reported tight undergarments are
associated with decreased sperm concentration
and motility. In addition, elevated scrotal temperatures can be associated with poor sperm
characteristics, and genital tucking could theoretically affect spermatogenesis and fertility
(Marsh, 2019) although there are no definitive
studies evaluating these adverse outcomes. Further
research is needed to determine the specific benefits and risks of tucking in youth.

\hypertarget{statement-6.7-we-recommend-providers-consider-prescribing-menstrual-suppression-agents-for-adolescents-experiencing-gender-incongruence-who-may-not-desire-testosterone-therapy-who-desire-but-have-not-yet-begun-testosterone-therapy-or-in-conjunction-with-testosterone-therapy-for-breakthrough-bleeding.}{%
\section*{Statement 6.7 We recommend providers consider prescribing menstrual suppression agents for adolescents experiencing gender incongruence who may not desire testosterone therapy, who desire but have not yet begun testosterone therapy, or in conjunction with testosterone therapy for breakthrough bleeding.}\label{statement-6.7-we-recommend-providers-consider-prescribing-menstrual-suppression-agents-for-adolescents-experiencing-gender-incongruence-who-may-not-desire-testosterone-therapy-who-desire-but-have-not-yet-begun-testosterone-therapy-or-in-conjunction-with-testosterone-therapy-for-breakthrough-bleeding.}}
\addcontentsline{toc}{section}{Statement 6.7 We recommend providers consider prescribing menstrual suppression agents for adolescents experiencing gender incongruence who may not desire testosterone therapy, who desire but have not yet begun testosterone therapy, or in conjunction with testosterone therapy for breakthrough bleeding.}

When discussing the available options of
menstrual-suppressing medications with gender
diverse youth, providers should engage in shared
decision-making, use gender-inclusive language
(e.g., asking patients which terms they utilize to
refer to their menses, reproductive organs, and
genitalia) and perform physical exams in a sensitive, gender-affirmative manner (Bonnington
et al., 2020; Krempasky et al., 2020). There is no
formal research evaluating how menstrual
suppression may impact gender incongruence
and/or dysphoria. However, the use of menstrual
suppression can be an initial intervention that
allows for further exploration of gender-related
goals of care, prioritization of other mental health
care, or both, especially for those who experience
a worsening of gender dysphoria from unwanted
uterine bleeding (see Statement 6.12d; Mehringer
\& Dowshen, 2019). When testosterone is not
used, menstrual suppression can be achieved via
a progestin. To exclude any underlying menstrual
disorders, it is important to obtain a detailed
menstrual history and evaluation prior to implementing menstrual-suppressing therapy (Carswell
\& Roberts, 2017). As part of the discussion about
menstrual-suppressing medications, the need for
contraception and information regarding the
effectiveness of menstrual-suppressing medications as methods of contraception also need to
be addressed (Bonnington et al., 2020). A variety
of menstrual suppression options, such as combined estrogen-progestin medications, oral progestins, depot and subdermal progestin, and
intrauterine devices (IUDs), should be offered to
allow for individualized treatment plans while
properly considering availability, cost and insurance coverage, as well as contraindications and
side effects (Kanj et al., 2019).

Progestin-only hormonal medication are
options, especially in trans masculine or nonbinary youth who are not interested in
estrogen-containing medical therapies as well as
those at risk for thromboembolic events or who
have other contraindications to estrogen therapy
(Carswell \& Roberts, 2017). Progestin-only hormonal medications include oral progestins,
depo-medroxyprogesterone injection, etonogestrel
implant, and levonorgestrel IUD (Schwartz et al.,
2019). Progestin-only hormonal options vary in
terms of efficacy in achieving menstrual suppression and have lower rates of achieving amenorrhea than combined oral contraception (Pradhan
\& Gomez-Lobo, 2019). A more detailed description of the relevant clinical studies is presented
in Chapter 12---Hormone Therapy. HCPs should
not make assumptions regarding the individual's
preferred method of administration as some trans
masculine youth may prefer vaginal rings or IUD
implants (Akgul et al., 2019). Although hormonal
medications require monitoring for potential
mood lability, depressive effects, or both, the benefits and risks of untreated menstrual suppression
in the setting of gender dysphoria should be evaluated on an individual basis. Some patients may
opt for combined oral contraception that includes
different combinations of ethinyl estradiol, with
ranging doses, and different generations of progestins (Pradhan \& Gomez-Lobo, 2019). Lower
dose ethinyl estradiol components of combined
oral contraceptive pills are associated with
increased breakthrough uterine bleeding.
Continuous combined oral contraceptives may be
used to allow for continuous menstrual suppression and can be delivered as transdermal or vaginal rings.

The use of gonadotropin releasing hormone
(GnRH) analogues may also result in menstrual
suppression. However, it is recommended gender
diverse youth meet the eligibility criteria (as outlined in Statement 6.12) before this medication
is considered solely for this purpose (Carswell \&
Roberts, 2017; Pradhan \& Gomez-Lobo, 2019).
Finally, menstrual-suppression medications may
be indicated as an adjunctive therapy for breakthrough uterine bleeding that may occur while
on exogenous testosterone or as a bridging medication while awaiting menstrual suppression with
testosterone therapy. When exogenous testosterone is employed as a gender-affirming hormone,
menstrual suppression is typically achieved in the
first six months of therapy (Ahmad \& Leinung,
2017). However, it is vital adolescents be counseled ovulation and pregnancy can still occur in
the setting of amenorrhea (Gomez et al., 2020;
Kanj et al., 2019).

\hypertarget{statement-6.8-we-recommend-health-care-professionals-maintain-an-ongoing-relationship-with-the-gender-diverse-and-transgender-adolescent-and-any-relevant-caregivers-to-support-the-adolescent-in-their-decision-making-throughout-the-duration-of-puberty-suppression-treatment-hormonal-treatment-and-gender-related-surgery-until-the-transition-is-made-to-adult-care.}{%
\section*{Statement 6.8: We recommend health care professionals maintain an ongoing relationship with the gender diverse and transgender adolescent and any relevant caregivers to support the adolescent in their decision-making throughout the duration of puberty suppression treatment, hormonal treatment, and gender-related surgery until the transition is made to adult care.}\label{statement-6.8-we-recommend-health-care-professionals-maintain-an-ongoing-relationship-with-the-gender-diverse-and-transgender-adolescent-and-any-relevant-caregivers-to-support-the-adolescent-in-their-decision-making-throughout-the-duration-of-puberty-suppression-treatment-hormonal-treatment-and-gender-related-surgery-until-the-transition-is-made-to-adult-care.}}
\addcontentsline{toc}{section}{Statement 6.8: We recommend health care professionals maintain an ongoing relationship with the gender diverse and transgender adolescent and any relevant caregivers to support the adolescent in their decision-making throughout the duration of puberty suppression treatment, hormonal treatment, and gender-related surgery until the transition is made to adult care.}

HCPs with expertise in child and adolescent
development, as described in Statement 6.1, play
an important role in the continuity of care for
young people over the course of their
gender-related treatment needs. Supporting adolescents and their families necessitates approaching care using a developmental lens through
which understanding a young person's evolving
emotional maturity and care needs can take place
over time. As gender-affirming treatment pathways differ based on the needs and experiences
of individual TGD adolescents, decision-making
for these treatments (puberty suppression, estrogens/androgens, gender-affirmation surgeries) can
occur at different points in time within a span
of several years. Longitudinal research demonstrating the benefits of pubertal suppression and
gender-affirming hormone treatment (GAHT)
was carried out in a setting where an ongoing
clinical relationship between the adolescents/families and the multidisciplinary team was maintained (de Vries et al., 2014).

Clinical settings that offer longer appointment
times provide space for adolescents and caregivers
to share important psychosocial aspects of emotional well-being (e.g., family dynamics, school,
romantic, and sexual experiences) that contextualize individualized gender-affirming treatment
needs and decisions as described elsewhere in
the chapter. An ongoing clinical relationship can
take place across settings, whether that be within
a multidisciplinary team or with providers in
different locations who collaborate with one
another. Given the wide variability in the ability
to obtain access to specialized gender care centers, particularly for marginalized groups who
experience disparities with access, it is important
for the HCP to appreciate the existence of any
barriers to care while maintaining flexibility when
defining how an ongoing clinical relationship can
take place in that specific context.

An ongoing clinical relationship that increases
resilience in the youth and provides support to
parents/caregivers who may have their own treatment needs may ultimately lead to increased
parental acceptance---when needed---which is
associated with better mental health outcomes in
youth (Ryan, Huebner et al., 2009).

\hypertarget{statement-6.9-we-recommend-health-care-professionals-involve-relevant-disciplines-including-mental-health-and-medical-professionals-to-reach-a-decision-about-whether-puberty-suppression-hormone-initiation-or-gender-related-surgery-for-gender-diverse-and-transgender-adolescents-are-appropriate-and-remain-indicated-throughout-the-course-of-treatment-until-the-transition-is-made-to-adult-care.}{%
\section*{Statement 6.9: We recommend health care professionals involve relevant disciplines, including mental health and medical professionals, to reach a decision about whether puberty suppression, hormone initiation, or gender-related surgery for gender diverse and transgender adolescents are appropriate and remain indicated throughout the course of treatment until the transition is made to adult care.}\label{statement-6.9-we-recommend-health-care-professionals-involve-relevant-disciplines-including-mental-health-and-medical-professionals-to-reach-a-decision-about-whether-puberty-suppression-hormone-initiation-or-gender-related-surgery-for-gender-diverse-and-transgender-adolescents-are-appropriate-and-remain-indicated-throughout-the-course-of-treatment-until-the-transition-is-made-to-adult-care.}}
\addcontentsline{toc}{section}{Statement 6.9: We recommend health care professionals involve relevant disciplines, including mental health and medical professionals, to reach a decision about whether puberty suppression, hormone initiation, or gender-related surgery for gender diverse and transgender adolescents are appropriate and remain indicated throughout the course of treatment until the transition is made to adult care.}

TGD adolescents with gender dysphoria/gender incongruence who seek gender-affirming
medical and surgical treatments benefit from
the involvement of health care professionals
(HCPs) from different disciplines. Providing care
to TGD adolescents includes addressing 1) diagnostic considerations (see Statements 6.3, 6.12a,
and 6.12b) conducted by a specialized gender
HCP (as defined in Statement 6.1) whenever
possible and necessary; and 2) treatment considerations when prescribing, managing, and
monitoring medications for gender-affirming
medical and surgical care, requiring the training
of the relevant medical/surgical professional. The
list of key disciplines includes but is not limited
to adolescent medicine/primary care, endocrinology, psychology, psychiatry, speech/language
pathology, social work, support staff, and the
surgical team.

The evolving evidence has shown a clinical
benefit for transgender youth who receive their
gender-affirming treatments in multidisciplinary
gender clinics (de Vries et al., 2014; Kuper et al.,
2020; Tollit et al., 2019). Finally, adolescents seeking gender-affirming care in multidisciplinary
clinics are presenting with significant complexity
necessitating close collaboration between mental
health, medical, and/or surgical professionals
(McCallion et al., 2021; Sorbara et al., 2020;
Tishelman et al., 2015).

As not all patients and families are in the position or in a location to access multidisciplinary
care, the lack of available disciplines should not
preclude a young person from accessing needed
care in a timely manner. When disciplines are
available, particularly in centers with existing
multidisciplinary teams, disciplines, or both, it is
recommended efforts be made to include the relevant providers when developing a gender care
team. However, this does not mean all disciplines
are necessary to provide care to a particular
youth and family.

If written documentation or a letter is required
to recommend gender-affirming medical and surgical treatment (GAMST) for an adolescent, only
one letter of assessment from a member of the
multidisciplinary team is needed. This letter needs
to reflect the assessment and opinion from the team
that involves both medical HCPs and MHPs
(American Psychological Association, 2015; Hembree
et al., 2017; Telfer et al., 2018). Further assessment
results and written opinions may be requested when
there is a specific clinical need or when team members are in different locations or choose to write
their own summaries. For further information see
Chapter 5---Assessment for Adults, Statement 5.5.

\hypertarget{statement-6.10-we-recommend-health-care-professionals-working-with-transgender-and-gender-diverse-adolescents-requesting-gender-affirming-medical-or-surgical-treatments-inform-them-prior-to-the-initiation-of-treatment-of-the-reproductive-effects-including-the-potential-loss-of-fertility-and-available-options-to-preserve-fertility-within-the-context-of-the-youths-stage-of-pubertal-development.}{%
\section*{Statement 6.10: We recommend health care professionals working with transgender and gender diverse adolescents requesting gender-affirming medical or surgical treatments inform them, prior to the initiation of treatment, of the reproductive effects, including the potential loss of fertility and available options to preserve fertility within the context of the youth's stage of pubertal development.}\label{statement-6.10-we-recommend-health-care-professionals-working-with-transgender-and-gender-diverse-adolescents-requesting-gender-affirming-medical-or-surgical-treatments-inform-them-prior-to-the-initiation-of-treatment-of-the-reproductive-effects-including-the-potential-loss-of-fertility-and-available-options-to-preserve-fertility-within-the-context-of-the-youths-stage-of-pubertal-development.}}
\addcontentsline{toc}{section}{Statement 6.10: We recommend health care professionals working with transgender and gender diverse adolescents requesting gender-affirming medical or surgical treatments inform them, prior to the initiation of treatment, of the reproductive effects, including the potential loss of fertility and available options to preserve fertility within the context of the youth's stage of pubertal development.}

While assessing adolescents seeking
gender-affirming medical or surgical treatments,
HCPs should discuss the specific ways in which
the required treatment may affect reproductive
capacity. Fertility issues and the specific preservation options are more thoroughly discussed in
Chapter 12---Hormone Therapy and Chapter 16---
Reproductive Health.

It is important HCPs understand what fertility
preservation options exist so they can relay the
information to adolescents. Parents are advised
to be involved in this process and should also
understand the pros and cons of the different
options. HCPs should acknowledge adolescents
and parents may have different views around
reproductive capacity and may therefore come to
different decisions (Quain et al., 2020), which is
why HCPs can be helpful in guiding this process.

HCPs should specifically pay attention to the
developmental and psychological aspects of fertility preservation and decision-making competency for the individual adolescent. While
adolescents may think they have made up their
minds concerning their reproductive capacity, the
possibility their opinions about having
biologically related children in the future might
change over time needs to be discussed with an
HCP who has sufficient experience, is knowledgeable about adolescent development, and has
experience working with parents.

Addressing the long-term consequences on fertility of gender-affirming medical treatments and
ensuring transgender adolescents have realistic
expectations concerning fertility preservation
options or adoption cannot not be addressed with
a one-time discussion but should be part of an
ongoing conversation. This conversation should
occur not only before initiating any medical
intervention (puberty suppression, hormones, or
surgeries), but also during further treatment and
during transition.

Currently, there are only preliminary results
from retrospective studies evaluating transgender
adults and the decisions they made when they
were young regarding the consequences of
medical-affirming treatment on reproductive
capacity. It is important not to make assumptions
about what future adult goals an adolescent may
have. Research in childhood cancer survivors
found participants who acknowledged missed
opportunities for fertility preservation reported
distress and regret surrounding potential infertility
(Armuand et al., 2014; Ellis et al., 2016; Lehmann
et al., 2017). Furthermore, individuals with cancer
who did not prioritize having biological children
before treatment have reported ``changing their
minds'' in survivorship (Armuand et al., 2014).

Given the complexities of the different fertility
preservation options and the challenges HCPs
may experience discussing fertility with the adolescent and the family (Tishelman et al., 2019),
a fertility consultation is an important consideration for every transgender adolescent who pursues medical-affirming treatments unless the local
situation is such that a fertility consultation is
not covered by insurance or public health care
plans, is not available locally, or the individual
circumstances make this unpreferable.

\hypertarget{statement-6.11-we-recommend-when-gender-affirming-medical-or-surgical-treatments-are-indicated-for-adolescents-health-care-professionals-working-with-transgender-and-gender-diverse-adolescents-involve-parentsguardians-in-the-assessment-and-treatment-process-unless-their-involvement-is-determined-to-be-harmful-to-the-adolescent-or-not-feasible.}{%
\section*{Statement 6.11: We recommend when gender-affirming medical or surgical treatments are indicated for adolescents, health care professionals working with transgender and gender diverse adolescents involve parent(s)/guardian(s) in the assessment and treatment process, unless their involvement is determined to be harmful to the adolescent or not feasible.}\label{statement-6.11-we-recommend-when-gender-affirming-medical-or-surgical-treatments-are-indicated-for-adolescents-health-care-professionals-working-with-transgender-and-gender-diverse-adolescents-involve-parentsguardians-in-the-assessment-and-treatment-process-unless-their-involvement-is-determined-to-be-harmful-to-the-adolescent-or-not-feasible.}}
\addcontentsline{toc}{section}{Statement 6.11: We recommend when gender-affirming medical or surgical treatments are indicated for adolescents, health care professionals working with transgender and gender diverse adolescents involve parent(s)/guardian(s) in the assessment and treatment process, unless their involvement is determined to be harmful to the adolescent or not feasible.}

When there is an indication an adolescent
might benefit from a gender-affirming medical
or surgical treatment, involving the parent(s) or
primary caregiver(s) in the assessment process is
recommended in almost all situations
(Edwards-Leeper \& Spack, 2012; Rafferty et al.,
2018). Exceptions to this might include situations
in which an adolescent is in foster care, child
protective services, or both, and custody and parent involvement would be impossible, inappropriate, or harmful. Parent and family support of
TGD youth is a primary predictor of youth
well-being and is protective of the mental health
of TGD youth (Gower, Rider, Coleman et al.,
2018; Grossman et al., 2019; Lefevor et al., 2019;
McConnell et al., 2015; Pariseau et al., 2019;
Ryan, 2009; Ryan et al., 2010; Simons et al., 2013;
Wilson et al., 2016). Therefore, including
parent(s)/caregiver(s) in the assessment process
to encourage and facilitate increased parental
understanding and support of the adolescent may
be one of the most helpful practices available.

Parent(s)/caregiver(s) may provide key information for the clinical team, such as the young person's gender and overall developmental, medical,
and mental health history as well as insights into
the young person's level of current support, general
functioning, and well-being. Concordance or divergence of reports given by the adolescent and their
parent(s)/caregiver(s) may be important information for the assessment team and can aid in
designing and shaping individualized youth and
family supports (De Los Reyes et al., 2019;
Katz-Wise et al., 2017). Knowledge of the family
context, including resilience factors and challenges,
can help providers know where special supports
would be needed during the medical treatment
process. Engagement of parent(s)/caregiver(s) is
also important for educating families about various
treatment approaches, ongoing follow-up and care
needs, and potential treatment complications.
Through psychoeducation regarding clinical gender
care options and participation in the assessment
process, which may unfold over time, parent(s)/
caregiver(s) may better understand their adolescent
child's gender-related experience and needs
(Andrzejewski et al., 2020; Katz-Wise et al., 2017).

Parent/caregiver concerns or questions regarding the stability of gender-related needs over time
and implications of various gender-affirming
interventions are common and should not be
dismissed. It is appropriate for parent(s)/caregiver(s) to ask these questions, and there are
cases in which the parent(s)/caregiver(s)' questions or concerns are particularly helpful in
informing treatment decisions and plans. For
example, a parent/caregiver report may provide
critical context in situations in which a young
person experiences very recent or sudden
self-awareness of gender diversity and a corresponding gender treatment request, or when there
is concern for possible excessive peer and social
media influence on a young person's current
self-gender concept. Contextualization of the parent/caregiver report is also critical, as the report
of a young person's gender history as provided
by parent(s)/caregiver(s) may or may not align
with the young person's self-report. Importantly,
gender histories may be unknown to parent(s)/
caregiver(s) because gender may be internal experience for youth, not known by others unless it
is discussed. For this reason, an adolescent's
report of their gender history and experience is
central to the assessment process.

Some parents may present with unsupportive
or antagonistic beliefs about TGD identities, clinical gender care, or both (Clark et al., 2020).
Such unsupportive perspectives are an important
therapeutic target for families. Although challenging parent perspectives may in some cases seem
rigid, providers should not assume this is the
case. There are many examples of parent(s)/caregiver(s) who, over time with support and psychoeducation, have become increasingly accepting
of their TGD child's gender diversity and
care needs.

Helping youth and parent(s)/caregiver(s) work
together on important gender care decisions is a
primary goal. However, in some cases, parent(s)/
caregiver(s) may be too rejecting of their adolescent child and their child's gender needs to be
part of the clinical evaluation process. In these
situations, youth may require the engagement of
larger systems of advocacy and support to move
forward with the necessary support and care
(Dubin et al., 2020).

\hypertarget{statement-6.12-we-recommend-health-care-professionals-assessing-transgender-and-gender-diverse-adolescents-only-recommend-gender-affirming-medical-or-surgical-treatments-requested-by-the-patient-when}{%
\section*{Statement 6.12: We recommend health care professionals assessing transgender and gender diverse adolescents only recommend gender-affirming medical or surgical treatments requested by the patient when:}\label{statement-6.12-we-recommend-health-care-professionals-assessing-transgender-and-gender-diverse-adolescents-only-recommend-gender-affirming-medical-or-surgical-treatments-requested-by-the-patient-when}}
\addcontentsline{toc}{section}{Statement 6.12: We recommend health care professionals assessing transgender and gender diverse adolescents only recommend gender-affirming medical or surgical treatments requested by the patient when:}

\hypertarget{statement-6.12.a-the-adolescent-meets-the-diagnostic-criteria-of-gender-incongruence-as-per-the-icd-11-in-situations-where-a-diagnosis-is-necessary-to-access-health-care.-in-countries-that-have-not-implemented-the-latest-icd-other-taxonomies-may-be-used-although-efforts-should-be-undertaken-to-utilize-the-latest-icd-as-soon-as-practicable.}{%
\subsection*{Statement 6.12.a: The adolescent meets the diagnostic criteria of gender incongruence as per the ICD-11 in situations where a diagnosis is necessary to access health care. In countries that have not implemented the latest ICD, other taxonomies may be used although efforts should be undertaken to utilize the latest ICD as soon as practicable.}\label{statement-6.12.a-the-adolescent-meets-the-diagnostic-criteria-of-gender-incongruence-as-per-the-icd-11-in-situations-where-a-diagnosis-is-necessary-to-access-health-care.-in-countries-that-have-not-implemented-the-latest-icd-other-taxonomies-may-be-used-although-efforts-should-be-undertaken-to-utilize-the-latest-icd-as-soon-as-practicable.}}
\addcontentsline{toc}{subsection}{Statement 6.12.a: The adolescent meets the diagnostic criteria of gender incongruence as per the ICD-11 in situations where a diagnosis is necessary to access health care. In countries that have not implemented the latest ICD, other taxonomies may be used although efforts should be undertaken to utilize the latest ICD as soon as practicable.}

When working with TGD adolescents, HCPs
should realize while a classification may give
access to care, pathologizing transgender identities may be experienced as stigmatizing (Beek
et al., 2016). Assessments related to gender health
and gender diversity have been criticized, and
controversies exist around diagnostic systems
(Drescher, 2016).

HCPs should assess the overall gender-related
history and gender care-related needs of youth.
Through this assessment process, HCPs may provide a diagnosis when it is required to get access
to transgender-related care.

Gender incongruence and gender dysphoria
are the two diagnostic terms used in the World
Health Organization's International Classification
of Diseases (ICD) and the American Psychiatric
Association's Diagnostic and Statistical Manual
of Mental Disorders (DSM), respectively. Of
these two widely used classification systems, the
DSM is for psychiatric classifications only and
the ICD contains all diseases and conditions
related to physical as well as mental health. The
most recent versions of these two systems, the
DSM-5 and the ICD-11, reflect a long history
of reconceptualizing and de-psychopathologizing
gender-related diagnoses (American Psychiatric
Association, 2013; World Health Organization,
2019a). Compared with the earlier version, the
DSM-5 replaced gender identity disorder with
gender dysphoria, acknowledging the distress
experienced by some people stemming from the
incongruence between experienced gender identity and the sex assigned at birth. In the most
recent revision, the DSM-5-TR, no changes in
the diagnostic criteria for gender dysphoria are
made. However, terminology was adapted into
the most appropriate current language (e.g.,
birth-assigned gender instead of natal-gender
and gender-affirming treatment instead of gender reassignment (American Psychiatric
Association, 2022). Compared with the ICD 10th
edition, the gender incongruence classification
was moved from the Mental Health chapter to
the Conditions Related to Sexual Health chapter
in the ICD-11. When compared with the DSM-5
classification of gender dysphoria, one important
reconceptualization is distress is not a required
indicator of the ICD-11 classification of gender
incongruence (WHO, 2019a). After all, when
growing up in a supporting and accepting environment, the distress and impairment criterion,
an inherent part of every mental health condition, may not be applicable (Drescher, 2012). As
such, the ICD-11 classification of gender incongruence may better capture the fullness of gender diversity experiences and related clinical
gender needs.

Criteria for the ICD-11 classification gender
incongruence of adolescence or adulthood require
a marked and persistent incongruence between an
individual´s experienced gender and the assigned
sex, which often leads to a need to ``transition'' to
live and be accepted as a person of the experienced gender. For some, this includes hormonal
treatment, surgery, or other health care services
to enable the individual´s body to align as much
as required, and to the extent possible, with the
person's experienced gender. Relevant for adolescents is the indicator that a classification cannot
be assigned ``prior to the onset of puberty.'' Finally,
it is noted ``that gender variant behaviour and
preferences alone are not a basis for assigning the
classification'' (WHO, ICD-11, 2019a).

Criteria for the DSM-5 and DSM-5-TR classification of gender dysphoria in adolescence and
adulthood denote ``a marked incongruence between
one's experienced/expressed gender and assigned
gender, of at least 6 months' duration' (criterion
A, fulfilled when 2 of 6 subcriteria are manifest;
DSM-5, APA, 2013; DSM 5-TR, APA, 2022).

Of note, although a gender-related classification is one of the requirements for receiving
medical gender-affirming care, such a classification alone does not indicate a person needs
medical-affirming care. The range of youth experiences of gender incongruence necessitates professionals provide a range of treatments or
interventions based on the individual's needs.
Counseling, gender exploration, mental health
assessment and, when needed, treatment with
MHPs trained in gender development may all be
indicated with or without the implementation of
medical-affirming care.

\hypertarget{statement-6.12.b-the-experience-of-gender-diversityincongruence-is-marked-and-sustained-over-time.}{%
\subsection*{Statement 6.12.b: The experience of gender diversity/incongruence is marked and sustained over time.}\label{statement-6.12.b-the-experience-of-gender-diversityincongruence-is-marked-and-sustained-over-time.}}
\addcontentsline{toc}{subsection}{Statement 6.12.b: The experience of gender diversity/incongruence is marked and sustained over time.}

Identity exploration and consolidation are
experienced by many adolescents (Klimstra et al.,
2010; Topolewska-Siedzik \& Cieciuch, 2018).
Identity exploration during adolescence may
include a process of self-discovery around gender
and gender identity (Steensma, Kreukels et al.,
2013). Little is known about how processes that
underlie consolidation of gender identity during
adolescence (e.g., the process of commitment to
specific identities) may impact a young person's
experience(s) or needs over time.

Therefore, the level of reversibility of a
gender-affirming medical intervention should be
considered along with the sustained duration of
a young person's experience of gender incongruence when initiating treatment. Given potential shifts in gender-related experiences and
needs during adolescence, it is important to
establish the young person has experienced several years of persistent gender diversity/incongruence prior to initiating less reversible
treatments such as gender-affirming hormones
or surgeries. Puberty suppression treatment,
which provides more time for younger adolescents to engage their decision-making capacities,
also raises important considerations (see
Statement 6.12f and Chapter 12---Hormone
Therapy) suggesting the importance of a sustained experience of gender incongruence/diversity prior to initiation. However, in this age
group of younger adolescents, several years is
not always practical nor necessary given the
premise of the treatment as a means to buy time
while avoiding distress from irreversible pubertal
changes. For youth who have experienced a
shorter duration of gender incongruence, social
transition-related and/or other medical supports
(e.g., menstrual suppression/androgen blocking)
may also provide some relief as well as furnishing additional information to the clinical team
regarding a young person's broad gender care
needs (see Statements 6.4, 6.6, and 6.7).

Establishing evidence of persistent gender
diversity/incongruence typically requires careful
assessment with the young person over time (see
Statement 6.3). Whenever possible and when
appropriate, the assessment and discernment process should also include the parent(s)/caregiver(s)
(see Statement 6.11). Evidence demonstrating
gender diversity/incongruence sustained over time
can be provided via history obtained directly
from the adolescent and parents/caregivers when
this information is not documented in the medical records.

The research literature on continuity versus
discontinuity of gender-affirming medical care
needs/requests is complex and somewhat difficult to interpret. A series of studies conducted
over the last several decades, including some
with methodological challenges (as noted by
Temple Newhook et al., 2018; Winters et al.,
2018) suggest the experience of gender incongruence is not consistent for all children as they
progress into adolescence. For example, a subset
of youth who experienced gender incongruence
or who socially transitioned prior to puberty
over time can show a reduction in or even full
discontinuation of gender incongruence (de
Vries et al., 2010; Olson et al., 2022; Ristori \&
Steensma, 2016; Singh et al., 2021; Wagner et al.,
2021). However, there has been less research
focused on rates of continuity and discontinuity
of gender incongruence and gender-related needs
in pubertal and adolescent populations. The data
available regarding broad unselected
gender-referred pubertal/adolescent cohorts
(from the Amsterdam transgender clinic) suggest
that, following extended assessments over time,
a subset of adolescents with gender incongruence presenting for gender care elect not to
pursue gender-affirming medical care

(Arnoldussen et al., 2019; de Vries, Steensma
et al., 2011). Importantly, findings from studies
of gender incongruent pubertal/adolescent
cohorts, in which participants who have undergone comprehensive gender evaluation over
time, have shown persistent gender incongruence
and gender-related need and have received referrals for medical gender care, suggest low levels
of regret regarding gender-related medical care
decisions (de Vries et al., 2014; Wiepjes et al.,
2018). Critically, these findings of low regret
can only currently be applied to youth who have
demonstrated sustained gender incongruence
and gender-related needs over time as established through a comprehensive and iterative
assessment (see Statement 6.3).

\hypertarget{statement-6.12.c-the-adolescent-demonstrates-the-emotional-and-cognitive-maturity-required-to-provide-informed-consentassent-for-the-treatment.}{%
\subsection*{Statement 6.12.c: The adolescent demonstrates the emotional and cognitive maturity required to provide informed consent/assent for the treatment.}\label{statement-6.12.c-the-adolescent-demonstrates-the-emotional-and-cognitive-maturity-required-to-provide-informed-consentassent-for-the-treatment.}}
\addcontentsline{toc}{subsection}{Statement 6.12.c: The adolescent demonstrates the emotional and cognitive maturity required to provide informed consent/assent for the treatment.}

The process of informed consent includes communication between a patient and their provider
regarding the patient's understanding of a potential intervention as well as, ultimately, the patient's
decision whether to receive the intervention. In
most settings, for minors, the legal guardian is
integral to the informed consent process: if a
treatment is to be given, the legal guardian (often
the parent{[}s{]}/caregiver{[}s{]}) provides the informed
consent to do so. In most settings, assent is a
somewhat parallel process in which the minor
and the provider communicate about the intervention and the provider assesses the level of
understanding and intention.

A necessary step in the informed consent/
assent process for considering gender-affirming
medical care is a careful discussion with qualified
HCPs trained to assess the emotional and cognitive maturity of adolescents. The reversible and
irreversible effects of the treatment, as well as
fertility preservation options (when applicable),
and all potential risks and benefits of the intervention are important components of the discussion. These discussions are required when
obtaining informed consent/assent. Assessment
of cognitive and emotional maturity is important
because it helps the care team understand the
adolescent's capacity to be informed.

The skills necessary to assent/consent to any
medical intervention or treatment include the
ability to 1) comprehend the nature of the treatment; 2) reason about treatment options, including the risks and benefits; 3) appreciate the nature
of the decision, including the long-term consequences; and 4) communicate choice
(Grootens-Wiegers et al., 2017). In the case of
gender- affirming medical treatments, a young
person should be well-informed about what the
treatment may and may not accomplish, typical
timelines for changes to appear (e.g., with
gender-affirming hormones), and any implications
of stopping the treatment. Gender-diverse youth
should fully understand the reversible, partially
reversible, and irreversible aspects of a treatment,
as well as the limits of what is known about certain treatments (e.g., the impact of pubertal suppression on brain development (Chen and Loshak,
2020)). Gender-diverse youth should also understand, although many gender-diverse youth begin
gender- affirming medical care and experience
that care as a good fit for them long-term, there
is a subset of individuals who over time discover
this care is not a fit for them (Wiepjes et al.,
2018). Youth should know such shifts are sometimes connected to a change in gender needs over
time, and in some cases, a shift in gender identity
itself. Given this information, gender diverse
youth must be able to reason thoughtfully about
treatment options, considering the implications of
the choices at hand. Furthermore, as a foundation
for providing assent, the gender-diverse young
person needs to be able to communicate
their choice.

The skills needed to accomplish the tasks
required for assent/consent may not emerge at
specific ages per se (Grootens-Wiegers et al.,
2017). There may be variability in these capacities
related to developmental differences and mental
health presentations (Shumer \& Tishelman, 2015)
and dependent on the opportunities a young person has had to practice these skills (Alderson,
2007). Further, assessment of emotional and cognitive maturity must be conducted separately for
each gender-related treatment decision
(Vrouenraets et al., 2021).

The following questions may be useful to consider in assessing a young person's emotional and
cognitive readiness to assent or consent to a specific gender-affirming treatment:

\begin{itemize}
\tightlist
\item
  Can the young person think carefully into the future and consider the implications of a partially or fully irreversible intervention?
\item
  Does the young person have sufficient self-reflective capacity to consider the possibility that gender-related needs and priorities can develop over time, and gender-related priorities at a certain point in time might change?
\item
  Has the young person, to some extent, thought through the implications of what they might do if their priorities around gender do change in the future?
\item
  Is the young person able to understand and manage the day-to-day short- and long-term aspects of a specific medical treatment (e.g., medication adherence, administration, and necessary medical follow-ups)?
\end{itemize}

Assessment of emotional and cognitive maturity may be accomplished over time as the care
team continues to engage in conversations about
the treatment options and affords the young person the opportunity to practice thinking into
the future and flexibly consider options and
implications. For youth with neurodevelopmental
and/or some types of mental health differences,
skills for future thinking, planning, big picture
thinking, and self-reflection may be less-well
developed (Dubbelink \& Geurts, 2017). In these
cases, a more careful approach to consent and
assent may be required, and this may include
additional time and structured opportunities for
the young person to practice the skills necessary
for medical decision-making (Strang, Powers
et al., 2018).

For unique situations in which an adolescent
minor is consenting for their own treatment without parental permission (see Statement 6.11),
extra care must be taken to support the adolescent's informed decision-making. This will typically require greater levels of engagement of and
collaboration between the HCPs working with
the adolescent to provide the young person
appropriate cognitive and emotional support to
consider options, weigh benefits and potential
challenges/costs, and develop a plan for any
needed (and potentially ongoing) supports associated with the treatment.

\hypertarget{statement-6.12.d-the-adolescents-mental-health-concerns-if-any-that-may-interfere-with-diagnostic-clarity-capacity-to-consent-andor-gender-affirming-medical-treatments-have-been-addressed.}{%
\subsection*{Statement 6.12.d: The adolescent's mental health concerns (if any) that may interfere with diagnostic clarity, capacity to consent, and/or gender-affirming medical treatments have been addressed.}\label{statement-6.12.d-the-adolescents-mental-health-concerns-if-any-that-may-interfere-with-diagnostic-clarity-capacity-to-consent-andor-gender-affirming-medical-treatments-have-been-addressed.}}
\addcontentsline{toc}{subsection}{Statement 6.12.d: The adolescent's mental health concerns (if any) that may interfere with diagnostic clarity, capacity to consent, and/or gender-affirming medical treatments have been addressed.}

Evidence indicates TGD adolescents are at
increased risk of mental health challenges, often
related to family/caregiver rejection, non-affirming
community environments, and neurodiversityrelated factors (e.g., de Vries et al., 2016; Pariseau
et al., 2019; Ryan et al., 2010; Weinhardt et al.,
2017). A young person's mental health challenges
may impact their conceptualization of their gender development histor y and gender
identity-related needs, the adolescent's capacity
to consent, and the ability of the young person
to engage in or receive medical treatment.
Additionally, like cisgender youth, TGD youth
may experience mental health concerns irrespective of the presence of gender dysphoria or gender incongruence. In particular, depression and
self-harm may be of specific concern; many studies reveal depression scores and emotional and
behavioral problems comparable to those reported
in populations referred to mental health clinics
(Leibowitz \& de Vries, 2016). Higher rates of
suicidal ideation, suicide attempts, and self-harm
have also been reported (de Graaf et al., 2020).
In addition, eating disorders occur more frequently than expected in non-referred populations (Khatchadourian et al., 2013; Ristori et al.,
2019; Spack et al., 2012). Importantly, TGD adolescents show high rates of autism spectrum disorder/characteristics (Øien et al., 2018; van der
Miesen et al., 2016; see also Statement 6.1d).
Other neurodevelopmental presentations and/or
mental health challenges may also be present,
(e.g., ADHD, intellectual disability, and psychotic
disorders (de Vries, Doreleijers et al., 2011; Meijer
et al., 2018; Parkes \& Hall, 2006).

Of note, many transgender adolescents are
well-functioning and experience few if any mental
health concerns. For example, socially transitioned pubertal adolescents who receive medical
gender- affirming treatment at specialized gender
clinics may experience mental health outcomes
equivalent to those of their cisgender peers (e.g.,
de Vries et al., 2014; van der Miesen et al., 2020).
A provider's key task is to assess the direction
of the relationships that exist between any mental
health challenges and the young person's
self-understanding of gender care needs and then
prioritize accordingly.

Mental health difficulties may challenge the
assessment and treatment of gender-related needs
of TGD adolescents in various ways:

\begin{enumerate}
\def\labelenumi{\arabic{enumi}.}
\tightlist
\item
  First, when a TGD adolescent is experiencing acute suicidality, self-harm, eating disorders, or other mental health crises that threaten physical health, safety must be prioritized. According to the local context and existing guidelines, appropriate care should seek to mitigate the threat or crisis so there is sufficient time and stabilization for thoughtful gender-related assessment and decision-making. For example, an actively suicidal adolescent may not be emotionally able to make an informed decision regarding gender-affirming medical/surgical treatment. If indicated, safety-related interventions should not preclude starting gender-affirming care.
\item
  Second, mental health can also complicate the assessment of gender development and gender identity-related needs. For example, it is critical to differentiate gender incongruence from specific mental health presentations, such as obsessions and compulsions, special interests in autism, rigid thinking, broader identity problems, parent/child interaction difficulties, severe
  developmental anxieties (e.g., fear of growing up and pubertal changes unrelated to gender identity), trauma, or psychotic thoughts. Mental health challenges that interfere with the clarity of identity development and gender-related decision-making should be prioritized and addressed.
\item
  Third, decision-making regarding gender-affirming medical treatments that have life-long consequences requires thoughtful, future-oriented thinking by the adolescent, with support from the parents/ caregivers, as indicated (see Statement 6.11). To be able to make such an informed decision, an adolescent should be able to understand the issues, express a choice, appreciate and give careful thought regarding the wish for medical-affirming treatment (see Statement 6.12c). Neurodevelopmental differences, such as autistic features or autism spectrum disorder (see Statement 6.1d, e.g., communication differences; a preference for concrete or rigid thinking; differences in self-awareness, future thinking and planning), may challenge the assessment and decision-making process; neurodivergent youth may require extra support, structure, psychoeducation, and time built into the assessment process (Strang, Powers et al., 2018). Other mental health presentations that involve reduced communication and self-advocacy, difficulty engaging in assessment, memory and concentration difficulties, hopelessness, and difficulty engaging in future-oriented thinking may complicate assessment and decision-making. In such cases, extended time is often necessary before any decisions regarding medical-affirming treatment can be made.
\item
  Finally, while addressing mental health concerns is important during the course of medical treatment, it does not mean all mental health challenges can or should be resolved completely. However, it is important any mental health concerns are addressed sufficiently so that gender -affirming medical treatment can be provided optimally (e.g., medication adherence, attending follow-up medical appointments, and self-care, particularly during a postoperative course).
\end{enumerate}

\hypertarget{statement-6.12.e-the-adolescent-has-been-informed-of-the-reproductive-effects-including-the-potential-loss-of-fertility-and-available-options-to-preserve-fertility-and-these-have-been-discussed-in-the-context-of-the-adolescents-stage-of-pubertal-development.}{%
\subsection*{Statement 6.12.e: The adolescent has been informed of the reproductive effects, including the potential loss of fertility, and available options to preserve fertility, and these have been discussed in the context of the adolescent's stage of pubertal development.}\label{statement-6.12.e-the-adolescent-has-been-informed-of-the-reproductive-effects-including-the-potential-loss-of-fertility-and-available-options-to-preserve-fertility-and-these-have-been-discussed-in-the-context-of-the-adolescents-stage-of-pubertal-development.}}
\addcontentsline{toc}{subsection}{Statement 6.12.e: The adolescent has been informed of the reproductive effects, including the potential loss of fertility, and available options to preserve fertility, and these have been discussed in the context of the adolescent's stage of pubertal development.}

For guidelines regarding the clinical approach,
the scientific background, and the rationale, see
Chapter 12---Hormone Therapy and Chapter 16---
Reproductive Health.

\hypertarget{statement-6.12.f-the-adolescent-has-reached-tanner-stage-2-of-puberty-for-pubertal-suppression-to-be-initiated.}{%
\subsection*{Statement 6.12.f: The adolescent has reached Tanner stage 2 of puberty for pubertal suppression to be initiated.}\label{statement-6.12.f-the-adolescent-has-reached-tanner-stage-2-of-puberty-for-pubertal-suppression-to-be-initiated.}}
\addcontentsline{toc}{subsection}{Statement 6.12.f: The adolescent has reached Tanner stage 2 of puberty for pubertal suppression to be initiated.}

The onset of puberty is a pivotal point for
many gender diverse youth. For some, it creates
an intensification of their gender incongruence,
and for others, pubertal onset may lead to gender
fluidity (e.g., a transition from binary to nonbinary gender identity) or even attenuation of a
previously affirmed gender identity (Drummond
et al., 2008; Steensma et al., 2011, Steensma,
Kreukels et al., 2013; Wallien \& Cohen-Kettenis,
2008). The use of puberty-blocking medications,
such as GnRH analogues, is not recommended
until children have achieved a minimum of
Tanner stage 2 of puberty because the experience
of physical puberty may be critical for further
gender identity development for some TGD adolescents (Steensma et al., 2011). Therefore,
puberty blockers should not be implemented in
prepubertal gender diverse youth (Waal \&
Cohen-Kettenis, 2006). For some youth, GnRH
agonists may be appropriate in late stages or in
the post-pubertal period (e.g., Tanner stage 4 or
5), and this should be highly individualized. See
Chapter 12---Hormone Therapy for a more comprehensive review of the use of GnRH agonists.

Variations in the timing of pubertal onset is
due to multiple factors (e.g., sex assigned at birth,
genetics, nutrition, etc.). Tanner staging refers to
five stages of pubertal development ranging from
prepubertal (Tanner stage 1) to post-pubertal,
and adult sexual maturity (Tanner stage 5)
(Marshall \& Tanner, 1969, 1970). For assigned
females at birth, pubertal onset (e.g., gonadarche)
is defined by the occurrence of breast budding
(Tanner stage 2), and for birth-assigned males,
the achievement of a testicular volume of greater
than or equal to 4 mL (Roberts \& Kaiser, 2020).
An experienced medical provider should be relied
on to differentiate the onset of puberty from
physical changes such as pubic hair and apocrine
body odor due to sex steroids produced by the
adrenal gland (e.g., adrenarche) as adrenarche
does not warrant the use of puberty-blocking
medications (Roberts \& Kaiser, 2020). Educating
parents and families about the difference between
adrenarche and gonadarche helps families understand the timing during which shared
decision-making about gender-affirming medical
therapies should be undertaken with their multidisciplinary team.

The importance of addressing other risks and
benefits of pubertal suppression, both hypothetical and actual, cannot be overstated. Evidence
supports the existence of surgical implications for
transgender girls who proceed with pubertal suppression (van de Grift et al., 2020). Longitudinal
data exists to demonstrate improvement in
romantic and sexual satisfaction for adolescents
receiving puberty suppression, hormone treatment
and surgery (Bungener et al., 2020). A study on
surgical outcomes of laparoscopic intestinal vaginoplasty (performed because of limited genital
tissue after the use of puberty blockers) in transgender women revealed that the majority experienced orgasm after surgery (84\%), although a
specific correlation between sexual pleasure outcomes and the timing of pubertal suppression
initiation was not discussed in the study (Bouman,
van der Sluis et al., 2016), nor does the study
apply to those who would prefer a different surgical procedure. This underscores the importance
of engaging in discussions with families about
the future unknowns related to surgical and sexual health outcomes.

\hypertarget{statement-6.12.g-the-adolescent-had-at-least-12-months-of-gender-affirming-hormone-therapy-or-longer-if-required-to-achieve-the-desired-surgical-result-for-gender-affirming-procedures-including-breast-augmentation-orchiectomy-vaginoplasty-hysterectomy-phalloplasty-metoidioplasty-and-facial-surgery-as-part-of-gender-affirming-treatment-unless-hormone-therapy-is-either-not-desired-or-is-medically-contraindicated.}{%
\subsection*{Statement 6.12.g: The adolescent had at least 12 months of gender-affirming hormone therapy or longer, if required, to achieve the desired surgical result for gender-affirming procedures, including breast augmentation, orchiectomy, vaginoplasty, hysterectomy, phalloplasty, metoidioplasty, and facial surgery as part of gender-affirming treatment unless hormone therapy is either not desired or is medically contraindicated.}\label{statement-6.12.g-the-adolescent-had-at-least-12-months-of-gender-affirming-hormone-therapy-or-longer-if-required-to-achieve-the-desired-surgical-result-for-gender-affirming-procedures-including-breast-augmentation-orchiectomy-vaginoplasty-hysterectomy-phalloplasty-metoidioplasty-and-facial-surgery-as-part-of-gender-affirming-treatment-unless-hormone-therapy-is-either-not-desired-or-is-medically-contraindicated.}}
\addcontentsline{toc}{subsection}{Statement 6.12.g: The adolescent had at least 12 months of gender-affirming hormone therapy or longer, if required, to achieve the desired surgical result for gender-affirming procedures, including breast augmentation, orchiectomy, vaginoplasty, hysterectomy, phalloplasty, metoidioplasty, and facial surgery as part of gender-affirming treatment unless hormone therapy is either not desired or is medically contraindicated.}

GAHT leads to anatomical, physiological, and
psychological changes. The onset of the anatomic
effects (e.g., clitoral growth, breast growth, vaginal mucosal atrophy) may begin early after the
initiation of therapy, and the peak effect is
expected at 1--2 years (T'Sjoen et al., 2019). To
ensure sufficient time for psychological adaptations to the physical change during an important
developmental time for the adolescent, 12 months
of hormone treatment is suggested. Depending
upon the surgical result required, a period of
hormone treatment may need to be longer (e.g.,
sufficient clitoral virilization prior to metoidioplasty/phalloplasty, breast growth and skin expansion prior to breast augmentation, softening of
skin and changes in facial fat distribution prior
to facial GAS) (de Blok et al., 2021).

For individuals who are not taking hormones
prior to surgical interventions, it is important
surgeons review the impact of hormone therapy
on the proposed surgery. In addition, for individuals undergoing gonadectomy who are not
taking hormones, a plan for hormone replacement can be developed with their prescribing
professional prior to surgery.

\hypertarget{consideration-of-ages-for-gender-affirming-medical-and-surgical-treatment-for-adolescents}{%
\section*{Consideration of ages for gender-affirming medical and surgical treatment for adolescents}\label{consideration-of-ages-for-gender-affirming-medical-and-surgical-treatment-for-adolescents}}
\addcontentsline{toc}{section}{Consideration of ages for gender-affirming medical and surgical treatment for adolescents}

Age has a strong, albeit imperfect, correlation
with cognitive and psychosocial development and
may be a useful objective marker for determining
the potential timing of interventions (Ferguson
et al., 2021). Higher (i.e., more advanced) ages
may be required for treatments with greater irreversibility, complexity, or both. This approach
allows for continued cognitive/emotional maturation that may be required for the adolescent
to fully consider and consent to increasingly complex treatments (see Statement 6.12c).

A growing body of evidence indicates providing gender-affirming treatment for gender diverse
youth who meet criteria leads to positive outcomes (Achille et al., 2020; de Vries et al., 2014;
Kuper et al., 2020). There is, however, limited
data on the optimal timing of gender-affirming
interventions as well as the long-term physical,
psychological, and neurodevelopmental outcomes
in youth (Chen et al., 2020; Chew et al., 2018;
Olson-Kennedy et al., 2016). Currently, the only
existing longitudinal studies evaluating gender
diverse youth and adult outcomes are based on
a specific model (i.e., the Dutch approach) that
involved a comprehensive initial assessment with
follow-up. In this approach, pubertal suppression
was considered at age 12, GAHT at age 16, and
surgical interventions after age 18 with exceptions
in some cases. It is not clear if deviations from
this approach would lead to the same or different
outcomes. Longitudinal studies are currently
underway to better define outcomes as well as
the safety and efficacy of gender-affirming treatments in youth (Olson-Kennedy, Garofalo et al.,
2019; Olson-Kennedy, Rosenthal et al., 2019).
While the long-term effects of gender-affirming
treatments initiated in adolescence are not fully
known, the potential negative health consequences
of delaying treatment should also be considered
(de Vries et al., 2021). As the evidence base
regarding outcomes of gender-affirming interventions in youth continues to grow, recommendations on the timing and readiness for these
interventions may be updated.

Previous guidelines regarding gender-affirming
treatment of adolescents recommended partially
reversible GAHT could be initiated at approximately 16 years of age (Coleman et al., 2012;
Hembree et al., 2009). More recent guidelines
suggest there may be compelling reasons to initiate GAHT prior to the age of 16, although
there are limited studies on youth who have
initiated hormones prior to 14 years of age
(Hembree et al., 2017). A compelling reason for
earlier initiation of GAHT, for example, might
be to avoid prolonged pubertal suppression,
given potential bone health concerns and the
psychosocial implications of delaying puberty as
described in more detail in Chapter 12---
Hormone Therapy (Klink, Caris et al., 2015;
Schagen et al., 2020; Vlot et al., 2017; Zhu \&
Chan, 2017). Puberty is a time of significant
brain and cognitive development. The potential
neurodevelopmental impact of extended pubertal
suppression in gender diverse youth has been
specifically identified as an area in need of continued study (Chen et al., 2020). While GnRH
analogs have been shown to be safe when used
for the treatment of precocious puberty, there
are concerns delaying exposure to sex hormones
(endogenous or exogenous) at a time of peak
bone mineralization may lead to decreased bone
mineral density. The potential decrease in bone
mineral density as well as the clinical significance of any decrease requires continued study
(Klink, Caris et al., 2015; Lee, Finlayson et al.,
2020; Schagen et al., 2020). The potential negative psychosocial implications of not initiating
puberty with peers may place additional stress
on gender diverse youth, although this has not
been explicitly studied. When considering the
timing of initiation of gender-affirming hormones, providers should compare the potential
physical and psychological benefits and risks of
starting treatment with the potential risks and
benefits of delaying treatment. This process can
also help identify compelling factors that may
warrant an individualized approach.

Studies carried out with trans masculine youth
have demonstrated chest dysphoria is associated
with higher rates of anxiety, depression, and distress and can lead to functional limitations, such
as avoiding exercising or bathing (Mehringer
et al., 2021; Olson-Kennedy, Warus et al., 2018;
Sood et al., 2021). Testosterone unfortunately
does little to alleviate this distress, although chest
masculinization is an option for some individuals
to address this distress long-term. Studies with
youth who sought chest masculinization surgery
to alleviate chest dysphoria demonstrated good
surgical outcomes, satisfaction with results, and
minimal regret during the study monitoring
period (Marinkovic \& Newfield, 2017;
Olson-Kennedy, Warus et al., 2018). Chest masculinization surgery can be considered in minors
when clinically and developmentally appropriate
as determined by a multidisciplinary team experienced in adolescent and gender development
(see relevant statements in this chapter). The
duration or current use of testosterone therapy
should not preclude surgery if otherwise indicated. The needs of some TGD youth may be
met by chest masculinization surgery alone.
Breast augmentation may be needed by trans
feminine youth, although there is less data about
this procedure in youth, possibly due to fewer
individuals requesting this procedure (Boskey
et al., 2019; James, 2016). GAHT, specifically
estrogen, can help with development of breast
tissue, and it is recommended youth have a minimum of 12 months of hormone therapy, or longer as is surgically indicated, prior to breast
augmentation unless hormone therapy is not
clinically indicated or is medically
contraindicated.

Data are limited on the optimal timing for initiating other gender-affirming surgical treatments
in adolescents. This is partly due to the limited
access to these treatments, which varies in different geographical locations (Mahfouda et al., 2019).
Data indicate rates of gender-affirming surgeries
have increased since 2000, and there has been an
increase in the number of TGD youth seeking
vaginoplasty (Mahfouda et al., 2019; Milrod \&
Karasic, 2017). A 2017 study of 20 WPATH-affiliated
surgeons in the US reported slightly more than
half had performed vaginoplasty in minors (Milrod
\& Karasic, 2017). Limited data are available on
the outcomes for youth undergoing vaginoplasty.
Small studies have reported improved psychosocial
functioning and decreased gender dysphoria in
adolescents who have undergone vaginoplasty
(Becker et al., 2018; Cohen-Kettenis \& van Goozen,
1997; Smith et al.,2001). While the sample sizes
are small, these studies suggest there may be a
benefit for some adolescents to having these procedures performed before the age of 18. Factors
that may support pursuing these procedures for
youth under 18 years of age include the increased
availability of support from family members,
greater ease of managing postoperative care prior
to transitioning to tasks of early adulthood (e.g.,
entering university or the workforce), and safety
concerns in public spaces (i.e., to reduce transphobic violence) (Boskey et al., 2018; Boskey et al.,
2019; Mahfouda et al., 2019). Given the complexity
and irreversibility of these procedures, an assessment of the adolescent's ability to adhere to postsurgical care recommendations and to comprehend
the long-term impacts of these procedures on
reproductive and sexual function is crucial (Boskey
et al., 2019). Given the complexity of phalloplasty,
and current high rates of complications in comparison to other gender-affirming surgical treatments, it is not recommended this surgery be
considered in youth under 18 at this time (see
Chapter 13---Surgery and Postoperative Care).

Additional key factors that should be taken
into consideration when discussing the timing of
interventions with youth and families are
addressed in detail in statements 6.12a-f.~For a
summary of the criteria/recommendations for
medically necessary gender-affirming medical
treatment in adolescents, see Appendix D.

\hypertarget{children}{%
\chapter{Children}\label{children}}

These Standards of Care pertain to prepubescent
gender diverse children and are based on research,
ethical principles, and accumulated expert knowledge. The principles underlying these standards
include the following 1) childhood gender diversity
is an expected aspect of general human development (Endocrine Society and Pediatric Endocrine
Society, 2020; Telfer et al., 2018); 2) childhood
gender diversity is not a pathology or mental
health disorder (Endocrine Society and Pediatric
Endocrine Society, 2020; Oliphant et al., 2018;
Telfer et al., 2018); 3) diverse gender expressions
in children cannot always be assumed to reflect a
transgender identity or gender incongruence
(Ehrensaft, 2016; Ehrensaft, 2018; Rael et al.,
2019); 4) guidance from mental health professionals (MHPs) with expertise in gender care for children can be helpful in supporting positive
adaptation as well as discernment of gender-related
needs over time (APA, 2015; Ehrensaft, 2018;
Telfer et al., 2018); 5) conversion therapies for
gender diversity in children (i.e., any ``therapeutic''
attempts to compel a gender diverse child through
words, actions, or both to identify with, or behave
in accordance with, the gender associated with the
sex assigned at birth are harmful and we repudiate
their use (APA, 2021; Ashley, 2019b, Paré, 2020;
SAMHSA, 2015; Telfer et al., 2018; UN Human
Rights Council, 2020).

Throughout the text, the term ``health care
professional'' (HCP) is used broadly to refer to
professionals working with gender diverse children. Unlike pubescent youth and adults, prepubescent gender diverse children are not eligible
to access medical intervention (Pediatric
Endocrine Society, 2020); therefore, when professional input is sought, it is most likely to be from
an HCP specialized in psychosocial supports and
gender development. Thus, this chapter is
uniquely focused on developmentally appropriate
psychosocial practices, although other HCPs, such
as pediatricians and family practice HCPs may
also find these standards useful as they engage
in professional work with gender diverse children
and their families.

This chapter employs the term ``gender diverse''
given that gender trajectories in prepubescent
children cannot be predicted and may evolve over
time (Steensma, Kreukels et al., 2013). At the
same time, this chapter recognizes some children
will remain stable in a gender identity they articulate early in life that is discrepant from the sex
assigned at birth (Olson et al., 2022). The term,
``gender diverse'' includes transgender binary and
nonbinary children, as well as gender diverse
children who will ultimately not identify as transgender later in life. Terminology is inherently
culturally bound and evolves over time. Thus, it
is possible terms used here may become outdated
and we will find better descriptors.

This chapter describes aspects of medical necessary care intended to promote the well-being
and gender-related needs of children (see medically necessary statement in the Global Applicability
chapter, Statement 2.1). This chapter advocates
everyone employs these standards, to the extent
possible. There may be situations or locations in
which the recommended resources are not fully
available. HCPs/teams lacking resources need to
work toward meeting these standards. However, if
unavoidable limitations preclude components of
these recommendations, this should not hinder
providing the best services currently available. In
those locations where some but not all recommended services exist, choosing not to implement
potentially beneficial care services risks harm to
a child (Murchison et al., 2016; Telfer et al., 2018;
Riggs et al., 2020). Overall, it is imperative to
prioritize a child's best interests.

A vast empirical psychological literature indicates
early childhood experiences frequently set the stage
for lifelong patterns of risk and/or resilience and
contribute to a trajectory of development more or
less conducive to well-being and a positive quality
of life (Anda et al., 2010; Masten \& Cicchetti, 2010;
Shonkoff \& Garner, 2012). The available research
indicates, in general, gender diverse youth are at
greater risk for experiencing psychological difficulties (Ristori \& Steensma, 2016) than age- matched
cisgender peers as a result of encountering destructive experiences, including trauma and maltreatment
stemming from gender diversity-related rejection
and other harsh, non-accepting interactions (Barrow
\& Apostle, 2018; Giovanardi et al., 2018; Gower,
Rider, Brown et al., 2018; Grossman \& D'Augelli,
2006; Hendricks \& Testa, 2012; Reisner, Greytak
S68 E. COLEMAN ET AL.
et al., 2015; Roberts et al., 2014; Tishelman \&
Neumann-Mascis, 2018). Further, literature indicates
prepubescent children who are well accepted in
their gender diverse identities are generally
well-adjusted (Malpas et al., 2018; Olson et al.,
2016). Assessment and treatment of children typically emphasizes an ecological approach, recognizing
children need to be safe and nurtured in each setting they frequent (Belsky, 1993; Bronfenbrenner,
1979; Kaufman \& Tishelman, 2018; Lynch \&
Cicchetti, 1998; Tishelman et al., 2010; Zielinski \&
Bradshaw, 2006). Thus, the perspective of this chapter draws on basic psychological literature and
knowledge of the unique risks to gender diverse
children and emphasizes the integration of an ecological approach to understanding their needs and
to facilitating positive mental health in all gender
care. This perspective prioritizes fostering well-being
and quality of life for a child throughout their
development. Additionally, this chapter also
embraces the viewpoint, supported by the substantial psychological research cited above, that psychosocial gender-affirming care (Hidalgo et al., 2013)
for prepubescent children offers a window of opportunity to promote a trajectory of well-being that
will sustain them over time and during the transition to adolescence. This approach potentially can
mitigate some of the common mental health risks
faced by transgender and gender diverse (TGD)
teens, as frequently described in literature (Chen
et al., 2021; Edwards-Leeper et al., 2017; Haas et al.,
2011; Leibowitz \& de Vries, 2016; Reisner, Bradford
et al., 2015; Reisner, Greytak et al., 2015).

Developmental research has focused on understanding various aspects of gender development
in the earliest years of childhood based on a
general population of prepubescent children.
This research has typically relied on the assumption that child research participants are cisgender (Olezeski et al., 2020) and has reported
gender identity stability is established in the
preschool years for the general population of
children, most of whom are likely not gender
diverse (Kohlberg, 1966; Steensma, Kreukels
et al., 2013). Recently, developmental research
has demonstrated gender diversity can be
observed and identified in young prepubescent
children (Fast \& Olson, 2018; Olson \& Gülgöz,
2018; Robles et al., 2016). Nonetheless, empirical
study in this area is limited, and at this time
there are no psychometrically sound assessment
measures capable of reliably and/or fully ascertaining a prepubescent child's self-understanding
of their own gender and/or gender-related needs
and preferences (Bloom et al., 2021). Therefore,
this chapter emphasizes the importance of a
nuanced and individualized clinical approach to
gender assessment, consistent with the recommendations from various guidelines and literature (Berg \& Edwards-Leeper, 2018; de Vries \&
Cohen-Kettenis, 2012; Ehrensaft, 2018; Steensma
\& Wensing-Kruger, 2019). Research and clinical
experience have indicated gender diversity in
prepubescent children may, for some, be fluid;
there are no reliable means of predicting an
individual child's gender evolution
(Edwards-Leeper et al., 2016; Ehrensaft, 2018;
Steensma, Kreukels et al., 2013), and the
gender-related needs for a particular child may
vary over the course of their childhood.

It is important to understand the meaning of
the term ``assessment'' (sometimes used synonymously with the term ``evaluation''). There are
multiple contexts for assessment (Krishnamurthy
et al., 2004) including rapid assessments that
take place during an immediate crisis (e.g., safety
assessment when a child may be suicidal) and
focused assessments when a family may have a
circumscribed question, often in the context of
a relatively brief consultation (Berg \&
Edwards-Leeper, 2018). The term assessment is
also often used in reference to ``diagnostic assessment,'' which can also be called an ``intake'' and
is for the purpose of determining whether there
is an issue that is diagnosable and/or could benefit from a therapeutic process. This chapter
focus on comprehensive assessments, useful for
understanding a child and family's needs and
goals (APA, 2015; de Vries \& Cohen-Kettenis,
2012; Srinath et al., 2019; Steensma \&
Wensing-Kruger, 2019). This type of psychosocial
assessment is not necessary for all gender diverse
children, but may be requested for a number of
reasons. Assessments may present a useful
opportunity to start a process of support for a
gender diverse child and their family, with the
understanding that gender diverse children benefit when their family dynamics include
acceptance of their gender diversity and parenting guidance when requested. Comprehensive
assessments are appropriate when solicited by a
family requesting a full understanding of the
child's gender and mental health needs in the
context of gender diversity.

In these circumstances, family member mental
health issues, family dynamics, and social and cultural contexts, all of which impact a gender diverse
child, should be taken into consideration (Barrow
\& Apostle, 2018; Brown \& Mar, 2018; Cohen-Kettenis
et al., 2003; Hendricks \& Testa, 2012; Kaufman \&
Tishelman, 2018; Ristori \& Steensma, 2016;
Tishelman \& Neumann-Mascis, 2018). This is further elaborated upon in the text below.

It is important HCPs working with gender
diverse children strive to understand the child and
the family's various aspects of identity and experience: racial, ethnic, immigrant/refugee status,
religious, geographic, and socio-economic, for
example, and be respectful and sensitive to cultural
context in clinical interactions (Telfer et al., 2018).
Many factors may be relevant to culture and gender, including religious beliefs, gender-related
expectations, and the degree to which gender
diversity is accepted (Oliphant et al., 2018).
Intersections between gender diversity, sociocultural diversity, and minority statuses can be sources
of strength, social stress, or both (Brown \& Mar,
2018; Oliphant et al., 2018; Riggs \& Treharne, 2016).

Each child, family member, and family dynamic
is unique and potentially encompasses multiple
cultures and belief patterns. Thus, HCPs of all
disciplines should avoid stereotyping based on
preconceived ideas that may be incorrect or
biased (e.g., that a family who belongs to a religious organization that is opposed to appreciating
gender diversity will necessarily be unsupportive
of their child's gender diversity) (Brown \& Mar,
2018). Instead, it is essential to approach each
family openly and understand each family member and family pattern as distinct.

All the statements in this chapter have been
recommended based on a thorough review of
evidence, an assessment of the benefits and
harms, values and preferences of providers and
patients, and resource use and feasibility. In some
cases, we recognize evidence is limited and/or
services may not be accessible or desirable.

\hypertarget{statement-7.1-we-recommend-the-health-care-professionals-working-with-gender-diverse-children-receive-training-and-have-expertise-in-gender-development-and-gender-diversity-in-children-and-possess-general-knowledge-of-gender-diversity-across-the-life-span.}{%
\section*{Statement 7.1: We recommend the health care professionals working with gender diverse children receive training and have expertise in gender development and gender diversity in children and possess general knowledge of gender diversity across the life span.}\label{statement-7.1-we-recommend-the-health-care-professionals-working-with-gender-diverse-children-receive-training-and-have-expertise-in-gender-development-and-gender-diversity-in-children-and-possess-general-knowledge-of-gender-diversity-across-the-life-span.}}
\addcontentsline{toc}{section}{Statement 7.1: We recommend the health care professionals working with gender diverse children receive training and have expertise in gender development and gender diversity in children and possess general knowledge of gender diversity across the life span.}

HCPs working with gender diverse children
should acquire and maintain the necessary training and credentials relevant to the scope of their
role as professionals. This includes licensure, certification, or both by appropriate national and/
or regional accrediting bodies. We recognize the
specifics of credentialing and regulation of professionals vary globally. Importantly, basic licensure, certification, or both may be insufficient in
and of itself to ensure competency working with
gender diverse children, as HCPs specifically
require in-depth training and supervised experience in childhood gender development and gender diversity to provide appropriate care.

\hypertarget{statement-7.2-we-recommend-health-care-professionals-working-with-gender-diverse-children-receive-theoretical-and-evidenced-based-training-and-develop-expertise-in-general-child-and-family-mental-health-across-the-developmental-spectrum.}{%
\section*{Statement 7.2: We recommend health care professionals working with gender diverse children receive theoretical and evidenced-based training and develop expertise in general child and family mental health across the developmental spectrum.}\label{statement-7.2-we-recommend-health-care-professionals-working-with-gender-diverse-children-receive-theoretical-and-evidenced-based-training-and-develop-expertise-in-general-child-and-family-mental-health-across-the-developmental-spectrum.}}
\addcontentsline{toc}{section}{Statement 7.2: We recommend health care professionals working with gender diverse children receive theoretical and evidenced-based training and develop expertise in general child and family mental health across the developmental spectrum.}

HCPs should receive training and supervised
expertise in general child and family mental
health across the developmental spectrum from
toddlerhood through adolescence, including
evidence-based assessment and intervention
approaches. Gender diversity is not a mental
health disorder; however, as cited above, we know
mental health can be adversely impacted for gender diverse children (e.g., through gender minority
stress) (Hendricks \& Testa, 2012) that may benefit
from exploration and support; therefore, mental
health expertise is highly recommended. Working
with children is a complex endeavor, involving
an understanding of a child's developmental needs
at various ages, the ability to comprehend the
forces impacting a child's well-being both inside
and outside the family (Kaufman \& Tishelman,
2018), and an ability to fully assess when a child
is unhappy or experiencing significant mental
health difficulties, related or unrelated to gender.
Research has indicated high levels of adverse
experiences and trauma in the gender diverse
community of children, including susceptibility
to rejection or even maltreatment (APA, 2015;
Barrow \& Apostle, 2018; Giovanardi et al., 2018;
Reisner, Greytak et al., 2015; Roberts et al., 2012;
Tishelman \& Neumann-Mascis, 2018). HCPs need
to be cognizant of the potential for adverse experiences and be able to initiate effective interventions to prevent harm and promote positive
well-being.

\hypertarget{statement-7.3-we-recommend-health-care-professionals-working-with-gender-diverse-children-receive-training-and-develop-expertise-in-autism-spectrum-disorders-and-other-neurodiversity-or-collaborate-with-an-expert-with-relevant-expertise-when-working-with-autisticneurodivergent-gender-diverse-children.}{%
\section*{Statement 7.3 We recommend health care professionals working with gender diverse children receive training and develop expertise in autism spectrum disorders and other neurodiversity or collaborate with an expert with relevant expertise when working with autistic/neurodivergent, gender diverse children.}\label{statement-7.3-we-recommend-health-care-professionals-working-with-gender-diverse-children-receive-training-and-develop-expertise-in-autism-spectrum-disorders-and-other-neurodiversity-or-collaborate-with-an-expert-with-relevant-expertise-when-working-with-autisticneurodivergent-gender-diverse-children.}}
\addcontentsline{toc}{section}{Statement 7.3 We recommend health care professionals working with gender diverse children receive training and develop expertise in autism spectrum disorders and other neurodiversity or collaborate with an expert with relevant expertise when working with autistic/neurodivergent, gender diverse children.}

The experience of gender diversity in autistic
children as well as in children with other forms
of neurodivergence may present extra clinical
complexities (de Vries et al., 2010; Strang,
Meagher et al., 2018). For example, autistic children may find it difficult to self-advocate for
their gender-related needs and may communicate
in highly individualistic ways (Kuvalanka et al.,
2018; Strang, Powers et al., 2018). They may have
varied interpretations of gender-related experiences given common differences in communication and thinking style. Because of the unique
needs of gender diverse neurodivergent children,
they may be at high risk for being misunderstood
(i.e., for their communications to be misinterpreted). Therefore, professionals providing support to these children can best serve them by
receiving training and developing expertise in
autism and related neurodevelopmental presentations and/or collaborating with autism specialists (Strang, Meagher et al., 2018). Such training
is especially relevant as research has documented
higher rates of autism among gender diverse
youth than in the general population (de Vries
et al., 2010; Hisle-Gorman et al., 2019; Shumer
et al., 2015).

\hypertarget{statement-7.4-we-recommend-health-care-professionals-working-with-gender-diverse-children-engage-in-continuing-education-related-to-gender-diverse-children-and-families.}{%
\section*{Statement 7.4: We recommend health care professionals working with gender diverse children engage in continuing education related to gender diverse children and families.}\label{statement-7.4-we-recommend-health-care-professionals-working-with-gender-diverse-children-engage-in-continuing-education-related-to-gender-diverse-children-and-families.}}
\addcontentsline{toc}{section}{Statement 7.4: We recommend health care professionals working with gender diverse children engage in continuing education related to gender diverse children and families.}

Continuing professional development regarding
gender diverse children and families may be
acquired through various means, including
through readings (journal articles, books, websites
associated with gender knowledgeable organizations), attending on-line and in person trainings,
and joining peer supervision/consultation groups
(Bartholomaeus et al., 2021).

Continuing education includes 1) maintaining
up-to-date knowledge of available and relevant
research on gender development and gender
diversity in prepubescent children and gender
diversity across the life span; 2) maintaining current knowledge regarding best practices for
assessment, support, and treatment approaches
with gender diverse children and families. This
is a relatively new area of practice and health
care professionals need to adapt as new information emerges through research and other avenues
(Bartholomaeus et al., 2021).

\hypertarget{statement-7.5-we-recommend-health-care-professionals-conducting-an-assessment-with-gender-diverse-children-access-and-integrate-information-from-multiple-sources-as-part-of-the-assessment.}{%
\section*{Statement 7.5: We recommend health care professionals conducting an assessment with gender diverse children access and integrate information from multiple sources as part of the assessment.}\label{statement-7.5-we-recommend-health-care-professionals-conducting-an-assessment-with-gender-diverse-children-access-and-integrate-information-from-multiple-sources-as-part-of-the-assessment.}}
\addcontentsline{toc}{section}{Statement 7.5: We recommend health care professionals conducting an assessment with gender diverse children access and integrate information from multiple sources as part of the assessment.}

A comprehensive assessment, when requested
by a family and/or an HCP can be useful for
developing intervention recommendations, as
needed, to benefit the well-being of the child and
other family members. Such an assessment can be
beneficial in a variety of situations when a child
and/or their family/guardians, in coordination with
providers, feel some type of intervention would
be helpful. Neither assessments nor interventions
should ever be used as a means of covertly or
overtly discouraging a child's gender diverse
expressions or identity. Instead, with appropriately
trained providers, assessment can be an effective
means of better understanding how to support a
child and their family without privileging any particular gender identity or expression. An assessment can be especially important for some children
and their families by collaborating to promote a
child's gender health, well-being, and self-fulfillment.

A comprehensive assessment can facilitate the
formation of an individualized plan to assist a
gender diverse prepubescent children and family
members (de Vries \& Cohen-Kettenis, 2012;
Malpas et al., 2018; Steensma \& Wensing-Kruger,
2019; Telfer et al., 2018; Tishelman \& Kaufman,
2018). In such an assessment, integrating information from multiple sources is important to 1)
best understand the child's gender needs and
make recommendations; and 2) identify areas of
child, family/caregiver, and community strengths
and supports specific to the child's gender status
and development as well as risks and concerns
for the child, their family/caregivers and environment. Multiple informants for both evaluation
and support/intervention planning purposes may
include the child, parents/caregivers, extended
family members, siblings, school personnel, HCPs,
the community, broader cultural and legal contexts and other sources as indicated (Berg \&
Edwards-Leeper, 2018; Srinath, 2019).

An HCP conducting an assessment of gender
diverse children needs to explore gender-related
issues but must also take a broad view of the
child and the environment, consistent with the
ecological model described above
(Bronfenbrenner, 1979) to fully understand the
factors impacting a child's well-being and areas
of gender support and risk (Berg \&
Edwards-Leeper, 2018; Hendricks \& Testa, 2012;
Kaufman \& Tishelman, 2018; Tishelman \&
Neumann-Mascis, 2018). This includes understanding the strengths and challenges experienced by the child/family and that are present
in the environment. We advise HCPs conducting
an assessment with gender diverse children to
consider incorporating multiple assessment
domains, depending on the child and the family's needs and circumstances. Although some
of the latter listed domains below do not directly
address the child's gender (see items 7--12
below), they need to be accounted for in a gender assessment, as indicated by clinical judgment, to understand the complex web of factors
that may be affecting the child's well-being in
an integrated fashion, including gender health,
consistent with evaluation best practices a (APA,
2015; Berg \& Edwards-Leeper, 2018; Malpas
et al., 2018) and develop a multi-pronged intervention when needed.

Summarizing from relevant research and clinical
expertise, assessment domains often include 1) a
child's asserted gender identity and gender expression, currently and historically; 2) evidence of
dysphoria, gender incongruence, or both; 3)
strengths and challenges related to the child, family, peer and others' beliefs and attitudes about
gender diversity, acceptance and support for child;
4) child and family experiences of gender minority
stress and rejection, hostility, or both due to the
child's gender diversity; 5) level of support related
to gender diversity in social contexts (e.g., school,
faith community, extended family); 6) evaluation
of conflict regarding the child's gender and/or
parental/caregiver/sibling concerning behavior
related to the child's gender diversity; 7) child
mental health, communication and/or cognitive
strengths and challenges, neurodivergence, and/or
behavioral challenges causing significant functional
difficulty; 8) relevant medical and developmental
history; 9) areas that may pose risks (e.g., exposure
to domestic and/or community violence, any form
of child maltreatment; history of trauma; safety
and/or victimization with peers or in any other
setting; suicidality); 10) co-occurring significant
family stressors, such as chronic or terminal illness, homelessness or poverty; 11) parent/caregiver
and/or sibling mental health and/or behavioral
challenges causing significant functional difficulty;
and 12) child's and family's strengths and challenges.

A thorough assessment incorporating multiple
forms of information gathering is helpful for
understanding the needs, strengths, protective
factors, and risks for a specific child and family
across environments (e.g., home/school). Methods
of information gathering often include 1) interviews with the child, family members and others
(e.g., teachers), structured and unstructured; 2)
caregiver and child completed standardized measures related to gender; general child well-being;
child cognitive and communication skills and
developmental disorders/disabilities; support and
acceptance by parent/caregiver, sibling, extended
family and peers; parental stress; history of childhood adversities; and/or other issues as appropriate (APA, 2020; Berg \& Edwards-Leeper, 2018;
Kaufman \& Tishelman, 2018; Srinath, 2019).

Depending on the family characteristics, the
developmental profile of the child, or both, methods of information gathering also may also benefit
from including the following 1) child and/or family observation, structured and unstructured; and
2) structured and visually supported assessment
techniques (worksheets; self-portraits; family drawings, etc.) (Berg \& Edwards-Leeper, 2018).

\hypertarget{statement-7.6-we-recommend-that-health-care-professionals-conducting-an-assessment-with-gender-diverse-children-consider-relevant-developmental-factors-neurocognitive-functioning-and-language-skills.}{%
\section*{Statement 7.6: We recommend that health care professionals conducting an assessment with gender diverse children consider relevant developmental factors, neurocognitive functioning and language skills.}\label{statement-7.6-we-recommend-that-health-care-professionals-conducting-an-assessment-with-gender-diverse-children-consider-relevant-developmental-factors-neurocognitive-functioning-and-language-skills.}}
\addcontentsline{toc}{section}{Statement 7.6: We recommend that health care professionals conducting an assessment with gender diverse children consider relevant developmental factors, neurocognitive functioning and language skills.}

Given the complexities of assessing young
children who, unlike adults, are in the process
of development across a range of domains (cognitive, social, emotional, physiological), it is
important to consider the developmental status
of a child and gear assessment modalities and
interactions to the individualized abilities of the
child. This includes tailoring the assessment to
a child's developmental stage and abilities (preschoolers, school age, early puberty prior to
adolescence), including using language and
assessment approaches that prioritize a child's
comfort, language skills, and means of
self-expression (Berg \& Edwards-Leeper, 2018;
Srinath, 2019). For example, relevant developmental factors, such as neurocognitive differences (e.g., autism spectrum conditions), and
receptive and expressive language skills should
be considered in conducting the assessment.
Health care professionals may need to consult
with specialists for guidance in cases in which
they do not possess the specialized skills themselves (Strang et al., 2021).

\hypertarget{statement-7.7-we-recommend-health-care-professionals-conducting-an-assessment-with-gender-diverse-children-consider-factors-that-may-constrain-accurate-reporting-of-gender-identitygender-expression-by-the-child-andor-familycaregivers.}{%
\section*{Statement 7.7: We recommend health care professionals conducting an assessment with gender diverse children consider factors that may constrain accurate reporting of gender identity/gender expression by the child and/or family/caregiver(s).}\label{statement-7.7-we-recommend-health-care-professionals-conducting-an-assessment-with-gender-diverse-children-consider-factors-that-may-constrain-accurate-reporting-of-gender-identitygender-expression-by-the-child-andor-familycaregivers.}}
\addcontentsline{toc}{section}{Statement 7.7: We recommend health care professionals conducting an assessment with gender diverse children consider factors that may constrain accurate reporting of gender identity/gender expression by the child and/or family/caregiver(s).}

HCPs conducting an assessment with gender
diverse children and families need to account for
developmental, emotional, and environmental factors that may constrain a child's, caregiver's, sibling or other's report or influence their belief
systems related to gender (Riggs \& Bartholomaeus,
2018). As with all child psychological assessments, environmental and family/caregiver reactions (e.g., punishment), and/or cognitive and
social factors may influence a child's comfort
and/or ability to directly discuss certain factors,
including gender identity and related issues
(Srinath, 2019). Similarly, family members may
feel constrained in freely expressing their concerns and ideas depending on family conflicts or
dynamics and/or other influences (e.g., cultural/
religious; extended family pressure) (Riggs \&
Bartholomaeus, 2018).

\hypertarget{statement-7.8-we-recommend-health-care-professionals-consider-consultation-psychotherapy-or-both-for-a-gender-diverse-child-and-familycaregivers-when-families-and-health-care-professionals-believe-this-would-benefit-the-well-being-and-development-of-a-child-andor-family.}{%
\section*{Statement 7.8: We recommend health care professionals consider consultation, psychotherapy, or both for a gender diverse child and family/caregivers when families and health care professionals believe this would benefit the well-being and development of a child and/or family.}\label{statement-7.8-we-recommend-health-care-professionals-consider-consultation-psychotherapy-or-both-for-a-gender-diverse-child-and-familycaregivers-when-families-and-health-care-professionals-believe-this-would-benefit-the-well-being-and-development-of-a-child-andor-family.}}
\addcontentsline{toc}{section}{Statement 7.8: We recommend health care professionals consider consultation, psychotherapy, or both for a gender diverse child and family/caregivers when families and health care professionals believe this would benefit the well-being and development of a child and/or family.}

The goal of psychotherapy should never be aimed
at modifying a child's gender identity (APA, 2021;
Ashley, 2019b; Paré, 2020; SAMHSA, 2015; UN
Human Rights Council, 2020), either covertly or
overtly. Not all gender diverse children or their
families need input from MHPs as gender diversity
is not a mental health disorder (Pediatric Endocrine
Society, 2020; Telfer et al., 2018). Nevertheless, it is
often appropriate and helpful to seek psychotherapy
when there is distress or concerns are expressed by
parents to improve psychosocial health and prevent
further distress (APA, 2015). Some of the common
reasons for considering psychotherapy for a gender
diverse child and family include the following 1)
A child is demonstrating significant conflicts, confusion, stress or distress about their gender identity
or needs a protected space to explore their gender
(Ehrensaft, 2018; Spivey and Edwards-Leeper, 2019);
2) A child is experiencing external pressure to
express their gender in a way that conflicts with
their self-knowledge, desires, and beliefs (APA,
2015); 3) A child is struggling with mental health
concerns, related to or independent of their gender
(Barrow \& Apostle, 2018); 4) A child would benefit
from strengthening their resilience in the face of
negative environmental responses to their gender
identity or presentation (Craig \& Auston, 2018;
Malpas et al., 2018); 5) A child may be experiencing
mental health and/or environmental concerns,
including family system problems that can be misinterpreted as gender congruence or incongruence
(Berg \& Edwards-Leeper, 2018); and 6) A child
expresses a desire to meet with an MHP to get
gender-related support. In these situations, the psychotherapy will focus on supporting the child with
the understanding that the child's parent(s)/caregiver(s) and potentially other family members will
be included as necessary (APA, 2015; Ehrensaft,
2018; McLaughlin \& Sharp, 2018). Unless contraindicated, it is extremely helpful for parents/guardians to participate in some capacity in the
psychotherapy process involving prepubescent children as family factors are often central to a child's
well-being. Although relatively unexplored in
research involving gender diverse children, it may
be important to attend to the relationship between
siblings and the gender diverse child (Pariseau
et al., 2019; Parker \& Davis-McCabe, 2021).

HCPs should employ interventions tailor-made
to the individual needs of the child that are
designed to 1) foster protective social and emotional coping skills to promote resilience in the
face of potential negative reactions to the child's
gender identity, expressions, or both (Craig \&
Austin, 2016; Malpas et al., 2018; Spencer, Berg
et al., 2021); 2) collaboratively problem-solve
social challenges to reduce gender minority stress
(Barrow \& Apostle, 2018; Tishelman \&
Neumann-Mascis, 2018); 3) strengthen environmental supports for the child and/or members of
the immediate and extended family (Kaufman \&
Tishelman, 2018); and 4) provide the child an
opportunity to further understand their internal
gender experiences (APA, 2015; Barrow\& Apostle,
2018; Ehrensaft, 2018; Malpas et al., 2018;
McLaughlin \& Sharp, 2018). It is helpful for HCPs
to develop a relationship with a gender diverse
child and family that can endure over time as
needed. This enables the child/family to establish
a long-term trusting relationship throughout
childhood whereby the HCP can offer support
and guidance as a child matures and as potentially

different challenges or needs emerge for the child/
family (Spencer, Berg et al., 2021; Murchison
et al., 2016). In addition to the above and within
the limits of available resources, when a child is
neurodivergent, an HCP who has the skill set to
address both neurodevelopmental differences and
gender is most appropriate (Strang et al., 2021).

As outlined in the literature, there are numerous reasons parents/caregivers, siblings, and
extended family members of a prepubescent child
may find it useful to seek psychotherapy for
themselves (Ehrensaft, 2018; Malpas et al., 2018;
McLaughlin \& Sharp, 2018). As summarized
below, some of these common catalysts for seeking such treatment occur when one or more family members 1) desire education around gender
development (Spivey \& Edwards-Leeper, 2019);
2) are experiencing significant confusion or stress
about the child's gender identity, expression, or
both (Ashley, 2019c; Ehrensaft, 2018); 3) need
guidance related to emotional and behavioral
concerns regarding the gender diverse child
(Barrow \& Apostle, 2018; 4) need support to
promote affirming environments outside of the
home (e.g., school, sports, camps) (Kaufman \&
Tishelman, 2018); 5) are seeking assistance to
make informed decisions about social transition,
including how to do so in a way that is optimal
for a child's gender development and health (Lev
\& Wolf-Gould, 2018); 6) are seeking guidance
for dealing with condemnation from others,
including political entities and accompanying legislation, regarding their support for their gender
diverse child (negative reactions directed toward
parents/caregivers can sometimes include rejection and/or harassment/abuse from the social
environment arising from affirming decisions
(Hidalgo \& Chen, 2019); 7) are seeking to process
their own emotional reactions and needs about
their child's gender identity, including grief about
their child's gender diversity and/or potential
fears or anxieties for their child's current and
future well-being (Pullen Sansfaçon et al., 2019);
and 8) are emotionally distressed and/or in conflict with other family members regarding the
child's gender diversity (as needed, HCPs can
provide separate sessions for parents/caregivers,
siblings and extended family members for support, guidance, and/or psychoeducation)
(McLaughlin \& Sharp, 2018; Pullen Sansfaçon
et al., 2019; Spivey \& Edwards-Leeper, 2019).

\hypertarget{statement-7.9-we-recommend-health-care-professionals-offering-consultation-psychotherapy-or-both-to-gender-diverse-children-and-familiescaregivers-work-with-other-settings-and-individuals-important-to-the-child-to-promote-the-childs-resilience-and-emotional-well-being.}{%
\section*{Statement 7.9: We recommend health care professionals offering consultation, psychotherapy, or both to gender diverse children and families/caregivers work with other settings and individuals important to the child to promote the child's resilience and emotional well-being.}\label{statement-7.9-we-recommend-health-care-professionals-offering-consultation-psychotherapy-or-both-to-gender-diverse-children-and-familiescaregivers-work-with-other-settings-and-individuals-important-to-the-child-to-promote-the-childs-resilience-and-emotional-well-being.}}
\addcontentsline{toc}{section}{Statement 7.9: We recommend health care professionals offering consultation, psychotherapy, or both to gender diverse children and families/caregivers work with other settings and individuals important to the child to promote the child's resilience and emotional well-being.}

Consistent with the ecological model described
above and, as appropriate, based on individual/
family circumstances, it can be extremely helpful
for HCPs to prioritize coordination with important others (e.g., teachers, coaches, religious leaders) in a child's life to promote emotional and
physical safety across settings (e.g., school settings, sports and other recreational activities,
faith-based involvement) (Kaufman \& Tishelman,
2018). Therapeutic and/or support groups are
often recommended as a valuable resource for
families/caregivers and/or gender diverse children
themselves (Coolhart, 2018; Horton et al., 2021;
Malpas et al., 2018; Murchison et al., 2016).

\hypertarget{statement-7.10-we-recommend-hcps-offering-consultation-psychotherapy-or-both-to-gender-diverse-children-and-familiescaregivers-provide-both-parties-with-age-appropriate-psycho-education-about-gender-development.}{%
\section*{Statement 7.10: We recommend HCPs offering consultation, psychotherapy, or both to gender diverse children and families/caregivers provide both parties with age appropriate psycho-education about gender development.}\label{statement-7.10-we-recommend-hcps-offering-consultation-psychotherapy-or-both-to-gender-diverse-children-and-familiescaregivers-provide-both-parties-with-age-appropriate-psycho-education-about-gender-development.}}
\addcontentsline{toc}{section}{Statement 7.10: We recommend HCPs offering consultation, psychotherapy, or both to gender diverse children and families/caregivers provide both parties with age appropriate psycho-education about gender development.}

Parents/caregivers and their gender diverse child
should have the opportunity to develop knowledge
regarding ways in which families/caregivers can
best support their child to maximize resilience,
self-awareness, and functioning (APA, 2015;
Ehrensaft, 2018; Malpas, 2018; Spivey \&
Edwards-Leeper, 2019). It is neither possible nor
is it the role of the HCP to predict with certainty
the child's ultimate gender identity; instead, the
HCP's task is to provide a safe space for the child's
identity to develop and evolve over time without
attempts to prioritize any particular developmental
trajectory with regard to gender (APA, 2015;
Spivey \& Edwards-Leeper, 2019). Gender diverse
children and early adolescents have different needs
and experiences than older adolescents, socially
and physiologically, and those differences should
be reflected in the individualized approach HCPs
provide to each child/family (Keo-Meir \&
Ehrensaft, 2018; Spencer, Berg et al., 2021).

Parents/caregivers and their children should
also have the opportunity to develop knowledge
about gender development and gender literacy
through age-appropriate psychoeducation (Berg
\& Edwards-Leeper, 2018; Rider, Vencill et al.,
2019; Spencer, Berg et al., 2021). Gender literacy
involves understanding the distinctions between
sex designated at birth, gender identity, and gender expression, including the ways in which these
three factors uniquely come together for a child
(Berg \& Edwards-Leeper, 2018; Rider, Vencill
et al., 2019; Spencer, Berg et al., 2021). As a child
gains gender literacy, they begin to understand
their body parts do not necessarily define their
gender identity and/or their gender expression
(Berg \& Edwards-Leeper, 2018; Rider, Vencill
et al., 2019; Spencer, Berg et al., 2021). Gender
literacy also involves learning to identify messages
and experiences related to gender within society.
As a child gains gender literacy, they may view
their developing gender identity and gender
expression more positively, promoting resilience
and self-esteem, and diminishing risk of shame
in the face of negative messages from the environment. Gaining gender literacy through psychoeducation may also be important for siblings
and/or extended family members who are important to the child (Rider, Vencill et al., 2019;
Spencer, Berg et al., 2021).

\hypertarget{statement-7.11-we-recommend-health-care-professionals-provide-information-to-gender-diverse-children-and-their-familiescaregivers-as-the-child-approaches-puberty-about-potential-gender-affirming-medical-interventions-the-effects-of-these-treatments-on-future-fertility-and-options-for-fertility-preservation.}{%
\section*{Statement 7.11: We recommend health care professionals provide information to gender diverse children and their families/caregivers as the child approaches puberty about potential gender-affirming medical interventions, the effects of these treatments on future fertility, and options for fertility preservation.}\label{statement-7.11-we-recommend-health-care-professionals-provide-information-to-gender-diverse-children-and-their-familiescaregivers-as-the-child-approaches-puberty-about-potential-gender-affirming-medical-interventions-the-effects-of-these-treatments-on-future-fertility-and-options-for-fertility-preservation.}}
\addcontentsline{toc}{section}{Statement 7.11: We recommend health care professionals provide information to gender diverse children and their families/caregivers as the child approaches puberty about potential gender-affirming medical interventions, the effects of these treatments on future fertility, and options for fertility preservation.}

As a child matures and approaches puberty,
HCPs should prioritize working with children and
their parents/caregivers to integrate psychoeducation about puberty, engage in shared
decision-making about potential gender-affirming
medical interventions, and discuss fertility-related
and other reproductive health implications of
medical treatments (Nahata, Quinn et al., 2018;
Spencer, Berg et al., 2021). Although only limited
empirical research exists to evaluate such interventions, expert consensus and developmental
psychological literature generally support the
notion that open communication with children
about their bodies and preparation for physiological changes of puberty, combined with
gender-affirming acceptance, will promote resilience and help to foster positive sexuality as a
child matures into adolescence (Spencer, Berg
et al., 2019). All these discussions may be extended
(e.g., starting earlier) to include neurodivergent
children, to ensure there is enough time for
reflection and understanding, especially as choices
regarding future gender- affirming medical care
potentially arise (Strang, Jarin et al., 2018). These
discussions could include the following topics:

\begin{itemize}
\tightlist
\item
  Review of body parts and their different functions;
\item
  The ways in which a child's body may change over time with and without medical intervention;
\item
  The impact of medical interventions on later sexual functioning and fertility;
\item
  The impact of puberty suppression on potential later medical interventions;
\item
  Acknowledgment of the current lack ofclinical data in certain areas related to the impacts of puberty suppression;
\item
  The importance of appropriate sex education prior to puberty.
\end{itemize}

These discussions should employ developmentally appropriate language and teaching styles,
and be geared to the specific needs of each individual child (Spencer, Berg et al., 2021).

\hypertarget{statement-7.12-we-recommend-parentscaregivers-and-health-care-professionals-respond-supportively-to-children-who-desire-to-be-acknowledged-as-the-gender-that-matches-their-internal-sense-of-gender-identity.}{%
\section*{Statement 7.12: We recommend parents/caregivers and health care professionals respond supportively to children who desire to be acknowledged as the gender that matches their internal sense of gender identity.}\label{statement-7.12-we-recommend-parentscaregivers-and-health-care-professionals-respond-supportively-to-children-who-desire-to-be-acknowledged-as-the-gender-that-matches-their-internal-sense-of-gender-identity.}}
\addcontentsline{toc}{section}{Statement 7.12: We recommend parents/caregivers and health care professionals respond supportively to children who desire to be acknowledged as the gender that matches their internal sense of gender identity.}

Gender social transition refers to a process by
which a child is acknowledged by others and has
the opportunity to live publicly, either in all situations or in certain situations, in the gender
identity they affirm and has no singular set of
parameters or actions (Ehrensaft et al., 2018).

Gender social transition has often been conceived in the past as binary---a girl transitions to
a boy, a boy to a girl. The concept has expanded
to include children who shift to a nonbinary or
individually shaped iteration of gender identity
(Chew et al., 2020; Clark et al., 2018). Newer
research indicates the social transition process may
serve a protective function for some prepubescent
children and serve to foster positive mental health
and well-being (Durwood et al., 2017; Gibson
et al., 2021; Olson et al., 2016). Thus, recognition
that a child's gender may be fluid and develop
over time (Edwards-Leeper et al., 2016; Ehrensaft,
2018; Steensma, Kreukels et al., 2013) is not sufficient justification to negate or deter social transition for a prepubescent child when it would be
beneficial. Gender identity evolution may continue
even after a partial or complete social transition
process has taken place (Ashley, 2019e;
Edwards-Leeper et al., 2018; Ehrensaft, 2020;
Ehrensaft et al., 2018; Spivey \& Edwards-Leeper,
2019). Although empirical data remains limited,
existing research has indicated children who are
most assertive about their gender diversity are
most likely to persist in a diverse gender identity
across time, including children who socially transition prior to puberty (Olson et al., 2022; Rae
et al., 2019; Steensma, McGuire et al., 2013). Thus,
when considering a social transition, we suggest
parents/caregivers and HCPs pay particular attention to children who consistently and often persistently articulate a gender identity that does not
match the sex designated at birth. This includes
those children who may explicitly request or desire
a social acknowledgement of the gender that better
matches the child's articulated gender identity and/
or children who exhibit distress when their gender
as they know it is experienced as incongruent with
the sex designated at birth (Rae et al., 2019;
Steensma, Kreukels et al., 2013).

Although there is a dearth of empirical literature regarding best practices related to the social
transition process, clinical literature and expertise
provides the following guidance that prioritizes
a child's best interests (Ashley, 2019e; Ehrensaft,
2018; Ehrensaft et al, 2018; Murchison et al.,
2016; Telfer et al., 2018): 1) social transition
should originate from the child and reflect the
child's wishes in the process of making the
decision to initiate a social transition process; 2)
an HCP may assist exploring the advantages/benefits, plus potential challenges of social transition;
3) social transition may best occur in all or in
specific contexts/settings only (e.g., school, home);
and 4) a child may or may not choose to disclose
to others that they have socially transitioned, or
may designate, typically with the help of their
parents/caregivers, a select group of people with
whom they share the information.

In summary, social transition, when it takes
place, is likely to best serve a child's well-being
when it takes place thoughtfully and individually
for each child. A child's social transition (and
gender as well) may evolve over time and is not
necessarily static, but best reflects the cross-section
of the child's established self-knowledge of their
present gender identity and desired actions to
express that identity (Ehrensaft et al., 2018).

A social transition process can include one or
more of a number of different actions consistent
with a child's affirmed gender (Ehrensaft et al.,
2018), including:

\begin{itemize}
\tightlist
\item
  Name change;
\item
  Pronoun change;
\item
  Change in sex/gender markers (e.g., birth certificate; identification cards; passport; school and medical documentation; etc.);
\item
  Participation in gender-segregated programs (e.g., sports teams; recreational clubs and camps; schools; etc.);
\item
  Bathroom and locker room use;
\item
  Personal expression (e.g., hair style; clothing choice; etc.);
\item
  Communication of affirmed gender to others (e.g., social media; classroom or school announcements; letters to extended families or social contacts; etc.).
\end{itemize}

\hypertarget{statement-7.13-we-recommend-health-care-professionals-and-parentscaregivers-support-children-to-continue-to-explore-their-gender-throughout-the-pre-pubescent-years-regardless-of-social-transition.}{%
\section*{Statement 7.13: We recommend health care professionals and parents/caregivers support children to continue to explore their gender throughout the pre-pubescent years, regardless of social transition.}\label{statement-7.13-we-recommend-health-care-professionals-and-parentscaregivers-support-children-to-continue-to-explore-their-gender-throughout-the-pre-pubescent-years-regardless-of-social-transition.}}
\addcontentsline{toc}{section}{Statement 7.13: We recommend health care professionals and parents/caregivers support children to continue to explore their gender throughout the pre-pubescent years, regardless of social transition.}

It is important children who have engaged in
social transition be afforded the same opportunities as other children to continue considering
meanings and expressions of gender throughout
their childhood years (Ashley 2019e; Spencer,
Berg et al., 2021). Some research has found children may experience gender fluidity or even
detransition after an initial social transition.
Research has not been conclusive about when in
the life span such detransition is most likely to
occur, or what percentage of youth will eventually
experience gender fluidity and/or a desire to
detransition---due to gender evolution, or potentially other reasons (e.g., safety concerns; gender
minority stress) (Olson et al., 2022; Steensma,
Kreukels et al., 2013). A recent research report
indicates in the US, detransition occurs with only
a small percentage of youth five years after a
binary social transition (Olson et al., 2022); further follow-up of these young people would be
helpful. Replication of these findings is important
as well since this study was conducted with a
limited and self-selected participant pool in the
US and thus may not be applicable to all gender
diverse children. In summary, we have limited
ability to know in advance the ways in which a
child's gender identity and expressions may evolve
over time and whether or why detransition may
take place for some. In addition, not all gender
diverse children wish to explore their gender
(Telfer et al., 2018). Cisgender children are not
expected to undertake this exploration, and therefore attempts to force this with a gender diverse
child, if not indicated or welcomed, can be experienced as pathologizing, intrusive and/or cisnormative (Ansara \& Hegarty, 2012; Bartholomaeus
et al., 2021; Oliphant et al., 2018).

\hypertarget{statement-7.14-we-recommend-health-care-professionals-discuss-the-potential-benefits-and-risks-of-a-social-transition-with-families-who-are-considering-it.}{%
\section*{Statement 7.14: We recommend health care professionals discuss the potential benefits and risks of a social transition with families who are considering it.}\label{statement-7.14-we-recommend-health-care-professionals-discuss-the-potential-benefits-and-risks-of-a-social-transition-with-families-who-are-considering-it.}}
\addcontentsline{toc}{section}{Statement 7.14: We recommend health care professionals discuss the potential benefits and risks of a social transition with families who are considering it.}

Social transition in prepubescent children consists of a variety of choices, can occur as a process over time, is individualized based on both
a child's wishes and other psychosocial considerations (Ehrensaft, 2018), and is a decision for
which possible benefits and challenges should be
weighted and discussed.

A social transition may have potential benefits
as outlined in clinical literature (e.g., Ehrensaft
et al., 2018) and supported by research (Fast \&
Olson, 2018; Rae et al., 2019). These include
facilitating gender congruence while reducing
gender dysphoria and enhancing psychosocial
adjustment and well-being (Ehrensaft et al., 2018).
Studies have indicated socially transitioned gender diverse children largely mirror the mental
health characteristics of age matched cisgender
siblings and peers (Durwood et al., 2017). These
findings differ markedly from the mental health
challenges consistently noted in prior research
with gender diverse children and adolescents
(Barrow \& Apostle, 2018) and suggest the impact
of social transition may be positive. Additionally,
social transition for children typically can only
take place with the support and acceptance of
parents/caregivers, which has also been demonstrated to facilitate well-being in gender diverse
children (Durwood et al., 2021; Malpas et al.,
2018; Pariseau et al., 2019), although other forms
of support, such as school-based support, have
also been identified as important (Durwood
et al., 2021; Turban, King et al., 2021). HCPs
should discuss the potential benefits of a social
transition with children and families in situations
in which 1) there is a consistent, stable articulation of a gender identity that is incongruent with
the sex assigned at birth (Fast \& Olson, 2018).
This should be differentiated from gender diverse
expressions/behaviors/interests (e.g., playing with
toys, expressing oneself through clothing or
appearance choices, and/or engaging in activities
socially defined and typically associated with the
other gender in a binary model of gender)
(Ehrensaft, 2018; Ehrensaft et al., 2018); 2) the
child is expressing a strong desire or need to
transition to the gender they have articulated as
being their authentic gender (Ehrensaft et al.,
2018; Fast \& Olson, 2018; Rae et al., 2019); and
3) the child will be emotionally and physically
safe during and following transition (Brown \&
Mar, 2018). Prejudice and discrimination should
be considerations, especially in localities where
acceptance of gender diversity is limited or prohibited (Brown \& Mar, 2018; Hendricks \& Testa,
2012; Turban, King et al., 2021). Of note, there
can also be possible risks to a gender diverse
child who does not socially transition, including
1) being ostracized or bullied for being perceived
as not conforming to prescribed community
gender roles and/or socially expected patterns of
behavior; and 2) living with the internal stress
or distress that the gender they know themselves
to be is incongruent with the gender they are
being asked to present to the world.

To promote gender health, the HCP should discuss the potential challenges of a social transition.
One concern often expressed relates to fear that a
child will preclude considering the possible evolution
of their gender identity as they mature or be reluctant to initiate another gender transition even if they
no longer feel their social transition matches their
current gender identity (Edwards-Leeper et al., 2016;
Ristori \& Steensma, 2016). Although limited, recent
research has found some parents/caregivers of children who have socially transitioned may discuss
with their children the option of new gender iterations (for example, reverting to an earlier expression of gender) and are comfortable about this
possibility (Olson et al., 2019). Another often identified social transition concern is that a child may
suffer negative sequelae if they revert to the former
gender identity that matches their sex designated at
birth (Chen et al., 2018; Edwards-Leeper et al., 2019;
Steensma \& Cohen-Kettenis, 2011). From this point
of view, parents/caregivers should be aware of the
potential developmental effect of a social transition
on a child.

HCPs should provide guidance to parents/caregivers and supports to a child when a social gender transition is being considered or taking place
by 1) providing consultation, assessment, and gender supports when needed and sought by the parents/caregivers; 2) aiding family members, as
needed, to understand the child's desires for a
social transition and the family members' own
feelings about the child's expressed desires; 3)
exploring with, and learning from, the parents/
caregivers whether and how they believe a social
transition would benefit their child both now and
in their ongoing development; 4) providing guidance when parents/caregivers are not in agreement
about a social transition and offering the opportunity to work together toward a consistent understanding of their child's gender status and needs;
5) providing guidance about safe and supportive
ways to disclose their child's social transition to
others and to facilitate their child transitioning in
their various social environments (e.g., schools,
extended family); 6) facilitating communication,
when desired by the child, with peers about gender
and social transition as well as fortifying positive
peer relationships; 7) providing guidance when
social transition may not be socially accepted or
safe, either everywhere or in specific situations, or
when a child has reservations about initiating a
transition despite their wish to do so; there may
be multiple reasons for reservations, including
fears and anxieties; 8) working collaboratively with
family members and MHPs to facilitate a social
transition in a way that is optimal for the child's
unfolding gender development, overall well-being,
and physical and emotional safety; and 9) providing psychoeducation about the many different trajectories the child's gender may take over time,
leaving pathways open to future iterations of gender for the child, and emphasizing there is no
need to predict an individual child's gender identity in the future (Malpas et al., 2018).

All of these tasks incorporate enhancing the
quality of communication between the child and
family members and providing an opportunity
for the child to be heard and listened to by all
family members involved. These relational processes in turn facilitate the parents/caregivers'
success in making informed decisions about the
advisability and/or parameters of a social transition for their child (Malpas et al., 2018).

One role of HCPs is to provide guidance and
support in situations in which children and parents/caregivers wish to proceed with a social transition but conclude that the social environment
would not be accepting of those choices, by 1)
helping parents/caregivers define and extend safe
spaces in which the child can express their authentic gender freely; 2) discussing with parents/caregivers ways to advocate that increase the likelihood
of the social environment being supportive in the
future, if this is a realistic goal; 3) intervening as
needed to help the child/family with any associated
distress and/or shame brought about by the continued suppression of authentic gender identity
and the need for secrecy; and 4) building both
the child's and the family's resilience, instilling the
understanding that if the social environment is
having difficulty accepting a child's social transition and affirmed gender identity, it is not because
of some shortcoming in the child but because of
insufficient gender literacy in the social environment (Ehrensaft et al., 2018).

\hypertarget{statement-7.15-we-suggest-health-care-professionals-consider-working-collaboratively-with-other-professionals-and-organizations-to-promote-the-well-being-of-gender-diverse-children-and-minimize-the-adversities-they-may-face.}{%
\section*{Statement 7.15: We suggest health care professionals consider working collaboratively with other professionals and organizations to promote the well-being of gender diverse children and minimize the adversities they may face.}\label{statement-7.15-we-suggest-health-care-professionals-consider-working-collaboratively-with-other-professionals-and-organizations-to-promote-the-well-being-of-gender-diverse-children-and-minimize-the-adversities-they-may-face.}}
\addcontentsline{toc}{section}{Statement 7.15: We suggest health care professionals consider working collaboratively with other professionals and organizations to promote the well-being of gender diverse children and minimize the adversities they may face.}

All children have the right to be supported
and respected in their gender identities (Human
Rights Campaign, 2018; Paré, 2020; SAMHSA,
2015). As noted above, gender diverse children
are a particularly vulnerable group (Barrow \&
Apostle, 2018; Cohen-Kettenis et al., 2003;
Giovanardi et al., 2018; Gower, Rider, Coleman
et al., 2018; Grossman \& D'Augelli, 2007;
Hendricks \& Testa, 2012; Reisner, Greytak et al.,
2015; Ristori \& Steensma, 2016; Roberts et al.,
2012; Tishelman \& Neumann-Mascis, 2018). The
responsibilities of HCPs as advocates encompass
acknowledging social determinants of health are
critical for marginalized minorities (Barrow \&
Mar, 2018; Hendricks \& Testa, 2012). Advocacy
is taken up by all HCPs in the form of child and
family support (APA, 2015; Malpas et al., 2018).
Some HCPs may be called on to move beyond
their individual offices or programs to advocate
for gender diverse children in the larger community, often in partnership with stakeholders,
including parents/caregivers, allies, and youth
(Kaufman \& Tishelman, 2018; Lopez et al., 2017;
Vanderburgh, 2009). These efforts may be instrumental in enhancing children's gender health and
promoting their civil rights (Lopez et al., 2017).

HCP's voices may be essential in schools, in
parliamentary bodies, in courts of law, and in
the media (Kuvalanka et al., 2019; Lopez et al.,
2017; Whyatt-Sames, 2017; Vanderburgh, 2009).
In addition, HCPs may have a more generalized
advocacy role in acknowledging and addressing
the frequent intentional or unintentional negating of the experience of gender diverse children
that may be transmitted or communicated by
adults, peers, and in media (Rafferty et al.,
2018). Professionals who possess the skill sets
and find themselves in appropriate situations
can provide clear de-pathologizing statements
on the needs and rights of gender diverse children and on the damage caused by discriminatory and transphobic rules, laws, and norms
(Rafferty et al., 2018).

\hypertarget{nonbinary}{%
\chapter{Nonbinary}\label{nonbinary}}

Nonbinary is used as an umbrella term referring
to individuals who experience their gender as
outside of the gender binary. The term nonbinary
is predominantly but not exclusively associated
with global north contexts and may sometimes
be used to describe indigenous and non-Western
genders. The term nonbinary includes people
whose genders are comprised of more than one
gender identity simultaneously or at different
times (e.g., bigender), who do not have a gender
identity or have a neutral gender identity (e.g.,
agender or neutrois), have gender identities that
encompass or blend elements of other genders
(e.g., polygender, demiboy, demigirl), and/or who
have a gender that changes over time (e.g., genderfluid) (Kuper et al., 2014; Richards et al., 2016;
Richards et al., 2017; Vincent, 2019). Nonbinary
people may identify to varying degrees with
binary-associated genders, e.g., nonbinary man/
woman, or with multiple gender terms, e.g., nonbinary and genderfluid (James et al., 2016; Kuper
et al., 2012). Nonbinary also functions as a gender identity in its own right (Vincent, 2020). It
is important to acknowledge this is not an
exhaustive list, the same identities can have different meanings for different people, and the use
of terms can vary over time and by location.

Genderqueer, first used in the 1990s, is an identity category somewhat older than nonbinary---
which first emerged in approximately the late 2000s
(Nestle et al., 2002; Wilchins, 1995). Genderqueer
may sometimes be used synonymously with nonbinary or may communicate a specific consciously
politicized dimension to a person's gender. While
transgender is used in many cultural contexts as
an umbrella term inclusive of nonbinary people,
not all nonbinary people consider themselves to
be transgender for a range of reasons, including
because they consider being transgender to be
exclusively within the gender binary or because
they do not feel ``trans enough'' to describe themselves as transgender (Garrison, 2018). Some nonbinary people are unsure or ambivalent about
whether they would describe themselves as transgender (Darwin, 2020; Vincent, 2019).

In the context of the English language, nonbinary people may use the pronouns they/them/
theirs, or neopronouns which include e/em/eir,
ze/zir/hir, er/ers/erself among others (Moser \&
Devereux, 2019; Vincent, 2018). Some nonbinary
people use a combination of pronouns (either
deliberately mixing usage, allowing free choice,
or changing with social context), or prefer to
avoid gendered pronouns entirely, instead using
their name. Additionally, some nonbinary people
use she/her/hers, or he/him/his, sometimes or
exclusively, whilst in some regions in the world
descriptive language for nonbinary people does
not (yet) exist. In contexts outside of English, a
wide range of culturally specific linguistic adaptations and evolutions can be observed (Attig,
2022; Kirey-Sitnikova, 2021; Zimman, 2020). Also
of note, some languages use one pronoun that is
not associated with sex or gender while others
gender all nouns. These variations in language
are likely to influence nonbinary people's experience of gender and how they interact with others.

Recent studies suggest nonbinary people comprise roughly 25\% to over 50\% of the larger
transgender population, with samples of youth
reporting the highest percentage of nonbinary
people (Burgwal et al., 2019; James et al., 2016;
Watson, 2020). In recent studies of transgender
adults, nonbinary people tend to be younger than
transgender men and transgender women and in
studies of both youth and adults, nonbinary people are more likely to have been assigned female
at birth (AFAB). However, these findings should
be interpreted with caution as there are likely a
number of complex, sociocultural factors influencing the quality, representativeness, and accuracy of this data (Burgwal et al., 2019; James
et al., 2016; Watson, 2020; Wilson \& Meyer, 2021)
(see also Chapter 3---Population Estimates).

\hypertarget{understanding-gender-identities-and-gender-expressions-as-a-non-linear-spectrum}{%
\section*{Understanding gender identities and gender expressions as a non-linear spectrum}\label{understanding-gender-identities-and-gender-expressions-as-a-non-linear-spectrum}}
\addcontentsline{toc}{section}{Understanding gender identities and gender expressions as a non-linear spectrum}

Nonbinary genders have long been recognized
historically and cross-culturally (Herdt, 1994;
McNabb, 2017; Vincent \& Manzano, 2017). Many
gender identity categories are culturally specific
and cannot be easily translated from their context, either linguistically or in relation to the
Western paradigm of gender. Historical settler
colonial interactions with indigenous people with
non-Western genders remain highly relevant as
cultural erasure and the intersections of racism
and cisnormativity may detrimentally inform the
social determinants of health of indigenous gender diverse people. From the 1950s, gender was
used to reference the socially constructed categorization of behaviors, activities, appearance, etc.
in relation to a binary model of male/man/masculine, and female/woman/feminine within contemporary Western contexts. However, gender
now has a wider range of possible meanings,
appreciating interrelated yet distinguishable concepts, including gendered biology (sex), gender
roles, gender expression, and gender identity
(Vincent, 2020). Aspects of gender expression
that might traditionally be understood culturally
as ``masculine'', ``feminine'', or ``androgynous'' may
be legitimately expressed among people of any
and all gender identities, whether nonbinary or
not. For example, a nonbinary individual presenting in a feminine manner cannot be taken
to imply they will necessarily later identify as a
woman or access interventions associated with
transgender women, such as vaginoplasty. A person's gender nonconformity in relation to cultural
expectations should neither be viewed as a cause
for concern nor assumed to be indicative of clinical complexity---for example, a nonbinary person
assigned male at birth (AMAB) wearing
feminine-coded clothing, using she/her pronouns,
but keeping a masculine-coded first name.

Modeling gender as a spectrum offers greater
nuance than a binary model. However, there
remain significant limitations in a linear spectrum
model that can lead to uncritical generalizations
about gender. For example, while it is intuitive
to position the ``binary options'' (man/male,
woman/female) at either end of such a continuum, doing so situates masculinity as oppositional
to femininity, failing to accommodate gender neutrality, the expression of masculinity and femininity simultaneously, and genderqueer or
non-Western concepts of gender. It is essential
HCPs do not view nonbinary genders as ``partial''
articulations of transgender manhood (in nonbinary people AFAB) or transgender womanhood
(in nonbinary people AMAB), or definitively as
``somewhere along the spectrum of masculinity/
femininity''; some nonbinary individuals consider
themselves outside male/female dichotomization
altogether. A non-linear spectrum indicates differences of gender expression, identity, or needs
around gender affirmation between clients should
not be compared for the purposes of situating
them along a linear spectrum. Additionally, the
interpretation of gender expression is subjective
and culturally defined, and what may be experienced or viewed as highly feminine by one person
may not be viewed as such by another (Vincent,
2020). HCPs benefit from avoiding assumptions
about how each client conceptualizes their gender
and by being prepared to be led by a given client's
personal understanding of gender as it relates to
the client's gender identity, expression, and any
need for medical care.

The gender development process experienced
by all transgender and gender diverse (TGD)
people regardless of their relationship to a gender
binary appear to share similar themes (e.g.,
awareness, exploration, meaning making, integration), but the timing, progression, and personal
experiences associated with each of these processes vary both within and across groups of
transgender and nonbinary people (Kuper, Wright
et al., 2018; Kuper, Lindley et al., 2019; Tatum
et al., 2020). Sociocultural and intersectional perspectives can be helpful at contextualizing gender
development and social transition, including how
individual experiences are shaped by the social
and cultural context and how they interact with
additional domains of identity and personal
experience.

\hypertarget{the-need-for-access-to-gender-affirming-care}{%
\section*{The need for access to gender-affirming care}\label{the-need-for-access-to-gender-affirming-care}}
\addcontentsline{toc}{section}{The need for access to gender-affirming care}

Some nonbinary people seek gender-affirming
care to alleviate gender dysphoria or incongruence and increase body satisfaction through medically necessary interventions (see medically
necessary statement in Chapter 2---Global
Applicability, Statement 2.1). Some nonbinary
people may feel a certain treatment is necessary
for them---see also Chapter 5---Assessment of
Adults (Beek et al., 2015; Jones et al., 2019;
Köhler et al., 2018), whilst others do not (Burgwal
\& Motmans, 2021; Nieder, Eyssel et al., 2020),
and the proportion of nonbinary people who seek
gender-affirming care and the specific goals of
that care, remains unclear. It is the role of the
health care professional to provide information
about existing medical options (and their availability) that might help alleviate gender dysphoria
or incongruence and increase body satisfaction
without making assumptions about which treatment options may best fit each individual person.

Motivations for accessing (or not accessing)
gender-affirming medical interventions, including
hormone treatment, surgeries, or both are heterogeneous and potentially complex (Burgwal \&
Motmans, 2021; Vincent, 2019, 2020) and should
be explored collaboratively before making decisions about physical interventions. The need of
an individual to access gender-affirming medical
procedures cannot be predicted by their gender
role, expression, or identity. For example, some
transgender women have no need of vaginoplasty,
while some nonbinary individuals AMAB may
need and benefit from that same intervention.
Further, nonbinary people seeking gender-affirming
care associated closely with a transition pathway
from their assigned sex/gender to the other
binarily-recognized category (i.e., estrogen therapy
and vaginoplasty for someone AMAB) does not
undermine the validity of their nonbinary identity.

While barriers to care remain widespread for
many transgender people, nonbinary people
appear to experience particularly high rates of
difficulty accessing both mental health and
gender-affirming medical care (Clark et al., 2018;
James, 2016). Many nonbinary people report having experiences with health care professionals
who were not affirming of their nonbinary gender, including experiences where health care professionals convey beliefs that their gender is not
valid, or they are fundamentally more difficult
to provide care for (Valentine, 2016; Vincent,
2020). Nonbinary people may face provider
assumptions that they do not need or want
gender-affirming treatment (Kcomt et al., 2020;
Vincent, 2020) and have described experiencing
pressure to present themselves as transgender
men or transgender women (within a binary
framework of gender) in order to access treatment (Bradford et al., 2019; Taylor et al., 2019).
At times, nonbinary people find themselves educating the provider from whom they are seeking
services despite the inappropriateness of providers
relying primarily on their patients for education
(Kcomt et al., 2020). In comparison to transgender men and transgender women, Burgwal and
Motmans (2021) found that nonbinary people
experienced more fear of prejudice from health
care providers, less confidence in the services
provided, and greater difficulty knowing where
to go to for care. Studies in both Europe and US
have shown that nonbinary individuals tend to
delay care more often than binary transgender
men or transgender women, with fear of insensitive or incompetent treatment being the most
cited reason (Burgwal \& Motmans, 2021; Grant
et al., 2011). Nonbinary people also appear less
likely to disclose their gender identity to their
health care providers than other transgender people (Kcomt et al., 2020).

\hypertarget{the-need-for-an-appropriate-level-of-support}{%
\section*{The need for an appropriate level of support}\label{the-need-for-an-appropriate-level-of-support}}
\addcontentsline{toc}{section}{The need for an appropriate level of support}

Providing gender-affirming care to nonbinary people goes beyond the provision of specific genderaffirming interventions such as hormone therapy
or surgery and involves supporting the overall
health and development of nonbinary people.
Minority stress models have been adapted to conceptualize how the gender-related stressors experienced by transgender people are associated with
physical and mental health disparities (Delozier
et al., 2020; Testa et al., 2017). Nonbinary people
appear to experience minority stressors that are
both similar to and unique from those experienced
by transgender men and transgender women.
Johnson (2020) reported that experiences of invalidation are particularly high among nonbinary
people, e.g., statements or actions conveying a
belief that nonbinary identities are not ``real'' or
are the result of a ``fad'' or ``phase,'' and nonbinary
people appear less likely than transgender men
and transgender women to have their correct pronouns used by others. Similarly, nonbinary people
have described feeling ``invisible'' to others (Conlin,
2019; Taylor, 2018) and one study found that nonbinary youth reported lower levels of self-esteem
in comparison to young transgender men and
transgender women (Thorne, Witcomb et al., 2019).

While many TGD people report experiences
of discrimination, victimization, and interpersonal
rejection (James, 2016) including bullying within
samples of youth (Human Rights Campaign,
2018; Witcomb et al., 2019), the prevalence of
these experiences may vary across groups and
appears influenced by additional intersecting
characteristics. For example, Newcomb (2020)
found transgender women and nonbinary youth
AMAB experienced higher levels of victimization
than transgender men and nonbinary youth
AFAB, with nonbinary youth AMAB reporting
the highest levels of traumatic stress. In a second
study, Poquiz (2021) found transgender men and
transgender women experienced higher levels of
discrimination than nonbinary people. This intersectional complexity is also likely contributing to
the variability in findings from studies comparing
the physical and mental health of nonbinary and
transgender men and transgender women, with
some studies indicating more physical and mental
health concerns among nonbinary people, some
reporting less concerns, and some reporting no
difference between groups (Scandurra, 2019).

Given nonbinary identity narratives may be less
widely available than more binary-oriented identity narratives, nonbinary people may have less
resources available to explore and articulate their
gender-related sense of self. For example, this
might include access to community spaces and
interpersonal relationships where nonbinary identity can be explored, or access to language and
concepts that allow more nuanced consideration
of nonbinary experiences (Bradford et al., 2018;
Fiani \& Han, 2019; Galupo et al., 2019). Clinical
guidance is now developing to assist providers in
adapting gender-affirming therapeutic care to
meet these unique experiences of nonbinary people (Matsuno, 2019; Rider, Vencill et al., 2019).

\hypertarget{gender-affirming-medical-interventions-for-nonbinary-people}{%
\section*{Gender-affirming medical interventions for nonbinary people}\label{gender-affirming-medical-interventions-for-nonbinary-people}}
\addcontentsline{toc}{section}{Gender-affirming medical interventions for nonbinary people}

In contexts where a particular medical intervention does not have established precedent, it is
important that before the intervention is considered, the individual is provided with an overview
of the available information, including recognition
of potential knowledge limits. It is equally important to undertake and document a comprehensive
discussion of the physical changes needed and
the potential limitations in achieving those
attributes, as well as the implication that any
given intervention may or may not enhance an
individual's ability to express their gender.

With regards to estrogen therapy for nonbinary
people AMAB, it is important to note the possibility of breast growth cannot be avoided (Seal,
2017). Although the extent of growth is highly
variable, this should be made clear if a nonbinary
person seeks some of the other changes associated
with estrogen therapy (such as softening of skin
and reduction in facial hair growth) but does not
want or is ambivalent about breast growth.
Likewise, for nonbinary people AFAB who may
wish to access testosterone to acquire some changes
but not others, it should be recognized that if
facial hair development is needed, genital growth
is inevitable (Seal, 2017). The time frame for taking testosterone means these changes are likely
also to be accompanied by an irreversible vocal
pitch drop, although the extent of each is individual (Vincent, 2019; Ziegler et al., 2018). A vocal
pitch drop without the development of body hair
is another such challenge. For some nonbinary
people, hair removal is a very important part of
their gender affirmation (Cocchetti, Ristori,
Romani et al., 2020).

If hormonal therapy is discontinued and gonads
are retained, many physical changes will revert
to pre-hormone therapy status as gonadal hormones once again take effect, including reversal
of amenorrhea and body hair development in
nonbinary people AFAB and reduction in muscular definition and erectile dysfunction in nonbinary people AMAB. Other changes will be
permanent such as ``male-pattern'' baldness, genital growth, and facial hair growth in nonbinary
people AFAB or breast development in nonbinary
people AMAB (Hembree et al., 2017). These will
require further interventions to reverse, such as
electrolysis or mastectomy and are sometimes
described as ``partially reversible'' (Coleman et al.,
2012). As the implications of using low-dose hormone therapy are not documented in this patient
population, it is important to consider monitoring
for cardiovascular risk and bone health if low-dose
hormone therapy is used. For more detailed
information see Chapter 12---Hormone Therapy.

If neither testosterone nor estrogen expression is
needed, inhibition of estrogen and/or testosterone
production is possible. The implications of this with
regards to increased cardiovascular risk, reduced
bone mineralization, and risk of depression should
be discussed and measures taken to mitigate risk
(Brett et al., 2007; Vale et al., 2010; Wassersug \&
Johnson, 2007). For more information see also
Chapter 9---Eunuchs and Chapter 12---Hormone
Therapy. Exploration of medical and/or social transition independently of each other and options to
explore hormones, surgery, or both independently
of each other should be available to everyone,
whether the person is a transgender man, transgender woman, or a nonbinary person.
All the statements in this chapter have been
recommended based on a thorough review of
evidence, an assessment of the benefits and
harms, values and preferences of providers and
patients, and resource use and feasibility. In some
cases, we recognize evidence is limited and/or
services may not be accessible or desirable.

\hypertarget{statement-8.1-we-recommend-health-care-professionals-provide-nonbinary-people-with-individualized-assessment-and-treatment-that-affirms-their-nonbinary-experiences-of-gender.}{%
\section*{Statement 8.1 We recommend health care professionals provide nonbinary people with individualized assessment and treatment that affirms their nonbinary experiences of gender.}\label{statement-8.1-we-recommend-health-care-professionals-provide-nonbinary-people-with-individualized-assessment-and-treatment-that-affirms-their-nonbinary-experiences-of-gender.}}
\addcontentsline{toc}{section}{Statement 8.1 We recommend health care professionals provide nonbinary people with individualized assessment and treatment that affirms their nonbinary experiences of gender.}

An individualized assessment with a nonbinary
person starts with an understanding of how they
experience their own gender and how this impacts
their goals for the care they are seeking. How
individuals conceptualize their gender-related
experiences are likely to vary across groups and
cultures and may incorporate experiences associated with other intersecting aspects of identity
(e.g., age, sexuality, race, ethnicity, socioeconomic
status, disability status) (Kuper et al., 2014;
Subramanian et al., 2016).

HCPs should avoid making a priori assumptions
about any client's gender identity, expression, or
needs for care. They should also be mindful that
a client's nonbinary experience of gender may or
may not be relevant to the assessment and
treatment-related goals. The extent to which the
client's gender is relevant to their treatment goals
should determine the level of detail at which their
gender identity is explored. For example, when
seeking care for a presenting concern wholly unrelated to gender, simply determining the correct
name and pronouns may be sufficient (Knutson
et al., 2019). When addressing a concern for which
current or past hormonal or surgical status is relevant, more detail may be needed, even if the
concern is not specifically gender-related.

Clinical settings need to be welcoming, reflective of the diversity of genders, and affirm the
experiences of gender of nonbinary people to be
culturally competent. Ensuring clinic and provider information (e.g., websites), forms (e.g.,
intake surveys), and other materials are inclusive
of nonbinary identities and experiences conveys
that nonbinary people are welcome and recognized (Hagen \& Galupo, 2014). Using free text
fields for gender identity and pronouns is more
inclusive than using a list of response options.
Ensuring privacy at the reception desk, setting
up alternatives for listing legal names in digital
databases (in cultural contexts where this is necessary), installing gender-neutral toilets, and setting up alternatives to calling out the legal name
in the waiting room are additional examples of
transgender and gender diverse (TGD) cultural
competency (Burgwal et al., 2021). In care settings, it is important preferences for names, pronouns, and other gender-related terms are asked
and used both initially and on a regular basis as
they may vary over time and circumstance.

HCPs are encouraged to adopt an approach
that focuses on strengths and resilience.
Increasingly, critiques are emerging regarding
HCPs over-focus on gender-related distress as it
is also important to consider experiences of
increased comfort, joy, and self-fulfilment that
can result from self-affirmation and access to care
(Ashley, 2019a; Benestad, 2010). In addition to
utilizing diagnoses when/where required to facilitate access to care, HCPs are encouraged to collaboratively explore with clients this broader
range of potential gender-related experiences and
how they may fit with treatment options
(Motmans et al., 2019). For all TGD people, resiliency factors such as supportive relationships,
participation in communities that include similar
others, and identity pride are essential to consider
as they are associated with a range of positive
health outcomes (Bowling et al., 2019; Budge,
2015; Johns et al., 2018).

Awareness of the limitations that exist in the
tools providers have historically used to assess
transgender people's experience of dysphoria is
important as they may be particularly pronounced
for many nonbinary people. Most gender-related
measures assume clients experience their gender
in a binary way, among other concerns (e.g.,
Recalled Gender Identity Scale, Utrecht Gender
Dysphoria Scale). While several newer measures
have been developed in an attempt to better capture the experiences of nonbinary people (McGuire
et al., 2018; McGuire et al., 2020), open-ended
discussion is likely to provide a deeper and more
accurate understanding of each individual's unique
experiences of dysphoria and their associated care
needs. Similarly, while more recent iterations of
diagnostic categories (i.e., ``gender dysphoria'' in
the DSM 5 and ``gender incongruence'' in ICD-11)
were intended to be inclusive of people with nonbinary experiences of gender, they may not adequately capture the full diversity and scope of
experiences of gender-related distress, particularly
for nonbinary people. In addition to distress associated with aspects of one's physical body and
presentation (including features that may be existing or absent), distress may arise from how one
experiences their own gender, how one's gender
is perceived within social situations, and from
experiences of minority stress associated with one's
gender (Winters \& Ehrbar, 2010). Nonbinary peoples' experiences in each of these areas may or
may not be similar to those of transgender men
or women.

A person-centered approach for affirming care
includes specific discussion of how different
interventions may or may not shift the client's
comfort with their own experience of gender, and
how their gender is perceived by others.
Nonbinary people can face challenges in reconciling their personal identities with the limits of
the medical treatments available and can also
encounter confusion and intolerance from society
regarding their gender presentations (Taylor et al.,
2019). Emerging research suggests the medical
treatment needs of nonbinary people are particularly diverse, with some reporting needs for
treatments that have typically been associated
with transition trajectories historically associated
with transgender men and women and some
reporting alternative approaches (e.g., low dose
hormone therapy, surgery without hormone therapy), some reporting a lack of interest in medical
treatment, and some reporting feeling unsure
about their needs (Burgwal \& Motmans, 2021;
James et al., 2016). Conceptualizing assessment
as an ongoing process is particularly important
given gender-related experiences and associated
needs may shift throughout the lifespan. Given
the ongoing evolution in treatment options and
knowledge of treatment effects, particularly for
nonbinary people, clients will benefit from providers who regularly seek up-to-date knowledge
and convey these updates to their clients.

\hypertarget{statement-8.2-we-recommend-health-care-professionals-consider-medical-interventions-hormonal-treatment-or-surgery-for-nonbinary-people-in-the-absence-of-social-gender-transition.}{%
\section*{Statement 8.2: We recommend health care professionals consider medical interventions (hormonal treatment or surgery) for nonbinary people in the absence of ``social gender transition.''}\label{statement-8.2-we-recommend-health-care-professionals-consider-medical-interventions-hormonal-treatment-or-surgery-for-nonbinary-people-in-the-absence-of-social-gender-transition.}}
\addcontentsline{toc}{section}{Statement 8.2: We recommend health care professionals consider medical interventions (hormonal treatment or surgery) for nonbinary people in the absence of ``social gender transition.''}

Previous requirements for accessing hormonal
treatment and surgery, such as ``living in a gender
role that is congruent with one's gender identity,''
do not reflect the lived experiences of many TGD
people (Coleman et al., 2012). Due to the entrenched
nature of the gender binary in most contemporary
Western cultures, one can typically only be understood by others as a man or woman within most
settings (Butler, 1993). Hence, the visibility and
understanding of nonbinary embodiments and
expressions is limited. This is due to gendered cues
being almost always understood in reference to a
gender binary (Butler, 1993). Presently, it can be
difficult for nonbinary people to be reliably recognized as their gender via visual cues associated with
their gender expression (e.g., clothing, hair).
However, androgyny or gender nonconformity may
be communicated by the mixing or combining of
cultural markers with traditionally masculine or
feminine connotations. Because there is no commonly recognized ``nonbinary category'' within most
contemporary Western, global north cultural contexts, nonbinary visibility often necessitates explicit
sharing of one's gender with others or the use of
cues that may be interpreted as gender nonconformity (but not necessarily nonbinary).

For these reasons, framing access to medical
care in the context of someone experiencing a
``social gender transition'' where they are ``living
in a gender role that is congruent with one's gender identity'' is not in line with the way many
TGD people understand themselves and their
personal transition process. For some, ``living in
a gender role that is congruent with one's gender
identity'' does not involve changes in name, pronouns, or gender expression even as medical
intervention may be necessary. Even if a person
is able to live in ways that are congruent with
their gender identity, it may be difficult for an
outside observer to assess this without learning
directly from that person how they understand
their own experience in this regard. Expectation
of ``social gender transition'' may be unhelpful
when considering eligibility for gender- affirming
care, such as hormones and surgery, and rigid
expectations of what a ``social gender role transition'' ``should'' look like can be a barrier to care
for nonbinary people. There is no logical requirement gender-affirming medical interventions can
only be done once a person legally changes their
name, changes the gender marker on their identity documents, or wears or refrains from wearing
particular items of clothing. Nonbinary people
may struggle to access recognition of their genders on formal documentation, which may negatively affect their mental health or well-being
(Goetz \& Arcomano, 2021). TGD people may
benefit from specific support in accessing (or
retaining) their gender marker of preference. A
requirement that someone disclose their gender
identity in all circles of their lives (family, work,
school, etc.) in order to access medical care may
not be consistent with their goals and can place
them at risk if it is not safe to do so.

\hypertarget{statement-8.3-we-recommend-health-care-professionals-consider-gender-affirming-surgical-interventions-in-the-absence-of-hormonal-treatment-unless-hormone-therapy-is-required-to-achieve-the-desired-surgical-result.}{%
\section*{Statement 8.3: We recommend health care professionals consider gender-affirming surgical interventions in the absence of hormonal treatment unless hormone therapy is required to achieve the desired surgical result.}\label{statement-8.3-we-recommend-health-care-professionals-consider-gender-affirming-surgical-interventions-in-the-absence-of-hormonal-treatment-unless-hormone-therapy-is-required-to-achieve-the-desired-surgical-result.}}
\addcontentsline{toc}{section}{Statement 8.3: We recommend health care professionals consider gender-affirming surgical interventions in the absence of hormonal treatment unless hormone therapy is required to achieve the desired surgical result.}

The trajectory of ``hormones before surgery'' is
an option across a range of surgical interventions.
Some nonbinary people will seek gender-affirming
surgical treatment to alleviate gender incongruence
and increase body satisfaction (Beek et al., 2015;
Burgwal \& Motmans, 2021; Jones et al., 2019;
Koehler et al., 2018), but do not want hormonal
treatment or are unable to undergo hormonal therapy due to other medical reasons (Nieder, Eyssel
et al., 2020). Currently, it is unknown for which
proportion of nonbinary people these options apply.
Perhaps the surgery which has some specific
association with nonbinary people (rather than
sought by transgender men or undergone by
some cisgender women) is mastectomy in nonbinary people AFAB who have not taken testosterone---although testosterone is not a requirement
for this type of surgery---and some nonbinary
people AFAB may need breast reduction
(McTernan et al., 2020). An example of a surgery
for which at least a period of hormone therapy
may be necessary is metoidioplasty that enhances
the enlarged clitoris produced by testosterone
therapy. See Chapter 13---Surgery and
Postoperative Care for more detail on whether
hormone therapy is necessary for various surgeries. Procedures addressing the internal reproductive system include hysterectomy, unilateral or
bilateral salpingo-oophorectomy, and vaginectomy.
Hormone therapy is not required for any of these
procedures, but hormone replacement therapy
(either with estrogens, testosterone, or both) is
advisable in those individuals undergoing a total
gonadectomy to prevent adverse effects on their
cardiovascular and musculoskeletal systems
(Hembree et al., 2017; Seal, 2017). For phalloplasty, while there is no surgical requirement
per se for a minimum period of testosterone
treatment, virilization (or the absence of virilization) of the clitoris and labia minora may impact
the choice of surgical technique and influence
surgical options. For more information see
Chapter 13---Surgery and Postoperative Care.

Nonbinary AMAB clients should be informed
commencing estrogen therapy post-surgically with
no prior history of estrogen therapy may influence
(perhaps adversely) the surgical result (Kanhai, Hage,
Asscheman et al., 1999; Kanhai, Hage, Karim et al.,
1999). Nonbinary people AMAB requesting a bilateral orchiedectomy do not require estrogen therapy
to achieve a better outcome (Hembree et al., 2017).
In these contexts, it is good practice to inform clients of the risks and benefits of hormone replacement therapy (estrogens, testosterone, or both) in
preventing adverse effects on the cardiovascular and
musculoskeletal system as well as alternative treatment options, such as calcium plus vitamin D supplementation to prevent osteoporosis (Hembree
et al., 2017; Seal, 2017; Weaver et al., 2016). See
also Chapter 9---Eunuchs for those who choose to
forgo hormone replacement therapy. In the case of
vaginoplasty, individuals should be advised lack of
testosterone-blocking therapy may cause postoperative hair growth in the vagina when hair-bearing
skin graft and flaps have been used (Giltay \&
Gooren, 2000).

Additional surgical requests for nonbinary people AMAB include penile-preserving vaginoplasty,
vaginoplasty with preservation of the testicle(s),
and procedures resulting in an absence of external primary sexual characteristics (i.e., penectomy, scrotectomy, orchiectomy, etc.). The surgeon
and individual seeking treatment are advised to
engage in discussions so as to understand the
individual's goals and expectations as well as the
benefits and limitations of the intended (or
requested) procedure, to make decisions on an
individualized basis and collaborate with other
health care providers who are involved (if any).

\hypertarget{statement-8.4-we-recommend-health-care-professionals-provide-information-to-nonbinary-people-about-the-effects-of-hormonal-therapiessurgery-on-future-fertility-and-discuss-the-options-for-fertility-preservation-prior-to-starting-hormonal-treatment-or-undergoing-surgery.}{%
\section*{Statement 8.4: We recommend health care professionals provide information to nonbinary people about the effects of hormonal therapies/surgery on future fertility and discuss the options for fertility preservation prior to starting hormonal treatment or undergoing surgery.}\label{statement-8.4-we-recommend-health-care-professionals-provide-information-to-nonbinary-people-about-the-effects-of-hormonal-therapiessurgery-on-future-fertility-and-discuss-the-options-for-fertility-preservation-prior-to-starting-hormonal-treatment-or-undergoing-surgery.}}
\addcontentsline{toc}{section}{Statement 8.4: We recommend health care professionals provide information to nonbinary people about the effects of hormonal therapies/surgery on future fertility and discuss the options for fertility preservation prior to starting hormonal treatment or undergoing surgery.}

All nonbinary individuals who seek
gender-affirming hormonal therapies should be
offered information and guidance about fertility
options (Hembree et al., 2017; De Roo et al.,
2016; Defreyne, Elaut et al., 2020; Defreyne, van
Schuvlenbergh et al., 2020; Nahata et al., 2017;
Quinn et al., 2021). It is important to discuss
the potential impact of hormone therapy on fertility prior to initiation. This discussion should
include fertility preservation options, the extent
to which fertility may or may not be regained
if hormone therapy is ceased, and the fact that
hormone therapy per se is not birth control. For
more information see Chapter 16---
Reproductive Health.

Recent studies suggest that nonbinary individuals are less likely to access care and make their
needs for potential interventions heard (Beek
et al., 2015; Taylor et al., 2019). As such, it stands
to reason that any gender diverse individual
should be offered information on current options
and techniques for fertility preservation, ideally
prior to commencing hormonal treatment as the
quality of the sperm or eggs may be impacted
by exposure to hormones (Hamada et al., 2015;
Payer et al., 1979). However, this should in no
way preclude making inquiries and seeking more
information at a later time, as there is evidence
that fertility is still possible for individuals taking
estrogen and testosterone (Light et al., 2014). A
decision by a nonbinary or gender diverse person
that fertility preservation or counseling is not
needed should not be used as a basis for denying
or delaying access to hormonal treatment.

\hypertarget{eunuchs}{%
\chapter{Eunuchs}\label{eunuchs}}

Among the many people who benefit from
gender-affirming medical care, those who identify
as eunuchs are among the least visible. The 8th
version of the Standards of Care (SOC) includes
a discussion of eunuch individuals because of
their unique presentation and their need for medically necessary gender-affirming care (see
Chapter 2---Global Applicability, Statement 2.1).

Eunuch individuals are those assigned male at
birth (AMAB) and wish to eliminate masculine
physical features, masculine genitals, or genital
functioning. They also include those whose testicles have been surgically removed or rendered
nonfunctional by chemical or physical means and
who identify as eunuch. This identity-based definition for those who embrace the term eunuch
does not include others, such as men who have
been treated for advanced prostate cancer and
reject the designation of eunuch. We focus here
on those who identify as eunuchs as part of the
gender diverse umbrella.

As with other gender diverse individuals,
eunuchs may also seek castration to better align
their bodies with their gender identity. As such,
eunuch individuals are gender nonconforming
individuals who have needs requiring medically
necessary gender-affirming care (Brett et al.,
2007; Johnson et al., 2007; Roberts et al., 2008).

Eunuch individuals identify their gender identities in various ways. Many eunuch individuals
see their status as eunuch as their distinct gender
identity with no other gender or transgender
affiliation. The focus of this chapter is on the
treatment and care for those who identify as
eunuchs. Health care professionals (HCPs) will
encounter eunuchs requesting hormonal interventions, castration, or both to become eunuchs.
These individuals may also benefit from a eunuch
community because of the identification---with
or without actual castration.

While there is a 4000-year history of eunuchs
in society, the greatest wealth of information
about contemporary eunuch-identified people is
found within the large online peer-support community that congregates on sites such as the
Eunuch Archive (www.eunuch.org), which was
established in 1998. The moderators of this site
attempt to maintain both medical and historical
accuracy in its discussion forums, although there
is certainly misinformation as well. According to
the website, as of January 2022, there have been
over 130,000 registered members from various
parts of the world and frequently over 90\% of
those reading the site are ``guests'' rather than
members. The website lists over 23,000 threads
and nearly 220,000 posts. For example, two threads
giving instructions for self-castration by injection
of different toxins directly into the testicles have
about 2,500 posts each, and each has been read
well over one million times. Beginning in 2001,
there have been 20 annual international gatherings
of the Eunuch Archive community in Minneapolis
in addition to many regional gatherings elsewhere.
While the topic of castration is of interest to the
great majority of people who participate in the
discussions, it is a minority of the membership
who seriously seek or have undergone castration.
Many former Eunuch Archive members have
achieved their goals and no longer participate.

Because of misconceptions and prejudice about
historic eunuchs, the invisibility of contemporary
eunuchs, and the social stigma that affects all
gender and sexual minorities, few eunuch individuals come out publicly as eunuch and many
will tell no one and will share only with
like-minded people in an online community or
are known as such only to close family and
friends (Wassersug \& Lieberman, 2010). The stereotypes of eunuchs are often highly negative
(Lieberman 2018), and eunuchs may suffer the
same minority stress as other stigmatized groups
(Wassersug \& Lieberman, 2010). Research into
minority stress affecting gender diverse people
should therefore include eunuchs.

The current set of recommendations is directed
at professionals working with individuals who
identify as eunuchs (Johnson \& Wassersug, 2016;
Vale et al., 2010) requesting medically necessary
gender-affirming medical and/or surgical treatments (GAMSTs). Although not a specific diagnostic category in the ICD or DSM, eunuch is a
useful construct as it speaks to the specifics of
eunuch experience while also connecting it to
the experience of gender incongruence more
broadly. Eunuch individuals will present themselves clinically in various ways. They wish for
a body that is compatible with their eunuch identity---a body that does not have fully functional
male genitalia. Some other eunuch individuals
feel acute discomfort with their male genitals and
need to have them removed to feel comfortable
in their bodies (Johnson et al., 2007; Roberts
et al., 2008). Others are indifferent to having
male external genitalia as long as they are only
physically present and do not function to produce
androgens and male secondary sexual features
(Brett et al., 2007). Hormonal means may be used
to suppress the production of androgens, although
orchiectomy provides a permanent solution for
those not wishing genital functioning (Wibowo
et al., 2016). Some eunuch individuals desire
lower testosterone levels achieved with orchiectomy, but many will elect some form of hormone
replacement to prevent adverse effects associated
with hypogonadism. Most who elect hormone
therapy choose either a full or partial replacement
dose of testosterone. A smaller number elect
estrogen.

All the statements in this chapter have been
recommended based on a thorough review of
evidence, an assessment of the benefits and
harms, values and preferences of providers and
patients, and resource use and feasibility. In some
cases, we recognize evidence is limited and/or
services may not be accessible or desirable.

\hypertarget{statement-9.1-we-recommend-health-care-professionals-and-other-users-of-the-standards-of-care-version-8-guidelines-should-apply-the-recommendations-in-ways-that-meet-the-needs-of-eunuch-individuals.}{%
\section*{Statement 9.1: We recommend health care professionals and other users of the Standards of Care, Version 8 guidelines should apply the recommendations in ways that meet the needs of eunuch individuals.}\label{statement-9.1-we-recommend-health-care-professionals-and-other-users-of-the-standards-of-care-version-8-guidelines-should-apply-the-recommendations-in-ways-that-meet-the-needs-of-eunuch-individuals.}}
\addcontentsline{toc}{section}{Statement 9.1: We recommend health care professionals and other users of the Standards of Care, Version 8 guidelines should apply the recommendations in ways that meet the needs of eunuch individuals.}

Eunuch individuals are part of the population
of gender diverse people who experience gender
incongruence and may also seek gender-affirming
care. Like other transgender and gender diverse
(TGD) individuals, eunuchs require access to
affirming care to gain comfort with their gendered self. Each section of the SOC addresses
the needs of diverse individuals, and eunuchs
can be included within that group. They may
have commonality with some nonbinary individuals in that social transition may not be a
desired option, and hormone therapy may not
play the same role as it might in a social transition or transition within the binary (Wassersug
\& Lieberman, 2010).

Like other gender diverse individuals, eunuch
individuals may be aware of their identity in
childhood or adolescence. Due to the lack of
research into the treatment of children who may
identify as eunuchs, we refrain from making specific suggestions.

Eunuch individuals may seek medical or surgical care (hormone suppression, orchiectomy,
and, in some cases, penectomy) to achieve physical, psychological, or sexual changes (Wassersug
\& Johnson, 2007). It is important all patients,
including both eunuchs and those seeking castration, establish and maintain a relationship with
an HCP that is built upon trust and mutual
understanding. Given a lack of awareness of
eunuchs within the general medical community
and the fear among many individuals seeking
castration they will not be accepted, many do
not receive appropriate primary care and screening tests (Jäggi et al., 2018). Increased awareness
and education among medical providers will help
address the need to be informed about the need
to include eunuchs in discussions of gender
diversity (Deutsch, 2016a). It goes without saying
that eunuchs require and deserve the same primary care services as the general population. The
topic of screening tests for cancers, such as prostate and breast, is an important area for
discussion as the risks of hormone-related cancers
are likely different among male-assigned people
whose testosterone and estrogen levels are not in
the male range. Due to a lack of studies looking
at the prevalence and incidence of hormone-related
cancers in the eunuch population, there is no
evidence to guide how often to screen for
hormone-related cancers with prostate exams,
PSA measurements, mammograms, etcetera.

The large literature on prostate cancer patients
who have been medically or surgically castrated
provides information about some of the effects
of post-pubertal castration (such as potential
osteoporosis, depression, or metabolic syndrome),
but voluntary eunuchs may interpret the results
very differently from those castrated for medical
reasons. Chemical or surgical castration may be
experienced as a source of distress to cis men
with prostate cancer, while the same treatment
may be affirming and a source of comfort for
eunuch individuals. Similarly, transmasculine people who have a mastectomy to gain comfort with
their bodies experience that surgery differently
from ciswomen who undergo mastectomy to treat
breast cancer (Koçan \& Gürsoy, 2016; van de
Grift et al., 2016). The prostate cancer information is well summarized by Wassersug et al.
(2021) who provide references that explore the
large literature on the subject. Such information
on the effects of castration should be made available to those seeking castration.
Following an assessment as per the SOC-8,
medical options requested by the patient can be
considered and prescribed, if appropriate. These
options can be tailored to the individual to create
a plan that reflects their specific needs and preferences. The number and type of interventions
applied and the order in which these take place
may differ from person to person. These options
are consistent with both the assessment and surgery chapters of the SOC-8. Treatment options
for eunuchs to consider include:

\begin{itemize}
\tightlist
\item
  Hormone suppression to explore the effects of androgen deficiency for eunuch individuals wishing to become asexual, nonsexual, or androgynous;
\item
  Orchiectomy to stop testicular production of testosterone;
\item
  Orchiectomy with or without penectomy to alter their body to match their self-image;
\item
  Orchiectomy followed by hormone replacement with testosterone or estrogen.
\end{itemize}

Per statement 5.6 in Chapter 5---Assessment
of Adults, eunuch individuals seeking gonadectomy consider a minimum of 6 months of hormone therapy as appropriate to the TGD person's
gender goals before the TGD person undergoes
irreversible surgical intervention (unless hormones are not clinically indicated for the
individual).

\hypertarget{statement-9.2-we-recommend-health-care-professionals-consider-medical-intervention-surgical-intervention-or-both-for-eunuch-individuals-when-there-is-a-high-risk-that-withholding-treatment-will-cause-individuals-harm-through-self-surgery-surgery-by-unqualified-practitioners-or-unsupervised-use-of-medications-that-affect-hormones.}{%
\section*{Statement 9.2: We recommend health care professionals consider medical intervention, surgical intervention, or both for eunuch individuals when there is a high risk that withholding treatment will cause individuals harm through self-surgery, surgery by unqualified practitioners, or unsupervised use of medications that affect hormones.}\label{statement-9.2-we-recommend-health-care-professionals-consider-medical-intervention-surgical-intervention-or-both-for-eunuch-individuals-when-there-is-a-high-risk-that-withholding-treatment-will-cause-individuals-harm-through-self-surgery-surgery-by-unqualified-practitioners-or-unsupervised-use-of-medications-that-affect-hormones.}}
\addcontentsline{toc}{section}{Statement 9.2: We recommend health care professionals consider medical intervention, surgical intervention, or both for eunuch individuals when there is a high risk that withholding treatment will cause individuals harm through self-surgery, surgery by unqualified practitioners, or unsupervised use of medications that affect hormones.}

The same assessment process recommended in
the SOC-8 ought to apply to eunuchs (see Chapter
5---Assessment of Adults). The Eunuch Archive
has a large number of posts from individuals
finding great difficulty in seeking medical providers who will perform castration surgery. There
are a large number of eunuch individuals who
have performed self-surgery or have had surgery
performed by people who are not credentialed
medical providers (Johnson \& Irwig, 2014). There
are also clinical reports of eunuch individuals
who have self-castrated and accounts of patients
who have misled medical providers to obtain castration (Hermann \& Thorstenson, 2015;
Mukhopadhyay \& Chowdhury, 2009). There is
no doubt when members of this population are
denied access to quality medical treatment, they
will take actions that may cause them great harm,
such as bleeding and infection that may require
hospital admission (Hay, 2021; Jackowich et al.,
2014; Johnson \& Irwig, 2014). Because of these
serious problems and harm caused through
self-surgery, surgery by unqualified practitioners
or the unsupervised use of medications that affect
hormones, it is important health care providers
create a welcoming environment and consider
various treatment options after careful assessment
to avoid the problems that lack of access to treatment and withholding treatment will cause.

When desired, castration can be achieved
either chemically or surgically. For some, chemical castration can be an appropriate trial prior
to undergoing surgical castration to determine
how the individual feels when hypogonadal (Vale
et al., 2010). Chemical castration is usually
reversible if the medications are discontinued
(Wassersug et al., 2021). The most common types
of medications used to lower testosterone levels
are antiandrogens and estrogen.

The two most commonly used antiandrogens,
cyproterone acetate and spironolactone, are oral.
Estrogen is sometimes prescribed for prostate
cancer patients to lower serum testosterone levels
via negative feedback at the hypothalamus and
pituitary gland. Estrogens and antiandrogens may
not fully suppress testosterone levels into the
female or castrate range, and oral estrogens
increase the risk of venous thromboembolism.
Although not commonly used due to cost, gonadotropin releasing hormone (GnRH) agonists are
a very effective method for suppressing the production of sex steroids and fertility (Hembree
et al., 2017). When selecting a medication, we
advise using those which have been studied in
multiple transgender populations (i.e., estrogen,
cyproterone acetate, GnRH agonists) rather than
medications with little to no peer-reviewed scientific studies (i.e., bicalutamide, rectal progesterone, etc.) (Angus et al., 2021; Butler et al.,
2017; Efstathiou et al., 2019; Tosun et al., 2019).

Many eunuch individuals pursue hormone
replacement therapy following castration as they
do not desire the complete suppression of hormone levels and consequent problems, such as
the increased risk of osteoporosis. The two main
options for replacement of sex steroids are testosterone and estrogen that may be used in full
or partial replacement doses. The majority elect
testosterone as they present as male and are not
interested in feminization. A minority elect estrogen at a high enough dose to prevent osteoporosis, but low enough avoid most feminization.
They may identify as nonbinary, agender, or other
(Johnson et al., 2007; Johnson \& Wassersug, 2016).

Although studies on hormone replacement
therapy in eunuchs are lacking, findings from
cisgender men treated for prostate cancer can be
informative regarding the effects of hormone
therapy. In a randomized controlled trial of 1,694
cisgender men treated for locally advanced or
metastatic prostate cancer, one group received a
GnRH agonist and the other received transdermal
estrogen (Langley et al., 2021). Cisgender men
who received the GnRH agonist developed signs
and symptoms of both androgen and estrogen
deficiency, whereas men who received the estrogen patch only developed androgen-depleting
symptoms. Both groups had high rates of sexual
side effects (91\%), and weight gain was similar
among the groups. Compared with cisgender men
receiving the GnRH agonist, cisgender men
treated with estrogen patches had a higher
self-reported quality of life, lower rates of hot
flushes (35\% vs.~86\%), and higher rates of gynecomastia (86\% vs.~38\%). Metabolically, cisgender
men receiving estrogen patches had favorable
changes with a lower mean fasting glucose, fasting total cholesterol, systolic and diastolic blood
pressure. Conversely, cisgender men receiving the
GnRH agonist experienced the opposite effects.
Based on this study, eunuchs may consider a low
dose of transdermal estrogen therapy to avoid
adverse estrogen-depleting effects, which include
hot flashes, fatigue, metabolic effects, and loss of
bone mineral density (Hembree et al., 2017;
Langley et al., 2021). For further information see
Chapter 12---Hormone Therapy.

\hypertarget{statement-9.3-we-recommend-health-care-professionals-who-are-assessing-eunuch-individuals-for-treatment-have-demonstrated-competency-in-assessing-them.}{%
\section*{Statement 9.3: We recommend health care professionals who are assessing eunuch individuals for treatment have demonstrated competency in assessing them.}\label{statement-9.3-we-recommend-health-care-professionals-who-are-assessing-eunuch-individuals-for-treatment-have-demonstrated-competency-in-assessing-them.}}
\addcontentsline{toc}{section}{Statement 9.3: We recommend health care professionals who are assessing eunuch individuals for treatment have demonstrated competency in assessing them.}

A frequent topic on the discussion boards of
the Eunuch Archive is the difficulty of finding
practitioners who are able to understand their
needs. Eunuchs and those seeking castration usually are less visible than other gender minorities
(Wassersug \& Lieberman, 2010). Due to stigma
and fear of rejection by the medical community,
they may not voluntarily disclose their identity
and desires to their medical or mental health
providers. In some environments, medical providers may not be aware eunuchs exist and may
not even know they have treated eunuch-identified
patients.

The SOC section on assessment is applicable to
eunuch individuals. Like other gender diverse individuals, those seeking castration can engage in an
informed consent process in which qualified providers conduct assessments to ensure individuals are
capable of providing informed consent prior to medical interventions and to ensure a mental health
problem is not the etiology of the desire. As with
other sexual and gender minorities, working with
eunuchs requires an understanding that they are a
diverse population, and that each person is eunuch
in their own way (Johnson et al., 2007). The person
seeking services benefits from the professional's
accepting stance, open inquiry, suspension of judgment, and flexible expectations, combined with professional competency and expertise.

To provide appropriate treatment, providers
must establish trust and respect by creating an
inclusive environment for eunuch-identified people. For eunuch-identified individuals, the ideal
intake form would ask the assigned sex and identified gender and offer multiple gender options,
including ``eunuch'' and ``other.'' Individuals may
identify with more than one option and should
be able to select more than one.

HCPs may be involved in the assessment, psychotherapy (if desired), preparation, and follow-up
for medical and surgical gender-affirming interventions. They may also provide support for partners and families. Eunuch-identified individuals
who want the support of a qualified mental
health provider will benefit from a therapist who
meets the experience and criteria set out in
Chapter 4---Education.

While some individuals seeking or considering
castration come to counseling or therapy because
they want emotional support or help with
decision-making, many come to providers for an
assessment in preparation for specific medical
interventions (Vale et al., 2010).

\hypertarget{statement-9.4-we-suggest-health-care-professionals-providing-care-to-eunuch-individuals-include-sexuality-education-and-counseling.}{%
\section*{Statement 9.4: We suggest health care professionals providing care to eunuch individuals include sexuality education and counseling.}\label{statement-9.4-we-suggest-health-care-professionals-providing-care-to-eunuch-individuals-include-sexuality-education-and-counseling.}}
\addcontentsline{toc}{section}{Statement 9.4: We suggest health care professionals providing care to eunuch individuals include sexuality education and counseling.}

Several research studies have contributed to
our knowledge of contemporary eunuch-identified
people and have explored demographic characteristics and sexuality (Handy et al., 2015; Vale
et al., 2013; Wibowo et al., 2012, 2016). Medical
and MHPs should assume eunuchs are sexual
people capable of sexual activity, pleasure, and
relationships, unless they report otherwise
(Wibowo et al., 2021). Research has shown there
is great diversity among eunuchs regarding the
level of desire, type of preferred physical or sexual contact, and nature of preferred relationships
(Brett et al., 2007; Johnson et al., 2007; Roberts
et al., 2008). While some enjoy active sex lives
with or without romantic relationships, others
identify as asexual or aromantic and are relieved
by the loss of libido achieved through surgical
or chemical castration (Brett et al., 2007). Each
person is different, and one's genital status does
not determine sexual or romantic attraction
(Walton et al., 2016; Yule et al., 2015).

Regardless of the type of chemical suppression
or surgery a person has undergone, they may be
capable of sexual pleasure and sexual activity.
Contrary to popular belief, eunuchs are not necessarily asexual or nonsexual (Aucoin \& Wassersug,
2006). Safe sex education is necessary for all people
who engage in sexual activity that could involve an
exchange of body fluids. See Chapter 17---Sexual
Health for information regarding sex education and
safe sex options for people with diverse genders and
sexualities. In addition, fertility preservation should
be discussed when considering medical interventions
that might impact the possibilities for future parenthood. For more considerations see Chapter 16---
Reproductive Health.

\hypertarget{intersex}{%
\chapter{Intersex}\label{intersex}}

The Standards of Care, Version 7 included a chapter on the applicability of the standards to people
with physical intersexuality who become
gender-dysphoric and/or change their gender
because they differ from transgender individuals
without intersexuality in phenomenological presentation, life trajectories, prevalence, etiology, and
stigma risks. The current chapter provides an
update and adds recommendations on the medically necessary clinical approach to the management of individuals with intersexuality in general
(see medical necessity statement in Chapter 2---
Global Applicability, Statement 2.1). Because a newborn with an atypical sexual differentiation may
already present with clinical challenges, including
the need for family education and support from
early on, the decision-making on gender assignment, subsequent clinical gender management,
components of which---especially genital surgery---
may be controversial, and a later risk of gender
dysphoria development and gender change that is
markedly increased (Sandberg \& Gardner, 2022).

\hypertarget{terminology-2}{%
\section*{Terminology}\label{terminology-2}}
\addcontentsline{toc}{section}{Terminology}

``Intersex'' (from Latin, literal translation ``between
the sexes'') is a term grounded in the binary
system of sex underlying mammalian (including
human) reproduction. In medicine, the term is
colloquially applied to individuals with markedly
atypical, congenital variations in the reproductive
tract. Some variations, often labeled ``genital
ambiguity,'' preclude the simple recognition of
somatic sex as male or female and, in resource-rich
societies, may require a comprehensive physical,
endocrine, and genetic work-up, before a sex/
gender is ``assigned.'' In recent years ``intersex''
has also become an identity label adopted by
some individuals with intersex conditions and a
subset of (non-intersex) individuals with a nonbinary gender identity (Tamar-Mattis et al., 2018).

At a 2005 international consensus conference
on intersex management, intersex conditions were
subsumed under a new standard medical term,
``Disorders of Sex Development'' (DSD), defined
as ``congenital conditions in which development
of chromosomal, gonadal, or anatomical sex is
atypical'' (Hughes et al., 2006). DSD covers a
much wider range of conditions than those traditionally included under intersexuality and comprises conditions such as Turner syndrome and
Klinefelter syndrome, which are much more prevalent. In addition, many affected individuals dislike the term ``disorder,'' viewing it as inherently
stigmatizing (Carpenter, 2018; Griffiths, 2018;
Johnson et al., 2017; Lin-Su, et al., 2015; Lundberg
et al., 2018; Tiryaki et al., 2018). Health care professionals (HCPs) also vary in their acceptance
of the term (Miller et al., 2018). The wide-spread
alternative reading of DSD as ``Differences in Sex
Development'' can be seen as less pathologizing,
but is semantically unsatisfactory as this term
does not distinguish the typical genital differences
between males and females from atypical sexual
differentiation. Other recent attempts to come up
with less obviously stigmatizing terms such as
``Conditions Affecting Reproductive Development''
(CARD; Delimata et al., 2018) or ``Variations of/
in Sex Characteristics'' (VSC; Crocetti, et al.,
2021) are also not specific to intersexuality.

Given these definitional issues, in this chapter
we are using the term ``intersexuality'' (or ``intersex'') to refer to congenital physical manifestations only. This is done for both descriptive
clarity and historical continuity. This choice is
not meant to indicate an intention on our part
to take sides in the ongoing discussion regarding
the concept of sex/gender as a bipolar system or
as a continuum, which may vary with considerations of context and utility (Meyer-Bahlburg,
2019). In 21st century societies, the concepts of
sex and gender are in a process of evolution.

\hypertarget{prevalence}{%
\section*{Prevalence}\label{prevalence}}
\addcontentsline{toc}{section}{Prevalence}

The prevalence of intersex conditions depends
on the definition used. Obvious genital atypicality
(``ambiguous genitalia'') occurs with an estimated
frequency ranging from approximately 1:2000---
1:4500 people (Hughes et al., 2007). The most
inclusive definitions of DSD estimate a prevalence
of up to 1.7\% (Blackless et al., 2000). Although
these numbers are high in aggregate, the individual conditions associated with the intersex
variations tend to be much rarer. For instance,
androgen insensitivity syndrome (AIS) occurs in
approximately 1 in 100,000 46,XY births (Mendoza
\& Motos, 2013), and classic congenital adrenal
hyperplasia (CAH) in approximately 1 in 15,000
46,XX births (Therrell, 2001). Prevalence figures
for individual syndromes may vary dramatically
between countries and ethnic groups.

\hypertarget{presentation}{%
\section*{Presentation}\label{presentation}}
\addcontentsline{toc}{section}{Presentation}

The presentation of individuals with intersex traits
varies widely. Intersexuality can be recognized
during prenatal ultrasound imaging, although most
individuals will be identified during genital examinations at birth. In resource-rich societies, such
children will undergo extensive medical diagnostic
procedures within the first weeks of life. Taking
into consideration the specific medical diagnosis,
physical and hormonal findings, and information
from long-term follow-up studies about gender
outcome, joint decision-making between the
health-care team and the parents generally leads
to the newborn being assigned to the male or
female sex/gender. Some individuals with intersexuality come to the attention of specialists only
around the age of puberty, for instance, when
female-raised adolescents are evaluated for primary
amenorrhea.

HCPs assisting individuals with both intersexuality and gender uncertainty need to be
aware that the medical context in which such
individuals have grown up is typically very different from that of non-intersex TGD people.
There are many different syndromes of intersexuality, and each syndrome can vary in its
degree of severity. Thus, hormonal and surgical
treatment approaches vary accordingly.

Some physical manifestations of intersexuality
may require early urgent intervention, as in cases
of urinary obstruction or of adrenal crisis in
CAH. Most physical variations among individuals
with intersexuality neither impair function, at
least in the early years, nor risk safety for the
individual. Yet, the psychosocial stigma associated
with atypical genital appearance often motivates
early genital surgery (commonly labeled `corrective' or `normalizing') long before the individual
reaches the age of consent. This approach is
highly controversial because it conflicts with ethical principles supporting a person's autonomy
(Carpenter, 2021; Kon, 2015; National Commission
for the Protection of Human Subjects of
Biomedical and Behavioral Research, 1979). In
addition, among the manifestations without
immediate safety concerns, some individuals,
when older, may opt for a range of medical interventions to optimize function and appearance.
The specifics of medical treatments are far beyond
the scope of what can be addressed in this chapter, and the interested reader should consult the
respective endocrine and surgical literature.

Some intersex conditions are associated with a
greater variability in long-term gender identity outcome than others (Dessens et al., 2005). For instance,
the incidence of a non-cisgender gender identity in
46,XX individuals with CAH assigned female may
be as high as 5--10\% (Furtado et al., 2012). The
substantial biological component underlying gender
identity is a critical factor that must be considered
when offering psychosocial, medical, and surgical
interventions for individuals with intersex conditions.

There is also ample evidence people with intersexuality and their families may experience psychosocial distress (de Vries et al., 2019;
Rosenwohl-Mack et al., 2020; Wolfe-Christensen
et al., 2017), in part related to psychosocial
stigma (Meyer-Bahlburg, Khuri et al., 2017;
Meyer-Bahlburg, Reyes-Portillo et al., 2017;
Meyer-Bahlburg et al., 2018).

\hypertarget{intersexuality-in-the-psychiatric-nomenclature}{%
\section*{Intersexuality in the psychiatric nomenclature}\label{intersexuality-in-the-psychiatric-nomenclature}}
\addcontentsline{toc}{section}{Intersexuality in the psychiatric nomenclature}

Since 1980, the American psychiatric nomenclature recognized individuals with intersexuality
who meet the criteria for gender identity variants;
however, their diagnostic categorization changed
with successive DSM editions. For instance, in
DSM-III (American Psychiatric Association,
1980), the Axis-I category of ``transsexualism''
could not be applied to such individuals in adulthood, but such children were labeled ``gender
identity disorder of childhood,'' with the medical
intersex condition to be specified in Axis III. In
DSM-IV-TR (American Psychiatric Association,
2000), individuals with intersexuality were
excluded from the Axis-I category of ``gender
identity disorder'' regardless of age and, instead,
grouped with other conditions under the category
``gender identity disorder not otherwise specified.''
In DSM-5 (American Psychiatric Association,
2013), which moved away from the multiaxial
system, ``gender identity disorder'' was re-defined
as ``gender dysphoria'' and applied regardless of
age and intersex status, but individuals with intersexuality received the added specification ``with
a disorder of sex development'' (Zucker et al.,
2013). The just published text revision of DSM-5
(American Psychiatric Association, 2022) keeps
the term gender dysphoria. Note, however, the
recent revision of the International Classification
of Diseases {[}ICD-11; World Health Organization,
2019a{]} has moved ``gender incongruence'' from
the chapter ``Mental, B ehavioral, or
Neurodevelopmental Disorders'' to a new chapter
``Conditions Related to Sexual Health.''

All the statements in this chapter have been
recommended based on a thorough review of
evidence, an assessment of the benefits and
harms, values and preferences of providers and
patients, and resource use and feasibility. In some
cases, we recognize evidence is limited and/or
services may not be accessible or desirable.

\hypertarget{statement-10.1-we-suggest-a-multidisciplinary-team-knowledgeable-in-diversity-of-gender-identity-and-expression-as-well-as-in-intersexuality-provide-care-to-individuals-with-intersexuality-and-their-families.}{%
\section*{Statement 10.1: We suggest a multidisciplinary team, knowledgeable in diversity of gender identity and expression as well as in intersexuality, provide care to individuals with intersexuality and their families.}\label{statement-10.1-we-suggest-a-multidisciplinary-team-knowledgeable-in-diversity-of-gender-identity-and-expression-as-well-as-in-intersexuality-provide-care-to-individuals-with-intersexuality-and-their-families.}}
\addcontentsline{toc}{section}{Statement 10.1: We suggest a multidisciplinary team, knowledgeable in diversity of gender identity and expression as well as in intersexuality, provide care to individuals with intersexuality and their families.}

Intersexuality, a subcategory of DSD, is a complex congenital condition that requires the
involvement of experts from various medical and
behavioral disciplines (Hughes et al., 2006). Team
composition and function can vary depending on
team location, local resources, diagnosis, and the
needs of the individual with intersexuality and
her/his/their family. The ideal team includes pediatric subspecialists in endocrinology, surgery and/
or urology, psychology/psychiatry, gynecology,
genetics, and, if available, personnel trained in
social work, nursing, and medical ethics (Lee
et al., 2006). The structure of the team can be
in line with 1) the traditional multidisciplinary
medical model; 2) the interprofessional model;
or 3) the transdisciplinary model. Although these
structures can appear similar, they are in fact
very different and can exert varying influences
on how the team functions (Sandberg \& Mazur,
2014). The 2006 Consensus Statement makes no
decision about which model is best---multidisciplinary, interdisciplinary, or transdisciplinary---
and only states the models ``imply different
degrees of collaboration and professional
autonomy'' (Lee, Nordenström et al., 2016). Since
the publication of the Consensus Statement in
2006, such teams have been created both in
Europe and in the US. A listing of teams in the
US can be found on the DSD-Translational
Network (DSD-TRN) website. There are also
teams in a number of European countries (Thyen
et al., 2018). While there are barriers to the creation of teams as noted by Sandberg and Mazur
(2014), multidisciplinary teams help address a
number of problems that have undermined the
successful care of individuals with an intersex
diagnosis and their families, such as the scattered
nature of services, the limited or absent communication between professionals, and the resulting
fragmented nature of the explanations individuals
receive that cause more confusion than clarity.

Most individuals born with intersexuality will
be identified at birth or shortly thereafter, while
others will be identified at later times in the life
cycle, for example at puberty (see Brain et al.,
2010, Table 1). When this happens the team
approach will be modified based on the diagnosis
and the age of the person. In some circumstances,
the composition of the team can be expanded to
include other specialists as needed.

It has been reported children seen by a multidisciplinary team were significantly more likely
to receive nearly the full range of services rather
than only those services offered by a single provider (Crerand et al., 2019). Parents who received
such care positively endorsed psychosocial services and the team approach and reported receiving more information than those who did not
interact with such a team (Crerand et al., 2019).

\hypertarget{statement-10.2-we-recommend-health-care-professionals-providing-care-for-transgender-youth-and-adults-seek-training-and-education-in-the-aspects-of-intersex-care-relevant-to-their-professional-discipline.}{%
\section*{Statement 10.2: We recommend health care professionals providing care for transgender youth and adults seek training and education in the aspects of intersex care relevant to their professional discipline.}\label{statement-10.2-we-recommend-health-care-professionals-providing-care-for-transgender-youth-and-adults-seek-training-and-education-in-the-aspects-of-intersex-care-relevant-to-their-professional-discipline.}}
\addcontentsline{toc}{section}{Statement 10.2: We recommend health care professionals providing care for transgender youth and adults seek training and education in the aspects of intersex care relevant to their professional discipline.}

Results from interviews with medical trainees
(Liang et al., 2017; Zelin et al., 2018) and from
programmatic self-audits and surveys (DeVita
et al., 2018; Khalili et al., 2015) suggest medical
training programs are not adequately preparing
practitioners to provide competent care to individuals presenting with gender dysphoria and
intersexuality. Professional and stakeholder
attendees of intersex-specific events have identified ongoing education and collaboration as
an important professional development need
(Bertalan et al., 2018; Mazur et al., 2007). This
may be especially true for adult care providers
who may have less clinical guidance or support
in assisting those individuals who are transitioning from pediatric to adult care (Crouch \&
Creighton, 2014).

However, there are few guidelines for training
or assessing practitioner competency in managing
these topics, and those that are available primarily
apply to mental health professionals (MHPs)
(Hollenbach et al., 2014), with the exception of
a primary care guide (National LGBTQIA+Health
Education Center, 2020).

For HCPs wanting to improve their competency, seeking consultation from experts may be
an option when formal education or empirical
guidelines are otherwise unavailable. Given the
relative widespread adoption of multidisciplinary
expert teams in the treatment of intersexuality
(Pasterski et al., 2010), individuals serving on
these teams are well positioned to consult with
and educate other health care staff who may not
have received adequate training (Hughes et al.,
2006). Therefore, it is recommended the training
of other professionals be a central component of
team development (Auchus et al., 2010) and
members of multidisciplinary teams receive training specific to team-based work, including strategies for engaging in interprofessional learning
(Bisbey, et al., 2019; Interprofessional Education
Collaborative Expert Panel, 2011).

\hypertarget{statement-10.3-we-suggest-health-care-professionals-educate-and-counsel-families-of-children-with-intersexuality-from-the-time-of-diagnosis-onward-about-the-childs-specific-intersex-condition-and-its-psychosocial-implications.}{%
\section*{Statement 10.3: We suggest health care professionals educate and counsel families of children with intersexuality from the time of diagnosis onward about the child's specific intersex condition and its psychosocial implications.}\label{statement-10.3-we-suggest-health-care-professionals-educate-and-counsel-families-of-children-with-intersexuality-from-the-time-of-diagnosis-onward-about-the-childs-specific-intersex-condition-and-its-psychosocial-implications.}}
\addcontentsline{toc}{section}{Statement 10.3: We suggest health care professionals educate and counsel families of children with intersexuality from the time of diagnosis onward about the child's specific intersex condition and its psychosocial implications.}

Full disclosure of medical information to families of children with intersex conditions through
education and counseling should begin at the time
of diagnosis and should be consistent with guidance from multiple international consensus guidelines. One of the most challenging issues presented
by a newborn with intersexuality, particularly
when associated with noticeable genital ambiguity,
is sex assignment and from the parents' perspective, the gender of rearing (Fisher, Ristori et al.,
2016). Given this is a very stressful situation for
most parents, it is generally recommended the
decisions about sex/gender should be made as
quickly as a thorough diagnostic evaluation permits (Houk \& Lee, 2010). However, the criteria
for sex/gender decisions have changed over time.
In the second half of the 20th century, the decisions were biased towards female assignment,
because feminizing genital surgery was seen as
easier and less side-effect prone than masculinizing surgery. Yet, in certain intersex conditions, for
instance 46,XY 5α-RD-2 deficiency, female sex/
gender assignment was found to be associated with
high rates of later gender dysphoria and gender
change (Yang et al., 2010). Therefore, since the
International Consensus Conference on Intersex
Management in 2005, sex/gender assignment takes
into consideration the gradually accumulating data
on long-term gender outcome in the diverse conditions of intersexuality.

The practice of disclosure seeks to enable more
fully informed decision-making about care.
Additionally, while shame and stigma surrounding intersexuality is associated with poorer psychosocial outcomes, open and proactive
communication of health information has been
proposed as a strategy to reduce those risks (de
Vries et al., 2019). Depending on the person's
diagnosis and developmental stage, intersex conditions may differentially impact individuals and
their health care needs. Intersex-health-related
communication must therefore be continuous and
tailored to the individual. Research on
decision-making in intersex care suggests families
are influenced by how clinical teams communicate (Timmermans et al., 2018). In keeping with
the SOC, we encourage providers to adopt normalizing, affirming language and attitudes across
education and counseling functions. For example,
describing genital atypia as a ``variation'' or ``difference'' is more affirming than using the terms
``birth defect'' or ``abnormality.''

All HCPs involved in an individual's care can
provide essential education and information to
families. In multidisciplinary teams, the type of
education may align with an HCP's area of
expertise, for example, a surgeon educating the
individual on their anatomy, an endocrinologist
teaching the specifics of hormonal development,
or an MHP conveying the spectrums of gender
and sexual identity. Other HCPs may need to
provide comprehensive education. Families
should receive information that is pertinent to
the individual's specific intersex variation, when
known. All HCPs can supplement this information with patient-centered resources available
from support groups. People with intersexuality
have also been hired as team members to provide education using their lived experience.

Consensus guidelines also recommend families
be offered ongoing peer and professional psychosocial support (Hughes et al., 2006) that may
involve counseling with a focus on problem-solving
and anticipatory guidance (Hughes et al., 2006).
For example, families may seek guidance in educating other people---siblings, extended family,
and caregivers---about the specific intersex condition of an individual. Other families may need
support or mental health care to manage the
stress of intersex treatment. Adolescents may benefit from guidance on how to disclose information to peers as well as from support when
navigating dating and sex. Providing counseling
may also involve guiding families and individuals
of all ages through a shared decision-making process around medical or surgical care. Providers
may employ decision aids to support this process
(Sandberg et al., 2019; Weidler et al., 2019).

\hypertarget{statement-10.4-we-suggest-both-providers-and-parents-engage-childrenindividuals-with-intersexuality-in-ongoing-developmentally-appropriate-communications-about-their-intersex-condition-and-its-psychosocial-implications.}{%
\section*{Statement 10.4: We suggest both providers and parents engage children/individuals with intersexuality in ongoing, developmentally appropriate communications about their intersex condition and its psychosocial implications.}\label{statement-10.4-we-suggest-both-providers-and-parents-engage-childrenindividuals-with-intersexuality-in-ongoing-developmentally-appropriate-communications-about-their-intersex-condition-and-its-psychosocial-implications.}}
\addcontentsline{toc}{section}{Statement 10.4: We suggest both providers and parents engage children/individuals with intersexuality in ongoing, developmentally appropriate communications about their intersex condition and its psychosocial implications.}

Communicating health information is a
multi-directional process that includes the transfer of information from providers to patients,
from parents to patients, as well as from patients
back to their providers (Weidler \& Peterson,
2019). While much emphasis has been placed on
communicating to parents around issues of diagnosis and surgical decision-making, youth with
DSD have reported barriers to engaging with
health care providers and may not always turn
to their parents for support (Callens et al., 2021).
To prepare individuals to be fully engaged and
autonomous in their treatment, it is critical both
providers and parents communicate continuously
with children/individuals.

Providers must set an expectation as soon as
possible for ongoing, open communication
between all parties, especially since parents may
experience distress due to the uncertainty associated with DSD and may seek quick fixes
(Crissman et al., 2011; Roberts et al., 2020).
Models of shared decision-making as well as
related decisional tools have been developed to
support ongoing communication between HCPs
and families/individuals (Karkazis et al., 2010;
Sandberg et al., 2019; Siminoff \& Sandberg, 2015;
Weidler et al., 2019). In addition to setting an
expectation for dialogue, providers can also set
the tone of communication. Providers can help
parents and individuals tolerate diagnostic uncertainty while simultaneously providing education
on anatomic variations, modeling openness to
gender and sexual identity, and welcoming the
child's/individual's questions. As they age, children/individuals may have questions or need
age-appropriate information on issues of sex,
menstruation, fertility, the need for hormone
treatment (adrenal/sex), bone health, and cancer risk.

Parents also play a critical role in educating
their children and may be the first people to
disclose health information to their child (Callens
et al., 2021). As part of expectation-setting around
communication, providers should prepare parents
to educate their child and members of their support system about the intersex diagnosis and
treatment history. Some parents report difficulties
in knowing how much to disclose to others as
well as to their own children (Crissman et al.,
2011; Danon \& Kramer, 2017). The stress parents
experience while raising children with an intersex
condition is increased when parents adopt an
approach that minimizes disclosure/discussion of
their child's diagnosis (Crissman et al., 2011).
The level of stress also varies by developmental
stage, with parents of adolescents reporting higher
rates of stress (Hullman et al., 2011). Therefore,
HCPs should assist parents in developing strategies specific to their child's developmental stage
that address their psychosocial or cultural concerns and values (Danon \& Kramer, 2017; Weidler
\& Peterson, 2019). Finally, broader research on
sexuality and gender variance has found---counter
to the associations between shame/stigma and
negative health outcomes---supportive family
behaviors (including talking with children about
their identity and connecting them with peers)
predicted greater self-esteem and better health
outcomes in individuals (Ryan et al., 2010).

\hypertarget{statement-10.5-we-suggest-health-care-professionals-and-parents-support-childrenindividuals-with-intersexuality-in-exploring-their-gender-identity-throughout-their-life.}{%
\section*{Statement 10.5: We suggest health care professionals and parents support children/individuals with intersexuality in exploring their gender identity throughout their life.}\label{statement-10.5-we-suggest-health-care-professionals-and-parents-support-childrenindividuals-with-intersexuality-in-exploring-their-gender-identity-throughout-their-life.}}
\addcontentsline{toc}{section}{Statement 10.5: We suggest health care professionals and parents support children/individuals with intersexuality in exploring their gender identity throughout their life.}

Psychological, social, and cultural constructs
all intersect with biological factors to form an
individual's gender identity. As a group, individuals with intersexuality show increased rates of
gender nonconforming behavior, genderquestioning, and cross-gender wishes in childhood, dependent in part on the discrepancy
between the prenatal sex-hormonal milieu in
which the fetal brain has differentiated and the
sex assigned at birth (Callens et al., 2016; Hines,
et al., 2015; Meyer-Bahlburg et al., 2016; Pasterski
et al., 2015). Gender identity problems are
observed at different rates in individuals with
different intersex conditions (de Vries et al.,
2007). More recently, some individuals have been
documented to develop a nonbinary identity, at
least privately (Kreukels et al., 2018). Although
the majority of people with intersexuality may
not experience gender dysphoria or wishes for
gender transition, they may still have feelings of
uncertainty and unanswered questions regarding
their gender (Kreukels et al., 2018). Questions
about gender identity may arise from such factors
as genital appearance, pubertal development, and
knowledge of items such as the diagnostic term
of the medical condition, gonadal status, sex
chromosome status, and a history of genital surgery. Therefore, HCPs need to be accessible for
clients to discuss such questions and feelings,
openly converse about gender diversity, and adopt
a less binary approach to gender. HCPs are
advised to guide parents as well in supporting
their children in exploring gender.

Furthermore, such support should not be confined to the childhood years. Rather, individuals
should be given the opportunity to explore their
gender identity throughout their lifetime, because
different phases may come with new questions
regarding gender (for example, puberty/adolescence, childbearing age). Children in general may
have questions regarding their gender identity at
salient points during their maturation and evolution. When faced with additional stressors, for
example, genital ambiguity, genital examinations
and procedures, as well as the intersectionality
of cultural bias and influences, individuals with
intersexuality may need support and should be
encouraged to seek educated professional assistance and guidance when needed. Also, HCPs
should inquire regularly to determine if their
clients with intersexuality need such support.
When people experience gender incongruence,
gender-affirming interventions may be considered. Procedures that should be applied in such
interventions are described in other chapters.

\hypertarget{statement-10.6-we-suggest-health-care-professionals-promote-well-being-and-minimize-the-potential-stigma-of-having-an-intersex-condition-by-working-collaboratively-with-both-medical-and-non-medical-individualsorganizations.}{%
\section*{Statement 10.6: We suggest health care professionals promote well-being and minimize the potential stigma of having an intersex condition by working collaboratively with both medical and non-medical individuals/organizations.}\label{statement-10.6-we-suggest-health-care-professionals-promote-well-being-and-minimize-the-potential-stigma-of-having-an-intersex-condition-by-working-collaboratively-with-both-medical-and-non-medical-individualsorganizations.}}
\addcontentsline{toc}{section}{Statement 10.6: We suggest health care professionals promote well-being and minimize the potential stigma of having an intersex condition by working collaboratively with both medical and non-medical individuals/organizations.}

Individuals with intersexuality are reported to
experience stigma, feelings of shame, guilt, anger,
sadness and depression (Carroll et al., 2020;
Joseph et al., 2017; Schützmann et al., 2009).
Higher levels of psychological problems are
observed in this population than in the general
population (Liao \& Simmonds, 2014; de Vries
et al., 2019). In addition, parental fear of stigmatization and adjustment to their child's diagnosis must not be overlooked by the clinical
team. Parents may benefit from supportive counseling to assist them both in managing clinical
decision-making (Fleming et al., 2017; Rolston
et al., 2015; Timmermans et al., 2019) as well as
understanding the impact of clinical decisions on
their view of their child (Crissman et al., 2011;
Fedele et al., 2010).

Thyen et al.~(2005) found repeated genital
examinations appear to be correlated with shame,
fear and pain and may increase the likelihood of
developing post-traumatic stress disorder (PTSD)
later in life (Alexander et al., 1997; Money \&
Lamacz, 1987). Exposure to repeated genital
examinations, fear of medical interventions, and
parental and physician secrecy about being intersex ultimately undermine the self-empowerment
and self-esteem of the person with intersexuality
(Meyer-Bahlburg et al., 2018; Thyen et al., 2005;
Tishelman et al., 2017; van de Grift, Cohen-Kettenis
et al., 2018). For recommendations on how to
conduct genital examinations to minimize adverse
psychological side effects see Tishelman
et al.~(2017).

There is an active movement within the intersex community to alleviate stigma and to return
human rights and dignity to intersex people
rather than viewing them as medical anomalies
and curiosities (Yogyakarta Principles, 2007,
2017). Chase (2003) summarizes the major reasons for the intersex advocacy movement and
outlines how stigma and emotional trauma are
the outcome of ignorance and the perceived need
for secrecy. Public awareness of intersex conditions is very limited, and images and histories of
individuals with intersexuality are still presented
as ``abnormalities of nature''. We, therefore, advise
HCPs to actively educate their colleagues, individuals with intersexuality, their families, and
communities, raise public awareness, and increase
knowledge about intersexuality. Societal awareness
and knowledge regarding intersexuality may help
reduce discrimination and stigmatization. Tools
and education/information materials may also
help individuals with intersexuality disclose their
condition, if desired (Ernst et al., 2016).

HCPs should be able to recognize and address
stigmatization in their clients (Meyer-Bahlburg
et al., 2018) and should encourage people with
intersexuality of various ages to connect via support groups. There is a need for developing specific techniques/methods for assisting clients to
cope with stigma related to intersex.

\hypertarget{statement-10.7-we-suggest-health-care-professionals-refer-childrenindividuals-with-intersexuality-and-their-families-to-mental-health-professionals-as-well-as-peer-and-other-psychosocial-supports-as-indicated.}{%
\section*{Statement 10.7: We suggest health care professionals refer children/individuals with intersexuality and their families to mental health professionals as well as peer and other psychosocial supports as indicated.}\label{statement-10.7-we-suggest-health-care-professionals-refer-childrenindividuals-with-intersexuality-and-their-families-to-mental-health-professionals-as-well-as-peer-and-other-psychosocial-supports-as-indicated.}}
\addcontentsline{toc}{section}{Statement 10.7: We suggest health care professionals refer children/individuals with intersexuality and their families to mental health professionals as well as peer and other psychosocial supports as indicated.}

For almost all parents, the birth of a child with
intersexuality is entirely unexpected and comes as
a shock. Their inability to respond immediately
to the ubiquitous question, ``Is your baby a boy
or a girl?'', their lack of knowledge about the
child's condition, the uncertainty regarding the
child's future, and the pervasive intersex stigma
are likely to cause distress, sometimes to the level
of PTSD and may lead to prolonged anxiety and
depression (Pasterski et al., 2014; Roberts et al.,
2020; Wisniewski \& Sandberg, 2015). This situation may affect parental care and long-term outcome of their child with intersexuality (Schweizer
et al., 2017). As these children grow up, they are
also at risk of experiencing intersex stigma in its
three major forms (enacted, anticipated, internalized) in all spheres of life (Meyer-Bahlburg et al.,
2018), along with other potential difficulties such
as body image problems, gender-atypical behavior,
and gender identity questioning. Many may face
the additional challenge presented by the awareness of the incongruence between their assigned
gender and biological characteristics such as sexual
karyotype, gonads, past and/or current
sex-hormonal milieu, and reproductive tract configuration. This situation may also adversely affect
the individuals' mental health (Godfrey, 2021;
Meyer-Bahlburg, 2022). A recent online study of
a very large sample of LGBTQ youth indicated
that LGBTQ youth who categorized themselves as
having a physical intersex variation had a rate of
mental health problems that was higher than the
rate in LGBTQ youth without intersexuality
(Trevor Project, 2021). As intersex conditions are
rare, parents of such children and later the individuals themselves may experience their situation
as unique and very difficult for others to understand. Thus, based on clinical experience, there is
a consensus among HCPs who are experienced in
intersex care, that social support is a crucial component of intersex care, not only through professional support by MHPs (Pasterski et al., 2010),
but also, importantly, through support groups of
individuals with intersex conditions (Baratz et al.,
2014; Cull \& Simmonds, 2010; Hughes et al., 2006;
Lampalzer et al., 2021). A detailed international
listing of DSD and intersex peer support and
advocacy groups with their websites has been provided by Lee, Nordenström et al.~(2016). Given
the heterogeneity of intersex conditions and treatment regimens, an individual with intersexuality
may find it most helpful to associate with a support group that includes members with the same
or similar condition as that of the individual. It
is important HCPs specializing in intersex care
also collaborate closely with such support groups
so that occasional differences in opinions regarding
specific aspects of care can be resolved through
detailed discussions. Close contacts between HCPs
and support groups also facilitate community-based
participatory research that benefits both sides.

\hypertarget{statement-10.8-we-recommend-health-care-professionals-counsel-individuals-with-intersexuality-and-their-families-about-puberty-suppression-andor-hormonal-treatment-options-within-the-context-of-the-individuals-gender-identity-age-and-unique-medical-circumstances.}{%
\section*{Statement 10.8: We recommend health care professionals counsel individuals with intersexuality and their families about puberty suppression and/or hormonal treatment options within the context of the individual's gender identity, age, and unique medical circumstances.}\label{statement-10.8-we-recommend-health-care-professionals-counsel-individuals-with-intersexuality-and-their-families-about-puberty-suppression-andor-hormonal-treatment-options-within-the-context-of-the-individuals-gender-identity-age-and-unique-medical-circumstances.}}
\addcontentsline{toc}{section}{Statement 10.8: We recommend health care professionals counsel individuals with intersexuality and their families about puberty suppression and/or hormonal treatment options within the context of the individual's gender identity, age, and unique medical circumstances.}

While many people with intersexuality have a
gender identity in line with their XX or XY karyotype, there is sufficient heterogeneity that HCPs
should be able to provide customized approaches.
For example, among XX individuals with virilizing
CAH, a larger than expected minority have a male
gender identity (Dessens et al., 2005). Among XY
individuals with partial androgen insensitivity syndrome, gender identity can vary significantly (Babu
\& Shah, 2021). Furthermore, among XY individuals
with 5α-reductase-2 (5α-RD-2) deficiency and with
17-beta-hydroxysteroid dehydrogenase-3 deficiency
who are assigned the female sex at birth, a large
fraction (56--63\% and 39--64\%, respectively) change
from a typical female gender role to a typical male
gender role as they age (Cohen-Kettenis, 2005).

People with intersexuality have a wide range
of medical options open to them depending on
their gender identity and its alignment with anatomy. These options include puberty suppression
medication, hormonal treatment, and surgeries,
all customized to the unique circumstances of
the individual (Weinand \& Safer, 2015; Safer \&
Tangpricha, 2019) (for further information see
Chapter 6---Adolescents and Chapter 12---
Hormone Therapy). Specifically, when functional
gonads are present, puberty may be temporarily
suspended by using gonadotropin-releasing hormone (GnRH) analogues. Such intervention can
facilitate the necessary passage of time needed
by the individual to explore gender identity and
to actively participate in sex designation, especially for conditions in which sex role change is
common (i.e., in female-raised individuals with
5α-RD-2 deficiency; Cocchetti, Ristori, Mazzoli
et al., 2020; Fisher, Castellini et al., 2016).
HCPs can counsel individuals and their families directly if the providers have sufficient
expertise and can leverage expertise needed to
determine both a course of treatment appropriate
for the individual and the logistics involved in
implementing the chosen therapeutic option.

\hypertarget{statement-10.9-we-suggest-health-care-professionals-counsel-parents-and-children-with-intersexuality-when-cognitively-sufficiently-developed-to-delay-gender-affirming-genital-surgery-gonadal-surgery-or-both-so-as-to-optimize-the-childrens-self-determination-and-ability-to-participate-in-the-decision-based-on-informed-consent.}{%
\section*{Statement 10.9: We suggest health care professionals counsel parents and children with intersexuality (when cognitively sufficiently developed) to delay gender-affirming genital surgery, gonadal surgery, or both, so as to optimize the children's self-determination and ability to participate in the decision based on informed consent.}\label{statement-10.9-we-suggest-health-care-professionals-counsel-parents-and-children-with-intersexuality-when-cognitively-sufficiently-developed-to-delay-gender-affirming-genital-surgery-gonadal-surgery-or-both-so-as-to-optimize-the-childrens-self-determination-and-ability-to-participate-in-the-decision-based-on-informed-consent.}}
\addcontentsline{toc}{section}{Statement 10.9: We suggest health care professionals counsel parents and children with intersexuality (when cognitively sufficiently developed) to delay gender-affirming genital surgery, gonadal surgery, or both, so as to optimize the children's self-determination and ability to participate in the decision based on informed consent.}

International human rights organizations have
increasingly expressed their concerns that surgeries performed before a child can participate
meaningfully in decision-making may endanger
the child's human rights to autonomy,
self-determination, and an open future (e.g.,
Human Rights Watch, 2017). Numerous medical
and intersex advocacy organizations as well as
several countries have joined these international
human rights groups in recommending the delay
of surgery when medically feasible (Dalke et al.,
2020; National Academies of Sciences, Engineering,
and Medicine, 2020). However, it is important to
note some anatomic variations, such as obstruction of urinary flow or exposure of pelvic organs,
pose an imminent risk to physical health
(Mouriquand et al., 2016). Others, such as menstrual obstruction or long-term malignancy risk
in undescended testes, have eventual physical
consequences. A third group of variations, i.e.,
variations in the appearance of external genitals
or vaginal depth, pose no immediate or long-term
physical risk. The above recommendation
addresses only those anatomic variations that, if
left untreated, have no immediate adverse physical consequences and where delaying surgical
treatment poses no physical health risk.

Non-urgent surgical care for individuals with
these variations is complex and often contested,
particularly when an individual is an infant or a
young child and cannot yet participate in the
decision-making process. Older people with intersexuality have reported psychosocial and sexual
health problems, including depression, anxiety,
and sexual and social stigma (de Vries et al.,
2019; Rosenwohl-Mack et al., 2020). Some studies
have suggested individuals with a specific variation (e.g., 46,XX CAH) agree with surgery being
performed before adolescence (Bennecke et al.,
2021). Recent studies suggest some adolescents
and adults are satisfied with the appearance and
function of the genitals after childhood surgery
(Rapp et al., 2021). A child's genital difference
can also become a source of stress for parents,
and there is research that reports a correlation
of surgery to create binary genitals with a limited
amount of reduction in parental distress
(Wolfe-Christensen et al., 2017), although a
minority of parents may report decisional regret
(Ellens et al., 2017). Consequently, some organizations recommend surgery be offered to very
young children (American Urological Association,
2019; Pediatric Endocrine Society, 2020).

This shows the division within the medical
field regarding its management guidelines for
early genital surgery. The authors of this chapter
also did not reach complete consensus. Some
intersex specialists consider it potentially harmful
to insist on a universal deferral of early genital
surgery for genital variations without immediate
medical risks. Reasons supporting this view
include 1) intersex conditions are highly heterogeneous with respect to type and severity as well
as associated gonadal structure, function, and
malignancy risk; 2) societies and families vary
tremendously in gender norms and intersex
stigma potential; 3) early surgery may present
certain technical advantages; and 4) a review of
surveys of individuals with intersexuality (most
of whom had previously undergone genital surgery) show the majority endorse surgery before
the age of consent, especially in the case of individuals with 46,XX CAH and less strongly for
individuals with XY intersex conditions
(Meyer-Bahlburg, 2022). Experts supporting this
view call for an individualized approach to
decisions regarding genital surgery and its timing.
This approach has been adopted by medical societies with high rates of intersex specialists
(Bangalore Krishna et al., 2021; Pediatric
Endocrine Society, 2020; Speiser et al., 2018;
Stark et al., 2019) and by certain support organizations (CARES Foundation; Krege et al., 2019).

Nonetheless, long-term outcome studies are
limited and most studies reporting positive outcomes lack a non-surgical comparison group
(Dalke, et al., 2020; National Academies of
Sciences, Engineering, and Medicine, 2020). There
is also no evidence surgery protects children with
intersex conditions from stigma (Roen, 2019).
Adults with intersexuality do experience stigma,
depression, and anxiety related to their genitalia,
but can also experience stigma whether or not
they have surgery (Ediati et al., 2017;
Meyer-Bahlburg, Khuri et al., 2017; Meyer-Bahlburg
et al., 2018). There is also evidence surgeries may
lead to significant cosmetic, urinary, and sexual
complications extending into adulthood (Gong \&
Cheng, 2017; National Academies of Sciences,
Engineering, and Medicine, 2020). Recent studies
suggest some groups of individuals may have particularly negative experiences with gonadectomy,
although this risk has to be weighed against that
of gonadal malignancy (Duranteau et al., 2020;
Rapp et al., 2021). People with intersex conditions
are also far more likely than the general population to be transgender, to be gender diverse, or
to have gender dysphoria (Almasri et al., 2018;
Pasterski et al., 2015). Genital surgeries of young
children may therefore irreversibly reinforce a
binary sex assignment that is not aligned with
the persons' future. These findings, together with
human rights perspectives, support the call for
the delay in the decision for surgery until the
individual can decide for him/her/themselves.

Systematic long-term follow-up studies are urgently
needed to compare individuals with the same intersex conditions who differ in the age at surgery or
have had no surgery with regard to gender identity,
mental health, and general quality of life.

\hypertarget{statement-10.10-we-suggest-only-surgeons-experienced-in-intersex-genital-or-gonadal-surgery-operate-on-individuals-with-intersexuality.}{%
\section*{Statement 10.10: We suggest only surgeons experienced in intersex genital or gonadal surgery operate on individuals with intersexuality.}\label{statement-10.10-we-suggest-only-surgeons-experienced-in-intersex-genital-or-gonadal-surgery-operate-on-individuals-with-intersexuality.}}
\addcontentsline{toc}{section}{Statement 10.10: We suggest only surgeons experienced in intersex genital or gonadal surgery operate on individuals with intersexuality.}

Intersex conditions are rare, and intersex genital and gonadal anatomy are heterogeneous.
Surgeries have been associated with a risk of significant long-term complications (e.g., National
Academies of Sciences, Engineering, and Medicine,
2020), and most surgical training programs do
not prepare trainees to provide this specialized
care (Grimstad, Kremen et al., 2021). In recognition of the complexity of surgical care across
the lifespan, standards produced by expert and
international consensus recommend this care be
provided by multidisciplinary teams of experts
(Krege et al, 2019; Lee, Nordenström et al., 2016;
Pediatric Endocrine Society, 2020). Therefore, we
advise surgical care be limited to
intersex-specialized, multidisciplinary settings that
include surgeons experienced in intersex care.

\hypertarget{statement-10.11-we-recommend-health-care-professionals-who-are-prescribing-or-referring-for-hormonal-therapiessurgeries-counsel-individuals-with-intersexuality-and-fertility-potential-and-their-families-about-a-known-effects-of-hormonal-therapiessurgery-on-future-fertility-b-potential-effects-of-therapies-that-are-not-well-studied-and-are-of-unknown-reversibility-c-fertility-preservation-options-and-d-psychosocial-implications-of-infertility.}{%
\section*{Statement 10.11: We recommend health care professionals who are prescribing or referring for hormonal therapies/surgeries counsel individuals with intersexuality and fertility potential and their families about a) known effects of hormonal therapies/surgery on future fertility; b) potential effects of therapies that are not well studied and are of unknown reversibility; c) fertility preservation options; and d) psychosocial implications of infertility.}\label{statement-10.11-we-recommend-health-care-professionals-who-are-prescribing-or-referring-for-hormonal-therapiessurgeries-counsel-individuals-with-intersexuality-and-fertility-potential-and-their-families-about-a-known-effects-of-hormonal-therapiessurgery-on-future-fertility-b-potential-effects-of-therapies-that-are-not-well-studied-and-are-of-unknown-reversibility-c-fertility-preservation-options-and-d-psychosocial-implications-of-infertility.}}
\addcontentsline{toc}{section}{Statement 10.11: We recommend health care professionals who are prescribing or referring for hormonal therapies/surgeries counsel individuals with intersexuality and fertility potential and their families about a) known effects of hormonal therapies/surgery on future fertility; b) potential effects of therapies that are not well studied and are of unknown reversibility; c) fertility preservation options; and d) psychosocial implications of infertility.}

Individuals with certain intersex conditions
may have reproductively functional genitalia but
experience infertility due to atypical gonadal
development. Others may have functioning
gonads with viable germ cells but an inability to
achieve natural fertility secondary to incongruent
internal or external genitalia (van Batavia \&
Kolon, 2016). Pubertal suppression, hormonal
treatment with sex steroid hormones, and gender
affirming surgeries may all have an adverse
impact on future fertility. The potential consequences of the treatment and fertility preservation
options should therefore be reviewed and
discussed.

Individuals with functioning testes should be
advised prolonged treatment with estrogen and
suppression of testosterone, as studied in TGD
people without intersexuality, may cause testicular
atrophy and a reduction in sperm count
(Mattawanon et al., 2018). Although interruption
of such gender affirming hormonal treatment
may improve sperm quality, a complete reversal
of semen impairment cannot be guaranteed
(Sermondade et al., 2021). The principal fertility
preservation option for individuals with functioning testes is cryopreservation of sperm collected
through masturbation or vibratory stimulation
(de Roo et al., 2016). Although there are no data
for success in humans, there is a proposal to
offer direct testicular extraction and cryopreservation of immature testicular tissue to adolescents
who have not yet undergone spermarche
(Mattawanon et al., 2018).

Individuals with functioning ovaries should be
advised testosterone therapy usually results in cessation of both menses and ovulation, often within
a few months of initiating therapy. There are major
gaps in knowledge regarding the potential effects
of testosterone on oocytes and subsequent fertility.
In transgender people, one study reported testosterone treatment may be associated with the development of polycystic ovarian morphology (Grynberg
et al., 2010). However, other researchers have not
found evidence of polycystic ovarian syndrome
(PCOS) among transgender men receiving gender
affirming hormone therapy based on metabolic
(Chan et al., 2018) or histologic parameters (de
Roo et al., 2017). Individuals with an intact uterus
and functioning ovaries may regain their fertility
potential if testosterone therapy is discontinued.

Fertility preservation options in post-pubertal
people with intersexuality and functioning ovaries
include hormonal stimulation for mature oocyte
cryopreservation or ovarian tissue cryopreservation. Alternatively, stimulated oocyte extraction
has been reported even for a transgender man
continuing testosterone therapy (Greenwald,
2021). Similarly, oocyte cryopreservation after
ovarian stimulation has been reported in a transgender boy receiving GnRHa therapy (Rothenberg
et al., 2019). It should be noted ovarian stimulation, temporary cessation of GnRHa, testosterone treatment, or both, as well as gynecological
procedures, can all be psychologically distressing
to individuals, with the stress reaction being
influenced by mental health, gender identity, and
other medical experience. Applicability of certain
interventions may depend on the support of other
people in the individual's social network, including potential partners.

\hypertarget{statement-10.12-we-suggest-health-care-professionals-caring-for-individuals-with-intersexuality-and-congenital-infertility-introduce-them-and-their-families-early-and-gradually-to-the-various-alternative-options-of-parenthood.}{%
\section*{Statement 10.12: We suggest health care professionals caring for individuals with intersexuality and congenital infertility introduce them and their families, early and gradually, to the various alternative options of parenthood.}\label{statement-10.12-we-suggest-health-care-professionals-caring-for-individuals-with-intersexuality-and-congenital-infertility-introduce-them-and-their-families-early-and-gradually-to-the-various-alternative-options-of-parenthood.}}
\addcontentsline{toc}{section}{Statement 10.12: We suggest health care professionals caring for individuals with intersexuality and congenital infertility introduce them and their families, early and gradually, to the various alternative options of parenthood.}

For people with intersex characteristics, the
likelihood of infertility may be recognized in
infancy, childhood, adolescence as well as in
adulthood, without first engaging in attempts
to conceive. For many individuals, a diagnosis
of infertility accompanies the intersex diagnosis
(Jones, 2019). For some individuals, assisted
heterologous fertilization (e.g., oocyte or sperm
donation) may be an option. Multiple adoption
pathways exist. Some may require commitment
and a considerable investment of time.
Individuals who are either not interested in
engaging in the efforts to achieve fertility previously described or for whom fertility is not
possible can benefit from early exposure to the
options available for adoption and alternative
parenthood. While uterus transplantation has
had preliminary success in people with
Mullerian agenesis (Richards et al., 2021), there
is no protocol to date that avoids exposure of
the developing fetus to the risks associated with
the medications used to avoid transplant
rejection.

\hypertarget{institutional-environments}{%
\chapter{Institutional Environments}\label{institutional-environments}}

This chapter addresses care for transgender and
gender diverse (TGD) individuals who reside in
institutions. By definition, institutions are facilities or establishments in which people live and
receive care in a congregate or large group setting, where individuals may or may not have
freedom of movement, individual consent, or
agency. Carceral facilities (correctional facilities,
immigration detention centers, jails, juvenile
detention centers) and noncarceral facilities
(long-term care facilities, in-patient psychiatric
facilities, domiciliaries, hospice/palliative care,
assisted living facilities) are residential institutions
where health care access for transgender persons
may be provided. Much of the evidence in support of proper care of TGD persons comes from
carceral settings. However, the recommendations
put forth here apply to all institutions that house
TGD individuals, both carceral and noncarceral
(Porter et al., 2016). All of the recommendations
of the Standards of Care apply equally to people
living in both types of institutions. People should
have access to these medically necessary treatments irrespective of their housing situation
within an institution (Brown, 2009). Care for an
institutionalized person must consider the individual does not have the access that
non-institutionalized persons have to securing
care on their own. For that reason,
institutionalized persons must be supported in
being able to receive the Standards of Care established by the World Professional Association for
Transgender Health (WPATH).

TGD residents in carceral facilities report the
lack of access to medically necessary
transgender-specific health care (see Chapter 2---
Global Applicability, Statement 2.1), which is
ranked as their number one concern while incarcerated (Brown, 2014; Emmer et al., 2011). The
systemic racial inequities inherent in many
carceral environments (Sawyer, 2020), racial disparities in health outcomes (Nowotny et al.,
2017), and the overrepresentation of TGD people
of color in some facilities (Reisner et al., 2014)
punctuate a need for facility leadership to attend
to transitional care access issues. Controlled studies show clinically significant health and mental
health disparities for justice-involved transgender
people compared to matched groups of transgender people who have not been incarcerated or
jailed (Brown and Jones, 2015). Too often the
agencies, structures, and personnel that provide
care are lacking in knowledge, training, and
capacity to care for gender diverse people (Clark
et al., 2017). Discrimination against TGD residents in palliative care settings, including hospice,
is common, and the needs of TGD patients or
their surrogates have been ignored in these settings (Stein et al., 2020). This is one reason why
lesbian, gay, bisexual and transgender (LGBT)
patients may choose to hide their sexual and/or
gender identity when they enter a nursing home,
despite the fact that prior to their admission to
the facility they had been living publicly as a
LGBT-identified person (Carroll, 2017; Serafin
et al., 2013).

All the statements in this chapter have been
recommended based on a thorough review of
evidence, an assessment of the benefits and
harms, values and preferences of providers and
patients, and resource use and feasibility. In some
cases, we recognize evidence is limited and/or
services may not be accessible or desirable. The
majority of the available literature related to institutions focuses on those who are incarcerated in
jails, prisons, or other carceral environments.
Literature about other institutional types were
also considered and referenced where available.
We hope future investigations will address this
relative lack of data from noncarceral institutions.
The recommendations summarized above are
generalizable to a variety of institutional settings
that have characteristics in common, including
extended periods of stay, loss of or limited agency,
and reliance on institutional staff for some or all
of the basic necessities of life.

\hypertarget{statement-11.1-we-recommend-health-care-professionals-responsible-for-providing-gender-affirming-care-to-individuals-residing-in-institutions-or-associated-with-institutions-or-agencies-recognize-the-entire-list-of-recommendations-of-the-soc-8-apply-equally-to-tgd-people-living-in-institutions.}{%
\section*{Statement 11.1: We recommend health care professionals responsible for providing gender-affirming care to individuals residing in institutions (or associated with institutions or agencies) recognize the entire list of recommendations of the SOC-8, apply equally to TGD people living in institutions.}\label{statement-11.1-we-recommend-health-care-professionals-responsible-for-providing-gender-affirming-care-to-individuals-residing-in-institutions-or-associated-with-institutions-or-agencies-recognize-the-entire-list-of-recommendations-of-the-soc-8-apply-equally-to-tgd-people-living-in-institutions.}}
\addcontentsline{toc}{section}{Statement 11.1: We recommend health care professionals responsible for providing gender-affirming care to individuals residing in institutions (or associated with institutions or agencies) recognize the entire list of recommendations of the SOC-8, apply equally to TGD people living in institutions.}

Just as people living in institutions require and
deserve mental and medical health care in general
and in specialty areas, we recognize TGD people
are in these institutions and thus need care specific to TGD concerns. We recommend the application of the Standards of Care (SOC) to people
living in institutions as basic principles of health
care and ethics (Beauchamp \& Childress, 2019;
Pope \& Vasquez, 2016). Additionally, numerous
courts have long upheld the need to provide
TGD-informed care based in the WPATH SOC
to people living in institutions as well (e.g.,
Koselik v. Massachusetts, 2002; Edmo v. Idaho
Department of Corrections, 2020). Agencies that
provide staffing for long-term, in-home services
should also be aware of the applicability of the
Standards of Care.

\hypertarget{statement-11.2-we-suggest-institutions-provide-all-staff-with-training-on-gender-diversity.}{%
\section*{Statement 11.2: We suggest institutions provide all staff with training on gender diversity.}\label{statement-11.2-we-suggest-institutions-provide-all-staff-with-training-on-gender-diversity.}}
\addcontentsline{toc}{section}{Statement 11.2: We suggest institutions provide all staff with training on gender diversity.}

Because TGD care affects a small percentage
of the population, it requires specialized training
as outlined in this SOC Version 8. While the
level of training will vary based on the staff
member's role within the institutional setting, all
staff will need training in addressing residents
appropriately while other clinical staff may need
more intensive training and/or consultation.
These training recommendations also apply to
agencies that supply staffing for in-home,
long-term care. Misgendering institutionalized
residents, not allowing for gender appropriate
clothing, shower facilities, or housing, and not
using chosen names communicates a lack of
respect for TGD residents who may experience
repeated indignities as emotionally traumatic,
depressing, and anxiety-producing. By providing
all institutional staff with training on gender
diversity and b asic comp etence in
transgender-related health care issues, these
harms can be prevented (Hafford-Letchfield
et al., 2017). Surveys indicate individuals working
with incarcerated individuals as well as in workers in noncarceral settings like palliative care
have significant knowledge gaps (Stein et al.,
2020; White et al., 2016). Hafford-Letchfied et al.
(2017) showed benefit to training residential
long-term care staff when such training began
with ``recognizing LBGT issues'' and existed in
``care homes''. If the assigned health care providers lack the expertise to assess and/or treat gender diverse persons under their charge, outside
consultation should be sought from professionals
with expertise in the provision of gender-affirming
health care (Brömdal et al., 2019; Sevelius and
Jenness, 2017).

\hypertarget{statement-11.3-we-recommend-medical-professionals-charged-with-prescribing-and-monitoring-hormones-for-tgd-individuals-living-in-institutions-who-need-gender-affirming-hormone-therapy-do-so-without-undue-delay-and-in-accordance-with-the-soc-8.}{%
\section*{Statement 11.3: We recommend medical professionals charged with prescribing and monitoring hormones for TGD individuals living in institutions who need gender-affirming hormone therapy do so without undue delay and in accordance with the SOC-8.}\label{statement-11.3-we-recommend-medical-professionals-charged-with-prescribing-and-monitoring-hormones-for-tgd-individuals-living-in-institutions-who-need-gender-affirming-hormone-therapy-do-so-without-undue-delay-and-in-accordance-with-the-soc-8.}}
\addcontentsline{toc}{section}{Statement 11.3: We recommend medical professionals charged with prescribing and monitoring hormones for TGD individuals living in institutions who need gender-affirming hormone therapy do so without undue delay and in accordance with the SOC-8.}

TGD persons may be admitted to institutions
in need of evaluation for gender-affirming hormonal care or may develop this need after they
have resided in an institutional setting for varying
degrees of time. It is not uncommon for TGD
persons to be denied access to hormonal care for
months or years after making such needs known
or to be undertreated and poorly monitored,
delaying the necessary titration of medications
for safety and efficacy (Keohane, 2018; Kosilek v.
Massachusetts, 2002; Monroe v. Baldwin et al.,
2019). This can result in significant negative mental health outcomes to include depression, anxiety,
suicidality, and surgical self-treatment risks
(Brown, 2010). As with all medically necessary
health care, access to gender-affirming hormone
therapies should be provided in a timely fashion
when indicated for a TGD resident, in both
carceral and noncarceral institutional environments. Medical professionals shall appropriately
titrate hormones based on laboratory results and
clinical outcomes to ensure results are within the
range of recommended standards within the field
of endocrinology. Such labs shall be taken at a
frequency so as not to delay appropriate titration.
TGD elderly people living in long-term care
facilities have unique needs (Boyd, 2019; Caroll,
2017; Porter, 2016). When elderly individuals
request hormonal treatment, while physicians
should assess pre-existing conditions, rarely do
such conditions absolutely contraindicate administering hormones in this population (Ettner,
2013). People with gender incongruence in institutions may also have coexisting mental health
conditions (Brown and Jones, 2015; Cole et al.,
1997). These conditions should be evaluated and
treated appropriately as part of the overall assessment. Persons receiving hormones must be closely
medically monitored to avoid potential drug
interactions and polypharmacy (Hembree
et al., 2017).

TGD persons who enter an institution on an
appropriate regimen of gender-affirming hormone
therapy should be continued on the same or similar therapies and monitored according to the
SOC Version 8. A ``freeze frame'' approach is
inappropriate and dangerous (Kosilek v.
Massachusetts, 2002). A ``freeze frame'' approach
is the outmoded practice of denying hormones
to people who are not already on them or keeping TGD persons on the same dose of hormones
throughout their institutionalization that they
were receiving upon admission, even if that dose
was an initiation (low) dose. TGD persons who
are deemed appropriate for de novo
gender-affirming hormone therapy should be
started on such therapy just as they would be
outside of an institution (Adams v. Federal Bureau
of Prisons, No.~09-10272 {[}D. MO June 7, 2010{]};
Brown 2009). The consequences of abrupt withdrawal of hormones or lack of initiation of hormone therapy when medically necessary include
a significant likelihood of negative outcomes
(Brown, 2010; Sundstrom and Fields v. Frank,
2011), such as surgical self-treatment by autocastration, depressed mood, increased gender dysphoria, and/or suicidality (Brown, 2010;
Maruri, 2011).

If an individual in an institution does receive
gender-affirming hormones and/or surgeries,
decisions regarding housing in sex-segregated
facilities may need to be reassessed for the safety
and well-being of the TGD person (Ministry of
Justice {[}UK{]}, 2016).

\hypertarget{statement-11.4-we-recommend-staff-and-professionals-charged-with-providing-health-care-to-tgd-individuals-living-in-institutions-recommend-and-support-gender-affirming-surgical-treatments-in-accordance-with-soc-8-when-sought-by-the-individual-without-undue-delay.}{%
\section*{Statement 11.4: We recommend staff and professionals charged with providing health care to TGD individuals living in institutions recommend and support gender-affirming surgical treatments in accordance with SOC-8, when sought by the individual, without undue delay.}\label{statement-11.4-we-recommend-staff-and-professionals-charged-with-providing-health-care-to-tgd-individuals-living-in-institutions-recommend-and-support-gender-affirming-surgical-treatments-in-accordance-with-soc-8-when-sought-by-the-individual-without-undue-delay.}}
\addcontentsline{toc}{section}{Statement 11.4: We recommend staff and professionals charged with providing health care to TGD individuals living in institutions recommend and support gender-affirming surgical treatments in accordance with SOC-8, when sought by the individual, without undue delay.}

TGD people with gender dysphoria should
have an appropriate treatment plan to provide
medically necessary surgical treatments that contain similar elements provided to persons who
reside outside institutions (Adams v. Federal
Bureau of Prisons, No.~09-10272 {[}D. MO June 7,
2010{]}; Brown 2009; Edmo v. Idaho Department
of Corrections, 2020). The consequences of denial
or lack of access to gender- affirming surgeries
for residents of institutions who cannot access
such care outside of their institutions may be
serious, including substantial worsening of gender
dysphoria symptoms, depression, anxiety, suicidality, and the possibility of surgical self-treatment
(e.g., autocastration or autopenectomy; Brown,
2010; Edmo v. Idaho Department of Corrections,
2020; Maruri, 2011). It is not uncommon for
residents of institutions to be denied access to
evaluation for gender-affirming surgery as well
as denial of the treatment itself, even when medically necessary (Kosilek v. Massachusetts/
Dennehy, 2012; Edmo v. Idaho Department of
Corrections, 2020). The denial of medically necessary evaluations for and the provision of
gender-affirming surgical treatments and necessary aftercare is inappropriate and inconsistent
with these Standards of Care.

\hypertarget{statement-11.5-we-recommend-administrators-health-care-professionals-and-all-others-working-in-institutions-charged-with-the-responsibility-of-caring-for-tgd-individuals-allow-those-individuals-who-request-appropriate-clothing-and-grooming-items-to-obtain-such-items-concordant-with-their-gender-expression.}{%
\section*{Statement 11.5: We recommend administrators, health care professionals, and all others working in institutions charged with the responsibility of caring for TGD individuals allow those individuals who request appropriate clothing and grooming items to obtain such items concordant with their gender expression.}\label{statement-11.5-we-recommend-administrators-health-care-professionals-and-all-others-working-in-institutions-charged-with-the-responsibility-of-caring-for-tgd-individuals-allow-those-individuals-who-request-appropriate-clothing-and-grooming-items-to-obtain-such-items-concordant-with-their-gender-expression.}}
\addcontentsline{toc}{section}{Statement 11.5: We recommend administrators, health care professionals, and all others working in institutions charged with the responsibility of caring for TGD individuals allow those individuals who request appropriate clothing and grooming items to obtain such items concordant with their gender expression.}

Gender expression refers to people having
hairstyles, grooming products, clothing, names,
and pronouns associated with their gender identity in their culture and/or community (American
Psychological Association, 2015; Hembree et al.,
2017). Gender expression is the norm among
most people within a culture or a community.
Social transition is the process of TGD persons
beginning and continuing to express their gender identity in ways that are authentic and
socially perceptible. Often, social transition
involves behavior and public presentation differing from what is usually expected for people
assigned a given legal gender marker at birth.
A gender marker is the legal label for a person's
sex that is typically assigned or designated at
birth on official documents (American
Psychological Association, 2015). This is most
commonly recorded as male or female but also
intersex or ``X'' in some nations and jurisdictions. TGD individuals need the same rights to
gender expression afforded cisgender people
living both outside and inside institutional settings. Staff acceptance of social transition also
sets a tone of respect and affirmation that may
enhance respect and affirmation with others
residing in the institution, thereby increasing
safety and reducing some aspects of gender
incongruence.

Research indicates social transition and congruent gender expression have a significant beneficial effect on the mental health of TGD people
(Bockting \& Coleman, 2007; Boedecker, 2018;
Devor, 2004; Glynn et al., 2016; Russell et al.,
2018). To allow for expressing gender identity,
these recommendations include being allowed to
wear gender congruent clothing and hairstyles,
to obtain and use gender-appropriate hygiene and
grooming products, to be addressed by a chosen
name or legal last name (even if unable to change
the assigned name legally yet), and to be
addressed by a pronoun consistent with one's
identity. These elements of gender expression and
social transition, individually or collectively as
indicated by the individual's needs, reduce gender
dysphoria/incongruence, depression, anxiety,
self-harm ideation and behavior, suicidal ideation
and attempts (Russell et al., 2018). Furthermore,
these elements of congruent gender expression
enhance well-being and functioning (Glynn
et al., 2016).

\hypertarget{statement-11.6-we-recommend-all-institutional-staff-address-tgd-individuals-by-their-chosen-names-and-pronouns-at-all-times}{%
\section*{Statement 11.6: We recommend all institutional staff address TGD individuals by their chosen names and pronouns at all times}\label{statement-11.6-we-recommend-all-institutional-staff-address-tgd-individuals-by-their-chosen-names-and-pronouns-at-all-times}}
\addcontentsline{toc}{section}{Statement 11.6: We recommend all institutional staff address TGD individuals by their chosen names and pronouns at all times}

Given that an increasing percentage of people
openly identify as gender diverse, there is a need
to develop and implement practices and policies
that meet the needs of these people irrespective
of where they live (McCauley et al., 2017). For
example, institutions should utilize medical and
administrative records systems for their residents
that track gender markers consistent with gender
identity and not solely sex assigned at birth. In
developing these recommendations, there was
recognition that gender expansiveness can challenge some institutional norms where TGD people live. However, all institutions have the
responsibility to provide for the safety and
well-being of all persons living therein (Australia,
2015; Corrective Services New South Wales, 2015;
Edmo v. Idaho Department of Corrections, 2020;
Kosilek v. Massachusetts, 2002; NCCHC, 2015).
Sevelius and colleagues (2020) demonstrated correct pronoun usage is gender-affirming for
transgender women and correlates with positive
mental health and HIV-related health outcomes.
If a resident of an institution has legally changed
names, the institutional records should be changed
to reflect those changes.

\hypertarget{statement-11.7-we-recommend-institutional-administrators-health-care-professionals-and-other-officials-responsible-for-making-housing-decisions-for-tgd-residents-consider-the-individuals-housing-preference-gender-identity-and-expression-and-safety-considerations-rather-than-solely-their-anatomy-or-sex-assignment-at-birth.}{%
\section*{Statement 11.7: We recommend institutional administrators, health care professionals, and other officials responsible for making housing decisions for TGD residents consider the individual's housing preference, gender identity and expression, and safety considerations, rather than solely their anatomy or sex assignment at birth.}\label{statement-11.7-we-recommend-institutional-administrators-health-care-professionals-and-other-officials-responsible-for-making-housing-decisions-for-tgd-residents-consider-the-individuals-housing-preference-gender-identity-and-expression-and-safety-considerations-rather-than-solely-their-anatomy-or-sex-assignment-at-birth.}}
\addcontentsline{toc}{section}{Statement 11.7: We recommend institutional administrators, health care professionals, and other officials responsible for making housing decisions for TGD residents consider the individual's housing preference, gender identity and expression, and safety considerations, rather than solely their anatomy or sex assignment at birth.}

The separation of people based on sex assigned
at birth, a policy almost universally implemented
in institutional settings (Brown and McDuffie,
2009; Routh et al., 2017), can create an inherently
dangerous environment (Ledesma \& Ford, 2020).
Gender diverse people are extremely vulnerable to
stigmatization, victimization, neglect, violence, and
sexual abuse (Banbury, 2004; Beck, 2014; Jenness
and Fenstermaker, 2016; Malkin \& DeJong, 2018;
Oparah, 2012; Stein et al., 2020). This systemic
sex-segregated rigidity often fails to keep TGD
people safe and may impede access to
gender-affirming health care (Stohr, 2015). As a
result, institutions should follow procedures that
routinely evaluate the housing needs and preferences of TGD inmates (e.g., Federal Bureau of
Prisons, 2016). Likewise, the Prison Rape
Elimination Act specifically cites TGD individuals
as a vulnerable population and directs prisons
nationwide in the US to consider the housing preferences of these inmates (Bureau of Justice
Assistance, 2017).

\hypertarget{statement-11.8-we-recommend-institutional-personnel-establish-housing-policies-that-ensure-the-safety-of-transgender-and-gender-diverse-residents-without-segregating-or-isolating-these-individuals.}{%
\section*{Statement 11.8: We recommend institutional personnel establish housing policies that ensure the safety of transgender and gender diverse residents without segregating or isolating these individuals.}\label{statement-11.8-we-recommend-institutional-personnel-establish-housing-policies-that-ensure-the-safety-of-transgender-and-gender-diverse-residents-without-segregating-or-isolating-these-individuals.}}
\addcontentsline{toc}{section}{Statement 11.8: We recommend institutional personnel establish housing policies that ensure the safety of transgender and gender diverse residents without segregating or isolating these individuals.}

Assigning placement for a TGD resident solely
on the basis of their genital anatomy or sex
assigned at birth is misguided and places people
at risk for physical and/or psychological harm
(Scott, 2013; Simopoulos \& Khin, 2014; Yona \&
Katri, 2020). It is well established within carceral
settings, transgender individuals are far more
likely than other prisoners to be sexually harassed,
assaulted, or both (James et al., 2016; Jenness \&
Fenstermaker, 2016; Malkin \& DeJong, 2019).
While placement decisions need to address security concerns, shared decision-making that
includes the input of the individual should be
made on a case-by-case basis (Federal Bureau of
Prisons, 2016; Jenness and Smyth, 2011). Some
transgender women prefer to reside in a male
facility while others feel safer in a female facility.
Given the range of gender identities, expression
and transition status is so heterogeneous among
gender diverse people, keeping residents safe
requires flexible decision-making processes (Yona
\& Katri, 2020). One of the fears older LBGT
individuals have living in long-term care is mistreatment by roommates (Jablonski et al., 2013).
Consequently, housing in nursing homes and
assisted living facilities should consider assigning
rooms to elders based on their self-identified
gender without regard to birth assignment or
surgical history and in collaboration with the
TGD patient.

Solitary confinement, sometimes referred to as
administrative segregation in carceral facilities,
refers to physical isolation of individuals during
which they are confined in their cells for approximately twenty-three hours each day. The use of
isolation is employed in some carceral facilities
as a disciplinary measure as well as a means of
protecting prisoners who are considered a risk
to themselves or others or who are at risk of
sexual assault by other inmates. However, isolating prisoners for safety concerns, if necessary,
should be brief, as isolation can cause severe
psychological harm and gross disturbances of
functioning (Ahalt et al., 2017; Scharff Smith,
2006). National prison standards organizations as
well as The United Nations consider isolation
longer than 15 days to be torture (NCCHC, 2016;
United Nations, 2015).

\hypertarget{statement-11.9-we-recommend-institutional-personnel-allow-transgender-and-gender-diverse-residents-the-private-use-of-shower-and-toilet-facilities-upon-request.}{%
\section*{Statement 11.9: We recommend institutional personnel allow transgender and gender diverse residents the private use of shower and toilet facilities, upon request.}\label{statement-11.9-we-recommend-institutional-personnel-allow-transgender-and-gender-diverse-residents-the-private-use-of-shower-and-toilet-facilities-upon-request.}}
\addcontentsline{toc}{section}{Statement 11.9: We recommend institutional personnel allow transgender and gender diverse residents the private use of shower and toilet facilities, upon request.}

The necessity and importance of privacy is
universal irrespective of gender identity. TGD
individuals report avoiding public restrooms,
limiting the amount they eat and drink so as
not to have to use a public facility, often leading to urinary tract infections and kidney-related
problems (James et al., 2016). TGD individuals
in institutions are often deprived of privacy in
bathroom and shower use, which can result in
psychological harm and/or physical and sexual
abuse (Bartels and Lynch, 2017; Brown, 2014;
Cook-Daniels, 2016; Mann, 2006). Similarly, in
carceral environments, pat downs, strip searches
and body cavity searches should be conducted
by staff members of the same sex with the
understanding this may not be possible in
extreme emergencies. The incidental viewing
of searches by other employees should be
avoided (Bureau of Justice Assistance, 2017).
Private use of shower and toilet facilities for
incarcerated transgender people is also required
by some laws, including for instance the United
States' federal Prison Rape Elimination Act
in the US.

The population of aging/older TGD persons
who need to be served by institutions is increasing (Carroll, 2017; Witten \& Eyler, 2016). Many
long-term care and other facilities catering to the
needs of the aging need to take into consideration the needs of their non-cisgender residents
(Ettner, 2016; Ettner \& Wiley, 2016). Surveys of
HCPs working with elders in hospice and palliative care settings as well as other long-term care
facilities report patients who identify as TGD
often do not get their basic needs met, are discriminated against in their medical care access,
or are physically and/or emotionally abused (Stein
et al., 2020) A survey of retirement and residential care providers in Australia found little experience with or understanding of the issues facing
this population. Indeed, many elderly TGD residents admitted to concealing their gender identity, bowing to the fear of insensitive treatment
or frank discrimination (Cartwright et al., 2012;
Cook-Daniels, 2016; Grant et al., 2012; Horner
et al., 2012; Orel \& Fruhauf, 2015).

\hypertarget{hormone-therapy}{%
\chapter{Hormone Therapy}\label{hormone-therapy}}

Transgender and gender diverse (TGD) persons
may require medically necessary gender-affirming
hormone therapy (GAHT) to achieve changes
consistent with their embodiment goals, gender
identity, or both (see medically necessary statement in Chapter 2---Global Applicability,
Statement 2.1). This chapter describes hormone
therapy recommendations for TGD adults and
adolescents. Please refer to Chapter 5---
Assessment of Adults and Chapter 6---Adolescents
for the assessment criteria related to initiation
of hormone therapy for adults and adolescents,
respectively. A summary of the recommendations and assessment criteria can be found in
Appendix D.

Ever since the first World Professional
Association for Transgender Health (WPATH)
Standards of Care (SOC) was published in 1979
and in subsequent updates of the SOC, including
SOC version 7, GAHT has been accepted as medically necessary (Coleman et al., 2012). WPATH
endorsed the Endocrine Society's guidelines for
GAHT for TGD persons in 2009 and 2017
(Hembree et al., 2009; Hembree et al., 2017). The
European Society for Sexual Medicine has also
published a position statement on hormone management in adolescent and adult TGD people
(T'Sjoen et al., 2020). When provided under medical supervision, GAHT in adults is safe
(Tangpricha \& den Heijer, 2017; Safer \&
Tangpricha, 2019). However, there are some
potential long-term risks, and careful monitoring
and screening are required to reduce adverse
events (Hembree et al., 2017; Rosenthal, 2021).

In general, the goal is to target serum levels
of the sex steroids to match the levels associated
with the individual's gender identity, although
optimal target ranges have not been established
(Hembree et al., 2017). Health care professionals
(HCPs) can use serum testosterone and/or estradiol levels to monitor most sex steroid treatments.
However, conjugated estrogens or synthetic estrogen use cannot be monitored. The assumption
that the estrone/estradiol ratio should be monitored was not supported in a recent cohort study
as there was no relationship between estrone
concentration and change in body fat or breast
development seen in a European cohort of 212
adult transgender women during a 1-year
follow-up of hormone treatment (Tebbens et al.,
2021). This study demonstrated higher estrone
concentrations or higher estrone/estradiol ratios
are not associated with antagonistic effects on
feminization (fat percentage and breast development) (Tebbens et al., 2021). Thus, monitoring
of the estrone to estradiol ratio is not supported
by the current published evidence. Previously
used conjugated estrogens have been abandoned
in favor of bioidentical estrogens. Even if several
studies have shown a significantly greater risk of
thromboembolic and cardiovascular complications
with the use of oral conjugated estrogens compared with oral estradiol in postmenopausal
women, no randomized controlled trials have
taken place, either in postmenopausal women or
in transgender people undergoing estrogen treatment (Smith et al., 2014).

The approach to GAHT differs and depends on
the developmental stage of the individual at the
time of initiation of hormone therapy as well as
their treatment goals. Hormone therapy is not recommended for children who have not begun
endogenous puberty. In eligible youth (as per
Chapter 6---Adolescents) who have reached the
early stages of puberty, the focus is usually to delay
further pubertal progression with gonadotropin
releasing hormone agonists (GnRHas) until an
appropriate time when GAHT can be introduced.
In these cases, pubertal suppression is considered
medically necessary. Eligible adults may initiate
GAHT if they fulfill the criteria as per Chapter
5---Assessment for Adults. In addition, health care
providers should discuss fertility goals and fertility
preservation procedures prior to initiating GAHT.
See Chapter 16---Reproductive Health.

GAHT with feminine embodiment goals typically consists of estrogen and an androgen-lowering
medication (Hembree et al., 2017). Although
there are anecdotal reports of progesterone use
for breast development and mood management,
there is currently insufficient evidence the potential benefits of progesterone administration outweigh the potential risks (Iwamoto, T'Sjoen et al.,
2019). Masculinizing GAHT typically consists of
testosterone. Both WPATH and the Endocrine
Society recommend monitoring levels of sex
hormones. While GAHT is customized to meet
the individual needs of the TGD person, typically
hormone levels are maintained at a concentration
sufficient to support good bone health and are
not supraphysiologic (Hembree et al., 2017; Rosen
et al., 2019).

In most cases, GAHT is maintained throughout
life. It is not known if doses of GAHT should be
reduced in older TGD people. Discontinuation of
hormone therapy may result in bone loss in TGD
individuals and will definitely do so in individuals
whose gonads have been removed (Wiepjes et al.,
2020). Routine primary care should also be performed (see Chapter 15---Primary Care).
Epidemiology studies have reported an increased
incidence of cardiovascular disease and venous
thromboembolism (VTE) in TGD people receiving
estrogen, most notably in older people and with
different preparations of GAHT (Irwig, 2018;
Maraka et al., 2017). TGD individuals treated with
testosterone may also have increased adverse cardiovascular risks and events, such as increased
myocardial infarction, blood pressure, decreased
HDL-cholesterol, and excess weight (Alzahrani
et al., 2019; Irwig, 2018; Kyinn et al., 2021). Health
care professionals (HCPs) should discuss lifestyle
and pharmacologic therapy with patients who are
at the highest risk of developing cardiovascular
disease (see Chapter 15---Primary Care).
Polycythemia is another disorder that may present
in TGD people taking testosterone (Antun et al.,
2020). Therefore, it is important to continuously
monitor for the development of conditions that
can be exacerbated by GAHT throughout life
(Hembree et al., 2017).

All the statements in this chapter have been
recommended based on a thorough review of
evidence, an assessment of the benefits and
harms, values and preferences of providers and
patients, and resource use and feasibility. In some
cases, we recognize evidence is limited and/or
services may not be accessible or desirable.

\hypertarget{gender-affirming-hormone-therapy-in-youth}{%
\section*{Gender-Affirming Hormone Therapy in Youth}\label{gender-affirming-hormone-therapy-in-youth}}
\addcontentsline{toc}{section}{Gender-Affirming Hormone Therapy in Youth}

The following sections will discuss hormone therapy in TGD youth. Depending on the developmental stage of the youth, this hormone therapy
generally comprises two phases, namely pubertal
suppression followed by the addition of GAHT.
During the first phase, pubertal development is
halted to allow the youth to explore their gender
identity and embodiment goals to prepare for the
next phase, which may include GAHT. This section
will discuss the recommendations for the use of
gonadotropin releasing hormone agonists (GnRHas)
as well as alternate approaches to pubertal suppression and will be followed by recommendations for
GAHT. Sections that are applicable to youth and
adults will follow in the next section.

\hypertarget{statement-12.1-we-recommend-health-care-professionals-begin-pubertal-hormone-suppression-in-eligible-transgender-and-gender-diverse-adolescents-only-after-they-first-exhibit-physical-changes-of-puberty-tanner-stage-2.}{%
\section*{Statement 12.1: We recommend health care professionals begin pubertal hormone suppression in eligible* transgender and gender diverse adolescents only after they first exhibit physical changes of puberty (Tanner stage 2).}\label{statement-12.1-we-recommend-health-care-professionals-begin-pubertal-hormone-suppression-in-eligible-transgender-and-gender-diverse-adolescents-only-after-they-first-exhibit-physical-changes-of-puberty-tanner-stage-2.}}
\addcontentsline{toc}{section}{Statement 12.1: We recommend health care professionals begin pubertal hormone suppression in eligible* transgender and gender diverse adolescents only after they first exhibit physical changes of puberty (Tanner stage 2).}

In general, the goal of GnRHa administration
in TGD adolescents is to prevent further development of the endogenous secondary sex characteristics corresponding to the sex designated at
birth. Since this treatment is fully reversible, it
is regarded as an extended time for adolescents
to explore their gender identity by means of an
early social transition (Ashley, 2019e). Treatment
with GnRHas also has therapeutic benefit since
it often results in a vast reduction in the level
of distress stemming from physical changes that
occur when endogenous puberty begins
(Rosenthal, 2014; Turban, King et al., 2020).

For those prepubertal TGD children who have
been persistent in their gender identity, any
amount of permanent development of secondary
sex characteristics could result in significant distress. While one might consider use of a GnRHa
to prevent initiation of puberty in such individuals
who remain at Tanner Stage 1, this use of GnRHa
has not been recommended (Hembree et al., 2017).
When a child reaches an age where pubertal development would normally begin (typically from 7-8
to 13 years for those with ovaries and from 9 to
14 years for those with testes), it would be appropriate to screen the child more frequently, perhaps
at 4-month intervals, for signs of pubertal development (breast budding or testicular volume \textgreater{}
4 cc). Given the typical tempo of pubertal development (3.5--4 years for completion), it would be
very unlikely for permanent pubertal changes to
develop if one is only in puberty for 4 months or
less. Thus, with frequent follow-up, the initiation
of puberty can easily be detected before there are
irreversible physical changes, and GnRHa can be
started at that time with great efficacy. Of note,
following initiation of a GnRHa, there is typically
a regression of one Tanner stage. Thus, if there is
only Tanner stage 2 breast development, it typically
fully regresses to the prepubertal Tanner stage 1;
the same is typically true with Tanner stage 2
testes (often not even discernable to the patient
and is not associated with development of secondary sex characteristics).

Given GnRHas work through GnRH receptor
desensitization, if there's no uptick in endogenous
GnRH stimulation of the pituitary (the first biochemical sign of puberty), there's no need for
GnRH receptor desensitization. In addition,
because of the wide variability in the timing of
the start of puberty (as noted above), it is hard
to justify using a GnRHa that might have some
unknown risk if there's no physiological benefit
before pubertal onset. Using a GnRHa with a
child at Tanner stage 1 would only be indicated
in cases of constitutional delay in growth and
puberty, likely alongside the start of GAHT.

However, the use of a GnRHa could be considered in a child who, due to a constitutional delay
in growth and puberty, starts GAHT while still in
Tanner Stage 1. Initiating GAHT may activate the
hypothalamic-pituitary gonadal axis in the beginning but may also mask the effects on the body
of this activation. To avoid body changes with the
potential to exacerbate an individual's gender
incongruence, the GnRHa can be started as an
adjunctive therapy to the GAHT shortly after the
initiation of the GAHT to provide for pubertal
development of the identified phenotype.

In addition, the suppression of the development of secondary sex characteristics is most
effective when sex hormonal treatment is initiated
in early to mid-puberty when compared with the
initiation of sex hormonal treatment after puberty
is completed (Bangalore-Krisha et al., 2019).
Correspondingly, for adolescents who have already
completed endogenous puberty and are considering starting GAHT, GnRHas can be used to
inhibit physical functions, such as menses or
erections, and can serve as a bridge until the
adolescent, guardian(s) (if the adolescent is not
able to consent independently), and treatment
team reach a decision (Bangalore-Krishna et al.,
2019; Rosenthal, 2021).

The onset of puberty occurs through reactivation of the hypothalamic-pituitary-gonadal axis.
Clinical assessment of the stages of puberty is
based on physical features that reflect that reactivation. In individuals with functioning ovaries,
Tanner stage 2 is characterized by the budding of
the mammary gland. The development of the
mammary gland occurs from exposure to estrogen
produced by the ovaries. In individuals with functioning testes, Tanner stage 2 is characterized by
an increase in testicular volume (typically greater
than 4ml). The growth of the testes is mediated
through the gonadotropins luteinizing hormone
(LH) and follicle stimulating hormone (FSH). In
the later stages, the testes produce enough testosterone to induce masculinization of the body.

\hypertarget{statement-12.2-we-recommend-health-care-professionals-use-gnrh-agonists-to-suppress-endogenous-sex-hormones-in-eligible-transgender-and-gender-diverse-people-for-whom-puberty-blocking-is-indicated.-for-supporting-text-see-statement-12.4.}{%
\section*{Statement 12.2: We recommend health care professionals use GnRH agonists to suppress endogenous sex hormones in eligible* transgender and gender diverse people for whom puberty blocking is indicated. For supporting text, see Statement 12.4.}\label{statement-12.2-we-recommend-health-care-professionals-use-gnrh-agonists-to-suppress-endogenous-sex-hormones-in-eligible-transgender-and-gender-diverse-people-for-whom-puberty-blocking-is-indicated.-for-supporting-text-see-statement-12.4.}}
\addcontentsline{toc}{section}{Statement 12.2: We recommend health care professionals use GnRH agonists to suppress endogenous sex hormones in eligible* transgender and gender diverse people for whom puberty blocking is indicated. For supporting text, see Statement 12.4.}

\hypertarget{statement-12.3-we-suggest-health-care-professionals-prescribe-progestins-oral-or-injectable-depot-for-pubertal-suspension-in-eligible-transgender-and-gender-diverse-youth-when-gnrh-agonists-are-not-available-or-are-cost-prohibitive.-for-supporting-text-see-statement-12.4.}{%
\section*{Statement 12.3: We suggest health care professionals prescribe progestins (oral or injectable depot) for pubertal suspension in eligible* transgender and gender diverse youth when GnRH agonists are not available or are cost prohibitive. For supporting text, see Statement 12.4.}\label{statement-12.3-we-suggest-health-care-professionals-prescribe-progestins-oral-or-injectable-depot-for-pubertal-suspension-in-eligible-transgender-and-gender-diverse-youth-when-gnrh-agonists-are-not-available-or-are-cost-prohibitive.-for-supporting-text-see-statement-12.4.}}
\addcontentsline{toc}{section}{Statement 12.3: We suggest health care professionals prescribe progestins (oral or injectable depot) for pubertal suspension in eligible* transgender and gender diverse youth when GnRH agonists are not available or are cost prohibitive. For supporting text, see Statement 12.4.}

\hypertarget{statement-12.4-we-suggest-health-care-professionals-prescribe-gnrh-agonists-to-suppress-sex-steroids-without-concomitant-sex-steroid-hormone-replacement-in-eligible-transgender-and-gender-diverse-adolescents-seeking-such-intervention-who-are-well-into-or-have-completed-pubertal-development-past-tanner-stage-3-but-are-unsure-about-or-do-not-wish-to-begin-sex-steroid-hormone-therapy.}{%
\section*{Statement 12.4: We suggest health care professionals prescribe GnRH agonists to suppress sex steroids without concomitant sex steroid hormone replacement in eligible transgender and gender diverse adolescents seeking such intervention who are well into or have completed pubertal development (past Tanner stage 3) but are unsure about or do not wish to begin sex steroid hormone therapy.}\label{statement-12.4-we-suggest-health-care-professionals-prescribe-gnrh-agonists-to-suppress-sex-steroids-without-concomitant-sex-steroid-hormone-replacement-in-eligible-transgender-and-gender-diverse-adolescents-seeking-such-intervention-who-are-well-into-or-have-completed-pubertal-development-past-tanner-stage-3-but-are-unsure-about-or-do-not-wish-to-begin-sex-steroid-hormone-therapy.}}
\addcontentsline{toc}{section}{Statement 12.4: We suggest health care professionals prescribe GnRH agonists to suppress sex steroids without concomitant sex steroid hormone replacement in eligible transgender and gender diverse adolescents seeking such intervention who are well into or have completed pubertal development (past Tanner stage 3) but are unsure about or do not wish to begin sex steroid hormone therapy.}

GnRHas reduce gonadotrophin and sex steroid
concentrations in TGD adolescents and thus halt
the further development of secondary sex characteristics (Schagen et al., 2016). Their use is
generally safe with the development of hypertension being the only short-term adverse event
reported in the literature (Delemarre-van de Waal
\& Cohen-Kettenis, 2006; Klink, Bokenkamp et al.,
2015). GnRHas prevent the pituitary gland from
secreting LH and FSH (Gava et al., 2020). When
the gonadotropins decrease, the gonad is no longer stimulated to produce sex hormones (estrogens or androgens), and the sex hormone levels
in the blood decrease to prepubertal levels.
GnRHa treatment leads to partial regression of
the initial stages of the already developed secondary sex characteristics (Bangalore et al., 2019).
TGD adolescents with functioning ovaries will
experience diminished growth of breast tissue,
and if treatment is started at Tanner stage 2, the
breast tissue may disappear completely (Shumer
et al., 2016). Menarche can be prevented or discontinued following the administration of GnRHas
in adolescents with a uterus. In TGD adolescents
with functioning testes, testicular volume will
regress to a lower volume.

When GnRHa treatment is started in adolescents at the later phases of pubertal development,
some physical changes of pubertal development,
such as late-stage breast development in TGD
adolescents with functioning ovaries and a lower
voice and growth of facial hair in TGD adolescents with functioning testes, will not regress
completely, although any further progression will
be stopped (Delemarre-van de Waal \&
Cohen-Kettenis, 2006). GnRHas have been used
since 1981 for the treatment of central precocious
puberty (Comite et al., 1981; Laron et al., 1981),
and their benefits are well established (please also
see the statements in Chapter 6---Adolescents).
The use of GnRHas in individuals with central
precocious puberty is regarded as both safe and
effective, with no known long-term adverse
effects (Carel et al., 2009). However, the use of
GnRHas in TGD adolescents is considered
off-label because they were not initially developed
for this purpose. Nonetheless, data from adolescents prescribed GnRHas in a similar dose and
fashion demonstrate effectiveness in delaying the
onset of puberty although the long-term effects
on bone mass have not been well established
(Klink, Caris et al., 2015). Although long-term
data are more limited in TGD adolescents than
in adolescents with precocious puberty, data collection specifically in this population are ongoing
(Klaver et al., 2020; Lee, Finlayson et al., 2020;
Millington et al., 2020; Olson-Kennedy, Garofalo
et al., 2019).

We recognize even though GnRHas are a medically necessary treatment, they may not be available for eligible adolescents because it is not
covered by health insurance plans in some countries or may be cost-prohibitive. Therefore, other
approaches should be considered in these cases,
such as oral or injectable progestin formulations.
In addition, for adolescents older than 14 years,
there are currently no data to inform HCPs
whether GnRHas can be administered as monotherapy (and for what duration) without posing
a significant risk to skeletal health. This is because
the skeleton will not have any exposure to adequate levels of sex steroid hormones
(Rosenthal, 2021).

A prolonged hypogonadal state in adolescence,
whether due to medical conditions such as hypergonadotropic hypogonadism, iatrogenic causes
such as GnRHa monotherapy or physiological
conditions such as conditional delay of growth
and development, is often associated with an
increased risk of poor bone health later in life
(Bertelloni et al., 1998; Finkelstein et al., 1996).
However, bone mass accrual is a multifactorial
process that involves a complex interplay between
endocrine, genetic, and lifestyle factors (Anai
et al., 2001). When deciding on the duration of
GnRHa monotherapy, all contributing factors
should be considered, including factors such as
pretreatment bone mass, bone age, and pubertal
stage from an endocrine perspective and height
gain, as well as psychosocial factors such as mental maturity and developmental stage relative to
one's adolescent cohort and the adolescent's individual treatment goals (Rosenthal, 2021). For
these reasons, a multidisciplinary team and an
ongoing clinical relationship with the adolescent
and the family should be maintained when initiating GnRHa treatment (see Statements 6.8, 6.9,
and 6.12 in Chapter 6---Adolescents). The clinical
course of the treatment, e.g., the development of
bone mass during GnRHa treatment and the adolescent's response to treatment, can help to determine the length of GnRHa monotherapy.

\hypertarget{statement-12.5-we-recommend-health-care-professionals-prescribe-sex-hormone-treatment-regimens-as-part-of-gender-affirming-treatment-in-eligible-transgender-and-gender-diverse-adolescents-who-are-at-least-tanner-stage-2-with-parentalguardian-involvement-unless-their-involvement-is-determined-to-be-harmful-or-unnecessary-to-the-adolescent.-for-supporting-text-see-statement-12.6.}{%
\section*{Statement 12.5: We recommend health care professionals prescribe sex hormone treatment regimens as part of gender-affirming treatment in eligible* transgender and gender diverse adolescents who are at least Tanner stage 2, with parental/guardian involvement unless their involvement is determined to be harmful or unnecessary to the adolescent. For supporting text, see Statement 12.6.}\label{statement-12.5-we-recommend-health-care-professionals-prescribe-sex-hormone-treatment-regimens-as-part-of-gender-affirming-treatment-in-eligible-transgender-and-gender-diverse-adolescents-who-are-at-least-tanner-stage-2-with-parentalguardian-involvement-unless-their-involvement-is-determined-to-be-harmful-or-unnecessary-to-the-adolescent.-for-supporting-text-see-statement-12.6.}}
\addcontentsline{toc}{section}{Statement 12.5: We recommend health care professionals prescribe sex hormone treatment regimens as part of gender-affirming treatment in eligible* transgender and gender diverse adolescents who are at least Tanner stage 2, with parental/guardian involvement unless their involvement is determined to be harmful or unnecessary to the adolescent. For supporting text, see Statement 12.6.}

\hypertarget{statement-12.6-we-recommend-health-care-professionals-measure-hormone-levels-during-gender-affirming-treatment-to-ensure-endogenous-sex-steroids-are-lowered-and-administered-sex-steroids-are-maintained-at-a-level-appropriate-for-the-treatment-goals-of-transgender-and-gender-diverse-people-according-to-the-tanner-stage.}{%
\section*{Statement 12.6: We recommend health care professionals measure hormone levels during gender-affirming treatment to ensure endogenous sex steroids are lowered and administered sex steroids are maintained at a level appropriate for the treatment goals of transgender and gender diverse people according to the Tanner stage.}\label{statement-12.6-we-recommend-health-care-professionals-measure-hormone-levels-during-gender-affirming-treatment-to-ensure-endogenous-sex-steroids-are-lowered-and-administered-sex-steroids-are-maintained-at-a-level-appropriate-for-the-treatment-goals-of-transgender-and-gender-diverse-people-according-to-the-tanner-stage.}}
\addcontentsline{toc}{section}{Statement 12.6: We recommend health care professionals measure hormone levels during gender-affirming treatment to ensure endogenous sex steroids are lowered and administered sex steroids are maintained at a level appropriate for the treatment goals of transgender and gender diverse people according to the Tanner stage.}

Sex steroid hormone therapy generally comprises two treatment regimens, depending on the
timing of the GnRHa treatment. When GnRHa
treatment is started in the early stages of endogenous pubertal development, puberty corresponding with gender identity or embodiment goals is
induced with doses of sex steroid hormones similar to those used in peripubertal hypogonadal
adolescents. In this context, adult doses of sex
steroid hormones are typically reached over
approximately a 2-year period (Chantrapanichkul
et al., 2021). When GnRHa treatment is started
in late- or postpubertal transgender adolescents,
sex steroid hormones can be given at a higher
starting dose and increased more rapidly until a
maintenance dose is achieved, resembling treatment protocols used in transgender adults
(Hembree et al., 2017). An additional advantage
of GnRHa treatment is sex steroid hormones do
not have to be administered in supraphysiological
doses, which would otherwise be needed to suppress endogenous sex steroid production (Safer
\& Tangpricha, 2019). For TGD individuals with
functioning testes, GnRHa treatment (or another
testosterone-blocking medication) should be continued until such time as the TGD adolescent/
young adult ultimately undergoes gonadectomy,
if this surgical procedure is pursued as a medically necessary part of their gender-affirming
care. Once adult levels of testosterone are reached
in TGD individuals with functioning ovaries who
have been initially suppressed with GnRHa's, testosterone alone at physiological doses is typically
sufficient to lower ovarian estrogen secretion, and
GnRHas can be discontinued as discussed below
(Hembree et al., 2017). For TGD adolescents with
functioning ovaries who are new to care, GAHT
can be accomplished with physiological doses of
testosterone alone without the need for concomitant GnRHa administration (Hembree et al., 2017).

Gender-affirming sex steroid hormone therapy
induces the development of secondary sex characteristics of the gender identity. Also, the rate
of bone mineralization, which decreases during
treatment with GnRHa's, rapidly recovers (Klink,
Caris et al., 2015). During GnRHa treatment in
early-pubertal TGD adolescents, the bone epiphyseal plates are still unfused (Kvist et al., 2020;
Schagen et al., 2020). Following the initiation of
sex steroid hormone treatment, a growth spurt
can occur, and bone maturation continues (Vlot
et al., 2017). In postpubertal TGD adolescents,
sex steroid hormone treatment will not affect
height since the epiphyseal plates have fused, and
bone maturation is complete (Vlot et al., 2017).
In TGD adolescents with functioning testes,
the use of 17-ß-estradiol for pubertal induction
is preferred over that of synthetic estrogens, such
as the more thrombogenic ethinyl estradiol (see
Appendix D (Asscheman et al., 2015). It is still
necessary to either continue GnRHa's to suppress
endogenous testosterone production or transition
to another medication that suppresses endogenous testosterone production (Rosenthal et al.,
2016). Breast development and a female-typical
fat distribution are among a number of physical
changes that occur in response to estrogen treatment. See Appendix C---Table 1.

For TGD adolescents seeking masculinizing
treatment, androgens are available as injectable
preparations, transdermal formulations, and subcutaneous pellets. For pubertal induction, the use
of testosterone-ester injection is generally recommended by most experts initially because of cost,
availability, and experience (Shumer et al., 2016).
It is advised to continue GnRHas at least until
a maintenance level of testosterone is reached. In
response to androgen treatment, virilization of
the body occurs, including a lowering of the
voice, more muscular development particularly
in the upper body, growth of facial and body
hair, and clitoral enlargement (Rosenthal et al.,
2016). See Appendix C---Table 1.

In almost all situations, parental/caregiver consent should be obtained. Exceptions to this recommendation, in particular when caregiver or
parental involvement is determined to be harmful
to the adolescent, are described in more detail
in Chapter 6---Adolescents (see Statement 6.11)
where the rationale for involving parents/caregivers in the consent process is also described.

\hypertarget{statement-12.7-we-recommend-health-care-professionals-prescribe-progestogens-or-a-gnrh-agonist-for-eligible-transgender-and-gender-diverse-adolescents-with-a-uterus-to-reduce-dysphoria-caused-by-their-menstrual-cycle-when-gender-affirming-testosterone-use-is-not-yet-indicated.}{%
\section*{Statement 12.7: We recommend health care professionals prescribe progestogens or a GnRH agonist for eligible* transgender and gender diverse adolescents with a uterus to reduce dysphoria caused by their menstrual cycle when gender-affirming testosterone use is not yet indicated.}\label{statement-12.7-we-recommend-health-care-professionals-prescribe-progestogens-or-a-gnrh-agonist-for-eligible-transgender-and-gender-diverse-adolescents-with-a-uterus-to-reduce-dysphoria-caused-by-their-menstrual-cycle-when-gender-affirming-testosterone-use-is-not-yet-indicated.}}
\addcontentsline{toc}{section}{Statement 12.7: We recommend health care professionals prescribe progestogens or a GnRH agonist for eligible* transgender and gender diverse adolescents with a uterus to reduce dysphoria caused by their menstrual cycle when gender-affirming testosterone use is not yet indicated.}

Menstrual suppression is a treatment option
commonly needed by TGD individuals who experience distress related to menses or the anticipation of menarche. Statement 6.7 in Chapter
6---Adolescents describes this in more detail. To
achieve amenorrhea, menstrual suppression can
be initiated as a solo option before initiating testosterone or alongside testosterone therapy
(Carswell \& Roberts, 2017). Some youth, who are
not ready for testosterone therapy or are not yet
at an appropriate pubertal/developmental stage
to begin such treatment, will benefit from the
induction of amenorrhea (Olson-Kennedy,
Rosenthal et al., 2018). Adolescents who experience an exacerbation of dysphoria related to the
onset of puberty may elect to be treated with
GnRHas for pubertal suppression (also see the
Adolescents chapter).

Progestogens may be effective in adolescents
whose goal is solely menstrual suppression.
Continuous administration of progestin-only oral
pills (including the contraceptive and noncontraceptive options), medroxyprogesterone injections,
or levonorgestrel intrauterine device can be used
for induction of amenorrhea (Pradhan \&
Gomez-Lobo, 2019). TGD individuals with functioning ovaries who start testosterone therapy
may have 1--5 menstrual cycles before amenorrhea is achieved (Taub et al., 2020). Once amenorrhea is achieved, some TGD individuals with
functioning ovaries may also choose to continue
progestin treatment for birth control if relevant
to their sexual practices.

TGD individuals with functioning ovaries and
a uterus should be counseled about the potential
for breakthrough menstrual bleeding in the first
few months after initiating menstrual suppression.
With GnRHa therapy, breakthrough bleeding may
occur 2--3 weeks after initiation of the medication. For individuals seeking contraception or for
those who continue to experience menstrual
bleeding on progestin therapy, an estrogen combination with progestin may be considered for
the maintenance of amenorrhea, yet they should
be counseled on the possible side effect of breast
development (Schwartz et al., 2019).

\hypertarget{statement-12.8-we-recommend-health-care-providers-involve-professionals-from-multiple-disciplines-who-are-experts-in-transgender-health-and-in-the-management-of-the-care-of-transgender-and-gender-diverse-adolescents.}{%
\section*{Statement 12.8: We recommend health care providers involve professionals from multiple disciplines who are experts in transgender health and in the management of the care of transgender and gender diverse adolescents.}\label{statement-12.8-we-recommend-health-care-providers-involve-professionals-from-multiple-disciplines-who-are-experts-in-transgender-health-and-in-the-management-of-the-care-of-transgender-and-gender-diverse-adolescents.}}
\addcontentsline{toc}{section}{Statement 12.8: We recommend health care providers involve professionals from multiple disciplines who are experts in transgender health and in the management of the care of transgender and gender diverse adolescents.}

As with the care of adolescents, we suggest
where possible a multidisciplinary expert team
of medical and mental health professionals
(MHPs) be assembled to manage this treatment.
In adolescents who pursue GAHT (given this is
a partly irreversible treatment), we suggest initiating treatment using a schedule of gradually
increasing doses after a multidisciplinary team
of medical and MHPs has confirmed the persistence of GD/gender incongruence and has
established the individual possesses the mental
capacity to give informed consent (Hembree
et al., 2017). Specific aspects concerning the
assessment of adolescents and the involvement
of their caregivers and a multidisciplinary team
are described in more detail in Chapter
6---Adolescents.

If possible, TGD adolescents should have access
to experts in pediatric transgender health from
multiple disciplines including primary care, endocrinology, fertility, mental health, voice, social
work, spiritual support, and surgery (Chen,
Hidalgo et al., 2016; Eisenberg et al., 2020;
Keo-Meier \& Ehrensaft, 2018). Individual providers are encouraged to form collaborative working
relationships with providers from other disciplines to facilitate referrals as needed for the
individual youth and their family (Tishelman
et al., 2015). However, the lack of available
experts and resources should not constitute a
barrier to care (Rider, McMorris et al., 2019).
Helpful support for adolescents includes access
to accurate, culturally informed information
related to gender and sexual identities, transition
options, the impact of family support, and connections to others with similar experiences and
with TGD adults through online and in person
support groups for adolescents and their family
members (Rider, McMorris et al., 2019).

Many TGD adolescents have been found to
experience mental health disparities and initial
mental health screening (e.g., PHQ-2, GAD) can
be employed as indicated (Rider, McMorris et al.,
2019). Providers should keep in mind being
transgender or questioning one's gender does not
constitute pathology or a disorder. Therefore,
individuals should not be referred for mental
health treatment exclusively on the basis of a
transgender identity. HCPs and MHPs who treat
these youths and make referrals should, at a minimum, be familiar with the impact of trauma,
gender dysphoria, and gender minority stressors
on any potential mental health symptomatology,
such as disordered eating, suicidal ideation, social
anxiety. These health care providers should also
be knowledgeable about the level of readiness of
inpatient mental health services in their region
to provide competent, gender-affirming care to
TGD youth (Barrow \& Apostle, 2018; Kuper,
Wright et al., 2018; Kuper, Mathews et al., 2019;
Tishelman \& Neumann-Mascis, 2018). Statements
6.3, 6.4, and 6.12d in Chapter 6---Adolescents
address this in more detail. Because parents of
these youth commonly experience high levels of
anxiety immediately after learning their youth is
TGD, and their response to their child predicts
that child's long-term physical and mental health
outcomes, appropriate referrals for mental health
support of the parents can be of great utility
(Coolhart et al., 2017; Pullen Sansfaçon et al.,
2015; Taliaferro et al., 2019).

\hypertarget{statement-12.9-we-recommend-health-care-professionals-organize-regular-clinical-evaluations-for-physical-changes-and-potential-adverse-reactions-to-sex-steroid-hormones-including-laboratory-monitoring-of-sex-steroid-hormones-every-3-months-during-the-first-year-of-hormone-therapy-or-with-dose-changes-until-a-stable-adult-dosing-is-reached-followed-by-clinical-and-laboratory-testing-once-or-twice-a-year-once-an-adult-maintenance-dose-is-attained.}{%
\section*{Statement 12.9: We recommend health care professionals organize regular clinical evaluations for physical changes and potential adverse reactions to sex steroid hormones, including laboratory monitoring of sex steroid hormones every 3 months during the first year of hormone therapy or with dose changes until a stable adult dosing is reached followed by clinical and laboratory testing once or twice a year once an adult maintenance dose is attained.}\label{statement-12.9-we-recommend-health-care-professionals-organize-regular-clinical-evaluations-for-physical-changes-and-potential-adverse-reactions-to-sex-steroid-hormones-including-laboratory-monitoring-of-sex-steroid-hormones-every-3-months-during-the-first-year-of-hormone-therapy-or-with-dose-changes-until-a-stable-adult-dosing-is-reached-followed-by-clinical-and-laboratory-testing-once-or-twice-a-year-once-an-adult-maintenance-dose-is-attained.}}
\addcontentsline{toc}{section}{Statement 12.9: We recommend health care professionals organize regular clinical evaluations for physical changes and potential adverse reactions to sex steroid hormones, including laboratory monitoring of sex steroid hormones every 3 months during the first year of hormone therapy or with dose changes until a stable adult dosing is reached followed by clinical and laboratory testing once or twice a year once an adult maintenance dose is attained.}

Sex steroid hormone therapy is associated with
a broad array of physical and psychological
changes (Irwig, 2017; Tangpricha \& den Heijer,
2017) (see Appendix C---Table 1). After sex steroid hormone therapy has been initiated, the
HCP should regularly assess the progress and
response of the individual to the treatment (also
see Chapter 6---Adolescents). This evaluation
should assess the presence of any physical changes
as well as the impact of treatment on gender
dysphoria (if present) and psychological well-being
(see Appendix C---Table 1). Clinical visits provide
important opportunities for HCPs to educate
patients about the typical time course required
for physical changes to manifest and encourage
realistic expectations. During the first year of
hormone therapy, sex steroid hormone doses are
often increased. A major factor guiding the dose
is the serum level of the corresponding sex steroid hormone. In general, the goal is to target
serum levels of the sex steroids to match the
levels associated with the individual's gender
identity, although optimal target ranges have not
been established (Hembree et al., 2017).

In addition to assessing the positive changes
associated with sex steroid hormone therapy, the
HCP should regularly assess whether the treatment has caused any adverse effects (see Appendix
C---Table 2). Examples of adverse signs and
symptoms include androgenic acne or bothersome
sexual dysfunction (Braun et al., 2021; Kerckhof
et al., 2019). GAHT also has the potential to
adversely influence several laboratory tests. For
example, spironolactone may cause hyperkalemia,
although it is an uncommon and transient phenomenon (Millington et al., 2019). Testosterone
increases the red blood cell count (hematocrit),
which may occasionally cause erythrocytosis
(Antun et al., 2020) (see Statement 12.17)
(Hembree et al., 2017). Both estrogen and testosterone can alter lipid parameters, such as
high-density protein lipoprotein (HDL) cholesterol and triglycerides (Maraka et al., 2017). See
Appendix C---Tables 3 and 4.

The frequency of clinical evaluations should
be individualized and guided by the individual's
response to treatment. We suggest clinical assessments be performed approximately every 3
months during the first year of hormone therapy
in patients who are stable and are not experiencing significant adverse effects (Appendix C---Table
5). We suggest rather than recommend testing
be carried out every 3 months in the first year
to allow some flexibility on the timing of these
tests as there is no strong evidence or evidence
from published studies supporting specific testing
intervals. If an individual does experience an
adverse effect, more frequent laboratory testing
and/or clinical visits are often needed. Given the
potential harm associated with sex hormone levels that exceed expected ranges in humans, we
strongly recommend regular testing be performed
as a standard practice when initiating GAHT in
TGD individuals. Once a person has reached a
stable adult dose of sex steroid hormone with no
significant adverse effects, the frequency of clinic
visits can be reduced to one to two per year
(Hembree et al., 2017).

\hypertarget{statement-12.10-we-recommend-health-care-professionals-inform-and-counsel-all-individuals-seeking-gender-affirming-medical-treatment-about-options-for-fertility-preservation-prior-to-initiating-puberty-suppression-and-prior-to-administering-hormone-therapy.}{%
\section*{Statement 12.10: We recommend health care professionals inform and counsel all individuals seeking gender-affirming medical treatment about options for fertility preservation prior to initiating puberty suppression and prior to administering hormone therapy.}\label{statement-12.10-we-recommend-health-care-professionals-inform-and-counsel-all-individuals-seeking-gender-affirming-medical-treatment-about-options-for-fertility-preservation-prior-to-initiating-puberty-suppression-and-prior-to-administering-hormone-therapy.}}
\addcontentsline{toc}{section}{Statement 12.10: We recommend health care professionals inform and counsel all individuals seeking gender-affirming medical treatment about options for fertility preservation prior to initiating puberty suppression and prior to administering hormone therapy.}

Pubertal suppression and hormone treatment
with sex steroid hormones may have potential
adverse effects on a person's future fertility
(Cheng et al., 2019) (see also Chapter 6---
Adolescents and Chapter 16---Reproductive
Health). Although some TGD people may not
have given much thought to their future reproductive potential at the time of their initial
assessment to begin medical therapy, the potential
implications of the treatment and fertility preservation options should be reviewed by the hormone prescriber and discussed with the person
seeking these therapies (Ethics Committee of the
American Society for Reproductive Medicine
et al., 2015; De Roo et al., 2016).

Individuals with testes should be advised prolonged treatment with estrogen often causes
testicular atrophy and a reduction in sperm count
and other semen parameters (Adeleye et al.,
2018). Nonetheless, there are major gaps in
knowledge, and findings regarding the fertility of
trans feminine people who take estrogen and
antiandrogens are inconsistent (Cheng et al.,
2019). In one study, heterogeneity in testicular
histology was evident whether patients discontinued or continued therapy prior to orchiectomies (Schneider et al., 2015). For example, the
discontinuation of estrogen and antiandrogens for
six weeks resulted in complete spermatogenesis
in 45\% of individuals with the remainder showing
meiotic arrest or spermatogonial arrest (Schneider
et al., 2015). However, serum testosterone levels
confirmed to be within female reference ranges
leads to complete suppression of spermatogenesis
in most transgender women (Vereecke et al.,
2020). The principal fertility preservation option
for patients with functioning testes is sperm cryopreservation, also known as sperm banking
(Mattawanon et al., 2018). For prepubertal
patients, suppression of puberty with GnRHs
pauses the maturation of sperm (Finlayson
et al., 2016).

Individuals with functioning ovaries should
be advised testosterone therapy usually results
in the cessation of menses and ovulation, often
within a few months of initiation (Taub et al.,
2020). There are also major gaps in knowledge
regarding the potential effects of testosterone
on oocytes and subsequent fertility of TGD
patients (Eisenberg et al., 2020; Stuyver et al.,
2020). One study found testosterone treatment
may be associated with polycystic ovarian morphology, whereas other studies reported no
metabolic (Chan et al., 2018) or histologic (De
Roo et al., 2017; Grynberg et al., 2010) evidence of polycystic ovary syndrome (PCOS)
following treatment with testosterone, and some
studies have found a pre-existing higher prevalence of PCOS in transgender patients with
ovaries (Baba, 2007; Gezer et al., 2021). TGD
patients with an intact uterus and ovaries often
regain their fertility potential if testosterone
therapy is discontinued (Light et al., 2014).
Indeed, a live birth after assisted reproductive
technology has been reported following
hormone-stimulated egg retrieval from a TGD
individual who did not discontinue testosterone
therapy (Greenwald et al., 2021; Safer and
Tangpricha, 2019). Other fertility preservation
options for TGD patients with ovaries are
oocyte cryopreservation and embryo cryopreservation with sperm from a partner or donor.
The above options require hormonal stimulation for egg retrieval and the use of assisted
reproductive technology.

For early pubertal transgender youth, suppression of puberty with GnRHa's pauses the maturation of germ cells, although a recent report
noted ovarian stimulation of a TGD adolescent
treated with a GnRHa's in early puberty (and
continued during ovarian stimulation) resulted in
a small number of mature oocytes that were cryopreserved (Rothenberg et al., 2019). Treating an
TGD adolescent with functioning testes in the
early stages of puberty with a GnRHa not only
pauses maturation of germ cells but will also
maintains the penis in a prepubertal size. This
will likely impact surgical considerations if that
person eventually undergoes a penile-inversion
vaginoplasty as there will be less penile tissue to
work with. In these cases, there is an increased
likelihood a vaginoplasty will require a more
complex surgical procedure, e.g., intestinal vaginoplasty (Dy et al., 2021; van de Grift et al.,
2020). Such considerations should be included in
any discussions with patients and families considering use of pubertal blockers in early pubertal
adolescents with functioning testes.

\hypertarget{statement-12.11-we-recommend-health-care-professionals-evaluate-and-address-medical-conditions-that-can-be-exacerbated-by-lowered-endogenous-sex-hormone-concentrations-and-treatment-with-exogenous-sex-hormones-before-beginning-treatment-in-transgender-and-gender-diverse-people.}{%
\section*{Statement 12.11: We recommend health care professionals evaluate and address medical conditions that can be exacerbated by lowered endogenous sex hormone concentrations and treatment with exogenous sex hormones before beginning treatment in transgender and gender diverse people.}\label{statement-12.11-we-recommend-health-care-professionals-evaluate-and-address-medical-conditions-that-can-be-exacerbated-by-lowered-endogenous-sex-hormone-concentrations-and-treatment-with-exogenous-sex-hormones-before-beginning-treatment-in-transgender-and-gender-diverse-people.}}
\addcontentsline{toc}{section}{Statement 12.11: We recommend health care professionals evaluate and address medical conditions that can be exacerbated by lowered endogenous sex hormone concentrations and treatment with exogenous sex hormones before beginning treatment in transgender and gender diverse people.}

TGD people seeking masculinization must be
informed about the possibilities, consequences,
limitations, and risks associated with testosterone
treatment. Testosterone therapy is contraindicated
during pregnancy or while attempting to become
pregnant given its potential iatrogenic effects on
the fetus. Relative contraindications to testosterone therapy include severe hypertension, sleep
apnea, and polycythemia since these conditions
can be exacerbated by testosterone. Monitoring
blood pressure and lipid profiles should be performed before and after the onset of testosterone
therapy. The increase in blood pressure typically
occurs within 2 to 4 months following the initiation of testosterone therapy (Banks et al.,
2021). Patients who develop hypercholesterolemia
and/or hypertriglyceridemia may require treatment with dietary modifications, medication,
or both.

TGD people seeking feminizing treatment
with a history of thromboembolic events, such
as deep vein thrombosis and pulmonary embolism, should undergo evaluation and treatment
prior to the initiation of hormone therapy. This
is because estrogen therapy is strongly associated
with an increased risk of thromboembolism, a
potentially life-threatening complication. In
addition, risk factors that can increase the risk
of thromboembolic conditions, such as smoking,
obesity, and sedentary lifestyle, should be modified. In patients with nonmodifiable risk factors,
such as a known history of thrombophilia, a
past history of thrombosis, or a strong family
history of thromboembolism, treatment with
transdermal estrogen concomitant with anticoagulants may decrease the risk of thromboembolism. However, there are limited data to guide
treatment decisions. The presence of a disease
at baseline such as a hormone sensitive cancer,
coronary artery disease, cerebrovascular disease,
hyperprolactinemia, hypertriglyceridemia, and
cholelithiasis should be evaluated prior to the
initiation of gender-affirming hormone therapy
as relative risks may be shifted in association
with exogenous hormone treatment (Hembree
et al., 2017).

\hypertarget{statement-12.12-we-recommend-health-care-professionals-educate-transgender-and-gender-diverse-people-undergoing-gender-affirming-treatment-about-the-onset-and-time-course-of-physical-changes-induced-by-sex-hormone-treatment.}{%
\section*{Statement 12.12: We recommend health care professionals educate transgender and gender diverse people undergoing gender-affirming treatment about the onset and time course of physical changes induced by sex hormone treatment.}\label{statement-12.12-we-recommend-health-care-professionals-educate-transgender-and-gender-diverse-people-undergoing-gender-affirming-treatment-about-the-onset-and-time-course-of-physical-changes-induced-by-sex-hormone-treatment.}}
\addcontentsline{toc}{section}{Statement 12.12: We recommend health care professionals educate transgender and gender diverse people undergoing gender-affirming treatment about the onset and time course of physical changes induced by sex hormone treatment.}

The effects of testosterone treatment are multiple and may include the appearance of increased
body and facial hair, male pattern baldness,
increased muscle mass and strength, decreased
fat mass, deepening of the voice, interruption of
menses (if still present), increased prevalence and
severity of acne, clitoral enlargement, and
increased sexual desire (Defreyne, Elaut et al.,
2020; Fisher, Castellini et al., 2016; Giltay \&
Gooren, 2000; T'Sjoen et al., 2019; Yeung et al.,
2020). Other testosterone-associated changes
include increased lean body mass, skin oiliness,
(de Blok et al., 2020; Hembree et al., 2017; Kuper,
Mathews et al., 2019; Taliaferro et al., 2019;
Tishelman \& Neumann-Mascis, 2018) (see
Appendix C---Table 1).

Estrogen treatment induces breast development.
However, fewer than 20\% of individuals reach
Tanner breast stages 4--5 after 2 years of treatment (de Blok et al., 2021). Additional changes
include decreases in testicular volume, lean body
mass, skin oiliness, sexual desire, spontaneous
erections, facial hair, and body hair along with
increased subcutaneous body fat) (see Appendix
C---Table 1). In adult patients, estrogen does not
alter a person's voice or height (Iwamoto, Defreyne
et al., 2019; Wiepjes et al., 2019).
The time course and extent of physical changes
vary among individuals and are related to factors
such as genetics, age of initiation, and overall
state of health (Deutsch, Bhakri et al., 2015; van
Dijk et al., 2019). Knowledge of the extent and
timing of sex hormone--induced changes, if available, may prevent the potential harm and expense
of unnecessary treatment changes, dosage
increases, and premature surgical procedures
(Dekker et al., 2016).

\hypertarget{statement-12.13-we-recommend-health-care-professionals-not-prescribe-ethinyl-estradiol-for-transgender-and-gender-diverse-people-as-part-of-a-gender-affirming-hormonal-treatment.-for-supporting-text-see-statement-12.15.}{%
\section*{Statement 12.13: We recommend health care professionals not prescribe ethinyl estradiol for transgender and gender diverse people as part of a gender-affirming hormonal treatment. For supporting text, see Statement 12.15.}\label{statement-12.13-we-recommend-health-care-professionals-not-prescribe-ethinyl-estradiol-for-transgender-and-gender-diverse-people-as-part-of-a-gender-affirming-hormonal-treatment.-for-supporting-text-see-statement-12.15.}}
\addcontentsline{toc}{section}{Statement 12.13: We recommend health care professionals not prescribe ethinyl estradiol for transgender and gender diverse people as part of a gender-affirming hormonal treatment. For supporting text, see Statement 12.15.}

\hypertarget{statement-12.14-we-suggest-health-care-professionals-prescribe-transdermal-estrogen-for-eligible-transgender-and-gender-diverse-people-at-higher-risk-of-developing-venous-thromboembolism-based-on-age-45-years-or-a-previous-history-of-venous-thromboembolism-when-gender-affirming-estrogen-treatment-is-recommended.-for-supporting-text-see-statement-12.15.}{%
\section*{Statement 12.14: We suggest health care professionals prescribe transdermal estrogen for eligible* transgender and gender diverse people at higher risk of developing venous thromboembolism based on age \textgreater45 years or a previous history of venous thromboembolism, when gender-affirming estrogen treatment is recommended. For supporting text, see Statement 12.15).}\label{statement-12.14-we-suggest-health-care-professionals-prescribe-transdermal-estrogen-for-eligible-transgender-and-gender-diverse-people-at-higher-risk-of-developing-venous-thromboembolism-based-on-age-45-years-or-a-previous-history-of-venous-thromboembolism-when-gender-affirming-estrogen-treatment-is-recommended.-for-supporting-text-see-statement-12.15.}}
\addcontentsline{toc}{section}{Statement 12.14: We suggest health care professionals prescribe transdermal estrogen for eligible* transgender and gender diverse people at higher risk of developing venous thromboembolism based on age \textgreater45 years or a previous history of venous thromboembolism, when gender-affirming estrogen treatment is recommended. For supporting text, see Statement 12.15).}

\hypertarget{statement-12.15-we-suggest-health-care-professionals-not-prescribe-conjugated-estrogens-in-transgender-and-gender-diverse-people-when-estradiol-is-available-as-part-of-a-gender--affirming-hormonal-treatment.}{%
\section*{Statement 12.15: We suggest health care professionals not prescribe conjugated estrogens in transgender and gender diverse people when estradiol is available as part of a gender- affirming hormonal treatment.}\label{statement-12.15-we-suggest-health-care-professionals-not-prescribe-conjugated-estrogens-in-transgender-and-gender-diverse-people-when-estradiol-is-available-as-part-of-a-gender--affirming-hormonal-treatment.}}
\addcontentsline{toc}{section}{Statement 12.15: We suggest health care professionals not prescribe conjugated estrogens in transgender and gender diverse people when estradiol is available as part of a gender- affirming hormonal treatment.}

Determining the safest and most efficacious
estrogen compound and route of administration
for TGD people is an important topic. The recommended estrogen-based regimens are presented
in Appendix C---Table 4. The Amsterdam Medical
Center (AMC) first reported 45 events of VTE
occurring in 816 transgender women, notably an
expected incidence ratio of VTE 20-fold higher
than that reported in a reference population (van
Kesteren et al., 1997). Following this report, the
AMC clinic recommended the use of transdermal
estradiol for transgender women older than 40
years of age, which subsequently lowered the
incidence of VTE (Nota et al., 2019; Toorians
et al., 2003). Other studies suggested ethinyl
estradiol is associated with a higher risk of blood
clotting due to an increased resistance to the
anticoagulating effects of activated protein C
(APC) and elevated concentrations of the clotting
factors protein C and protein S (Toorians et al.,
2013). Other studies published within the past
15 years from other clinics reported transgender
women taking other forms of estrogen had lower
rates of VTE than transgender women taking
ethinyl estradiol (Asscheman et al., 2013).
Furthermore, a 2019 systematic review concluded
ethinyl estradiol administration was associated
with the highest risk of VTE in transgender
women, while an association between progesterone use and VTE was also identified (Goldstein
et al., 2019).

The 2017 Endocrine Society guidelines did not
recommend conjugated equine estrogens (CEEs)
as a treatment option because blood levels of
conjugated estrogens cannot be measured in
transgender women making it difficult to prevent
supraphysiologic dosing of estrogen and thereby
increasing the potential risk of VTE (Hembree
et al., 2017). A retrospective study from the UK
examined the risks of oral CEE versus oral estradiol valerate versus oral ethinyl estradiol and
found up to a 7-fold increase in the percentage
of transgender women in the oral CEE group
who developed VTE compared with transgender
women using other forms of estrogen (Seal et al.,
2012). In a nested, case-control study, over 80,000
cisgender women aged 40--79 who developed a
VTE were matched to approximately 390,000 cisgender women without VTE; the results showed
oral estradiol use had a lower risk of VTE than
conjugated estrogens, and transdermal estrogen
was not associated with an increased risk of VTE
(Vinogradova et al., 2019).

A systematic review evaluated several formulations of estrogen and identified a retrospective
and a cross-sectional study that made head-tohead comparisons of the risks associated with
different formulations (Wierckx, Mueller et al.,
2012; Wierckx et al., 2013). No identified studies
evaluating the risk of different formulations of
estrogen employed a prospective interventional
design. The retrospective study examined 214
transgender women taking transdermal estradiol
(17β-estradiol gel 1.5 mg/d or estradiol patch
50 mcg/d) or a daily intake of oral estrogens
(estradiol 2 mg/d, estriol 2 mg/d, ethinyl estradiol
50 mcg/day, or ethinyl estradiol 30--50 mcg in an
oral contraceptive) (Wierckx et al., 2013). Within
a 10-year observation period, 5\% of the cohort
developed a VTE, 1.4\% (3 of 214) experienced
a myocardial infarction (MI), and 2.3\% (5 of 214)
a transient ischemic attack or cerebrovascular
accident (TIA/CVA). The prevalence of VTE, MI
and TIA/CVA was increased following the initiation of estrogen therapy. However, the authors
did not report differences between regimens of
estrogen in terms of these endpoints.

The same group of investigators conducted a
cross-sectional study that examined 50 transgender women (mean age 43 ± 10) taking oral estrogen (estradiol valerate 2 mg/d, estriol 2 mg/d or
ethinyl estradiol 50--120 mcg/day) or using transdermal estradiol (17β-estradiol 1.5 mg/day or
estradiol 50 mcg/day) over a follow-up duration
of 9.2 years (Wierckx, Mueller et al., 2012).
Twelve percent (n = 6) developed either a VTE,
MI, or a TIA/CVA. Two of the participants were
taking conjugated estrogen 0.625 mg/d (one person in combination with cyproterone acetate), 2
participants were taking ethinyl estradiol
20--50 mcg/d, 1 was taking cyproterone acetate
50 mg/d, while the estrogen regimen used by the
sixth participant was not defined. None of the
subjects taking oral estradiol or transdermal
estradiol developed a VTE, MI, or TIA/CVA.

One prospective study examined the route of
estrogen administration in 53 transgender women
in a multicenter study carried out throughout
Europe. Transgender women younger than 45
years of age (n = 40) received estradiol valerate
4mg/d in combination with cyproterone acetate
(CPA) 50mg/d and transgender women older than
45 years of age (n = 13) received transdermal
17β-estradiol, also with CPA. No VTE, MI, or
TIA/CVA was reported after a 1-year follow-up in
either the oral or transdermal estrogen group. An
additional retrospective study from Vienna found
no occurrences of VTE among 162 transgender
women using transdermal estradiol who were followed for a mean of 5 years (Ott et al., 2010).

We are strongly confident in our recommendation against the use of ethinyl estradiol based on
historical data from the Amsterdam clinic demonstrating a reduction in the incidence of VTE after
discontinuing the use of ethinyl estradiol and the
recent systematic review demonstrating an increased
risk of VTE in transgender women taking ethinyl
estradiol (Weinand \& Safer, 2015). We are confident
in our recommendation against the use of CEE
based on the 2012 study by Seal et al.~demonstrating an increased risk of VTE in transgender women
taking CEE compared with other formulations of
estrogen and with data from cisgender women on
hormone replacement therapy (Canonico et al.,
2007; Seal et al., 2012). Prospective and retrospective studies in transgender women have reported
occurrences of VTE/MI/CVA only in those taking
CEE or ethinyl estradiol. Since estradiol is inexpensive, more widely available, and appears safer
than CEE in limited studies, the committee recommends against using CEE when estradiol is an
available treatment option. The quality of studies
may be limited to prospective, cohort or
cross-sectional study designs; however, the stronger
level of recommendation is based on the consistent
evidence supporting the association between the
use of ethinyl estradiol and CEE and a greater risk
of VTE/MI/CVA in transgender women.
We are also confident in our recommendation
for the administration of transdermal preparations of estrogen in older transgender women
(age \textgreater{} 45 years) or those with a previous history
of VTE. The confidence in our recommendation
is based on the decreased incidence of VTE
reported from the Amsterdam clinic when transgender women are switched to using transdermal
preparations after age 40 (van Kesteren et al.,
1997). Furthermore, the prospective, multicenter
cohort study ENIGI found no incidence of VTE/
MI/CVA in transgender women who are routinely
switched to transdermal estrogen at age 45
(Dekker et al., 2016). In addition, a study by Ott
et al.~demonstrated no incidence of VTE in 162
transgender women treated with estradiol patches
(Ott et al., 2010).

With the exception of cyproterone acetate (note
this is not approved for use in the US because
of concerns of potential hepatotoxicity), the use
of progestins in hormone therapy regimens
remains controversial. To date, there have been
no quality studies evaluating the role of progesterones in hormone therapy for transgender
patients.

We are aware some practitioners who prescribe
progestins, including micronized progesterone,
are under the impression there may be improvements in breast and/or areolar development,
mood, libido, and overall shape for those seeking
it along with other benefits yet to be demonstrated (Deutsch, 2016a; Wierckx, van Caenegem
et al., 2014). However, these improvements remain
anecdotal, and there are no quality data to support such progestin use. An attempted systematic
review we commissioned for this version of the
SOC failed to identify enough data to make a
recommendation in favor of any progestins.
Instead, existing data suggest harm is associated
with extended progestin exposure (Safer, 2021).

For cisgender women who have a uterus, progestins in combination with estrogens are necessary to avoid the endometrial cancer risk
associated with the administration of unopposed
estrogen. For cisgender women who do not have
a uterus, progestins are not used. The best data
for the concerns related to progestin use come
from comparisons between the above two cisgender populations, which we acknowledge is not
necessarily generalizable to this population.
Although not definitive of a class effect for all
progestins, medroxyprogesterone added to
combined equine estrogens is associated with
greater breast cancer and cardiac risks (Chlebowski
2020; Manson, 2013). It is important to note data
from the Women's Health Initiative (WHI) studies
may not be generalizable to transgender populations. Compared with the cisgender women in
the studies, transgender populations seeking hormone therapy tend to be younger, do not use
equine estrogen, and hormone therapy in these
cases address current mental health and quality
of life and not solely risk prevention
(Deutsch, 2016a).

Potential adverse effects of progestins include
weight gain, depression, and lipid changes.
Micronized progesterone may be better tolerated
and may have a more favorable impact on the
lipid profile than medroxyprogesterone (Fitzpatrick
et al., 2000). When paired with estrogens for
transgender women, the progestin cyproterone
acetate is associated with elevated prolactin,
decreased HDL cholesterol, and rare meningiomas---none of which are seen when estrogens are
paired with GnRH agonists or spironolactone
(Bisson, 2018; Borghei-Razavi, 2014; Defreyne,
Nota et al., 2017; Sofer et al., 2020).

Thus, data to date do not include quality evidence supporting a benefit of progestin therapy
for transgender women. However, the literature
does suggest a potential harm of some progestins,
at least in the setting of multi-year exposure. If,
after a discussion of the risks and benefits of progesterone treatment, there is a collaborative decision to begin a trial of progesterone therapy, the
prescriber should evaluate the patient within a year
to review the patient's response to this treatment.

\hypertarget{statement-12.16-we-recommend-health-care-professionals-prescribe-testosterone-lowering-medications-either-cyproterone-acetate-spironolactone-or-gnrh-agonists-for-eligible-transgender-and-gendered-diverse-people-with-testes-taking-estrogen-as-part-of-a-hormonal-treatment-plan-if-their-individual-goal-is-to-approximate-levels-of-circulating-sex-hormone-in-cisgender-women.}{%
\section*{Statement 12.16: We recommend health care professionals prescribe testosterone-lowering medications (either cyproterone acetate, spironolactone, or GnRH agonists) for eligible* transgender and gendered diverse people with testes taking estrogen as part of a hormonal treatment plan if their individual goal is to approximate levels of circulating sex hormone in cisgender women.}\label{statement-12.16-we-recommend-health-care-professionals-prescribe-testosterone-lowering-medications-either-cyproterone-acetate-spironolactone-or-gnrh-agonists-for-eligible-transgender-and-gendered-diverse-people-with-testes-taking-estrogen-as-part-of-a-hormonal-treatment-plan-if-their-individual-goal-is-to-approximate-levels-of-circulating-sex-hormone-in-cisgender-women.}}
\addcontentsline{toc}{section}{Statement 12.16: We recommend health care professionals prescribe testosterone-lowering medications (either cyproterone acetate, spironolactone, or GnRH agonists) for eligible* transgender and gendered diverse people with testes taking estrogen as part of a hormonal treatment plan if their individual goal is to approximate levels of circulating sex hormone in cisgender women.}

Most gender clinics in the US and Europe prescribe estrogen combined with a
testosterone-lowering medication (Mamoojee
et al., 2017) (see Appendix C---Table 5). In the
US, spironolactone is the most commonly prescribed testosterone-lowering medication, while
GnRHas are commonly used in the UK, and
cyproterone acetate are most often prescribed in
the rest of Europe (Angus et al., 2021; Kuijpers
et al., 2021). The rationale for adding a
testosterone-lowering medication is two-fold 1)
to lower testosterone levels to within the reference range of cisgender women; and 2) to reduce
the amount of estrogen needed to achieve adequate physical effects. Each testosterone-lowering
medication has a different side effect profile.
Spironolactone is an antihypertensive and
potassium-sparing diuretic, and thus may lead to
hyperkalemia, increased frequency of urination,
and a reduction in blood pressure (Lin et al.,
2021). Cyproterone acetate has been associated
with the development of meningioma and hyperprolactinemia (Nota et al., 2018). GnRHa's, while
very effective in lowering testosterone levels, can
result in osteoporosis if doses of estrogen given
concurrently are insufficient (Klink, Caris
et al., 2015).

One systematic review identified one study that
reported findings from a head-to-head comparison of the testosterone-lowering medications
cyproterone acetate and leuprolide (Gava et al.,
2016). Two studies compared a group of transgender women taking estrogen plus
testosterone-lowering medications with a group
who received only estrogen. The systematic
review did not provide sufficient evidence to suggest any of the three testosterone-lowering medications had a better safety profile in terms of
improved outcomes in bone health, testosterone
levels, potassium levels, or in the incidence of
hyperprolactinemia or meningiomas (Wilson
et al., 2020). Therefore, no recommendation can
be given. There view did report
spironolactone-based regimens were associated
with a 45\% increase in prolactin levels, whereas
cyproterone-based regimens increased prolactin
levels by more than 100\%. However, the clinical
significance of elevated prolactin levels is not
clear because the rates of prolactinomas were not
significantly elevated in either the spironolactoneor CPA-treated groups (Wilson et al., 2020). One
retrospective, cohort study from a single center
in the US reported no clinically significant
increases in prolactin levels in 100 transgender
women treated with estrogen plus spironolactone
(Bisson et al., 2018). A retrospective study from
the Netherlands of 2,555 transgender women taking primarily CPA with various formulations of
estrogen reported an increased standardized incidence ratio of meningiomas in patients who used
cyproterone acetate after gonadectomy for many
years when compared with the general Dutch
population (Nota et al., 2018). Furthermore, in
a shorter study in Belgium, 107 transgender
women had transient elevations in prolactin levels
following treatment with cyproterone acetate,
which declined to normal after discontinuation
(Defreyne, Nota et al., 2017). A recent publication, not included in the systematic review, examined 126 transgender women taking spironolactone,
GnRHas, or cyproterone and concluded cyproterone was associated with higher prolactin levels
and a worse lipid profile than spironolactone or
GnRHas (Sofer et al., 2020). After balancing the
costs and accessibility of measuring prolactin levels against the clinical significance of an elevated
level, a decision was made not to make a recommendation for or against monitoring prolactin
levels at this time. HCPs should therefore make
individualized clinical decisions about the necessity to measure prolactin levels based on the type
of hormone regimen and/or the presence of
symptoms of hyperprolactinemia or a pituitary
tumor (e.g., galactorrhea, visual field changes).

Cyproterone has also been linked to meningiomas. Nine cases of meningioma have been
reported in the literature among transgender
women primarily taking cyproterone acetate
(Mancini et al., 2018). This increased risk has also
been identified in cisgender populations. In 2020,
the European Medicines Agency published a report
recommending cyproterone products with daily
doses of 10 mg or more should be restricted
because of the risk of developing meningioma
(European Medicines Agency, 2020). Most likely
this association is a specific effect of cyproterone
acetate and has not been extrapolated to include
other testosterone-lowering drugs. In the US,
where cyproterone acetate is not available, the
North American Association of Central Cancer
Registries (NAACCRs) database did not identify
an increased risk of brain tumors (not specific to
meningiomas) among transgender women (Nash
et al., 2018). Furthermore, there was not an
increase in the hazard ratio of brain tumors in
the Kaiser cohort of 2,791 transgender women
compared with cisgender controls (Silverberg et al.,
2017). No long-term studies have reported on the
risk of meningiomas and prolactinomas in transgender women taking GnRHas.

Our strong recommendation for the use of
testosterone-lowering medications as part of a
hormone regimen for transgender individuals
with testes is based on the global practice of
using these medications in addition to estrogen
therapies as well as the relatively minimal risk
associated with these therapies. However, we are
not able to make a recommendation favoring
one testosterone-lowering medication over
another at this time. The published data thus
far raises some concerns about the risk of
meningiomas with the prolonged use (\textgreater2 years)
and higher doses (\textgreater10mg daily) of cyproterone
acetate (Nota et al., 2018; Ter Wengel et al.,
2016; Weill et al., 2021).

Bicalutamide is an antiandrogen that has been
used in the treatment of prostate cancer. It competitively binds to the androgen receptor to block
the binding of androgens. Data on the use of
bicalutamide in trans feminine populations is
very sparse and safety data is lacking. One small
study looked at the use of bicalutamide 50 mg
daily as a puberty blocker in 23 trans feminine
adolescents who could not obtain treatment with
a GnRH analogue (Neyman et al., 2019). All adolescents experienced breast development which is
also commonly seen in men with prostate cancer
who are treated with bicalutamide. Although rare,
fulminant hepatotoxicity resulting in death has
been described with bicalutamide (O'Bryant et al.,
2008). Given that bicalutamide has not been adequately studied in trans feminine populations, we
do not recommend its routine use.

The administration of 5α-reductase inhibitors
block the conversion of testosterone to the more
potent androgen dihydrotestosterone. The Food
\& Drug Administration (FDA) approved indications of finasteride administration include benign
prostatic hypertrophy and androgenetic alopecia.
Data on the use of 5α-reductase inhibitors in
trans feminine populations is very sparse (Irwig,
2021). It is unclear whether this class of medication could have any clinical benefit in trans
feminine individuals whose testosterone and dihydrotestosterone levels have already been lowered
with estrogen and an antiandrogen. We therefore
do not recommend their routine use in trans
feminine populations. Finasteride may be an
appropriate treatment option in trans masculine
individuals experiencing bothersome alopecia
resulting from higher dihydrotestosterone levels.
Nonetheless, treatment with a 5α-reductase inhibitor may impair clitoral growth and the development of facial and body hair in trans masculine
individuals. Studies are needed to assess the efficacy and safety of 5α-reductase inhibitors in
transgender populations.

\hypertarget{statement-12.17-we-recommend-health-care-professionals-monitor-hematocrit-or-hemoglobin-levels-in-transgender-and-gender-diverse-people-treated-with-testosterone.}{%
\section*{Statement 12.17: We recommend health care professionals monitor hematocrit (or hemoglobin) levels in transgender and gender diverse people treated with testosterone.}\label{statement-12.17-we-recommend-health-care-professionals-monitor-hematocrit-or-hemoglobin-levels-in-transgender-and-gender-diverse-people-treated-with-testosterone.}}
\addcontentsline{toc}{section}{Statement 12.17: We recommend health care professionals monitor hematocrit (or hemoglobin) levels in transgender and gender diverse people treated with testosterone.}

There are good quality data suggesting a rise
in hematocrit (or hemoglobin) is associated with
TGD persons treated with testosterone (Defreyne
et al., 2018). The testosterone regimens in the systematic review included testosterone esters ranging
from the equivalent of 25--250mg SC/IM weekly,
testosterone undecanoate 1000mg every 12 weeks,
or testosterone gel 50mg applied daily to the skin
(Defreyne et al., 2018; Gava et al., 2018; Giltay
et al., 2000; Meriggiola et al., 2008; Pelusi et al.,
2014; T'Sjoen et al., 2005; Wierckx, van Caenegem
et al., 2014; Wierckx, van de Peer et al., 2014).
The expected rise should be consistent with reference ranges in cisgender males.

\hypertarget{statement-12.18-we-suggest-health-care-professionals-collaborate-with-surgeons-regarding-hormone-use-before-and-after-gender-affirmation-surgery.-for-supporting-text-see-statement-12.19.}{%
\section*{Statement 12.18: We suggest health care professionals collaborate with surgeons regarding hormone use before and after gender-affirmation surgery. For supporting text, see Statement 12.19.}\label{statement-12.18-we-suggest-health-care-professionals-collaborate-with-surgeons-regarding-hormone-use-before-and-after-gender-affirmation-surgery.-for-supporting-text-see-statement-12.19.}}
\addcontentsline{toc}{section}{Statement 12.18: We suggest health care professionals collaborate with surgeons regarding hormone use before and after gender-affirmation surgery. For supporting text, see Statement 12.19.}

\hypertarget{statement-12.19-we-suggest-health-care-professionals-counsel-eligible-transgender-and-gender-diverse-people-about-the-various-options-for-gender-affirmation-surgery-unless-surgery-is-either-not-indicated-or-is-medically-contraindicated.}{%
\section*{Statement 12.19: We suggest health care professionals counsel eligible* transgender and gender diverse people about the various options for gender-affirmation surgery unless surgery is either not indicated or is medically contraindicated.}\label{statement-12.19-we-suggest-health-care-professionals-counsel-eligible-transgender-and-gender-diverse-people-about-the-various-options-for-gender-affirmation-surgery-unless-surgery-is-either-not-indicated-or-is-medically-contraindicated.}}
\addcontentsline{toc}{section}{Statement 12.19: We suggest health care professionals counsel eligible* transgender and gender diverse people about the various options for gender-affirmation surgery unless surgery is either not indicated or is medically contraindicated.}

Despite the absence of evidence, perioperative
clinical standards for gender-affirmation surgeries
have included cessation of hormone therapy for
1--4 weeks before and after surgery, most commonly genital surgeries (Hembree et al., 2009).
Such practice was meant to mitigate the risk of
VTE associated with exogenous estrogen administration (Hembree et al., 2009). Estrogen and
testosterone could then be resumed at some point
postoperatively.

After careful examination, investigators have
found no perioperative increase in the rate of
VTE among transgender individuals undergoing
surgery, while being maintained on sex steroid
treatment throughout when compared with that
among patients whose sex steroid treatment was
discontinued preoperatively (Gaither et al., 2018;
Hembree et al., 2009; Kozato et al., 2021; Prince
\& Safer, 2020). Sex steroid treatment is especially
important after gonadectomy to avoid the sequelae
of hypogonadism, the risk of developing osteoporosis, and for the maintenance of mental health
and quality of life (Fisher, Castellini et al., 2016;
Rosen et al., 2019). Thus, hormone providers and
surgeons should educate patients about the necessity for continuous exogenous hormone therapy
after gonadectomy.

To be able to educate patients and serve as
clinical advocates, HCPs should be knowledgeable
about the risks and benefits of gender-affirmation
surgeries and should also be cognizant of the
performance measures and surgical outcomes of
the surgeons to whom they might refer patients
(Beek, Kreukels et al., 2015; Colebunders et al.,
2017; Wiepjes et al., 2018). In general, most medically necessary surgeries can be thought of as
involving three regions: the face, chest/breasts,
and genitalia (internal and external). Additional
medically necessary procedures include body contouring and voice surgery. See medical necessity
statement in Chapter 2---Global Applicability,
Statement 2.1).

Multiple procedures are available for facial
gender-affirming surgeries including, but not limited to chondrolanryngoplasty, rhinoplasty, contouring or augmentation of the jaw, chin, and
forehead, facelift, hair removal and hair transplantation (see Chapter 13---Surgery and
Postoperative Care). Procedures available for
chest/breast surgery include breast augmentation,
double mastectomy with nipple grafts, periareolar
mastectomy, and liposuction. The most common
gender-affirmation surgery for TGD individuals
with endogenous breast development is masculinizing chest surgery (mastectomy) (Horbach
et al., 2015; Kailas et al., 2017).

Internal genital surgery procedures include but
are not limited to orchiectomy, hysterectomy,
salpingo-oophorectomy, vaginoplasty, and colpectomy/vaginectomy (Horbach et al., 2015; Jiang
et al., 2018). The inner lining in vaginoplasty is
typically constructed from penile skin, skin grafts,
a combination of both, or a bowel segment.
Removal of the uterus/ovaries can be performed
individually or all at once (hysterectomy,
salpingo-oophorectomy, and colpectomy). If
colpectomy is performed, a hysterectomy must
also be performed. The ovaries may remain in
situ, upon patient request. A potential benefit of
leaving one or both ovaries is fertility preservation, while the downside is the potential for the
development of ovarian pathology, including cancer (De Roo et al., 2017).

External genital surgery procedures include but
are not limited to vulvoplasty, metoidioplasty, and
phalloplasty (Djordjevic et al., 2008; Frey et al.,
2016). Hair removal is generally necessary before
performing external genital procedures (Marks
et al., 2019). Vulvoplasty can include the creation
of the mons, labia, clitoris, and urethral opening.
Urethral lengthening is an option for both
metoidioplasty and phalloplasty, but is associated
with a greatly increased complication rate
(Schechter \& Safa, 2018). Wound care and physical therapy are necessary for managing wounds
resulting from the donor sites for phalloplasty
(van Caenegem, Verhaeghe et al., 2013). Pelvic
physical therapy can also be an important adjunct
intervention after surgery for managing voiding
and sexual function (Jiang et al., 2019). Dialogue,
mutual understanding, and clear communication
in a common language between patients, HCPs,
and surgeons will contribute to well-considered
decisions about the available surgical procedures.

\hypertarget{statement-12.20-we-recommend-health-care-professionals-initiate-and-continue-gender-affirming-hormone-therapy-for-eligible-transgender-and-gender-diverse-people-who-wish-this-treatment-due-to-demonstrated-improvement-in-psychosocial-functioning-and-quality-of-life.-for-supporting-text-see-statement-12.21.}{%
\section*{Statement 12.20: We recommend health care professionals initiate and continue gender-affirming hormone therapy for eligible* transgender and gender diverse people who wish this treatment due to demonstrated improvement in psychosocial functioning and quality of life. For supporting text, see Statement 12.21.}\label{statement-12.20-we-recommend-health-care-professionals-initiate-and-continue-gender-affirming-hormone-therapy-for-eligible-transgender-and-gender-diverse-people-who-wish-this-treatment-due-to-demonstrated-improvement-in-psychosocial-functioning-and-quality-of-life.-for-supporting-text-see-statement-12.21.}}
\addcontentsline{toc}{section}{Statement 12.20: We recommend health care professionals initiate and continue gender-affirming hormone therapy for eligible* transgender and gender diverse people who wish this treatment due to demonstrated improvement in psychosocial functioning and quality of life. For supporting text, see Statement 12.21.}

\hypertarget{statement-12.21-we-recommend-health-care-professionals-maintain-existing-hormone-therapy-if-the-transgender-and-gender-diverse-individuals-mental-health-deteriorates-and-assess-the-reason-for-the-deterioration-unless-contraindicated.}{%
\section*{Statement 12.21: We recommend health care professionals maintain existing hormone therapy if the transgender and gender diverse individual's mental health deteriorates and assess the reason for the deterioration, unless contraindicated.}\label{statement-12.21-we-recommend-health-care-professionals-maintain-existing-hormone-therapy-if-the-transgender-and-gender-diverse-individuals-mental-health-deteriorates-and-assess-the-reason-for-the-deterioration-unless-contraindicated.}}
\addcontentsline{toc}{section}{Statement 12.21: We recommend health care professionals maintain existing hormone therapy if the transgender and gender diverse individual's mental health deteriorates and assess the reason for the deterioration, unless contraindicated.}

Several mental health disparities have been
documented in the transgender population
including depression, suicidality, anxiety,
decreased self-esteem, and post-traumatic stress
disorder (Arcelus et al., 2016; Becerra-Culqui et
al, 2018; Bouman et al., 2017; Eisenberg et al.,
2017; Heylens, Elaut et al., 2014; Witcomb et al.,
2018). The gender minority stress model provides evidence of several mediators and moderators of these disparities (Hendricks \& Testa,
2012; Meyer, 2003). Mediators and moderators
of mental health disparities unique to transgender people include experiences of discrimination,
victimization, misgendering, family rejection, and
internalized transphobia (Hendricks \& Testa,
2012). Factors that have a positive effect on mental health include family acceptance, supportive
social and romantic relationships, transgender
community connectedness, protection by affirming and inclusive policies, policies of affirmation
and inclusion, possession of updated legal name/
gender documentation, and achievement of physical gender transition based on individualized
embodiment goals (Bauer et al., 2015; Bockting
et al., 2013; Bouman et al., 2016; Davey et al.,
2014; de Vries et al., 2014; Du Bois et al., 2018;
Gower, Rider, Brown et al., 2018; Hendricks \&
Testa, 2012; Keo-Meier et al., 2015; Meier et al.,
2013; Pflum et al., 2015; Ryan et al., 2010; Smith
et al., 2018).

Hormone therapy has been found to positively
impact the mental health and quality of life of
TGD youth and adults who embark on this treatment (Aldridge et al., 2020; Allen et al., 2019;
Bauer et al., 2015; Nobili et al., 2018; Russell
et al., 2018; Ryan, 2009). In many cases, hormone
therapy is considered a lifesaving intervention
(Allen et al., 2019; Grossman \& D'Augelli, 2006;
Moody et al., 2015). Several studies have found
associations between the initiation of hormone
therapy and improved mental health in youth
and adults (Aldridge et al., 2020; Costa et al.,
2016; de Vries et al., 2014; Kuper et al., 2020;
Nguyen et al., 2018; White Hughto \& Reisner,
2016), including improvements in quality of life
(Gorin-Lazard et al., 2012; Gorin-Lazard et al.,
2013; Murad et al., 2010; Newfield et al., 2006;
Nobili et al., 2018; White Hughto \& Reisner,
2016), a reduction in anxiety and depression
(Aldridge et al., 2020; Colizzi et al., 2014; Davis
\& Meier, 2014; de Vries, Steensma et al., 2011;
Gómez-Gil et al., 2012; Rowniak et al., 2019),
decreased stress, and decreased paranoia
(Keo-Meier \& Fitzgerald, 2017). A prospective,
controlled trial using the Minnesota Multiphasic
Personality Inventory-2 (MMPI-2) demonstrated
significant improvement in multiple domains of
psychological functioning in transgender men
after only 3 months of testosterone treatment
(Keo-Meier et al., 2015). Although there are
higher rates of autism symptoms in the transgender population, these symptoms have not been
found to increase after the initiation of hormone
therapy (Nobili et al., 2020).

As a reduction in depressive symptoms may
correlate with a decrease in the risk of suicide,
withholding hormone therapy based on the presence of depression or suicidality may cause harm
(Keo-Meier et al., 2015; Levy et al., 2003). Turban,
King et al.~(2020) found a decrease in the odds
of lifetime suicidal ideation in adolescents who
required pubertal suppression and had access to
this treatment compared with those with a similar
desire with no such access (Turban, King et al.,
2020). A recent systematic review found pubertal
suppression in TGD adolescents was associated
with an improved social life, decreased suicidality
in adulthood, improved psychological functioning
and quality of life (Rew et al., 2020). Because evidence suggests hormone therapy is directly linked
to decreased symptoms of depression and anxiety,
the practice of withholding hormone therapy until
these symptoms are treated with traditional psychiatry is considered to have iatrogenic effects
(Keo-Meier et al., 2015). If psychiatric treatment
is indicated, it can be started or adjusted concurrently without discontinuing hormone therapy.
\emph{For eligibility criteria for adolescents and adults,
please refer to Chapter 5---Assessment for Adults and
Chapter 6---Adolescents as well as Appendix D.}

\hypertarget{surgery-and-postoperative-care}{%
\chapter{Surgery and Postoperative Care}\label{surgery-and-postoperative-care}}

Medically necessary gender-affirmation surgery
(GAS) refers to a constellation of procedures
designed to align a person's body with their gender
identity (see Chapter 2---Global Applicability for
medical necessity, Statement 2.1). This chapter
describes surgery and postoperative care recommendations for TGD adults and adolescents. Please
refer to Chapter 5---Assessment of Adults and
Chapter 6---Adolescents for the assessment criteria
related to surgery for adults and adolescents,
respectively. A summary of the recommendations
and assessment criteria can be found in Appendix D.

Recognizing the diverse and heterogeneous community of individuals who identify as transgender
and gender diverse (TGD), gender-affirming surgical
interventions may be categorized along a spectrum
of procedures for individuals assigned male at birth
(AMAB) and assigned female at birth (AFAB).

In appropriately selected TGD individuals, the
current literature supports the benefits of GAS.
While complications following GAS occur, many
are either minor or can be treated with local care
on an outpatient basis (Canner et al., 2018;
Gaither et al., 2018; Morrison et al., 2016). In
addition, complication rates are consistent with
those of similar procedures performed for different diagnoses (i.e., non-gender-affirming
procedures).

In individuals AFAB, gender-affirming chest
surgery or ``top surgery'' (i.e.~``subcutaneous mastectomy'') has been studied in prospective
(Agarwal et al., 2018; Frederick et al., 2017; Top
\& Balta, 2017; van de Grift, Elaut et al., 2017;
van de Grift et al., 2016), retrospective (Bertrand
et al., 2017; Claes et al., 2018; Esmonde et al.,
2019; Lo Russo et al., 2017; Marinkovic \&
Newfield, 2017; Poudrier et al., 2019; Wolter
et al., 2015; Wolter et al., 2018), and cross-sectional
cohort studies (Olson-Kennedy, Warus et al.,
2018; Owen-Smith et al., 2018; van de Grift, Elaut
et al., 2018; van de Grift, Elfering et al., 2018).
The efficacy of top surgery has been demonstrated in multiple domains, including a consistent and direct increase in health-related quality
of life, a significant decrease in gender dysphoria,
and a consistent increase in satisfaction with
body and appearance. Additionally, rates of regret
remain very low, varying from 0 to 4\%. While
the effect of top surgery on additional outcome
measures such as depression, anxiety, and sexual
function also demonstrated a benefit, the studies
were of insufficient strength to draw definitive
conclusions. Although further investigation is
needed to draw more robust conclusions, the evidence demonstrates top surgery to be a safe and
effective intervention.

In individuals AMAB, fewer studies have been
published regarding gender-affirming breast surgery (``breast augmentation'') and include 2 prospective (Weigert et al., 2013; Zavlin et al., 2018),
1 retrospective cohort (Fakin et al., 2019), and 3
cross-sectional cohort studies (Kanhai et al., 2000;
Owen-Smith et al., 2018; van de Grift, Elaut
et al., 2018). All the studies reported a consistent
and direct improvement in patient satisfaction,
including general satisfaction, body image satisfaction, and body image following surgery.
Owen-Smith et al.~(2018) demonstrated a positive
trend toward improvement in both depression
and anxiety scores with increasing levels of
gender-affirming interventions. However, there
was no statistical comparison between individuals
who underwent top surgery and any other group.

Gender-affirming vaginoplasty is one of the
most frequently reported gender-affirming surgical
interventions; 8 prospective (Buncamper et al.,
2017; Cardoso da Silva et al., 2016; Kanhai, 2016;
Manero Vazquez et al., 2018; Papadopulos, Zavlin
et al., 2017; Tavakkoli Tabassi et al., 2015; Wei
et al., 2018; Zavlin et al., 2018), 15 retrospective
cohort (Bouman, van der Sluis et al., 2016;
Buncamper et al., 2015; Hess et al., 2016; Jiang
et al., 2018; LeBreton et al., 2017; Manrique et al.,
2018; Massie et al., 2018; Morrison et al., 2015;
Papadopulos, Lelle et al., 2017; Raigosa et al., 2015;
Salgado et al., 2018; Seyed-Forootan et al., 2018;
Sigurjonsson et al., 2017; Simonsen et al., 2016;
Thalaivirithan et al., 2018), and 3 cross-sectional
cohort studies have recently been reported
(Castellano et al., 2015; Owen-Smith et al., 2018;
van de Grift, Elaut et al., 2018).

Although different assessment measurements
were used, the results from all studies consistently
reported both a high level of patient satisfaction
(78--100\%) as well as satisfaction with sexual
function (75--100\%). This was especially evident
when using more recent surgical techniques.
Gender-affirming vaginoplasty was also associated
with a low rate of complications and a low incidence of regret (0--8\%).

Recent literature reflects the increased clinical
interest in metoidioplasty and phalloplasty as
reflected by 3 prospective cohort (Garaffa et al.,
2010; Stojanovic et al., 2017; Vukadinovic et al.,
2014), 6 retrospective cohort (Cohanzad, 2016;
Garcia et al., 2014; Simonsen et al., 2016; van de
Grift, Pigot et al., 2017; van der Sluis et al., 2017;
Zhang et al., 2015), and 4 cross-sectional studies
(Castellano et al., 2015; Owen-Smith et al., 2018;
van de Grift, Elaut et al., 2018; Wierckx, Van
Caenegem et al., 2011), which reviewed the risks
and benefits of these procedures.

In terms of urinary function, between 75 and
100\% of study participants were able to void
while standing. In terms of sexual function,
between 77 and 95\% of study participants
reported satisfaction with their sexual function.
Most of these studies report high overall levels
of postoperative satisfaction (range 83--100\%),
with higher rates of satisfaction in studies involving newer surgical techniques. Two prospective
and two retrospective cohort studies specifically
assessed regret following surgery and found no
transgender men experienced regret. While study
limitations were identified, the reported results
were consistent and direct.

In recent years, facial GAS (FGAS) has received
increased attention, and current literature supports
its benefits. Eight recent publications include 1 prospective cohort (Morrison et al., 2020), 5 retrospective cohort (Bellinga et al., 2017; Capitán et al.,
2014; Noureai et al., 2007; Raffaini et al., 2016;
Simon et al., 2022), and 2 cross-sectional studies
(Ainsworth \& Spiegel, 2010; van de Grift, Elaut
et al., 2018). All 8 studies clearly demonstrated
individuals were very satisfied with their surgical
results (between 72\% and 100\% of individuals).
Additionally, individuals were significantly more
satisfied with the appearance of their face compared
with individuals who had not undergone surgery.
One prospective, international, multicenter, cohort
study found facial GAS significantly improves both
mid- and long-term quality of life (Morrison et al.,
2020). The results were direct and consistent, but
somewhat imprecise because of certain study limitations. While gender-affirming facial surgery for
AFAB individuals is an emerging field, current limited data points toward equal benefits in select
patients. Future studies are recommended.

Additional procedures and/or interventions
such as hair removal (prior to facial and/or genital surgery) may be required as part of the preoperative process. See Chapter 15---Primary Care.
Furthermore, consultation with pelvic floor physical therapy may be important (or required) both
before and after surgery.

\textbf{Representative surgical interventions include
(for complete list, see appendix E and the end
of this chapter):}

AMAB: facial feminization surgery (including
chondrolaryngoplasty/vocal cord surgery),
gender-affirming breast surgery, body contouring
procedures, orchiectomy, vagino/vulvoplasty
(with/without depth), aesthetic procedures, and
procedures designed to prepare individuals for
surgery (i.e., hair removal).

AFAB: facial masculinization surgery,
gender-affirming chest surgery, hysterectomy/
oophorectomy, metoidioplasty (including placement of testicular prosthesis), phalloplasty
(including placement of testicular/penile prostheses), body contouring procedures, aesthetic procedures, and procedures designed to prepare
individuals for surgery (i.e., hair removal).

It is important surgeons understand the indication(s) and the timing for GAS. This is especially important when caring for adolescents (see
Chapter 6---Adolescents).

It is important the surgeon and the patient participate in a shared decision-making approach that
includes 1) a multidisciplinary approach; 2) an
understanding of the patient's goals and
expectations; 3) a discussion regarding the surgical
options and associated risks and benefits; and 4)
an informed plan for aftercare (see Chapter 5---
Assessment for Adults). These recommendations are
designed to facilitate an individualized approach
to care.

Appropriate aftercare is essential for optimizing
outcomes (Buncamper et al., 2015; Lawrence,
2003), and it is important patients are informed
about postoperative needs (including local wound
care, activity restrictions, time off from work or
school, etc.). In addition, it is important the surgeon is available to provide and facilitate postoperative care, refer to specialty services, or both
as needed. This may include the need for ongoing
support (i.e., both from the caregiver as well as
the primary care provider, mental health professionals (MHPs), or both), as well as the need for
routine primary care (i.e., breast/chest cancer
screening, urologic/gynecologic care, etc.).

With the increase both in public interest and
in the number of gender-affirming surgical procedures (Canner et al., 2018; Ross, 2017; Shen
et al., 2019), additional training, tracking of
outcomes, and continuing medical education
for surgeons are necessar y (Schechter
et al., 2017).

All the statements in this chapter have been
recommended based on a thorough review of
evidence, an assessment of the benefits and
harms, values and preferences of providers and
patients, and resource use and feasibility. In some
cases, we recognize evidence is limited and/or
services may not be accessible or desirable.

\hypertarget{statement-13.1-we-recommend-surgeons-who-perform-gender-affirming-surgical-procedures-have-the-following-credentials}{%
\section*{Statement 13.1: We recommend surgeons who perform gender-affirming surgical procedures have the following credentials:}\label{statement-13.1-we-recommend-surgeons-who-perform-gender-affirming-surgical-procedures-have-the-following-credentials}}
\addcontentsline{toc}{section}{Statement 13.1: We recommend surgeons who perform gender-affirming surgical procedures have the following credentials:}

\textbf{a. Training and documented supervision in gender-affirming procedures;
b. Maintenance of an active practice in gender-affirming surgical procedures;
c.~Knowledge about gender diverse identities and expressions;
d.~Continuing education in the field of gender-affirmation surgery;
e. Tracking of surgical outcomes.}

Surgeons offering GAS may have a variety of
surgical specialty training and backgrounds. The
most common surgical specialties include plastic
surgery, urology, gynecology, otolaryngology and
oro-maxillofacial surgery (Jazayeri et al., 2021).
Consistent with other surgical domains, we recommend only surgeons who are certified or eligible
to be certified by their respective national professional boards offer GAS. Furthermore, it is recommended surgeons offering care for TGD people
have received documented training in
gender-affirming procedures and principles of
gender-affirming care (Schechter et al., 2017;
Schechter \& Schechter, 2019). The latter includes,
but is not limited, to knowledge about gender
diverse identities and expressions, and how those
affect patient goals, expectations, and outcomes. It
is important surgeons offering GAS be familiar with
the available procedures and can provide informed
consent. If surgeons do not offer a requested procedure, they may offer a referral for a second opinion. Surgeons offering GAS are expected to
participate in continuing education activities in the
field of GAS (i.e., meetings, conferences, seminars,
etc.) to maintain current knowledge. We further
recommend surgical outcomes be tracked and communicated to the patients as part of the informed
consent (Schechter et al., 2017).

In addition, hospitals, institutions, and physician offices that offer GAS need to be knowledgeable regarding cultural competencies (i.e.,
language, terminology, etc.). This may require
ongoing and regular staff education.

\hypertarget{statement-13.2-we-recommend-surgeons-assess-transgender-and-gender-diverse-people-for-risk-factors-associated-with-breast-cancer-prior-to-breast-augmentation-or-mastectomy.}{%
\section*{Statement 13.2: We recommend surgeons assess transgender and gender diverse people for risk factors associated with breast cancer prior to breast augmentation or mastectomy.}\label{statement-13.2-we-recommend-surgeons-assess-transgender-and-gender-diverse-people-for-risk-factors-associated-with-breast-cancer-prior-to-breast-augmentation-or-mastectomy.}}
\addcontentsline{toc}{section}{Statement 13.2: We recommend surgeons assess transgender and gender diverse people for risk factors associated with breast cancer prior to breast augmentation or mastectomy.}

Prior to breast augmentation or mastectomy,
individuals need to be informed about and
assessed for breast cancer risk factors, including
genetic mutations (i.e., BRCA1, BRCA2), family
history, age, radiation, exposure to estrogen, and
the amount of breast tissue anticipated to remain
after surgery (Brown, Lourenco et al., 2021;
Brown \& Jones, 2015; Colebunders et al., 2014;
Gooren et al., 2013; Salibian et al., 2021; Weyers
et al., 2010). Breast cancer screening balances the
identification of cancer with the selection of
appropriate imaging, tests, and procedures.
Currently, evidence-based screening guidelines
specific for TGD individuals do not exist (Salibian
et al., 2021), however, recent guidelines have been
proposed by the American College of Radiology
(Brown, Lourenco et al., 2021). Because the risk
of cancer in individuals seeking gender-affirming
breast augmentation or mastectomy is similar to
that in the general population (even in the setting
of hormone use), existing cancer screening guidelines need to be followed (Brown \& Jones, 2015;
Gooren et al., 2013; Salibian et al., 2021; Weyers
et al., 2010). Professionals need to be familiar
with updates to these guidelines as they are subject to change. Individuals who undergo
gender-affirming surgery of the chest should have
ongoing breast cancer surveillance, which should
be overseen by their primary care providers.

\hypertarget{statement-13.3-we-recommend-surgeons-inform-transgender-and-gender-diverse-people-undergoing-gender-affirming-surgical-procedures-about-aftercare-requirements-travel-and-accommodations-and-the-importance-of-postoperative-follow-up-during-the-preoperative-process.}{%
\section*{Statement 13.3: We recommend surgeons inform transgender and gender diverse people undergoing gender-affirming surgical procedures about aftercare requirements, travel and accommodations, and the importance of postoperative follow-up during the preoperative process.}\label{statement-13.3-we-recommend-surgeons-inform-transgender-and-gender-diverse-people-undergoing-gender-affirming-surgical-procedures-about-aftercare-requirements-travel-and-accommodations-and-the-importance-of-postoperative-follow-up-during-the-preoperative-process.}}
\addcontentsline{toc}{section}{Statement 13.3: We recommend surgeons inform transgender and gender diverse people undergoing gender-affirming surgical procedures about aftercare requirements, travel and accommodations, and the importance of postoperative follow-up during the preoperative process.}

Details about the timing, technique, and duration of the aftercare requirements are shared with
patients in the preoperative period such that
appropriate planning may be undertaken. This
includes a discussion regarding the anticipated
staging of surgical procedures (and associated
travel requirements). Given the small number of
surgeons who specialize in GAS, it is common
for patients to travel for their procedures. Prior
to surgery, surgeons should provide patients with
a postoperative follow-up schedule. The surgeon
should discuss the duration of the patient's travel
dates, the anticipated inpatient versus outpatient
stay, and the potential need for flexibility in travel
arrangements (especially if complications occur).
Given the complexity and cost of travel and lodging, changes in the care plan should be shared
with the patient as early as possible. Surgeons
should facilitate continuity of care with a local
provider upon returning home.

Aftercare and postsurgical follow-up are
important. Gender-affirming surgical procedures
often have specific aftercare requirements, such
as postsurgery resources (stable, safe housing;
resources for travel and follow-up care), instructions in health-positive habits (e.g., personal
hygiene, healthy living, prevention of urinary
tract infections (UTIs) and sexually-transmitted
infections (STIs) (Wierckx, Van Caenegem et al.,
2011)), postsurgery precautions or limitations
on activities of daily life (e.g., bathing, physical
activity, exercise, nutritional guidance, resumption of sexual activity) (Capitán et al., 2020),
postsurgery resumption of medications (i.e.,
anticoagulants, hormones, etc.), and detailed
postsurgery self-care activities (e.g., postvaginoplasty dilation and douching regimens, activation of a penile prosthesis, strategies to optimize
postphalloplasty urination, recommendations for
hair transplant care) (Capitán et al., 2017;
Falcone et al., 2018; Garcia, 2018; Hoebeke
et al., 2005). Some aspects of postsurgery
self-care activities may be introduced prior to
surgery and are reinforced after surgery (Falcone
et al., 2018). As issues such as wound disruptions, difficulty with dilation, and UTIs may
occur (Dy et al., 2019), the follow-up period
provides an opportunity to intervene, mitigate,
and prevent complications (Buncamper et al.,
2016; Garcia, 2021).

\hypertarget{statement-13.4-we-recommend-surgeons-confirm-reproductive-options-have-been-discussed-prior-to-gonadectomy-in-transgender-and-gender-diverse-people.}{%
\section*{Statement 13.4: We recommend surgeons confirm reproductive options have been discussed prior to gonadectomy in transgender and gender diverse people.}\label{statement-13.4-we-recommend-surgeons-confirm-reproductive-options-have-been-discussed-prior-to-gonadectomy-in-transgender-and-gender-diverse-people.}}
\addcontentsline{toc}{section}{Statement 13.4: We recommend surgeons confirm reproductive options have been discussed prior to gonadectomy in transgender and gender diverse people.}

Infertility is often a consequence of both
gender-affirming hormone therapy (temporary)
and GAS (permanent), and fertility preservation
is discussed prior to medical interventions, surgical interventions, or both (Defreyne, van
Schuylenbergh et al., 2020; Jahromi et al., 2021;
Jones et al., 2021). Surgical interventions that
alter reproductive anatomy or function may limit
future reproductive options to varying degrees
(Nahata et al., 2019). It is thus critical to discuss
infertility risk and fertility preservation (FP)
options with transgender individuals and their
families prior to initiating any of these interventions and on an ongoing basis thereafter (Hembree
et al., 2017).

For specific recommendations regarding reproductive options, see Chapter 16---Reproductive Health.

\hypertarget{statement-13.5-we-suggest-surgeons-consider-offering-gonadectomy-to-eligible-transgender-and-gender-diverse-adults-when-there-is-evidence-they-have-tolerated-a-minimum-of-6-months-of-hormone-therapy-unless-hormone-replacement-therapy-or-gonadal-suppression-is-not-clinically-indicated-or-the-procedure-is-inconsistent-with-the-patients-desires-goals-or-expressions-of-individual-gender-identity.-for-supporting-text-see-statement-13.6.}{%
\section*{Statement 13.5: We suggest surgeons consider offering gonadectomy to eligible* transgender and gender diverse adults when there is evidence they have tolerated a minimum of 6 months of hormone therapy (unless hormone replacement therapy or gonadal suppression is not clinically indicated or the procedure is inconsistent with the patient's desires, goals, or expressions of individual gender identity). For supporting text, see Statement 13.6.}\label{statement-13.5-we-suggest-surgeons-consider-offering-gonadectomy-to-eligible-transgender-and-gender-diverse-adults-when-there-is-evidence-they-have-tolerated-a-minimum-of-6-months-of-hormone-therapy-unless-hormone-replacement-therapy-or-gonadal-suppression-is-not-clinically-indicated-or-the-procedure-is-inconsistent-with-the-patients-desires-goals-or-expressions-of-individual-gender-identity.-for-supporting-text-see-statement-13.6.}}
\addcontentsline{toc}{section}{Statement 13.5: We suggest surgeons consider offering gonadectomy to eligible* transgender and gender diverse adults when there is evidence they have tolerated a minimum of 6 months of hormone therapy (unless hormone replacement therapy or gonadal suppression is not clinically indicated or the procedure is inconsistent with the patient's desires, goals, or expressions of individual gender identity). For supporting text, see Statement 13.6.}

\hypertarget{statement-13.6-we-suggest-health-care-professionals-consider-gender-affirming-genital-procedures-in-eligible-transgender-and-gender-diverse-adults-seeking-these-interventions-when-there-is-evidence-the-individual-has-been-stable-on-their-current-treatment-regime-which-may-include-at-least-6-months-of-hormone-treatment-or-a-longer-period-if-required-to-achieve-the-desired-surgical-result-unless-hormone-therapy-is-either-not-desired-or-is-medic-ally-contraindicated.}{%
\section*{Statement 13.6: We suggest health care professionals consider gender-affirming genital procedures in eligible* transgender and gender diverse adults seeking these interventions when there is evidence the individual has been stable on their current treatment regime (which may include at least 6 months of hormone treatment or a longer period if required to achieve the desired surgical result unless hormone therapy is either not desired or is medic ally contraindicated).}\label{statement-13.6-we-suggest-health-care-professionals-consider-gender-affirming-genital-procedures-in-eligible-transgender-and-gender-diverse-adults-seeking-these-interventions-when-there-is-evidence-the-individual-has-been-stable-on-their-current-treatment-regime-which-may-include-at-least-6-months-of-hormone-treatment-or-a-longer-period-if-required-to-achieve-the-desired-surgical-result-unless-hormone-therapy-is-either-not-desired-or-is-medic-ally-contraindicated.}}
\addcontentsline{toc}{section}{Statement 13.6: We suggest health care professionals consider gender-affirming genital procedures in eligible* transgender and gender diverse adults seeking these interventions when there is evidence the individual has been stable on their current treatment regime (which may include at least 6 months of hormone treatment or a longer period if required to achieve the desired surgical result unless hormone therapy is either not desired or is medic ally contraindicated).}

GAHT leads to anatomical, physiological, and
psychological changes. The onset of the anatomic effects (e.g., clitoral growth, vaginal mucosal atrophy) may begin early after the initiation
of therapy, and the peak effect is expected at
1--2 years (T'Sjoen et al., 2019). Depending upon
the surgical result required, a period of hormone
treatment may be required (e.g., sufficient clitoral virilization prior to metoidioplasty/phalloplasty) or preferred for psychological reasons,
anatomical reasons, or both (breast growth and
skin expansion prior to breast augmentation,
softening of skin and changes in facial fat distribution prior to facial GAS) (de Blok
et al., 2021).

For individuals who are not taking hormones
prior to surgical interventions, it is important
surgeons review the impact of this on the proposed surgery.

For individuals undergoing gonadectomy who
are not taking hormones, a plan for hormone
replacement can be developed with their prescribing professional prior to surgery.

\hypertarget{statement-13.7-we-recommend-surgeons-consider-genderaffirming-surgical-interventions-for-eligible-transgender-and-gender-diverse-adolescents-when-there-is-evidence-a-multidisciplinary-approach-that-includes-mental-health-and-medical-professionals-has-been-involved-in-the-decision-making-process.}{%
\section*{Statement 13.7: We recommend surgeons consider genderaffirming surgical interventions for eligible* transgender and gender diverse adolescents when there is evidence a multidisciplinary approach that includes mental health and medical professionals has been involved in the decision-making process.}\label{statement-13.7-we-recommend-surgeons-consider-genderaffirming-surgical-interventions-for-eligible-transgender-and-gender-diverse-adolescents-when-there-is-evidence-a-multidisciplinary-approach-that-includes-mental-health-and-medical-professionals-has-been-involved-in-the-decision-making-process.}}
\addcontentsline{toc}{section}{Statement 13.7: We recommend surgeons consider genderaffirming surgical interventions for eligible* transgender and gender diverse adolescents when there is evidence a multidisciplinary approach that includes mental health and medical professionals has been involved in the decision-making process.}

Substantial evidence (i.e., observational studies
(Monstrey et al., 2001; Stojanovic et al., 2017),
literature reviews and expert opinions (Esteva
de Antonio et al., 2013; Frey et al., 2017;
Hadj-Moussa et al., 2019; Pan \& Honig, 2018),
established guidelines (Byne et al., 2018; Chen,
Fuqua et al., 2016; Hembree et al., 2017; Karasic
\& Fraser, 2018; Klein, Paradise et al., 2018;
Weissler et al., 2018), and a thematic content
analysis (Gerritse et al., 2018), support the
importance of a multidisciplinary (i.e., medical,
mental health, and surgery) approach to transgender health care.

A multidisciplinary approach is especially
important in managing mental health issues if
these are experienced by a TGD person undergoing GAS (de Freitas et al., 2020; Dhejne et al.,
2016; van der Miesen et al., 2016). In addition,
primary care providers and medical specialists
can help support decisions regarding the timing
of surgery, surgical outcomes and expectations,
perioperative hormone management, and optimization of medical conditions (Elamin et al., 2010;
Hembree et al., 2017).

For specific recommendations regarding presurgical assessment in adolescents, see Chapter
6---Adolescents.

\hypertarget{statement-13.8-we-recommend-surgeons-consult-a-comprehensive-multidisciplinary-team-of-professionals-in-the-field-of-transgender-health-when-eligible-transgender-and-gender-diverse-people-request-individually-customized-previously-termed-non-standard-surgeries-as-part-of-a-gender-affirming-surgical-intervention.}{%
\section*{Statement 13.8: We recommend surgeons consult a comprehensive, multidisciplinary team of professionals in the field of transgender health when eligible* transgender and gender diverse people request individually customized (previously termed ``non-standard'') surgeries as part of a gender-affirming surgical intervention.}\label{statement-13.8-we-recommend-surgeons-consult-a-comprehensive-multidisciplinary-team-of-professionals-in-the-field-of-transgender-health-when-eligible-transgender-and-gender-diverse-people-request-individually-customized-previously-termed-non-standard-surgeries-as-part-of-a-gender-affirming-surgical-intervention.}}
\addcontentsline{toc}{section}{Statement 13.8: We recommend surgeons consult a comprehensive, multidisciplinary team of professionals in the field of transgender health when eligible* transgender and gender diverse people request individually customized (previously termed ``non-standard'') surgeries as part of a gender-affirming surgical intervention.}

Gender identities may present along a spectrum, and the expression of a person's identity
may vary quite widely amongst individuals (Beek
et al., 2015; Koehler et al., 2018). While the overall goal of a particular procedure usually includes
reduction of gender dysphoria (van de Grift,
Elaut et al., 2017) or achieving gender congruence, gender diverse presentations may lead to
individually customized surgical requests some
may consider ``non-standard'' (Beek et al., 2015;
Bizic et al., 2018). Individually customized surgical requests can be defined as 1) a procedure
that alters an individual's gender expression without necessarily aiming to express an alternative,
binary gender; 2) the ``non-standard'' combination
of well-established procedures; or 3) both.

This is designed to help counsel and inform
the patient as well as to ensure their goals can
be achieved. The patient and their surgeon need
to work together to ensure the patient's expectations are realistic and achievable, and the proposed interventions are safe and technically
feasible. The patient and their surgical team need
to engage in a shared decision-making process
(Cavanaugh et al., 2016). This informed consent
process needs to address the irreversibility of
some procedures, the newer nature of some procedures, and the limited information available
about the long-term outcomes of some
procedures.

\hypertarget{statement-13.9-we-suggest-surgeons-caring-for-transgender-men-and-gender-diverse-people-who-have-undergone-metoidioplastyphalloplasty-encourage-lifelong-urological-follow-up.}{%
\section*{Statement 13.9: We suggest surgeons caring for transgender men and gender diverse people who have undergone metoidioplasty/phalloplasty encourage lifelong urological follow-up.}\label{statement-13.9-we-suggest-surgeons-caring-for-transgender-men-and-gender-diverse-people-who-have-undergone-metoidioplastyphalloplasty-encourage-lifelong-urological-follow-up.}}
\addcontentsline{toc}{section}{Statement 13.9: We suggest surgeons caring for transgender men and gender diverse people who have undergone metoidioplasty/phalloplasty encourage lifelong urological follow-up.}

Postoperative complications following metoidioplasty/phalloplasty comprise the urinary tract
and sexual function (Kang et al., 2019; Monstrey
et al., 2009; Santucci, 2018; Schardein et al.,
2019). Reported urethral complications (related
to urethral lengthening) include urethral strictures 35--58\%, urethral fistulae 15--70\% (Monstrey
et al., 2009; Santucci, 2018; Schardein et al.,
2019), diverticulae, mucocele due to vaginal
remnant, and hair growth within the neourethra
(Berli et al., 2021; Veerman et al., 2020).
Complications related to sexual function include
limited to absent tactile and/or erogenous sensation, difficulties with orgasm function, and
complications with penile prosthetics (Kang
et al., 2019; Santucci, 2018). Penile
prosthesis-related complications are estimated to
involve infection (incidence 8--12\%),
malfunction, urethral erosion, skin extrusion,
and dislocation of its bone fixation (Falcone
et al., 2018; Kang et al., 2019; Morrison et al.,
2016). Although most urethral and prosthetic
complications occur in the immediate and intermediate postoperative period, complications can
occur at any time. Early detection may reduce
morbidity (e.g., urethral strictures resulting in
fistulae, pending erosion of a penile prosthetic
leading to infection and requiring total explant)
(Blecher et al., 2019).

Routine follow-up to assess for early evidence
of urethral stricture (or other urinary issues)
includes bladder ultrasound measurement of postvoid residual volume (to screen for and stage neourethral stricture), fluoroscopic urethrography (to
identify and stage neourethral strictures, fistulae,
and diverticulae), and cystourethroscopy to examine the urethra and bladder. TGD men may also
have routine urologic issues that need not be
related to gender transition (urinary calculi, hematuria, and genitourinary malignancies; fertility
preservation) (Sterling \& Garcia, 2020a, 2020b).

\hypertarget{statement-13.10-we-recommend-surgeons-caring-for-transgender-women-and-gender-diverse-people-who-have-undergone-vaginoplasty-encourage-follow-up-with-their-primary-surgeon-primary-care-physician-or-gynecologist.}{%
\section*{Statement 13.10: We recommend surgeons caring for transgender women and gender diverse people who have undergone vaginoplasty encourage follow-up with their primary surgeon, primary care physician, or gynecologist.}\label{statement-13.10-we-recommend-surgeons-caring-for-transgender-women-and-gender-diverse-people-who-have-undergone-vaginoplasty-encourage-follow-up-with-their-primary-surgeon-primary-care-physician-or-gynecologist.}}
\addcontentsline{toc}{section}{Statement 13.10: We recommend surgeons caring for transgender women and gender diverse people who have undergone vaginoplasty encourage follow-up with their primary surgeon, primary care physician, or gynecologist.}

Vaginoplasty is a safe procedure (Hontscharuk,
Alba, Hamidian Jahromi et al., 2021). While
complications may occur, most are self-limited
or can be treated with minor interventions
(Hontscharuk, Alba, Hamidian Jahromi et al.,
2021). Minor complications include issues such
as the formation of granulation tissue, intravaginal hair growth, delayed wound healing or
wound disruption (or both), aesthetic concerns,
and introital stenosis (Ferrando, 2020; Kloer
et al., 2021). While these complications are usually self-limited, they may impact patient
well-being after surgery. Additionally, these issues
may go either undiagnosed or may be misdiagnosed if patients are not able to access care provided by professionals with expertise in the field
of transgender health. We recommend patients
be followed by their primary surgeon in person
and at regular intervals---for example at two
weeks, three months, six months, and one year
after surgery---although more follow-up may be
indicated for some individuals.

Additional gynecologic care is conducted
throughout the TGD person's lifetime and can be
managed in many settings. A speculum exam to
check for granulation tissue, hair, and lesions can
be performed by the primary care provider, gynecologist, or GAS surgeon and may be necessary
outside of the immediate postoperative period
(Grimstad, McLaren et al., 2021; Suchak et al.,
2015; van der Sluis et al., 2020). After confirmation by laboratory testing, UTIs, STIs, and other
fluctuations in the vaginal microbiome may be
treated following relevant guidelines formulated
for cisgender populations (Hooton, 2012; Sherrard
et al., 2018). Manual prostate checks are performed based on relevant guidelines formulated
for cisgender populations via the vaginal canal,
as the prostate is located on the anterior wall of
the vagina (Carter et al., 2013).

Other complications include issues such as stenosis of the neovaginal canal, rectovaginal fistulae, and inflammation (intestinal vaginoplasty)
(Bustos et al., 2021). These require a combination
of nonsurgical and surgical treatment with consultation and possible referral back to the primary surgeon with other surgical consultants (i.e.,
colorectal surgeon), if required. In addition, as
pelvic floor dysfunction may affect 30--40\% of
patients both prior to and following vaginoplasty,
the availability of pelvic floor physical therapists
is an important adjunct in the postoperative
period (Jiang et al., 2019).

\hypertarget{statement-13.11-we-recommend-patients-who-regret-their-gender-related-surgical-intervention-be-managed-by-an-expert-multidisciplinary-team.}{%
\section*{Statement 13.11: We recommend patients who regret their gender-related surgical intervention be managed by an expert multidisciplinary team.}\label{statement-13.11-we-recommend-patients-who-regret-their-gender-related-surgical-intervention-be-managed-by-an-expert-multidisciplinary-team.}}
\addcontentsline{toc}{section}{Statement 13.11: We recommend patients who regret their gender-related surgical intervention be managed by an expert multidisciplinary team.}

The percentage of individuals who regret their
GAS is very low (between 0.3\% and 3.8\%) (De
Cuypere \& Vercruysse, 2009; Defreyne, Motmans
et al., 2017; Hadj-Moussa et al., 2019;
Hadj-Moussa, Agarwal et al., 2018; Hadj-Moussa,
Ohl et al., 2018; Landén et al., 1998; Narayan
et al., 2021; van de Grift, Elaut et al., 2018;
Wiepjes et al., 2018). The highest incidence of
regret was reported at a time when surgical techniques were less refined, the role of multidisciplinary care was less established, and the
Standards of Care did not exist or were not
widely known (Landén et al., 1998). Regret can
be temporarily or permanent and may be classified as (Narayan et al., 2021) social regret (caused
by difficulties in familial, religious, social, or professional life), medical regret (due to long-term
medical complications, disappointment in surgical
results or inadequate preoperative
decision-making), and true gender-related regret
(mostly based on patient experienced misdiagnosis, insufficient exploration of gender identity, or
both). This classification is in accordance with
previously discussed positive and negative
predictive factors (De Cuypere \& Vercruysse,
2009; Gils \& Brewaeys, 2007; Pfäfflin \&
Junge, 1998).

A multidisciplinary team can help identify the
etiology of regret as well as the temporal stability
of the surgical request (Narayan et al., 2021).
Following this evaluation and in consideration of
the individual's circumstances, medical and/or surgical interventions with the intent of either continuing transition or performing surgical procedures to
return anatomy to that of the sex assigned at birth
may be indicated. For further information see
Chapter 5---Assessment of Adults.

\emph{For eligibility criteria for adolescents and
adults, please refer to the Assessment for Adults
and Adolescent chapters and Appendix D.}

\hypertarget{gender-affirming-surgical-procedures}{%
\section*{GENDER-AFFIRMING SURGICAL PROCEDURES}\label{gender-affirming-surgical-procedures}}
\addcontentsline{toc}{section}{GENDER-AFFIRMING SURGICAL PROCEDURES}

As the field's understanding of the many facets
of gender incongruence expands, and as technology develops which allows for additional treatments, it is imperative to understand this list is
not intended to be exhaustive. This is particularly
important given the often lengthy time periods
between updates to the SOC, during which evolutions in understanding and treatment modalities
may occur.

FACIAL SURGERY
Brow • Brow reduction
• Brow augmentation
• Brow lift
Hair line advancement and/or hair transplant
Facelift/mid-face lift (following alteration of the underlying skeletal
structures)
Facelift/mid-face lift (following alteration of the underlying skeletal
structures)
• Platysmaplasty
Blepharoplasty • Lipofilling
Rhinoplasty (+/- fillers)
Cheek • Implant
• Lipofilling
Lip • Upper lip shortening
• Lip augmentation (includes autologous and non-autologous)
Lower jaw • Reduction of mandibular angle
• Augmentation
Chin reshaping • Osteoplastic
• Alloplastic (implant-based)
Chondrolaryngoplasty • Vocal cord surgery (see voice chapter)
BREAST/CHEST SURGERY
Mastectomy • Mastectomy with nipple-areola preservation/reconstruction as determined
medically necessary for the specific patient
• Mastectomy without nipple-areola preservation/reconstruction as
determined medically necessary for the specific patient
Liposuction
Breast reconstruction (augmentation) • Implant and/or tissue expander
• Autologous (includes flap-based and lipofilling)
GENITAL SURGERY
Phalloplasty (with/without scrotoplasty) • With/without urethral lengthening
• With/without prosthesis (penile and/or testicular)
• With/without colpectomy/colpocleisis
Metoidioplasty (with/without scrotoplasty) • With/without urethral lengthening
• With/without prosthesis (penile and/or testicular)
• With/without colpectomy/colpocleisis
Vaginoplasty (inversion, peritoneal, intestinal) • May include retention of penis and/or testicle
Vulvoplasty • May include procedures described as ``flat front''
GONADECTOMY
Orchiectomy
Hysterectomy and/or salpingo-oophorectomy
BODY CONTOURING
Liposuction
Lipofilling
Implants • Pectoral, hip, gluteal, calf
Monsplasty/mons reduction
ADDITIONAL PROCEDURES
Hair removal: Hair removal from the face, body, and genital areas
for gender affirmation or as part of a preoperative preparation
process. (see Statement 15.14 regarding hair removal)
• Electrolysis
• Laser epilation
Tattoo (i.e., nipple-areola)
Uterine transplantation
Penile transplantation

\hypertarget{voice-and-communication}{%
\chapter{Voice and Communication}\label{voice-and-communication}}

Human beings engage in communication practices not only to exchange ideas about the outside
world, but also to present themselves as sociocultural beings and to negotiate forms of address,
referral and treatment by others that allow them
to feel safe and respected (Azul et al., 2022). The
human voice is widely regarded as one of the
key modalities that contributes to the communication of gender as one of the dimensions of
human diversity. However, other aspects and ways
of communicating (e.g., articulation, word choice,
gesture, listener perceptions and attributions)
need to be considered as well (Azul, 2015; Azul
\& Hancock, 2020). Throughout this chapter
``voice and communication'' is used as a phrase
encompassing the meaning-making practices in
which each of the participants of a social encounter engage according to their own needs, wishes,
identifications, and capacities.

While a binary understanding of gender has
dominated the research literature in this area, the
approach recommended in this chapter implies
a broadly inclusive view of gender identification
(e.g., trans feminine, trans masculine, gender
fluid, nonbinary, genderqueer, agender) and the
understanding that gender does not exist in isolation, but intersects with other aspects of human
diversity (e.g., First Nation status, ethnicity/race,
sexuality, dis/ability, faith/religion/spirituality).
The recommendations in this chapter apply to
all transgender and gender diverse (TGD) people
who are seeking professional voice and communication support, including children, adolescents,
younger and older adults, and people who wish
to transition or detransition, irrespective of their
intervention choices.

Not every TGD person experiences challenges
with or wants professional support for their voice
and communication, but those who do often
encounter barriers in accessing care. Although
the percentages vary by country and TGD subpopulation, the statistics support the concern
TGD people are not able to access voice and
communication services when and how they
desire (Eyssel et al., 2017; James et al., 2016;
Oğuz et al., 2021; Södersten et al., 2015; Veale
et al., 2019). In these studies, the percentage of
TGD people wishing to receive voice and
communication training or voice surgery is generally higher than the percentage of people who
have undergone these interventions. With few
exceptions, access to voice training is usually
greater than access to voice surgery. Groups of
TGD people who are further marginalized in
their societies, such as TGD people of marginalized race/ethnicity, experience discrimination
and limited access to care at even greater rates
(James et al., 2016; Xavier et al., 2005).

Cost, not knowing where to access services,
and services not being available are amongst the
most common barriers cited by research participants. According to studies in the US (Hancock
\& Downs, 2021; Kennedy \& Thibeault, 2020),
Turkey (Oğuz et al., 2021), and Aotearoa/New
Zealand (Veale et al., 2019), lack of accurate
information about options for voice and communication services among TGD people is a significant and ubiquitous barrier to care. Notably, in
Sweden, all TGD people are offered support for
their voice and communication when a diagnosis
of gender incongruence is made (Södersten et al.,
2015). Additionally, cultural responsiveness of
providers is only slowly improving (Hancock \&
Haskin, 2015; Jakomin et al., 2020; Matthews
et al., 2020; Sawyer et al., 2014). Hancock and
Downs (2021) have conducted preliminary work
to identify specific barriers to voice and communication services and develop effective means for
eliminating them.

This chapter is intended to provide guidance
for health care professionals (HCPs) to support
and foster well-being in all TGD people who are
experiencing challenges or distress regarding their
own voice and communication practices and/or
regarding responses and attributions they receive
from others (Azul et al., 2022).

A number of different approaches TGD people
can use to modify their voice and communication, either individually or in combination include
self-initiated change, which may be supported by
resources TGD people use to guide their voice
use and communication practice; behavioral
change supported by voice and communication
specialists (hereafter referred to as ``voice and
communication training''); and change as a result
of androgen hormonal treatment and/or laryngeal
surgery. The currently existing research evidence
does not include self-initiated change, but is
focused on the latter three approaches.

A ``voice and communication specialist'' is
someone who has knowledge regarding the ongoing and dynamic agency of speaker and listener
practices, relevant professional interventions
including behavioral, hormonal, and surgical, and
relevant processes related to biophysiology, sociocultural meaning-making, and external material
forces (Azul \& Hancock, 2020). This specialist is
capable of conducting appropriate assessments to
inform the TGD person's choice and support the
exploration of goals and intervention options by
providing guidance in a culturally responsive,
person-centered approach. This specialist has
knowledge and skills in behavioral voice and
communication intervention approaches.

Practices amenable to behavioral change
include: speaking and singing voice, mindfulness,
relaxation, respiration, pitch and pitch range,
voice quality, resonance/timbre, loudness, projection, facial expression, gesture, posture, movement, introducing self to others, describing
identifications and requesting culturally responsive treatment and forms of address by others,
assertive and resilient responses to misattributions, practicing implementation of voice use and
communication practices with different people
and in different everyday settings (e.g., Hancock
\& Siegfriedt, 2020; Mills \& Stoneham, 2017).

Voice and communication services are offered
as part of a complete and coordinated approach
to health, including support for medical, psychological, and social needs (Södersten et al., 2015);
however, there are no prerequisites (e.g., hormone
use, pursuit of surgeries, or duration living in a
gender role). The overall purposes of voice and
communication support for TGD people are:

\begin{itemize}
\tightlist
\item
  To educate clients about the factors that influence functional voice and communication practices and the communication of the speaker's identity (speaker, listener, professional practices, external material, biophysiological, and sociocultural factors);
\item
  To enable clients to communicate their sense of sociocultural belonging (e.g., in terms of gender) in everyday encounters in a manner that matches the client's desired self-presentation and to develop, maintain and habituate voices, vocal qualities, and communication practices that support the clients' goals in a manner that does not harm the voice production mechanism;
\item
  To provide training in functional voice production for clients who present with restrictions of voice function (e.g., as a result of overextending their voice production mechanism);
\item
  To support clients with developing the capacity to assertively negotiate desired forms of address and referral from others (e.g., names, pronouns, titles) and to respond to misattributions in a skillful manner that contributes to increasing and maintaining the client's well-being;
\item
  To support clients to develop the problem-solving skills needed to manage anxiety, stress, and dysphoria in collaboration with mental health providers; and to navigate barriers to practice or real-life use of one's preferred voice and communication.
\item
  To provide, or refer clients to, supportive resources that facilitate developing voice and communication skills, vocal awareness, and well-being.
\item
  To refer clients to, or collaborate with, other specialists such as mental health practitioners, laryngeal surgeons, and endocrinologists, who may be more equipped to meet the specific needs of that client. This may be especially relevant in cases where clients face unique challenges due to multiple barriers to their health and well-being or when the client wishes to pursue laryngeal surgery or hormone therapy.
\end{itemize}

Two types of laryngeal surgeries are relevant for
TGD populations: those for raising voice pitch
(e.g., glottoplasty with retro-displacement of the
anterior commissure, cricothyroid approximation
(CTA), feminization laryngoplasty, laser-assisted
voice adjustment (LAVA)) (Anderson, 2007;
Anderson, 2014; Brown, 2000; Casado, 2017;
Geneid, 2015; Gross, 1999; Kelly et al., 2018;
Kanagalingam, 2005; Kim, 2017; Kim, 2020; Kocak,
2010; Kunachak, 2000; Mastronikolis, 2013;
Mastronikolis et al., 2013; Matai, 2003; Meister,
2017; Mora, 2018; Neumann, 2004; Nuyen et al.,
2022; Orloff, 2006; Pickuth, 2000; Remacle, 2011;
Thomas \& MacMillan, 2013; Tschan, 2016; Van
Borsel, 2008; Wagner, 2003; Wendler, 1990; Yang,
2002) and for lowering voice pitch (e.g., thyroplasty type III, vocal fold injection augmentation)
(Bultynck et al, 2020; Isshiki et al., 1983; Kojima,
et al.~2008; Webb et al., 2021). Reported acoustic
benefits of pitch-raising surgery include increased
voice pitch (average frequency (fo)) and increased
Min fo(the lowest frequency in physiological voice
range). TGD people's self-rating ratings show general satisfaction with voice postsurgery, although
individuals who are interested in more comprehensive changes to vocal self-presentation may
need to engage in behavioral interventions with a
voice and communication specialist in addition to
laryngeal surgery (Brown, Chang et al.~2021; Kelly
et al., 2018; Nuyen et al., 2022). Potential harms
of pitch-raising surgery can be assessed and
addressed in voice training by a voice and communication specialist. Reported harms of
pitch-raising surgery include voice problems such
as dysphonia, weak voice, restricted speaking voice
range especially upper range (lowered Max fo, in
the physiological voice range), hoarseness, vocal
instability, and lowering of frequency values over
time (Kelly et al., 2018; Song \& Jiang, 2017),
although the rate of these outcomes is inconsistent.

Research on pitch-lowering surgeries is limited. However, studies including eight TGD people who elected to undergo thyroplasty type III
after continued dissatisfaction with hormonal
treatment (Bultynck et al., 2020) and one person
who received injection augmentation after testosterone therapy and voice training (Webb
et al., 2020), reported statistically significant
lowering of fundamental frequency, perceived
as pitch.

Estrogen treatment in TGD people has not
been associated with measurable voice changes
(Mészáros et al., 2005), while testosterone treatment in TGD people has been found to result
in both desired and undesired changes in genderand function-related aspects of voice production
(Azul, 2015; Azul et al., 2017, 2018, 2020; Azul
\& Neuschaefer-Rube, 2019; Cosyns et al., 2014;
Damrose, 2008; Deuster, Di Vicenzo et al., 2016;
Deuster, Matulat et al.~2016; Hancock et al., 2017;
Irwig et al., 2017; Nygren et al., 2016; Van Borsel
et al., 2000; Yanagi et al., 2015; Ziegler et al.,
2018). Desired changes associated with testosterone treatment include lowered voice pitch,
increased male attributions to voice, and increased
satisfaction with voice. Reported dissatisfaction
with testosterone treatment include lack of or
insufficient lowering of voice pitch, dysphonia,
weak voice, restricted singing pitch range, and
vocal instability. These areas can be assessed and
addressed in voice training by a voice and communication specialist.

All the statements in this chapter have been
recommended based on a thorough review of
evidence, an assessment of the benefits and
harms, values and preferences of providers and
patients, and resource use and feasibility. In some
cases, we recognize evidence is limited and/or
services may not be accessible or desirable.

\hypertarget{statement-14.1-we-recommend-voice-and-communication-specialists-assess-current-and-desired-vocal-and-communication-function-of-transgender-and-gender-diverse-people-and-develop-appropriate-intervention-plans-for-those-dissatisfied-with-their-voice-and-communication.}{%
\section*{Statement 14.1: We recommend voice and communication specialists assess current and desired vocal and communication function of transgender and gender diverse people and develop appropriate intervention plans for those dissatisfied with their voice and communication.}\label{statement-14.1-we-recommend-voice-and-communication-specialists-assess-current-and-desired-vocal-and-communication-function-of-transgender-and-gender-diverse-people-and-develop-appropriate-intervention-plans-for-those-dissatisfied-with-their-voice-and-communication.}}
\addcontentsline{toc}{section}{Statement 14.1: We recommend voice and communication specialists assess current and desired vocal and communication function of transgender and gender diverse people and develop appropriate intervention plans for those dissatisfied with their voice and communication.}

Voice and communication specialists may
assess satisfaction with the presentation of sociocultural positionings in communicative encounters, including gender and other intersecting
identifications, taking into consideration that
these may or may not be static over time; attributions received from others, and how these
relate to the individual's identifications, wishes,
and well-being; ratings of voice and speech naturalness; and voice and communication function
in relation to vocal demands. Assessments may
vary in nature (e.g., client-reported outcome measures, perceptual, acoustic, aerodynamic, endoscopic) according to their purpose (Davies et al.,
2015; Leyns et al., 2021; Oates \& Dacakis, 1983).
For example, laryngeal visualization is used when
individuals present with a concomitant voice
problem, (e.g., muscle tension dysphonia) (Palmer
et al., 2011) or experience voice difficulties, which
may or may not be secondary to medical
gender-affirming interventions of androgen therapy or laryngeal surgery (Azul et al., 2017).

Voice and communication specialists inform
intervention-seeking TGD people who are dissatisfied with their voice and communication about
available interventions that support TGD people
with their voice, communication, and well-being.
The nature of each option, including potential
outcomes and permanence, is presented objectively to provide the TGD person respect and
autonomy in decision-making. Appropriate intervention plans are individualized and feasible and
should be inclusive of any professional services
available. Goals may evolve over the course of
the support period as the TGD person explores
modifications to voice and communication,
assesses their satisfaction with achieved change
and refines their goals.

\hypertarget{statement-14.2-we-recommend-voice-and-communication-specialists-working-with-transgender-and-gender-diverse-people-receive-specific-education-to-develop-expertise-in-supporting-vocal-functioning-communication-and-well-being-in-this-population.}{%
\section*{Statement 14.2: We recommend voice and communication specialists working with transgender and gender diverse people receive specific education to develop expertise in supporting vocal functioning, communication, and well-being in this population.}\label{statement-14.2-we-recommend-voice-and-communication-specialists-working-with-transgender-and-gender-diverse-people-receive-specific-education-to-develop-expertise-in-supporting-vocal-functioning-communication-and-well-being-in-this-population.}}
\addcontentsline{toc}{section}{Statement 14.2: We recommend voice and communication specialists working with transgender and gender diverse people receive specific education to develop expertise in supporting vocal functioning, communication, and well-being in this population.}

Academic and licensing credentials of voice and
communication specialists (e.g., speech-language
pathologists, speech therapists, singing voice teachers, voice coaches) vary by location but typically
do not specify criteria for working with specific
populations. Standard curricula in formal education for these professions often do not include
specific or adequate training for working with
TGD populations (Jakomin et al., 2020; Matthews
et al., 2020). General knowledge and skills related
to the vocal mechanism and interpersonal communication are foundational but insufficient for
conducting culturally responsive, person-centered
care for TGD people that is effective, efficient,
inclusive, and accessible (Hancock, 2017; Russell
\& Abrams, 2019).

Professionals in this area should receive comprehensive education that invites them to develop
self-awareness, cultural humility, and cultural
responsiveness in order to be respectful of and
attentive to gender diversity and other aspects
of a client's identifications that can take a variety
of forms and imply a range of different support
needs (Azul, 2015; Azul et al., 2022). Client
preferences for use of names, formal forms of
address, gender entry, and pronouns need to be
respected in all communication with and about
the client (including medical records, reports,
emails). Education also needs to inform the setting up of a training space or clinic and administrative practices that are designed to be
welcoming to TGD people and allow TGD people to feel safe and respected when raising concerns or issues with the voice and communication
support team.

Voice and communication specialists working
with TGD people will need working knowledge
of applicable intervention principles, mechanisms,
and effectiveness, competence in teaching and
modeling voice and communication modification
skills, and a basic understanding of transgender
health, including hormonal and surgical treatments and trans-specific psychosocial issues.
Education needs to include methodologies and
practices that have been developed within TGD
communities and shown to be effective and
should ideally be presented by or in collaboration
with TGD people with lived experience of voice
and communication support.

\hypertarget{statement-14.3-we-recommend-health-care-professionals-in-transgender-health-working-with-transgender-and-gender-diverse-people-who-are-dissatisfied-with-their-voice-or-communication-consider-offering-a-referral-to-voice-and-communication-specialists-for-voice-related-support-assessment-and-training.}{%
\section*{Statement 14.3: We recommend health care professionals in transgender health working with transgender and gender diverse people who are dissatisfied with their voice or communication consider offering a referral to voice and communication specialists for voice-related support, assessment, and training.}\label{statement-14.3-we-recommend-health-care-professionals-in-transgender-health-working-with-transgender-and-gender-diverse-people-who-are-dissatisfied-with-their-voice-or-communication-consider-offering-a-referral-to-voice-and-communication-specialists-for-voice-related-support-assessment-and-training.}}
\addcontentsline{toc}{section}{Statement 14.3: We recommend health care professionals in transgender health working with transgender and gender diverse people who are dissatisfied with their voice or communication consider offering a referral to voice and communication specialists for voice-related support, assessment, and training.}

A voice and communication specialist is well
positioned to provide information and guidance
to the TGD person expressing dissatisfaction
with their voice or communication when available. There is evidence voice and communication specialists provide support in such a way
that a client's satisfaction with voice and communication can be achieved, thereby reducing
gender dysphoria and improving
communication-related quality of life (Azul,
2016; Block, 2017; Deuster, Di Vincenzo et al.,
2016; Hancock, 2017; Hancock et al., 2011;
Hardy et al., 2013; Kelly et al., 2018; McNamara,
2007; McNeill et al., 2008; Owen \& Hancock,
2010; Pasricha et al., 2008; Söderpalm et al.,
2004; Watt et al., 2018).

There is empirical evidence that behavioral
voice support for TGD AMAB people is effective
with regard to achieving the targeted voice
changes (Oates, 2019). Seven studies prior to
2020 provide empirical evidence for the effectiveness of voice training, although it is somewhat
weak (Carew et al., 2007; Dacakis, 2000; Gelfer
\& Tice, 2013; Hancock et al., 2011;Hancock \&
Garabedian, 2013; McNeill et al., 2008; Mészáros
et al., 2005). Voice training methods across these
seven studies were similar and indicated voice
training can be effective at increasing average
fundamental frequency (average pitch), fundamental frequency range (pitch range), satisfaction
with voice, self-perception and listener perception
of vocal femininity, voice-related quality of life,
and social and vocational participation.
Weaknesses of the identified studies include lack
of randomized controlled trials evaluating voice
training, small sample sizes, inadequate long-term
follow-up, and lack of control of confounding
variables. In 2021, another systematic review of
the effects of behavioral speech training for
AMAB people reached similar conclusions (Leyns
et al., 2021).

Until recently, there was almost no research
exploring the effectiveness of voice training with
TGD AFAB people. There is, however, some
promising, although weak evidence of effectiveness from a case study (Buckley et al., 2020) and
one uncontrolled prospective study of group voice
training (Mills et al., 2019).

\hypertarget{statement-14.4-we-recommend-health-care-professionals-working-with-transgender-and-gender-diverse-people-who-are-considering-undergoing-voice-surgery-consider-offering-a-referral-to-a-voice-and-communication-specialist-who-can-provide-pre--and-or-postoperative-support.}{%
\section*{Statement 14.4: We recommend health care professionals working with transgender and gender diverse people who are considering undergoing voice surgery consider offering a referral to a voice and communication specialist who can provide pre- and/ or postoperative support.}\label{statement-14.4-we-recommend-health-care-professionals-working-with-transgender-and-gender-diverse-people-who-are-considering-undergoing-voice-surgery-consider-offering-a-referral-to-a-voice-and-communication-specialist-who-can-provide-pre--and-or-postoperative-support.}}
\addcontentsline{toc}{section}{Statement 14.4: We recommend health care professionals working with transgender and gender diverse people who are considering undergoing voice surgery consider offering a referral to a voice and communication specialist who can provide pre- and/ or postoperative support.}

This statement does not intend to require TGD
people receive presurgical voice training. Rather,
it is recommended that every available support
be offered to provide individualized informational
counseling critical to person-centered care. The
recommendation is for the TGD person's consideration to be informed as necessary by individualized informational counseling based on voice
assessment, trial voice training, and discussion
of expected voice outcomes and risks of surgery
with a voice and communication specialist.

For most types of laryngeal surgery, voice
training is recommended both prior to surgery
to ensure preparation of the vocal mechanism
for the surgical intervention and postsurgery to
ensure a return to functional voice production
(Branski et al., 2006; Park et al., 2021). For
pitch-raising surgery in particular, another reason
a trial of voice training is recommended is
because there are indications certain measures
improve with training but not with pitch-raising
surgery (e.g., factors relevant to intonation and
naturalness, such as maximum f0 pitch in speech
range; Kelly et al., 2018).

The number and quality of research studies
evaluating pitch-lowering surgeries are currently
insufficient, particularly with regard to comparing
outcomes with and without other interventions
(i.e., testosterone) (Bultynck et al., 2020). There
are more techniques and studies of pitch-raising
surgeries, but the quality of the evidence is still
low. Outcomes from pitch-raising surgeries have
been compared to outcomes from having no surgery (Anderson, 2007, 2014; Brown et al., 2000;
Geneid et al., 2015; Gross, 1999; Kim, 2017;
Kocak et al., 2010; Kunachak et al., 2000; Matai
et al., 2003; Meister et al., 2017; Neumann \&
Welzel, 2004; Orloff et al., 2006; Pickuth et al.,
2000; Remacle et al., 2011; Thomas \& Macmillan,
2013; Tschan et al., 2016; Van Borsel et al., 2008;
Yang et al., 2002), another type of surgical technique (Mora, 2018), voice training alone
(Kanagalingam, 2005; Mastronikolis, 2013;
Wagner, 2003) and surgery in conjunction with
voice training (Casado, 2017; Kelly et al., 2018).

In the 11 studies reporting whether participants had voice training prior to pitch-raising
surgery, most participants had prior voice training, but remained dissatisfied with voice and
sought surgical intervention. Thus, most studies
of surgical outcomes reflect the combined effects
of voice training and surgical intervention.
Attributes predicting which clients will pursue
surgery after training are unknown.

\hypertarget{statement-14.5-we-recommend-health-care-professionals-in-transgender-health-inform-transgender-and-gender-diverse-people-commencing-testosterone-therapy-of-the-potential-and-variable-effects-of-this-treatment-on-voice-and-communication.}{%
\section*{Statement 14.5: We recommend health care professionals in transgender health inform transgender and gender diverse people commencing testosterone therapy of the potential and variable effects of this treatment on voice and communication.}\label{statement-14.5-we-recommend-health-care-professionals-in-transgender-health-inform-transgender-and-gender-diverse-people-commencing-testosterone-therapy-of-the-potential-and-variable-effects-of-this-treatment-on-voice-and-communication.}}
\addcontentsline{toc}{section}{Statement 14.5: We recommend health care professionals in transgender health inform transgender and gender diverse people commencing testosterone therapy of the potential and variable effects of this treatment on voice and communication.}

The research on the effects of androgen treatment on voice and communication of TGD people points to diverse and unpredictable effects
on individual clients. While a number of studies
have revealed effects on voice that matched TGD
people's expectations and wishes, there is high
quality evidence demonstrating TGD people are
not always satisfied with the vocal outcomes of
testosterone therapy, and many experience difficulties such as inadequate pitch lowering, compromised voice quality, vocal loudness, vocal
endurance, pitch range, and flexibility (Azul,
2015, 2016, 2017, 2018; Cosyns et al., 2014;
Nygren et al., 2016; Ziegler et al., 2018). A
recent meta-analysis of 19 studies examining the
effects of at least 1 year of testosterone therapy
estimated 21\% of participants did not achieve
cisgender male normative frequencies, 21\% of
participants reported incomplete voice-gender
congruence and voice problems, and 16\% were
not completely satisfied with their voice
(Ziegler, 2018).

For people who wish to be treated with androgens, accurate informational counseling prior to
commencing treatment should enable the development of realistic expectations to avoid disappointment regarding the permanent impact of
hormone treatment on voice and communication. In addition, TGD people who do not have
access to or do not wish to be treated with testosterone, but want to change their voice and
those who are dissatisfied with the outcomes of
testosterone treatment can be advised by a voice
and communication specialist of alternative and
additional support options (e.g., behavioral voice
and communication training; pitch-lowering
surgery).

\hypertarget{primary-care}{%
\chapter{Primary Care}\label{primary-care}}

Primary care is the broadest of health care disciplines and is defined as the ``provision of integrated, accessible health care services by health
care professionals who are accountable for
addressing a large majority of personal health
care needs, developing a sustained partnership
with patients, and practicing in the context of
family and community.'' (Institute of
Medicine, 1996).

Primary care providers (PCPs) encompass a
wide range of health care professionals (HCPs)
who deliver this care, including general and family medical practitioners, nurse practitioners,
advanced practice nurses, physician associates/
assistants, and internists. PCPs are represented
by a variety of educational backgrounds, training,
and specialties. Given the type of degree and the
nature of the specialty, the scope of practice varies, and not all providers may be trained or qualified to directly provide the full breadth of
transgender health care, such as mental health,
genital/pelvic care, or postoperative care, following gender-affirming procedures. Physicians and
other providers receive little education in transgender and gender-diverse (TGD) health at any
time during their training (Dubin et al., 2018),
and thus most skills are currently acquired in
practice, either informally or through brief continuing education opportunities, see also Chapter
4---Education. However, if providers are competent to deliver similar care for cisgender patients,
they should develop competency in caring for
TGD patients. The competencies outlined below
are all to be understood as being within the provider's scope of licensure and practice. However,
all PCPs should be able to manage the comprehensive health of TGD patients either directly or
by appropriate referral to other HCPs, including
other specialists, for evaluation and treatment.
There is no evidence competency in caring for
TGD patients can only be achieved through a
formal or certification process. In explicitly stating recommended competencies, however, PCP's
and TGD persons across all settings can share a
standard set of expectations of the knowledge,
skills, and cultural competence required for the
care of TGD persons.

Due to the unique medical, surgical, and social
conditions faced by TGD people, PCPs need distinct competencies in the care of TGD persons,
apart from what is expected of all PCP's who
may otherwise care for a diverse population that
includes ethnic, racial, or sexual minorities.
Professional bodies from a range of generalist
disciplines have issued position statements and
guidelines specific to the care of TGD people
(American College of Obstetricians and
Gynecology, 2021; Italian Society of Gender,
Identity and Health (SIGIS); the Italian Society
of Andrology and Sexual Medicine (SIAMS); the
Italian Society of Endocrinology (SIE), 2021;
Polish Sexological Society, 2021; the Southern
African HIV Clinicians' Society, 2021). Wylie
et al.~(2016) state ``For the most part, the general
health and well-being of transgender people
should be attended to within the primary care
setting, without differentiation from services
offered to cisgender (non-transgender) people for
physical, psychological, and sexual health issues.
Specific care for gender transition is also possible
in primary care.'' There are many examples of
these services being provided safely and effectively outside of specialist care in diverse cities
such as Toronto and Vancouver in Canada, New
York and Boston in the US, and in Sydney,
Australia, (Radix \& Eisfeld, 2014; Reisner, Radix
et al., 2016; Spanos et al., 2021).

\hypertarget{hormone-therapy-1}{%
\section*{Hormone therapy}\label{hormone-therapy-1}}
\addcontentsline{toc}{section}{Hormone therapy}

Whether TGD patients receive medically necessary gender-affirming hormone therapy (GAHT)
from a specialist, e.g., an endocrinologist, or a
PCP may depend on the availability of knowledgeable and welcoming providers and
country-level factors, such as health care regulations and health services funding (see medically
necessary statement in Chapter 2---Global
Applicability, Statement 2.1). In much of the
world, specialty services for TGD people are
partly or wholly unavailable, which reinforces the
need for all health providers to undertake
training in the provision of gender-affirming care.
In some countries, PCPs may be required to refer
TGD patients to specialist services (e.g., gender
identity clinics) resulting in unacceptable delays
to access GAHT (Royal College of General
Practitioners, 2019).

Hormone-related therapy encompasses a range
of interventions, such as puberty suppression and
hormone initiation or hormone maintenance.
With training, gender-affirming hormone therapy
can be managed by most PCPs. Regardless of
whether they serve as the primary hormone prescriber, all PCPs should be familiar with the
medications, suggested monitoring, and potential
side effects associated with GAHT (see Chapter
12---Hormone Therapy). PCPs should be able to
make appropriate referrals to appropriate providers for all transition-related services they do not
themselves provide.

This chapter supports the argument GAHT can
be prescribed by PCPs or other
non-specialists---``Considering barriers to health
care access and the importance of GAHT to this
population, it is imperative that PCPs are able
and willing to provide GAHT for TGD patients.''
(Shires, 2017).

PCPs are commonly called upon to provide
care for a broad range of conditions and needs,
including those with which they may have had
limited or no prior experience. Often this involves
accessing commonly used and readily available
reference sources, such as professional society
guidelines or obtaining a subscription to online
knowledge bases. PCPs are advised to use a similar approach when asked to provide basic GAHT
care by using the Standards of Care as well as
other readily accessed resources (Cheung et al.,
2019; Hembree et al., 2017; Oliphant et al., 2018;
T'Sjoen et al., 2020). It should be noted most of
the commonly used medications in genderaffirming regimens are familiar to everyday primary care practice, including, but not limited to,
testosterone, estradiol, progesterone and other
progestagens, and spironolactone.

\hypertarget{mental-health}{%
\section*{Mental health}\label{mental-health}}
\addcontentsline{toc}{section}{Mental health}

PCPs should be able and willing to assess and
provide mental health support for TGD
people and GAHT that can alleviate gender
dysphoria and allow gender expression. At the
very least, they should be aware of these
needs and consult additional specialty support
if needed.

\hypertarget{preventive-care}{%
\section*{Preventive care}\label{preventive-care}}
\addcontentsline{toc}{section}{Preventive care}

General practitioners are versed to provide comprehensive primary and secondary cancer prevention as a part of routine primary care.
Evidence-based cancer prevention guidelines
vary globally due to differences in national
guidelines and levels of access to screening
modalities at the local level. To date, research
on the long-term impact of GAHT on cancer
risk is limited (Blondeel et al., 2016; Braun
et al., 2017). We have insufficient evidence to
estimate the prevalence of cancer of the breast
or reproductive organs among TGD populations
(Joint et al., 2018). However, cancer screening
should commence, in general, according to local
guidelines. Several modifications are discussed
in detail, below, depending on the type and
duration of hormone use, surgical intervention,
or both. In caring for transgender patients, the
PCP should maintain an updated record of
which organs are present in TGD patients so
that appropriate, routine screening can be
offered.

This organ inventory should be updated based
on the surgical history or any development that
has occurred due to taking gender-affirming hormones. Not all PCP's provide care across the
lifespan. However, if providers routinely care for
children, adolescents, or elder cisgender persons,
they should develop competency in transgender
care that is applicable to these age groups. If they
are unable to do so, then PCPs should be able
to make appropriate referrals to other HCPs who
care for these populations.

All the statements in this chapter have been
recommended based on a thorough review of
evidence, an assessment of the benefits and
harms, values and preferences of providers and
patients, and resource use and feasibility. In
some cases, we recognize evidence is limited
and/or services may not be accessible or
desirable.

\hypertarget{statement-15.1-we-recommend-health-care-professionals-obtain-a-detailed-medical-history-from-transgender-and-gender-diverse-people-that-includes-past-and-present-use-of-hormones-gonadal-surgeries-as-well-as-the-presence-of-traditional-cardiovascular-and-cerebrovascular-risk-factors-with-the-aim-of-providing-regular-cardiovascular-risk-assessment-according-to-established-locally-used-guidelines.-for-supporting-text-see-statement-15.3.}{%
\section*{Statement 15.1: We recommend health care professionals obtain a detailed medical history from transgender and gender diverse people, that includes past and present use of hormones, gonadal surgeries, as well as the presence of traditional cardiovascular and cerebrovascular risk factors with the aim of providing regular cardiovascular risk assessment according to established, locally used guidelines. For supporting text, see Statement 15.3.}\label{statement-15.1-we-recommend-health-care-professionals-obtain-a-detailed-medical-history-from-transgender-and-gender-diverse-people-that-includes-past-and-present-use-of-hormones-gonadal-surgeries-as-well-as-the-presence-of-traditional-cardiovascular-and-cerebrovascular-risk-factors-with-the-aim-of-providing-regular-cardiovascular-risk-assessment-according-to-established-locally-used-guidelines.-for-supporting-text-see-statement-15.3.}}
\addcontentsline{toc}{section}{Statement 15.1: We recommend health care professionals obtain a detailed medical history from transgender and gender diverse people, that includes past and present use of hormones, gonadal surgeries, as well as the presence of traditional cardiovascular and cerebrovascular risk factors with the aim of providing regular cardiovascular risk assessment according to established, locally used guidelines. For supporting text, see Statement 15.3.}

\hypertarget{statement-15.2-we-recommend-health-care-professionals-assess-and-manage-cardiovascular-health-in-transgender-and-gender-diverse-people-using-a-tailored-risk-factor-assessment-and-cardiovascularcerebrovascular-management-methods.-for-supporting-text-see-statement-15.3.}{%
\section*{Statement 15.2: We recommend health care professionals assess and manage cardiovascular health in transgender and gender diverse people using a tailored risk factor assessment and cardiovascular/cerebrovascular management methods. For supporting text, see Statement 15.3.}\label{statement-15.2-we-recommend-health-care-professionals-assess-and-manage-cardiovascular-health-in-transgender-and-gender-diverse-people-using-a-tailored-risk-factor-assessment-and-cardiovascularcerebrovascular-management-methods.-for-supporting-text-see-statement-15.3.}}
\addcontentsline{toc}{section}{Statement 15.2: We recommend health care professionals assess and manage cardiovascular health in transgender and gender diverse people using a tailored risk factor assessment and cardiovascular/cerebrovascular management methods. For supporting text, see Statement 15.3.}

\hypertarget{statement-15.3-we-recommend-health-care-professionals-tailor-sex-based-risk-calculators-used-for-assessing-medical-conditions-to-the-needs-of-transgender-and-gender-diverse-people-taking-into-consideration-the-length-of-hormone-use-dosing-serum-hormone-levels-current-age-and-the-age-at-which-hormone-therapy-was-initiated.}{%
\section*{Statement 15.3: We recommend health care professionals tailor sex-based risk calculators used for assessing medical conditions to the needs of transgender and gender diverse people, taking into consideration the length of hormone use, dosing, serum hormone levels, current age, and the age at which hormone therapy was initiated.}\label{statement-15.3-we-recommend-health-care-professionals-tailor-sex-based-risk-calculators-used-for-assessing-medical-conditions-to-the-needs-of-transgender-and-gender-diverse-people-taking-into-consideration-the-length-of-hormone-use-dosing-serum-hormone-levels-current-age-and-the-age-at-which-hormone-therapy-was-initiated.}}
\addcontentsline{toc}{section}{Statement 15.3: We recommend health care professionals tailor sex-based risk calculators used for assessing medical conditions to the needs of transgender and gender diverse people, taking into consideration the length of hormone use, dosing, serum hormone levels, current age, and the age at which hormone therapy was initiated.}

Cardiovascular disease (CVD) and stroke are
the leading causes of mortality worldwide (World
Health Organization, 2018). Extensive data among
racial, ethnic, and sexual minorities in multiple
settings demonstrate significant disparities in the
prevalence of CVD and its risk factors as well as
in the outcomes to medical interventions.
Structural factors such as access to care, socioeconomic status, and allostatic load related to
minority stress contribute to these disparities
(Flentje et al., 2020; Havranek et al., 2015; Streed
et al., 2021). TGD people often experience social,
economic, and discriminatory conditions similar
to other minority populations with known
increased cardiovascular risk (Carpenter et al.,
2020; James et al., 2016; Reisner, Radix et al.,
2016). TGD persons of racial, ethnic, and sexual
minorities have been shown to experience
increased impact related to intersectional stress.
Conversely, access to gender-affirming care,
including GAHT, may buffer against the elevation
of CVD risk due to the improvement in quality
of life and reduction in gender dysphoria and
incongruence (Defreyne et al., 2019; Martinez
et al., 2018). PCPs can significantly improve TGD
health through screening and prevention of CVD
and its associated risk conditions---such as
tobacco use, diabetes mellitus, hypertension, dyslipidemia, and obesity.

The few, primarily US based, studies evaluating
the prevalence of CVD, stroke, or CVD risk in
TGD persons independent of GAHT indicate an
elevated CV risk, including high rates of undiagnosed and untreated CV risk factors with inadequate CV prevention when compared with
cisgender populations (Denby et al., 2021;
Malhotra et al., 2022; Nokoff et al., 2018). In one
population-based study, TGD people had greater
odds of discrimination, psychological distress, and
adverse childhood experience, and these were
associated with increased odds of having a cardiovascular condition (Poteat et al., 2021).

In US studies that are based on data from the
Behavioral Risk Factor Surveillance System, both
transgender men and transgender women show a
higher prevalence of myocardial infarction (MI),
stroke, or any CVD compared with cisgender men,
cisgender women or both. Results vary based on
the adjustment of data for additional variables,
including race, income, or cardiovascular risk factors (Alzahrani et al., 2019; Caceres et al., 2020;
Nokoff et al., 2018). Gender nonbinary persons
also have higher odds of CVD (Downing \&
Przedworski, 2018). Data on hormone use was not
collected in these studies, which are also limited
by the use of self-reported health histories. In the
US, TGD individuals presenting for GAHT may
have higher rates of undiagnosed and untreated
CVD risk factors compared with the cisgender
population (Denby et al., 2021), although this may
not be applicable globally.

A large 2018 case control study from several
US centers that used 10:1 cisgender matched controls found no statistically significant difference
in rates of MI or stroke between transgender
women and cisgender men, and no difference in
rates of MI, stroke, or venous thromboembolism
(VTE) between transgender men and cisgender
men or women. There was a statistically significant hazard ratio of 1.9 for VTE among transgender women when compared with cisgender
men. A subcohort of transgender women who
initiated GAHT during (versus prior to) the
6-year study window did show an increased risk
of stroke. Increases in rates of VTE in the overall
cohort of transgender women and in rates of
stroke in the initiation subcohort of transgender
women demonstrated calculated
numbers-needed-to-harm (not reported in the
paper) between 71-123 (Getahun et al., 2018).
Other studies have demonstrated no increase in
CV events or stroke among transgender men
undergoing testosterone therapy, although studies
are limited by their small sample size, relatively
short follow-up, and the younger age of the sample population (Martinez et al., 2020; Nota
et al., 2019).

European and US studies in transgender
women who have accessed feminizing GAHT
increasingly indicate a higher risk of CVD, stroke,
or both, compared with cisgender women and,
in some studies, cisgender men (Getahun et al.,
2018; Nota et al., 2019; Wierckx et al., 2013).
Many of these studies had significant limitations,
such as variably adjusting for CV-related risk
factors, small sample sizes---especially involving
older transgender women---and variable duration
and types of GAHT (Connelly et al., 2019;
Defreyne et al., 2019, Martinez et al., 2020).
Furthermore, the overall increased risk was small.
In many of these studies, the majority of transgender women who experienced cardiac events
or stroke were over 50 years old, had one or
more CVD risk factors, and were taking a variety
of hormone regimens, including, but not limited,
to ethinyl estradiol, a synthetic estrogen that confers significant elevations in thrombotic risk and
is not recommended for use in feminizing regimens (Gooren et al., 2014; Martinez et al., 2020).
Current limited evidence suggests estrogen-based
GAHT is associated with an increased risk of
myocardial infarction and stroke, but whether
this small risk is a result of GAHT or an effect
of pre-existing CV risk is unclear. There are no
known studies that specifically address CVD and
International Journal of Transgender Health S147
related conditions in nonbinary individuals, individuals who use subphysiologic doses of
gender-affirming hormones, or in adults previously treated with puberty suppression.

PCPs can best address CVD risk during GAHT
by assessing TGD people for CVD and modifiable
CVD risk factors, such as diabetes mellitus, hypertension, hyperlipidemia, obesity, and smoking, as
well as by addressing the impact of minority stress
on cardiovascular risk (Streed et al., 2021). In
addition, PCPs can mitigate transgender cardiovascular health disparities by providing a timely
diagnosis and treatment of risk conditions and by
tailoring their management in a way that supports
ongoing gender-affirming interventions.

Risk assessment guidelines vary based on the
national or international context and scientific
affiliation of guideline developers. CVD prevention guidelines also vary in terms of the nature
and frequency of the risk assessment for otherwise healthy adults under age 40 (Arnett et al.,
2019; Piepoli et al., 2020; Précoma et al., 2019;
Streed et al., 2021; WHO, 2007). Over age 40,
when cardiovascular risk increases, guidelines
clearly recommend scheduled risk assessments
using a calculated prediction of ten-year total
CVD risk based on risk prediction equations
from large population samples. Examples of risk
calculators include SCORE (recommended by the
European Guidelines on CVD Prevention), Pooled
Cohort Studies Equations (2013 AHA ACC
Guideline on the Assessment of CVD risk),
Framingham Risk scores, and the World Health
Organization (WHO) Risk Prediction Charts. The
WHO charts were developed based on information from the countries in each WHO subregion.
In many low resource settings, facilities are not
available to measure cholesterol or serum glucose,
and alternative predication charts are available
without these measures.

Of note, all current cardiovascular risk calculators are gendered, using sex as a significant risk
variable. There is currently insufficient data on
cardiovascular risk interventions across the lifespan in TGD persons with medical and surgical
interventions to adjust these predictive equations.
Nonetheless, it is clear both sex assigned at birth
and medical transition can affect the parameters
used to calculate cardiovascular risk (Connelly
et al., 2019; Defreyne et al., 2019; Maraka et al.,
2017; Martinez et al., 2020). Providers can take
a variety of approaches to using cardiovascular
risk calculators in TGD persons, including
employing the risk calculator for the sex assigned
at birth, affirmed gender, or a weighted average
of the two, taking into consideration total lifetime
exposure to GAHT. Although data are lacking,
using the affirmed gender for transgender adults
with a history of pubertal-age GAHT initiations
is likely to be most appropriate. Patients with a
history of submaximal GAHT use or prolonged
periods of time postgonadectomy without hormone replacement before roughly age 50 may
require an even more nuanced approach. Providers
should be aware of the characteristics and limitations of the risk calculator in use and should
engage patients in shared decision-making regarding these specific considerations.

There are currently no studies comparing the
prevalence of dyslipidemia between transgender
and cisgender samples, while controlling for hormone use. As noted previously, data in other
populations demonstrate the presence of psychosocial stress during childhood and remote adulthood favor adiposity and abnormal lipid
metabolism. Both testosterone- and estrogen-based
GAHT affect lipid metabolism, although evidence
is limited by the variety of hormone regimens
and additional variables (Connelly et al., 2019;
Defreyne et al., 2019; Deutsch, Glidden et al.,
2015; Maraka et al., 2017; Martinez et al., 2020;).
On balance, estrogen tends to increase
high-density lipoprotein (HDL) cholesterol and
triglycerides with variable effects on low density
lipoprotein (LDL) cholesterol, while testosterone
variably affects triglycerides, decreases HDL cholesterol and increases LDL cholesterol. The
method of administration may also affect this
pattern, particularly in relation to oral versus
transdermal estrogen and their impact on triglycerides (Maraka et al., 2017). In general, the
effect sizes of these differences are minimal, and
the overall impact on cardio- and cerebrovascular
outcomes is unclear. There are no studies examining hormone effects in TGD people with
pre-existing dyslipidemia with hormone use starting over age 50, or investigating effects beyond
2-5 years of therapy.

Studies comparing the prevalence of hypertension between TGD and cisgender samples that
controlled for hormone use are lacking. Data in
other populations demonstrate chronic and acute
psychosocial stress, including experiences of discrimination can mediate hypertension
(Din-Dzietham et al., 2004; Spruill, 2010). In US
studies that were based on the Behavioral Risk
Factor Surveillance System, a large national US
health survey, there were no differences in
reported hypertension between transgender men
or women compared with cisgender samples
(Alzahrani et al., 2019; Nokoff et al., 2018).

Studies of testosterone---and estrogen-based
GAHT have shown inconsistent effects on systolic
and diastolic blood pressure. A retrospective
study of the effects of estrogen- and testosteronebased GAHT regimens on blood pressure found
a slight reduction in systolic blood pressure with
the initiation of estrogen-based regimens; while
there was a slight elevation (4 mm Hg) in mean
systolic blood pressure on long term follow-up
of testosterone-based regimens, this difference
was at the margin of statistical significance and
of limited clinical relevance (Banks et al., 2021).
A systematic review concluded, given the limited
quality of the studies, there is insufficient data
to reach conclusions on the effects of
gender-affirming hormone therapy on blood pressure (Connelly et al., 2021). Spironolactone, often
used as an androgen blocker in feminizing GAHT,
is a potassium sparing diuretic and may increase
potassium when used in conjunction with ACE
inhibitors or angiotensin receptor blocker medications, as well as salt substitutes. There are no
studies examining hormone effects in TGD people with pre-existing hypertension with hormone
use starting over age 50, or investigating effects
beyond 2--5 years of therapy. Transgender persons
receiving GAHT should undergo any additional
blood pressure screening or monitoring indicated
by WPATH guidelines for GAHT.

There are limited data comparing the prevalence
of diabetes mellitus between TGD and cisgender
samples independent of hormone use. Recent data
from the STRONG cohort study (Islam et al.,
2021) found the prevalence and incidence of type
2 diabetes was more common in the trans feminine cohort compared with cisgender females but
not cisgender male controls. No significant differences in the prevalence or incidence of type 2
diabetes were observed in the trans masculine
cohort and in TGD persons overall after starting
hormone therapy. However, the mean follow-up
for both cohorts was 2.8 and 3.1 years, respectively
(Islam et al., 2021). Data in other populations,
including sexual minorities, indicates chronic and
acute psychosocial stress can mediate the development and control of type 2 diabetes (Beach
et al., 2018; Kelly \& Mubarak, 2015).

US studies based on the Behavioral Risk Factor
Surveillance System found no differences in
reported diabetes between transgender men,
transgender women and nonbinary persons compared with cisgender persons (Alzahrani et al.,
2019; Caceres et al., 2020; Nokoff et al., 2018).
Several small studies have shown a higher-than-expected prevalence of polycystic ovarian syndrome
or hyperandrogenemia among transgender men
(Feldman et al., 2016), conditions associated with
insulin resistance and diabetes risk. While studies
of both testosterone- and estrogen-based GAHT
show varying effects on weight/body fat, glucose
metabolism, and insulin resistance (Defreyne
et al., 2019), most do not demonstrate any
increase in prediabetes or diabetes (Chan et al.,
2018; Connelly et al., 2019). There are no studies
examining hormone effects in TGD people with
pre-existing diabetes, with hormone use starting
over age 50, or investigating effects beyond 2--5
years of therapy. There are currently no studies
specifically addressing diabetes in adults previously treated with puberty suppression.

While intermediate-outcome studies of the
effects of GAHT on blood pressure and lipids
are helpful for hypothesis generation and for
studying etiology, future studies should focus on
cardiovascular outcomes of interest, with a specific focus on individual predictors such as age,
route and dose of hormones used, and total lifetime exposure to GAHT. Interpretation of data
should always consider whether cisgender controls were of the same natal sex or identified gender.

\hypertarget{statement-15.4-we-recommend-health-care-professionals-counsel-transgender-and-gender-diverse-people-about-their-tobacco-use-and-advise-tobacconicotine-abstinence-prior-to-gender-affirming-surgery.}{%
\section*{Statement 15.4: We recommend health care professionals counsel transgender and gender diverse people about their tobacco use and advise tobacco/nicotine abstinence prior to gender-affirming surgery.}\label{statement-15.4-we-recommend-health-care-professionals-counsel-transgender-and-gender-diverse-people-about-their-tobacco-use-and-advise-tobacconicotine-abstinence-prior-to-gender-affirming-surgery.}}
\addcontentsline{toc}{section}{Statement 15.4: We recommend health care professionals counsel transgender and gender diverse people about their tobacco use and advise tobacco/nicotine abstinence prior to gender-affirming surgery.}

Tobacco use is a leading contributor to cardiovascular disease, pulmonary disease, and cancer worldwide (World Health Organization, 2020).
TGD persons have a higher prevalence of tobacco
use compared with cisgender individuals, which
varies across the gender spectrum (Azagba et al.,
2019; Buchting et al., 2017). This pattern is consistent with other populations experiencing
minority stress (Gordon et al., 2021). PCPs can
promote protective factors against tobacco use,
including reducing exposure to personal or structural discrimination, having gender-affirming
identification, and having health insurance (Kidd
et al., 2018; Shires \& Jafee, 2016).

The health risks of tobacco use affect TGD
persons disproportionately, primarily due to
decreased access to culturally competent, affordable screening, and treatment of tobacco-related
diseases (Shires \& Jafee, 2016). Smoking may
further increase cardiovascular and VTE risk for
TGD individuals taking feminizing GAHT
(Hontscharuk, Alba, Manno et al., 2021). Smoking
also doubles or triples the risk of general surgery
complications, such as wound healing, scarring,
and infection (Yoong et al., 2020) and increases
these risks for those accessing gender-affirming
surgeries. Data in cisgender populations show
quitting smoking prior to surgery and maintaining abstinence for six weeks postoperatively significantly reduces complications (Yoong
et al., 2020).

There are currently few studies of smoking
cessation programs specifically focused on TGD
persons (Berger \& Mooney-Somers, 2017).
However, limited evidence suggests PCPs can
enhance smoking cessation efforts by addressing
the effects of minority stress (Gamarel et al.,
2015) and incorporating gender-affirming interventions, such as GAHT (Myers \& Safer, 2016).

HCPs should take into consideration the significant barriers people habituated to nicotine
encounter when attempting cessation. Nicotine
replacement therapy and/or other cessation
adjuncts should be made available, with an emphasis on individual preferences and a recognition of
underlying behavioral health factors that contribute
to continued nicotine use. Decision-making
regarding approaches to GAHT or surgery should
include consideration of the ``first do no harm''
principle of medical practice, with the realities of
an individual patient's abilities and needs.

\hypertarget{statement-15.5-we-recommend-health-care-professionals-discuss-and-address-aging-related-psychological-medical-and-social-concerns-with-transgender-and-gender-diverse-people.}{%
\section*{Statement 15.5: We recommend health care professionals discuss and address aging-related psychological, medical, and social concerns with transgender and gender diverse people.}\label{statement-15.5-we-recommend-health-care-professionals-discuss-and-address-aging-related-psychological-medical-and-social-concerns-with-transgender-and-gender-diverse-people.}}
\addcontentsline{toc}{section}{Statement 15.5: We recommend health care professionals discuss and address aging-related psychological, medical, and social concerns with transgender and gender diverse people.}

Aging presents specific social, physical, and
mental health challenges for TGD persons. While
the literature on aging and transgender elders is
limited, many older TGD adults have experienced
a lifetime of stigma, discrimination, and repression of identified gender (Fabbre \& Gaveras,
2020; Witten, 2017). This experience affects TGD
elders' interactions with health care systems
(Fredriksen-Goldsen et al., 2014; Kattari \&
Hasche, 2016; Walker et al., 2017). Transgender
elders are more likely than cisgender LGB peers
to report poor physical health, even when controlling for socio-demographic factors
(Fredriksen-Goldsen 2011; Fredriksen-Goldsen
et al., 2014). Reduced access to culturally competent care and the sequelae of minority stress
often result in delayed care, potentially exacerbating chronic conditions common with aging
(Bakko \& Kattari, 2021; Fredriksen-Goldsen
et al., 2014).

Although there are few studies on
gender-affirming medical interventions among
TGD elders, evidence suggests older adults experience a significantly higher quality of life with
medical transition even when compared with
younger TGD adults (Cai et al., 2019). Although
age itself is not an absolute contraindication or
limitation to gender-affirming medical or surgical
interventions, TGD elders may not be aware of
the current range of social, medical or surgical
options available that can help them meet their
individual needs (Hardacker et al., 2019;
Houlberg, 2019).

While studies on mental health among TGD
elders are limited, those over age fifty experience
significantly higher rates of depressive symptoms
and perceived stress compared with cisgender
LGB and heterosexual older adults
(Fredriksen-Goldsen 2011, Fredriksen-
Goldsen et al., 2014). Risk factors specific to
TGD elders include gender- and age-related discrimination, general stress, identity concealment,
victimization, and internalized stigma, while
social support and community belonging appear
protective (Fredriksen-Goldsen et al., 2014;
Hoy-Ellis \& Fredriksen-Goldsen, 2017; White
Hughto \& Reisner, 2018). PCPs can assist patients
by encouraging spirituality, self-acceptance and
self-advocacy, and an active healthy lifestyle, all
of which are associated with resilience and successful aging (McFadden et al., 2013; Witten, 2014).

TGD elders often face social isolation, loss of
support systems, and disconnection from close
friends and children (Fredriksen-Goldsen 2011;
Witten, 2017). The most common aging concerns
among TGD persons are losing the ability to care
for themselves followed by having to go into a
nursing home or assisted living facility (Henry
et al., 2020). While long-term care settings offer
the helpful needed assistance, they also have the
potential for physical or emotional abuse, for denial
of GAHT and routine care, for being ``outed,'' and
being prevented from living and dressing according
to one's affirmed gender (Auldridge et al., 2012;
Pang et al., 2019; Porter et al., 2016). TGD elders
identify senior housing, transportation, social
events, support groups as being the most needed
services (Auldridge et al., 2012; Witten, 2014).

Despite barriers, most TGD persons engage in
successful aging strengthened by self-acceptance, caring relationships, and advocacy (Fredriksen-Goldsen
2011; Witten, 2014). PCPs should address core health
issues facing TGD elders, including mental health,
gender-affirming medical interventions, social support, and end of life/long-term care.

Beyond the independent impact of factors such
as minority stress and social determinants of
health in later years, data are lacking on specific
health issues facing transgender people who use
GAHT later in life, individuals who began GAHT
at a younger age, and those seeking to continue
or begin GAHT in their sixth, seventh, eighth,
or later decades. With an increasing proportion
of transgender people beginning GAHT at
younger ages, including some who begin at the
time of puberty, studies to examine the impact
of decades of such treatment on long-term health
are ever more important.

\hypertarget{statement-15.6-we-recommend-health-care-professionals-follow-local-breast-cancer-screening-guidelines-developed-for-cisgender-women-in-their-care-of-transgender-and-gender-diverse-people-who-have-received-estrogens-taking-into-consideration-length-of-time-of-hormone-use-dosing-current-age-and-the-age-at-which-hormones-were-initiated.}{%
\section*{Statement 15.6: We recommend health care professionals follow local breast cancer screening guidelines developed for cisgender women in their care of transgender and gender diverse people who have received estrogens, taking into consideration length of time of hormone use, dosing, current age, and the age at which hormones were initiated.}\label{statement-15.6-we-recommend-health-care-professionals-follow-local-breast-cancer-screening-guidelines-developed-for-cisgender-women-in-their-care-of-transgender-and-gender-diverse-people-who-have-received-estrogens-taking-into-consideration-length-of-time-of-hormone-use-dosing-current-age-and-the-age-at-which-hormones-were-initiated.}}
\addcontentsline{toc}{section}{Statement 15.6: We recommend health care professionals follow local breast cancer screening guidelines developed for cisgender women in their care of transgender and gender diverse people who have received estrogens, taking into consideration length of time of hormone use, dosing, current age, and the age at which hormones were initiated.}

TGD individuals taking estrogen-based GAHT
will develop breasts, and therefore warrant consideration for breast cancer screening. Exogenous
estrogen may be one of multiple factors that contribute to breast cancer risk in cisgender people.
Two cohort studies have been published evaluating breast cancer prevalence among transgender
women in the Netherlands (Gooren et al., 2013)
and the US (Brown \& Jones, 2015). Both were
retrospective cohorts of clinical samples using a
diagnosis of breast cancer as the outcome of
interest and cisgender controls as a comparison
group. Neither study involved prospective screening for breast cancer, and both had significant
methodological limitations. Numerous guidelines
have been published (Deutsch, 2016a) recommending some combination of ``age plus length
of estrogen exposure'' as the determinant of need
to commence screening. These recommendations
are based on expert consensus only and are evidentiarily weak.

BRCA1 and 2 mutations increase the risk of
breast cancer, however the role sex hormone
exposure plays, if any, in this increased risk is
unclear (Rebbeck et al., 2005) The degree of
increase in risk, if any, from gender-affirming
estrogen therapy is unknown. Patients with a
known BRCA1 mutation should be counseled
about the unknowns and shared decision-making
with informed consent should occur between the
patient and provider, recognizing the numerous
benefits of GAHT.

Breast cancer screening among transgender
women should also take into consideration the
likelihood that a transgender woman's breasts may
be denser on mammography. Dense breasts, a
history of injecting breasts with fillers such as
silicone, and breast implants may complicate the
interpretation of mammographic findings
(Sonnenblick et al., 2018). Therefore, special
International Journal of Transgender Health S151
techniques should be used accordingly. People
who have injected particles such as silicone or
other fillers for breast augmentation may also
develop complications, such as sclerosing lipogranulomas, which obscure normal tissue on mammography or ultrasound.

\hypertarget{statement-15.7-we-recommend-health-care-professionals-follow-local-breast-cancer-screening-guidelines-developed-for-cisgender-women-in-their-care-of-transgender-and-gender-diverse-people-with-breasts-from-natal-puberty-who-have-not-had-gender-affirming-chest-surgery.}{%
\section*{Statement 15.7: We recommend health care professionals follow local breast cancer screening guidelines developed for cisgender women in their care of transgender and gender diverse people with breasts from natal puberty who have not had gender-affirming chest surgery.}\label{statement-15.7-we-recommend-health-care-professionals-follow-local-breast-cancer-screening-guidelines-developed-for-cisgender-women-in-their-care-of-transgender-and-gender-diverse-people-with-breasts-from-natal-puberty-who-have-not-had-gender-affirming-chest-surgery.}}
\addcontentsline{toc}{section}{Statement 15.7: We recommend health care professionals follow local breast cancer screening guidelines developed for cisgender women in their care of transgender and gender diverse people with breasts from natal puberty who have not had gender-affirming chest surgery.}

For TGD people assigned female at birth and
who developed breasts via natal puberty, there are
theoretical concerns about whether direct exposure
to testosterone and exposure to aromatized estrogen resulting from testosterone therapy are risk
factors for the development of breast cancer.
Limited retrospective data has not demonstrated
increased risk for breast cancer among transgender
men (Gooren et al., 2013; Grynberg et al., 2010),
however prospective and comparison data are
lacking. Most people in this group will have some
breast tissue remaining, and therefore it is important for providers to be aware breast cancer risk
is not zero in this population. The timing and
approach to breast cancer screening in this group
who have had chest surgery is currently not established, and, similar to cisgender men with significant family history or BRCA gene mutation,
screening via MRI or ultrasound may be appropriate. Because the utility and performance of
these approaches have not been studied and
because self- and HCP-led chest/breast screening
exams are not recommended in cisgender women
due to potential harms of both false-positive
results and over-detection (detection of a cancer
which would have regressed on its own with no
need for intervention), any approach to screening
in this group should occur in the context of
shared decision-making between patients and providers regarding the potential harms, benefits, and
unknowns of these approaches.

\hypertarget{statement-15.8-we-recommend-health-care-professionals-apply-the-same-respective-local-screening-guidelines-including-the-recommendation-not-to-screen-developed-for-cisgender-women-at-average-and-elevated-risk-for-developing-ovarian-or-endometrial-cancer-in-their-care-of-transgender-and-gender-diverse-people-who-have-the-same-risks.}{%
\section*{Statement 15.8: We recommend health care professionals apply the same respective local screening guidelines (including the recommendation not to screen) developed for cisgender women at average and elevated risk for developing ovarian or endometrial cancer in their care of transgender and gender diverse people who have the same risks.}\label{statement-15.8-we-recommend-health-care-professionals-apply-the-same-respective-local-screening-guidelines-including-the-recommendation-not-to-screen-developed-for-cisgender-women-at-average-and-elevated-risk-for-developing-ovarian-or-endometrial-cancer-in-their-care-of-transgender-and-gender-diverse-people-who-have-the-same-risks.}}
\addcontentsline{toc}{section}{Statement 15.8: We recommend health care professionals apply the same respective local screening guidelines (including the recommendation not to screen) developed for cisgender women at average and elevated risk for developing ovarian or endometrial cancer in their care of transgender and gender diverse people who have the same risks.}

Current consensus guidelines do not recommend routine ovarian cancer screening for cisgender women. Case reports of ovarian cancer among
transgender men have been reported (Dizon et al.,
2006; Hage et al., 2000). There is currently no
evidence testosterone therapy leads to an increased
risk of ovarian cancer, although long-term prospective studies are lacking (Joint et al., 2018).

\hypertarget{statement-15.9-we-recommend-against-routine-oophorectomy-or-hysterectomy-solely-for-the-purpose-of-preventing-ovarian-or-uterine-cancer-for-transgender-and-gender-diverse-people-undergoing-testosterone-treatment-and-who-have-an-otherwise-average-risk-of-malignancy.}{%
\section*{Statement 15.9: We recommend against routine oophorectomy or hysterectomy solely for the purpose of preventing ovarian or uterine cancer for transgender and gender diverse people undergoing testosterone treatment and who have an otherwise average risk of malignancy.}\label{statement-15.9-we-recommend-against-routine-oophorectomy-or-hysterectomy-solely-for-the-purpose-of-preventing-ovarian-or-uterine-cancer-for-transgender-and-gender-diverse-people-undergoing-testosterone-treatment-and-who-have-an-otherwise-average-risk-of-malignancy.}}
\addcontentsline{toc}{section}{Statement 15.9: We recommend against routine oophorectomy or hysterectomy solely for the purpose of preventing ovarian or uterine cancer for transgender and gender diverse people undergoing testosterone treatment and who have an otherwise average risk of malignancy.}

TGD people with ovaries who are taking
testosterone-based GAHT are often in an oligo- or
anovulatory state, or otherwise experience shifts
in luteal phase function and progesterone production. This condition combined with the possible
increased estrogen exposure from aromatization
of exogenous testosterone raises the concern for
excessive or unopposed endometrial estrogen
exposure, although the clinical significance is
unknown. Histologic studies of the endometrium
in TGD people taking testosterone have found
atrophy rather than hyperplasia (Grimstad et al.,
2018; Grynberg et al., 2010; Perrone et al., 2009).
In a large cohort of trans masculine people who
underwent a hysterectomy with oophorectomy,
benign ovarian histopathology was noted in all
cases (n = 85) (Grimstad et al., 2020). While prospective outcome data are lacking, there is insufficient evidence at this time to support a
recommendation transgender men undergo routine
hysterectomy or oophorectomy solely to prevent
endometrial or ovarian cancer. Certainly, unexplained signs/symptoms of endometrial or ovarian
cancer should be evaluated appropriately.

\hypertarget{statement-15.10-we-recommend-health-care-professionals-offer-cervical-cancer-screening-to-transgender-and-gender-diverse-people-who-currently-have-or-previously-had-a-cervix-following-local-guidelines-for-cisgender-women.}{%
\section*{Statement 15.10: We recommend health care professionals offer cervical cancer screening to transgender and gender diverse people who currently have or previously had a cervix, following local guidelines for cisgender women.}\label{statement-15.10-we-recommend-health-care-professionals-offer-cervical-cancer-screening-to-transgender-and-gender-diverse-people-who-currently-have-or-previously-had-a-cervix-following-local-guidelines-for-cisgender-women.}}
\addcontentsline{toc}{section}{Statement 15.10: We recommend health care professionals offer cervical cancer screening to transgender and gender diverse people who currently have or previously had a cervix, following local guidelines for cisgender women.}

Individuals with a cervix should undergo routine cervical cancer screening and prevention
according to age-based regional practices and
guidelines. This includes vaccination against the
human papilloma virus (HPV) and screening
according to local guidelines, including cytologic,
high-HPV co-testing if available. It is important
HCPs be mindful of performing pelvic speculum
examinations in a manner that minimizes pain
and distress for transgender masculine people.

TGD people with a cervix are less likely to
have had conventional cervical cancer screening,
either because the exam can cause worsening of
dysphoria and/or because general practitioners
and patients are misinformed about the need for
this screening (Agenor et al., 2016; Potter et al.,
2015). In addition, testosterone therapy can result
in atrophic changes of the genital tract, and the
duration of testosterone use has been associated
with a greater likelihood of obtaining an inadequate sample for cytologic screening of cervical
cancer (Peitzmeier et al., 2014). Alternatives to
speculum exams and cervical cytology, such as
provider- or self-collected high-risk HPV swabs,
may be of particular benefit for screening people
with a cervix. Research underway in the US is
investigating the use of self-collected vaginal
high-risk HPV testing among transgender masculine populations. HPV swabs were found to be
highly acceptable among transgender men with
a sensitivity to high-risk HPV of 71.4\% (negative
predictive value of 94.7\%) and a specificity of
98.2\% (Reisner et al., 2018). Further study is
needed to evaluate the harms of HPV primary
screening in transgender men in terms of the
potential increased harms associated with invasive
examinations and colposcopies.

\hypertarget{statement-15.11-we-recommend-health-care-professionals-counsel-transgender-and-gender-diverse-people-that-the-use-of-antiretroviral-medications-is-not-a-contraindication-to-gender-affirming-hormone-therapy.}{%
\section*{Statement 15.11: We recommend health care professionals counsel transgender and gender diverse people that the use of antiretroviral medications is not a contraindication to gender-affirming hormone therapy.}\label{statement-15.11-we-recommend-health-care-professionals-counsel-transgender-and-gender-diverse-people-that-the-use-of-antiretroviral-medications-is-not-a-contraindication-to-gender-affirming-hormone-therapy.}}
\addcontentsline{toc}{section}{Statement 15.11: We recommend health care professionals counsel transgender and gender diverse people that the use of antiretroviral medications is not a contraindication to gender-affirming hormone therapy.}

Human immunodeficiency virus (HIV) prevalence is disproportionately high in TGD
populations. A recent large metanalysis found a
global odds ratio for HIV infection of sixty-six
for trans feminine individuals and 6.8 for trans
masculine individuals (Stutterheim et al., 2021).
PCPs have unique opportunities to provide crucial education and implement prevention strategies, especially related to decreasing HIV
burden among TGD people. Mistrust of health
care providers due to past experiences of discrimination and transphobia impacts HIV prevention and disrupts the linkage to care efforts
(Sevelius et al., 2016). Stigma, lack of adequate
training, and innate power hierarchies within
medical establishments, all contribute to ambivalence and uncertainty among HCPs when caring for TGD people (Poteat et al., 2013). Finally,
a lack of inclusiveness and gender-affirming
practices in the health care setting may lead to
TGD people feeling unsafe discussing sensitive
topics, such as HIV diagnosis and avoiding care
out of fear (Bauer et al., 2014; Gibson et al.,
2016; Seelman et al., 2017).

HCPs should be aware of this broader context
within which many TGD people are seeking care
for either gender-affirming hormones, HIV
pre-exposure chemoprophylaxis/treatment (PrEP),
or both. There may be various misconceptions
about the safety of taking gender-affirming hormones concurrently with antiretroviral therapy
for HIV chemoprophylaxis or treatment.

Direct study of antiretroviral/gender-affirming
hormone therapy (ART/GAHT) interactions has
been limited. A subanalysis of transgender women
and trans feminine persons in the multinational
iPrEx trial found poor effectiveness in this group
in the intention-to-treat analysis, although effectiveness was similar to that in cisgender gay men
among those transgender participants who
adhered to the medication as prescribed, suggesting that uptake and adherence to PrEP remain
challenging in this population. Two studies of
the effects of GAHT on tenofovir diphosphate
(Grant et al., 2021) and tenofovir diphosphate
and emtricitabine (Shieh et al., 2019) found the
significantly lowered ART drug levels were
unlikely to be of clinical significance. Overall,
data on the interactions between hormonal contraceptives and antiretrovirals are reassuring in
terms of the impact of hormones on ART (Nanda
International Journal of Transgender Health S153
et al., 2017). Because estradiol is partially metabolized by cytochrome P450 (CYP) 3A4 and 1A2
enzymes, potential drug interactions with other
medications that induce or inhibit these pathways, such as non-nucleoside reverse transcriptase
inhibitors (NNRTIs, e.g., efavirenz (EFV) and
nevirapine (NVP)), may exist (Badowski et al.,
2021). However, the preferred first-line ART regimens in most countries include integrase inhibitors, which have minimal to no drug interactions
with gender-affirming hormones and can be used
safely (Badowski, 2021; Department of Health
and Human Services. Panel on Antiretroviral
Guidelines for Adults and Adolescents, 2021). If
concerns exist about potential interactions, HCPs
should monitor blood hormone levels as needed.
Therefore, TGD people living with HIV and taking antiretroviral medications should be counseled that taking antiretrovirals alongside GAHT
is safe.

\hypertarget{statement-15.12-we-recommend-health-care-professionals-obtain-a-detailed-medical-history-from-transgender-and-gender-diverse-people-that-includes-past-and-present-use-of-hormones-gonadal-surgeries-as-well-as-the-presence-of-traditional-osteoporosis-risk-factors-to-assess-the-optimal-age-and-necessity-for-osteoporosis-screening.-for-supporting-text-see-statement-15.13.}{%
\section*{Statement 15.12: We recommend health care professionals obtain a detailed medical history from transgender and gender diverse people that includes past and present use of hormones, gonadal surgeries as well as the presence of traditional osteoporosis risk factors, to assess the optimal age and necessity for osteoporosis screening. For supporting text, see Statement 15.13.}\label{statement-15.12-we-recommend-health-care-professionals-obtain-a-detailed-medical-history-from-transgender-and-gender-diverse-people-that-includes-past-and-present-use-of-hormones-gonadal-surgeries-as-well-as-the-presence-of-traditional-osteoporosis-risk-factors-to-assess-the-optimal-age-and-necessity-for-osteoporosis-screening.-for-supporting-text-see-statement-15.13.}}
\addcontentsline{toc}{section}{Statement 15.12: We recommend health care professionals obtain a detailed medical history from transgender and gender diverse people that includes past and present use of hormones, gonadal surgeries as well as the presence of traditional osteoporosis risk factors, to assess the optimal age and necessity for osteoporosis screening. For supporting text, see Statement 15.13.}

\hypertarget{statement-15.13-we-recommend-health-care-professionals-discuss-bone-health-with-transgender-and-gender-diverse-people-including-the-need-for-active-weight-bearing-exercise-healthy-diet-calcium-and-vitamin-d-supplementation.}{%
\section*{Statement 15.13: We recommend health care professionals discuss bone health with transgender and gender diverse people including the need for active weight bearing exercise, healthy diet, calcium, and vitamin D supplementation.}\label{statement-15.13-we-recommend-health-care-professionals-discuss-bone-health-with-transgender-and-gender-diverse-people-including-the-need-for-active-weight-bearing-exercise-healthy-diet-calcium-and-vitamin-d-supplementation.}}
\addcontentsline{toc}{section}{Statement 15.13: We recommend health care professionals discuss bone health with transgender and gender diverse people including the need for active weight bearing exercise, healthy diet, calcium, and vitamin D supplementation.}

Estrogen and testosterone both support bone
formation and turnover. Decreased sex hormone
levels are associated with a greater risk of osteoporosis in older age (Almeida et al., 2017). TGD
individuals may receive medical and/or surgical
interventions that have the potential to influence
bone health, such as sex hormone treatment,
androgen blockade, and gonadectomy. Therefore,
a detailed medical history, including past and
present use of hormones along with gonadal surgeries, is necessary to establish the need for osteoporosis screening.

Several observational studies have compared
bone mineral density (BMD) of TGD adults
before and after gender-affirming hormone therapy along with in TGD individuals compared
with sex-at-birth matched cisgender controls.

Low BMD may exist before the initiation of
hormones. One study showed a lower mean areal
BMD at the femoral neck, total hip, and spine
in transgender women than in age-matched cisgender male controls (Van Caenegem, Taes et al.,
2013). Another study revealed a high prevalence
of low BMD scores among TGD youth before
starting puberty blockers (Lee, Finlayson et al.,
2020). The authors of both studies concluded low
rates of physical activity may be an important
contributor to these findings.

Acceleration of bone loss can occur after gonadectomy if hormones are stopped or if hormones
levels are suboptimal. In one study, thirty percent
of transgender women who had undergone gonadectomy had low bone mass, and this correlated
with lower 17-ß estradiol levels and adherence
to GAHT (Motta et al., 2020).

Investigation of the effects of GAHT on BMD
have revealed TGD women receiving estrogen
therapy show improvements in BMD. A systematic review and meta-analysis on the impact of
sex hormones on bone health of transgender
individuals included 9 eligible studies in transgender women (n = 392) and 8 eligible studies in
transgender men (n = 247) published between
2008 and 2015. The meta-analysis revealed transgender women showed a statistically significant
increase in lumbar spine BMD (but not femoral
neck BMD) compared with baseline measures.
Among transgender men, there were no statistically significant changes in the lumbar spine,
femoral neck, and total hip BMD at 12 and 24
months after starting testosterone compared with
baseline measures (Singh-Ospina et al., 2017).
Since the publication of this study, the European
Network for Investigation of Gender Incongruence
(ENIGI) study, a multicenter prospective observational study (Belgium, Norway, Italy, and the
Netherlands) published results on BMD outcomes
for 231 transgender women and 199 transgender
men one year after initiating GAH (Wiepjes
et al., 2017). Transgender women had an increase
in BMD of the lumbar spine, total hip and
femoral neck, and increased BMD of the total
hip occurred in transgender men. One study
reported no fractures in transgender individuals
at 12 months following initiation of hormones in
53 transgender men and 53 transgender women
(Wierckx, van Caenegem et al., 2014). No studies
suggest GAHT should be an indication for
enhanced osteoporosis screening. Rather, gaps in
GAHT in those who have undergone prior gonadectomy would be a consideration for such
screening.

Clinical practice guidelines include recommendations for osteoporosis screening in TGD individuals (Deutsch, 2016a; Hembree et al., 2017;
Rosen et al., 2019). For TGD people, both the
International Society for Clinical Densitometry
and the Endocrine Society suggest consideration
of baseline BMD screening before initiation of
hormones. Further recommendations for BMD
screening are based on several factors including
sex reported at birth and age along with the presence of traditional risk factors for osteoporosis,
such as prior fracture, high risk medication use,
conditions associated with bone loss, and low
body weight (Rosen et al., 2019). Specifically, the
ISCD guidelines state BMD testing is indicated
for TGD individuals if they have a history of
gonadectomy or therapy that lowers endogenous
gonadal steroid levels prior to the initiation of
GAHT, hypogonadism with no plan to take
GAHT or known indications for BMD testing
(Rosen et al., 2019). However, the evidentiary
basis for these recommendations is weak.

The recommended screening modality for osteoporosis is dual energy x-ray absorptiometry
(DXA) of the lumbar spine, total hip, and femoral
neck (Kanis, 1994). However in many low- and
middle-income countries, BMD tests using DXA
are not available, and routine DXA-based screening is conducted in few countries, the US being
an exception.

PCPs should discuss ways to optimize bone
health with TGD people. In addition, PCPs
should provide information about the importance
of nutrition and exercise on maintaining bone
health. TGD individuals with (or at risk) for osteoporosis should be informed about the benefits
of weight bearing exercise along with strength
and resistance exercises in limiting bone loss
(Benedetti et al., 2018). Nutrition is integral to
bone health. Nutritional deficiencies, including
insufficient calcium intake and low vitamin D,
can result in low bone mineralization. Vitamin
D and calcium supplementation have been shown
to reduce hip as well as total fracture incidence
(Weaver et al., 2016). Although relevant to all
populations, this discussion is pertinent as a high
prevalence of hypovitaminosis D has been
observed in TGD populations (Motta et al., 2020;
Van Caenegem, Taes et al., 2013).

\hypertarget{statement-15.14-we-recommend-health-care-professionals-offer-transgender-and-gender-diverse-people-referrals-for-hair-removal-from-the-face-body-and-genital-areas-for-gender-affirmation-or-as-part-of-a-preoperative-preparation-process.}{%
\section*{Statement 15.14: We recommend health care professionals offer transgender and gender diverse people referrals for hair removal from the face, body, and genital areas for gender-affirmation or as part of a preoperative preparation process.}\label{statement-15.14-we-recommend-health-care-professionals-offer-transgender-and-gender-diverse-people-referrals-for-hair-removal-from-the-face-body-and-genital-areas-for-gender-affirmation-or-as-part-of-a-preoperative-preparation-process.}}
\addcontentsline{toc}{section}{Statement 15.14: We recommend health care professionals offer transgender and gender diverse people referrals for hair removal from the face, body, and genital areas for gender-affirmation or as part of a preoperative preparation process.}

Hair removal is necessary both for the elimination of facial hair (Marks et al., 2019) as well
as in preparation for certain gender-affirming
surgeries (GAS) such as vaginoplasty, phalloplasty,
and metoidioplasty (Zhang et al., 2016).
Preoperative permanent hair removal is required
for any skin area that will either be brought into
contact with urine (e.g., used to construct a neourethra) or be moved to reside within a partially
closed cavity within the body (e.g., used to line
the neovagina) (Zhang et al., 2016). Hair removal
techniques used in gender-affirming care are electrolysis hair removal (EHR) and laser hair
removal (LHR) (Fernandez et al., 2013). EHR is
currently the only US Food and Drug
Administration--approved method of permanent
hair removal, whereas LHR is approved for permanent hair reduction (Thoreson et al., 2020).

EHR involves the use of an electric current
with a very fine probe that is manually inserted
sequentially into individual hair follicles (Martin
et al., 2018). Since this method uses direct
mechanical destruction of the blood supply to
the hair, it can be used on all hair colors and
skin types (Martin et al., 2018). EHR is time
consuming and costly as it requires each hair
follicle to be treated individually, but is effective
for permanent hair removal. For genital permanent hair removal prior to GAS, this treatment
needs to be performed by a practitioner competent in genital hair removal as this method differs
International Journal of Transgender Health S155
from that of the face and body. EHR is more
painful than LHR, with possible side effects of
erythema, crusting, and swelling (Harris et al.,
2014). Postinflammatory hyperpigmentation is a
risk for dark-skinned individuals (Richards \&
Meharg, 1995). Pain can be controlled with topical local anesthetic and cooling techniques, and
tolerance to EHR does develop to some degree
with many persons able to tolerate longer sessions
(Richards \& Meharg, 1995).

LHR uses laser energy to target hair follicles. It
is beneficial for larger surface areas. The mechanism is photo-thermolysis, whereby light from a
laser selectively targets melanin in the hair shaft
(Gao et al., 2018). This energy is converted to
heat, which damages the follicles within the skin
that produce hairs and results in the destruction
of hair growth. Further treatments are needed to
achieve best results and are typically spaced six
weeks apart to allow for hair cycling (Zhang et al.,
2016). Because LHR targets melanin, results may
be limited for those with grey, blonde, or red hair.

There are specific considerations for using
LHR in dark-skinned individuals (Fitzpatrick skin
types IV to VI) (Fayne et al., 2018)). The higher
melanin content of the epidermis can compete
with the target chromophore of the light or laser,
which is the melanin in the hair shaft of the hair
follicle. For selective thermolysis to occur, heat
diffuses from the hair shaft to the follicular stem
cells to cause damage. In darker skin types, rather
than reaching the target melanin in the hair shaft,
light is absorbed in the epidermis where it is
then converted to heat. This may result in poorer
clinical outcomes and a higher rate of thermally
induced adverse effects, such as hypo- or hyperpigmentation, blistering, and crust formation
(Fayne et al., 2018). The selection of laser wavelength is critical in reducing this risk, with longer
wavelength recommended to minimize the
absorption of light in epidermal melanin and thus
maximize efficacy and minimize adverse effects
in patients with dark skin (Zhang et al., 2016).
Side effects from LHR can include the feeling of
sunburnt after treatment, as well as inflammation,
redness, hyperpigmentation, and swelling. Flashing
lights have been known to induce seizures in
susceptible patients, so patients should be
screened for this risk. Pain and discomfort during
the procedure can also represent a significant
barrier, and PCPs should be prepared to prescribe
topical or systemic analgesics, such as a eutectic
mixture of local anesthetics (EMLA) or a low
dose systemic opioid. For genital GAS, some have
recommended a 3-month wait after the last
planned hair removal treatment before proceeding
with surgery to confirm that no further hair
regrowth will occur (Zhang et al., 2016).

\hypertarget{reproductive-health}{%
\chapter{Reproductive Health}\label{reproductive-health}}

All humans, including transgender individuals,
have the reproductive right to decide whether
or not to have children (United Nations
Population Fund, 2014). Medically necessary
gender-affirming hormonal treatments (GAHTs)
and surgical interventions (see medically necessary statement in Chapter 2---Global Applicability,
Statement 2.1) that alter reproductive anatomy
or function may limit future reproductive options
to varying degrees (Hembree et al., 2017; Nahata
et al., 2019). It is thus critical to discuss infertility risk and fertility preservation (FP) options
with transgender individuals and their families
prior to initiating any of these treatments and
to continue these conversations on an ongoing
basis thereafter (Hembree et al., 2017). Established
FP options, such as embryo, oocyte, and sperm
cryopreservation, may be available for postpubertal transgender individuals (Nahata et al.,
2019). Research protocols for ovarian and testicular tissue cryopreservation have also been
developed and studied (Borgström et al., 2020;
Nahata et al., 2019; Rodriguez-Wallberg, et al.,
2019). Whereas the use of embryos, mature
oocytes, and sperm have all proven to be efficacious when employed within clinical treatments, cryopreserved gonadal tissues would
require either future retransplantation aimed at
obtaining fully functional gametes or the application of laboratory methods for culture, which
are still under development in basic science
research settings. Of note, recent American
Society for Reproductive Medicine guidelines
have lifted the experimental label on ovarian
tissue cryopreservation, but evidence remains
limited in prepubertal children (Practice
Committee of the American Society for
Reproductive Medicine, 2019).

Individualized care should be provided in the
context of each person's parenthood goals. Some
research suggests transgender and gender diverse
(TGD) people may be less likely to desire genetically related children or children at all when
compared with cisgender peers (Defreyne, van
Schuvlenbergh et al., 2020; Russell et al., 2016;
von Doussa et al., 2015). Yet, several other studies
have shown many TGD individuals 1) desire
genetically related children; 2) regret missed
opportunities for FP; and 3) are willing to delay
or interrupt hormone therapy to preserve fertility
and/or conceive (Armuand, Dhejne et al., 2017;
Auer et al., 2018; De Sutter et al., 2002; Defreyne,
van Schuylenbergh et al., 2020; Tornello \&
Bos, 2017).

Many barriers to FP have been reported, such
as cost (which is exacerbated when insurance
coverage is lacking), urgency to start treatment,
inability to make future-oriented decisions, inadequate provider knowledge/provider biases that
affect offering FP, and difficulties accessing FP
(Baram et al., 2019; Defreyne, van Schuylenbergh
et al., 2020). Additionally, transgender individuals may have worsening dysphoria due to various
steps in the FP process that are inseparably connected with the gender assigned at birth
(Armuand, Dhejne, et al., 2017; Baram et al.,
2019). When available, a multidisciplinary team
approach, where both medical and mental health
providers collaborate with gender-affirming fertility specialists, can help overcome some of
these barriers (Tishelman et al., 2019). TGD
individuals should be educated about the distinction between fertility (utilizing one's own
gametes/reproductive tissues) and pregnancy. In
addition to fertility considerations, efforts to
ensure equitable high-quality care for all forms
of family planning and building throughout the
full reproductive continuum must be maintained.
This includes procreative options such as perinatal care, pregnancy, delivery, and postpartum
care, as well as family planning and contraceptive
options to prevent unplanned pregnancies, and
pregnancy termination if sanctioned (Bonnington
et al., 2020; Cipres et al., 2017; Krempasky et al.,
2020; Light et al., 2018; Moseson, Fix et al.,
2020). TGD people who wish to carry a pregnancy should undergo standard of care preconception care and prenatal counseling and
should receive counseling about breast/chest
feeding in environments supportive of people
with diverse gender identities and experiences
(MacDonald et al., 2016; Obedin-Maliver \&
Makadon, 2016).

All the statements in this chapter have been
recommended based on a thorough review of
evidence, an assessment of the benefits and
harms, values and preferences of providers and
patients, and resource use and feasibility. In some
cases, we recognize evidence is limited and/or
services may not be accessible or desirable.

\hypertarget{statement-16.1-we-recommend-health-care-professionals-who-are-treating-transgender-and-gender-diverse-people-and-prescribing-or-referring-patients-for-hormone-therapiessurgeries-advise-their-patients-about}{%
\section*{Statement 16.1: We recommend health care professionals who are treating transgender and gender diverse people and prescribing or referring patients for hormone therapies/surgeries advise their patients about:}\label{statement-16.1-we-recommend-health-care-professionals-who-are-treating-transgender-and-gender-diverse-people-and-prescribing-or-referring-patients-for-hormone-therapiessurgeries-advise-their-patients-about}}
\addcontentsline{toc}{section}{Statement 16.1: We recommend health care professionals who are treating transgender and gender diverse people and prescribing or referring patients for hormone therapies/surgeries advise their patients about:}

\textbf{a. Known effects of hormone therapies/surgeries on future fertility;
b. Potential effects of therapies that are not well studied and are of unknown reversibility;
c.~Fertility preservation (FP) options (both established and experimental;
d.~Psychosocial implications of infertility.}

\hypertarget{tgd-individuals-assigned-female-at-birth}{%
\subsection*{TGD individuals assigned female at birth}\label{tgd-individuals-assigned-female-at-birth}}
\addcontentsline{toc}{subsection}{TGD individuals assigned female at birth}

GAHT may negatively impact future reproductive capacity (Hembree et al., 2017). Based on
current evidence in transgender men and gender
diverse people assigned female at birth, these
risks are as follows:

Gonadotropin-releasing hormone agonists
(GnRHas) may be used for pubertal suppression
to prevent further pubertal progression until
adolescents are ready for masculinizing treatment. GnRHas may also be used for menstrual
suppression. GnRHas impact the maturation of
gametes but do not cause permanent damage to
gonadal function. Thus, if GnRHas are discontinued, oocyte maturation would be expected
to resume.

There are few studies detailing the effects of
testosterone therapy on reproductive function in
transgender men (Moravek et al., 2020).
Restoration of normal ovarian function with
oocyte maturation after testosterone interruption
has been demonstrated in transgender men who
have achieved natural conception. A retrospective
study on oocyte cryopreservation showed no differences in the total number of oocytes retrieved
or in the number of mature oocytes between
transgender men and age- and BMI-matched cisgender women (Adeleye et al., 2018, 2019). The
first results have recently been published evaluating live birth rates after controlled ovarian stimulation in transgender men compared with
cisgender women (Leung et al., 2019). Testosterone
was discontinued prior to ovarian stimulation.
Overall, the results concerning the influence of
testosterone on reproductive organs and their
function appear to be reassuring. However, there
have been no prospective studies to date evaluating the effect of long-term hormone therapy
on fertility (i.e., started in adolescence) or in
those treated with GnRHas in early puberty followed by testosterone therapy. It is important to
take into consideration that required medications
and procedures for cryopreserving oocytes (a
pelvic examination, vaginal ultrasound monitoring, and oocyte retrievals) may lead to increasing
gender dysphoria in transgender men (Armuand,
Dhejne et al., 2017).

Surgical interventions among transgender men
will have obvious implications for reproductive
capacity. If patients desire a hysterectomy, the
option should be offered of preserving the ovaries
to retain the possibility of having a genetically
related child. Alternatively, if the ovaries are
removed either separately or concurrently with
the hysterectomy, egg freezing should be offered
prior to surgery and/or ovarian tissue cryopreservation can be done at the time of oophorectomy. Although this procedure is no longer
considered experimental, many transgender men
may desire in vitro maturation of primordial follicles, which is still investigational. Studies evaluating oocyte function have shown oocytes
isolated from transgender men with testosterone
exposure at the time of oophorectomy can be
matured in vitro to develop normal metaphase II
meiotic spindle structure (De Roo et al., 2017;
Lierman et al., 2017).

\hypertarget{tgd-individuals-assigned-male-at-birth}{%
\subsection*{TGD individuals assigned male at birth}\label{tgd-individuals-assigned-male-at-birth}}
\addcontentsline{toc}{subsection}{TGD individuals assigned male at birth}

Based on current evidence in transgender
women and gender diverse people assigned male
at birth (AMAB), the influence of medical treatment is as follows:

GnRHas inhibit spermatogenesis. Data suggest
discontinuation of treatment results in a
re-initiation of spermatogenesis, although this
may take at least 3 months and most likely longer
(Bertelloni et al., 2000). Furthermore, the psychological burden of re-exposure to testosterone
should be considered.

Anti-androgens and estrogens result in an
impaired sperm production (de Nie et al., 2020;
Jindarak et al., 2018; Kent et al., 2018).
Spermatogenesis might resume after discontinuation of prolonged treatment with anti-androgens
and estrogens, but data are limited (Adeleye
et al., 2019; Alford et al., 2020; Schneider et al.,
2017). Testicular volumes diminish under the
influence of gender-affirming hormone treatment
(Matoso et al., 2018). Semen quality in transgender women may also be negatively affected by
specific life-style factors, such as a low frequency
of masturbation, wearing the genitals tight against
the body (e.g., with use of tight undergarments
for tucking) (Jung \& Schuppe, 2007; Mieusset
et al., 1985, 1987; Rodriguez-Wallberg, Häljestig
et al., 2021).

\hypertarget{statement-16.2-we-recommend-health-care-professionals-refer-transgender-and-gender-diverse-people-interested-in-fertility-preservation-to-providers-with-expertise-in-fertility-preservation-for-further-discussion.}{%
\section*{Statement 16.2: We recommend health care professionals refer transgender and gender diverse people interested in fertility preservation to providers with expertise in fertility preservation for further discussion.}\label{statement-16.2-we-recommend-health-care-professionals-refer-transgender-and-gender-diverse-people-interested-in-fertility-preservation-to-providers-with-expertise-in-fertility-preservation-for-further-discussion.}}
\addcontentsline{toc}{section}{Statement 16.2: We recommend health care professionals refer transgender and gender diverse people interested in fertility preservation to providers with expertise in fertility preservation for further discussion.}

Research shows many transgender adults desire
biological children (De Sutter et al., 2002;
Defreyne, van Schuylenbergh et al., 2020;
Wierckx, Van Caenegem et al., 2012), yet FP
rates remain widely variable, particularly in youth
(\textless{} 5\%--40\%) (Brik et al., 2019; Chen et al., 2017;
Chiniara et al., 2019; Nahata et al., 2017;
Segev-Becker et al., 2020). In a recent survey,
many youth acknowledged their feelings about
having a biological child might change in the
future (Strang, Jarin et al., 2018). Non-elective
sterilization is a violation of human rights (Ethics
Committee of the American Society for
Reproductive Medicine, 2015; Equality and
Human Rights Commission, 2021; Meyer III
et al., 2001) and due to advances in social attitudes, fertility medicine, and affirmative transgender health care, opportunities for biological
parenthood during transition should be supported for transgender people. Due to the influence clinical opinion may have on transgender
or nonbinary people's FP and on parenting decisions, FP options should be explored by health
care providers alongside options such as fostering, adoption, coparenting, and other parenting
alternatives (Bartholomaeus \& Riggs, 2019).
Transgender patients who have been offered this
type of discussion and have been given the
choice to undergo procedures for FP have
reported the experience to be an overall positive
one (Armuand, Dhejne et al., 2017; De Sutter
et al., 2002; James-Abra et al., 2015).

In other patient populations, fertility referrals
and formal fertility programs have been shown
to increase FP rates and improve patient satisfaction (Kelvin et al., 2016; Klosky, Anderson
et al., 2017; Klosky, Wang et al., 2017;
Shnorhavorian et al., 2012) Physician attitudes
have been investigated, and recent studies indicate both an awareness and a desire to provide
fertility-related information to children and their
families (Armuand et al., 2020). However, barriers have also been identified, including lack
of knowledge, comfort, and resources (Armuand,
Nilsson et al., 2017; Frederick et al., 2018).
Thus, the need for appropriate training of health
care providers has been highlighted, with
emphasis placed on fertility counseling and
offering FP options to all at-risk individuals in
an unbiased way (Armuand, Nilsson et al.,
2017). Parents' recommendations have also been
shown to significantly influence FP rates in adolescent and young adult males with cancer
(Klosky, Flynn et al., 2017). While there are
clear clinical differences in these populations,
these findings can help inform best practices
for fertility counseling and FP referrals for
transgender individuals.

\hypertarget{statement-16.3-we-recommend-transgender-care-teams-partner-with-local-reproductive-specialists-and-facilities-to-provide-specific-and-timely-information-and-fertility-preservation-services-prior-to-offering-medical-and-surgical-interventions-that-may-impact-fertility.}{%
\section*{Statement 16.3: We recommend transgender care teams partner with local reproductive specialists and facilities to provide specific and timely information and fertility preservation services prior to offering medical and surgical interventions that may impact fertility.}\label{statement-16.3-we-recommend-transgender-care-teams-partner-with-local-reproductive-specialists-and-facilities-to-provide-specific-and-timely-information-and-fertility-preservation-services-prior-to-offering-medical-and-surgical-interventions-that-may-impact-fertility.}}
\addcontentsline{toc}{section}{Statement 16.3: We recommend transgender care teams partner with local reproductive specialists and facilities to provide specific and timely information and fertility preservation services prior to offering medical and surgical interventions that may impact fertility.}

Cryopreservation of sperm and oocytes are
established FP techniques and can be offered to
pubertal, late pubertal, and adult birth assigned
males and birth assigned females, respectively,
preferably prior to the initiation of GAHT
(Hembree et al., 2017; Practice Committee of the
American Society for Reproductive Medicine,
2019). Cryopreservation of embryos can be
offered to adult (post-pubertal) TGD people who
wish to have a child and have an available partner. The future use of cryopreserved gametes is
also dependent on the gametes and reproductive
organs of the future partner (Fischer, 2021;
Maxwell et al., 2017)

Although semen parameters have been shown
to be compromised when FP is performed after
initiation of GAH medication (Adeleye et al.,
2019), one small study showed when the treatment was discontinued, semen parameters were
comparable to those in TGD patients who had
never undergone GAH treatment. With regard to
ovarian stimulation, oocyte vitrification yield and
subsequent use of the oocytes in in-vitro fertilization (IVF), there is no reason to anticipate a
different outcome in assisted reproductive technology (ART) treatments for TGD patients than
that obtained in cisgender patients undergoing
ART---other than individual confounding factors
related to (in)fertility---when gametes are banked
prior to any medical treatment (Adeleye et al.,
2019). The use of oocytes in ART treatment
resulted in similarly successful outcomes in TGD
compared with controlled, matched cisgender
patients (Adeleye et al., 2019; Leung et al., 2019;
Maxwell et al., 2017).

Although these are established options, few
pubertal, late pubertal or adult TGD people
undergo FP (Nahata et al., 2017), and many experience challenges while undergoing FP interventions. Not only is access and cost of these
methods a barrier (particularly in regions without
insurance coverage), but these procedures are
often physically and emotionally uncomfortable,
and many express concerns about postponing the
transitioning process (Chen et al., 2017; De Sutter
et al., 2002; Nahata et al., 2017; Wierckx, Stuyver
et al., 2012). Especially for the birth assigned
females, the invasiveness of endovaginal ultrasound follow-up of the ovarian stimulation and
oocyte retrieval procedures (and associated psychological distress) have been cited as a barrier
(Armuand, Dhejne et al., 2017; Chen et al., 2017).
There is also the concern young adults going
through transitioning may not have a clear vision
of parenting and are therefore likely to decline
the opportunity to use FP at that time---while as
adults, they may have different opinions about
parenthood (Cauffman \& Steinberg, 2000). The
reduction of gender dysphoria during transitioning could also influence the decision-making
process surrounding FP (Nahata et al., 2017).
Based on research showing TGD youths' fertility
perspectives may change over time (Nahata et al.,
2019; Strang, Jarin et al., 2018), FP options should
be discussed on an ongoing basis.

\hypertarget{statement-16.4-we-recommend-health-care-professionals-counsel-pre--or-early-pubertal-transgender-and-gender-diverse-youth-seeking-gender-affirming-therapy-and-their-families-that-currently-evidence-basedestablished-fertility-preservation-options-are-limited.}{%
\section*{Statement 16.4: We recommend health care professionals counsel pre- or early-pubertal transgender and gender diverse youth seeking gender-affirming therapy and their families that currently evidence-based/established fertility preservation options are limited.}\label{statement-16.4-we-recommend-health-care-professionals-counsel-pre--or-early-pubertal-transgender-and-gender-diverse-youth-seeking-gender-affirming-therapy-and-their-families-that-currently-evidence-basedestablished-fertility-preservation-options-are-limited.}}
\addcontentsline{toc}{section}{Statement 16.4: We recommend health care professionals counsel pre- or early-pubertal transgender and gender diverse youth seeking gender-affirming therapy and their families that currently evidence-based/established fertility preservation options are limited.}

For prepubertal and early-pubertal children,
FP options are limited to the storage of gonadal
tissue. Although this option is available for TGD
children in the same way that it is available for
cisgender prepubertal and early-pubertal oncological patients, there is no literature describing
the utilization of this approach in the transgender
population. Ovarian tissue autotransplantation has
resulted in over 130 live births in cisgender
women. Most of these patients conceived naturally without ART (Donnez \& Dolmans, 2015;
Jadoul et al., 2017), and the majority stored their
ovarian tissue either as adults or during puberty.
Although the recent American Society for
Reproductive Medicine guideline has lifted the
experimental label from ovarian tissue cryopreservation (Practice Committee of the American
Society for Reproductive Medicine, 2019), there
are very few case reports describing a successful
pregnancy in a woman following the transplantation of ovarian tissue cryopreserved before
puberty. Demeestere et al.~(2015) and
Rodriguez-Wallberg, Milenkovic et al.~(2021)
described cases of successful pregnancies following transplantation of tissue procured at the age
of 14, and recently Matthews et al.~(2018)
described the case of a girl diagnosed with thalassemia who had ovarian tissue stored at the age
of 9 and transplantation 14 years late. She subsequently conceived through IVF and delivered
a healthy baby.

Currently, the only future clinical application
for storing ovarian tissue is autotransplantation,
which might be undesirable in a transgender man
(due to the potentially undesirable effects of estrogen). A laboratory procedure that would make it
possible to mature oocytes in vitro starting with
ovarian tissue would be the ideal future application
of stored ovarian tissue for transgender people,
but this technique is currently only being investigated and optimized in basic science research settings (Ladanyi et al., 2017; Oktay et al., 2010).

Prepubertal procurement of testicular tissue
has been documented as a low-risk procedure
(Borgström et al., 2020; Ming et al., 2018). Some
authors have also described this approach as a
theoretical option in transgender people (De Roo
et al., 2016; Martinez et al., 2017; Nahata, Curci
et al., 2018). However, there are no reports in
the literature describing the clinical or investigational utilization of this FP option for TGD
patients. Moreover, the viability of the clinical
application of autotransplantation of testicular
tissue remains unknown in humans, and in vitro
maturation techniques are still in the realm of
basic science research. Thus, specialists currently
consider this technique experimental (Picton
et al., 2015). The possibility of storing gonadal
tissue should be discussed prior to any genital
surgery that would result in sterilization, although
the probability of being able to use this tissue
must be clearly addressed.

\hypertarget{statement-16.5-we-recommend-transgender-and-gender-diverse-people-with-a-uterus-who-wish-to-carry-a-pregnancy-undergo-preconception-care-and-prenatal-counseling-regarding-the-use-and-cessation-of-gender-affirming-hormones-pregnancy-care-labor-and-delivery-chestbreast-feeding-supportive-services-and-postpartum-support-according-to-local-standards-of-care-in-a-genderaffirming-way.}{%
\section*{Statement 16.5: We recommend transgender and gender diverse people with a uterus who wish to carry a pregnancy undergo preconception care and prenatal counseling regarding the use and cessation of gender-affirming hormones, pregnancy care, labor and delivery, chest/breast feeding supportive services, and postpartum support according to local standards of care in a genderaffirming way.}\label{statement-16.5-we-recommend-transgender-and-gender-diverse-people-with-a-uterus-who-wish-to-carry-a-pregnancy-undergo-preconception-care-and-prenatal-counseling-regarding-the-use-and-cessation-of-gender-affirming-hormones-pregnancy-care-labor-and-delivery-chestbreast-feeding-supportive-services-and-postpartum-support-according-to-local-standards-of-care-in-a-genderaffirming-way.}}
\addcontentsline{toc}{section}{Statement 16.5: We recommend transgender and gender diverse people with a uterus who wish to carry a pregnancy undergo preconception care and prenatal counseling regarding the use and cessation of gender-affirming hormones, pregnancy care, labor and delivery, chest/breast feeding supportive services, and postpartum support according to local standards of care in a genderaffirming way.}

Most transgender men and gender diverse people (AFAB) retain their uterus and ovaries and
thus can conceive and carry a pregnancy even
after long-term testosterone use (Light et al., 2014).
Many transgender men desire children (Light
et al., 2018; Wierckx, van Caenegem et al., 2012)
and are willing to carry a pregnancy (Moseson,
Fix, Hastings et al., 2021; Moseson, Fix, Ragosta
et al., 2021). ART has expanded the opportunity
for many transgender men to conceive and fulfill
their family planning wishes (De Roo et al., 2017;
Ellis et al., 2015; Maxwell et al., 2017). Some transgender men report psychological isolation, dysphoria related to the gravid uterus and chest
changes, and depression (Charter, 2018; Ellis et al.,
2015; Hoffkling et al., 2017; Obedin-Maliver \&
Makadon, 2016). Conversely, other studies have
reported some positive experiences during pregnancy as well (Fischer, 2021; Light et al., 2014).
Mental health providers should be involved to
provide support, and counseling should be
provided addressing when to stop and when to
resume gender-affirming hormones, what options
are available for the mode of delivery and for
chest/breast feeding (Hoffkling et al., 2017). Finally,
system-level and interpersonal-level interventions
should be implemented to ensure person-centered
reproductive health care for all people (Hahn
et al., 2019; Hoffkling et al., 2017; Moseson,
Zazanis et al., 2020; Snowden et al., 2018).

Given the potential harmful effects of testosterone on the developing embryo, discontinuing
testosterone or masculinizing hormone therapy
prior to conception and during the entire pregnancy is recommended. However, the optimal
time for both the discontinuation of testosterone
prior to pregnancy and its resumption after pregnancy is unknown. Since stopping gender-affirming
hormones may cause distress and exacerbate dysphoria in transgender men, when and how to
stop this therapy should be discussed during prenatal counseling (Hahn et al., 2019). Because
information about the duration of testosterone
exposure and the risk of teratogenicity is lacking,
testosterone use should be discontinued prior to
attempting pregnancy and before stopping contraception. Moreover, there is limited information
regarding health outcomes of infants born to
transgender men. Small case series attempting to
evaluate this question have revealed no adverse
physical or psychosocial differences between
infants born to transgender men and infants in
the general population (Chiland et al., 2013).

\hypertarget{chestbreast-feeding}{%
\subsection*{Chest/Breast feeding}\label{chestbreast-feeding}}
\addcontentsline{toc}{subsection}{Chest/Breast feeding}

In the limited studies evaluating lactation and
chest/breast feeding, the majority of transgender
men and TGD individuals AFAB who chose to
chest/breast feed postpartum were successful,
with research suggesting induction of lactation
is in part dependent on preconception counseling
and experienced lactation nursing support
(MacDonald et al., 2016; Wolfe-Roubatis \& Spatz,
2015). Specifically, transgender men and TGD
people who use testosterone should be informed
1) although quantities are small, testosterone
does pass through chest/breast milk; and 2) the
impact on the developing neonate/child is
unknown, and therefore gender-affirming testosterone use is not recommended during lactation
but may be resumed after discontinuation of
chest/breast feeding (Glaser et al., 2009).
Transgender men and other TGD individuals
AFAB should be made aware some patients who
carry a pregnancy may experience undesired
chest growth and/or lactation even after chest
reconstruction and should therefore be supported
if they desire to suppress lactation (MacDonald
et al., 2016).

There is limited information concerning lactation in transgender women as well as other TGD
AMAB but many also express the desire to chest/
breast feed. While there is a case report of a
transgender woman successfully lactating and
chest/breast feeding her infant after hormonal
support using a combination of estrogen, progesterone, domperidone, and breast pumping
(Reisman \& Goldstein, 2018), the nutritional and
immunological profile of chest/breast milk under
these conditions has not been studied. Therefore,
patients need to be informed about the risks and
benefits of this approach to child feeding
(Reisman \& Goldstein, 2018).

\hypertarget{statement-16.6-we-recommend-medical-providers-discuss-contraception-methods-with-transgender-and-gender-diverse-people-who-engage-in-sexual-activity-that-can-result-in-pregnancy.}{%
\section*{Statement 16.6: We recommend medical providers discuss contraception methods with transgender and gender diverse people who engage in sexual activity that can result in pregnancy.}\label{statement-16.6-we-recommend-medical-providers-discuss-contraception-methods-with-transgender-and-gender-diverse-people-who-engage-in-sexual-activity-that-can-result-in-pregnancy.}}
\addcontentsline{toc}{section}{Statement 16.6: We recommend medical providers discuss contraception methods with transgender and gender diverse people who engage in sexual activity that can result in pregnancy.}

Many TGD individuals may retain reproductive
capacity, and they (if they retain a uterus, ovaries,
and tubes) or their sexual partners (for sperm
producing individuals) may experience unplanned
pregnancies (James et al., 2016; Light et al., 2014;
Moseson, Fix et al., 2020). Therefore, intentional
family planning counseling, including contraception and abortion conducted in gender-expansive
ways is needed (Klein, Berry-Bibee et al., 2018;
Obedin-Maliver, 2015; Stroumsa \& Wu, 2018).
TGD people AFAB may not use contraception due
to an erroneous assumption that testosterone is a
reliable form of contraception (Abern \& Maguire,
2018; Ingraham et al., 2018; Jones, Wood et al.,
2017; Potter et al., 2015). However, based on current understanding, testosterone should not be
considered a reliable form of contraception because
of its incomplete suppression of the
hypothalamic-pituitary-adrenal axis (Krempasky
et al., 2020). Furthermore, pregnancies have
occurred while individuals are amenorrheic due
to testosterone use, which may outlast active periods of administration (Light et al., 2014). Pregnancy
can also occur in TGD people after long-term
testosterone use (at least up to 10 years), although
the effect on oocytes and baseline fertility is still
unknown (Light et al., 2014).

TGD people AFAB may use a variety of contraceptive methods (Abern \& Maguire, 2018;
Bentsianov et al., 2018; Bonnington et al., 2020;
Chrisler et al., 2016; Cipres et al., 2017; Jones,
Wood et al., 2017; Krempasky et al., 2020; Light
et al., 2018).These methods may be used explicitly
for pregnancy prevention, menstrual suppression,
abnormal bleeding, or other gynecological needs
(Bonnington et al., 2020; Chrisler et al., 2016;
Krempasky et al., 2020; Schwartz et al., 2019).
Contraceptive research gaps within this population
are profound. No studies have examined how the
use of exogenous androgens (e.g., testosterone)
may modify the efficacy or safety profile of hormonal contraceptive methods (e.g., combined
estrogen and progestin hormonal contraceptives,
progestin-only based contraceptives) or
non-hormonal and barrier contraceptive methods
(e.g., internal and external condoms, non-hormonal
intrauterine devices, diaphragms, sponges, etc.).

Gender diverse individuals who currently have
a penis and testicles may engage in sexual activity
with individuals who have a uterus, ovaries, and
tubes of any gender. Gender diverse people who
have a penis and testicles can produce sperm
even while on gender-affirming hormones (i.e.,
estrogen), and although semen parameters are
diminished among those who are currently using
or who have previously used gender-affirming
hormones, azoospermia is not complete and
sperm activity is not totally suppressed (Adeleye
et al., 2019; Jindarak et al., 2018; Kent et al.,
2018). Therefore, contraception needs to be considered if pregnancy is to be avoided in penis-invagina sexual activity between a person with a
uterus, ovaries, and tubes and one with a penis
and testicles, irrespective of the use of
gender-affirming hormones by either partner.
Currently, contraceptive methods available for use
by the sperm-producing partner are primarily
mechanical barriers (i.e., external condoms, internal condoms), permanent sterilization (i.e., vasectomy), and gender-affirming surgery (e.g.,
orchiectomy, which also results in sterilization).
Contraceptive counseling that considers sperm
producing, egg producing, and gestating partners
(as relevant) is recommended.

\hypertarget{statement-16.7-we-recommend-providers-who-offer-pregnancy-termination-services-ensure-procedural-approaches-are-gender-affirming-and-serve-transgender-people-and-those-of-diverse-genders.}{%
\section*{Statement 16.7: We recommend providers who offer pregnancy termination services ensure procedural approaches are gender-affirming and serve transgender people and those of diverse genders.}\label{statement-16.7-we-recommend-providers-who-offer-pregnancy-termination-services-ensure-procedural-approaches-are-gender-affirming-and-serve-transgender-people-and-those-of-diverse-genders.}}
\addcontentsline{toc}{section}{Statement 16.7: We recommend providers who offer pregnancy termination services ensure procedural approaches are gender-affirming and serve transgender people and those of diverse genders.}

Unplanned pregnancies and abortions have
been reported among TGD individuals with a
uterus (Abern \& Maguire, 2018; Light et al., 2014;
Light et al., 2018; Moseson, Fix et al., 2020) and
documented through surveys of abortion-providing
facilities (Jones et al., 2020). However, the
population-based epidemiology of abortion provision and the experiences and preferences of
TGD individuals AFAB undergoing abortion still
represents a critical gap in research (Fix et al.,
2020; Moseson, Fix et al., 2020; Moseson, Lunn
et al., 2020). Nonetheless, given that pregnancy
capacity exists among many TGD people and
pregnancies may not always be planned or
desired, access to safe, legal, and gender-affirming
pregnancy medical and surgical termination services is necessary.

\hypertarget{sexual-health}{%
\chapter{Sexual Health}\label{sexual-health}}

Sexual health has a profound impact on physical
and psychological well-being, regardless of one's
sex, gender, or sexual orientation. However,
stigma about sex, gender and sexual orientation
influences individual's opportunities to live out
their sexuality and to receive appropriate sexual
health care. Specifically, in most societies, cisnormativity and heteronormativity lead to the
assumption that all people are cisgender and heterosexual (Bauer et al., 2009), and that this combination is superior to all other genders and
sexual orientations (Nieder, Güldenring et al.,
2020; Rider, Vencill et al., 2019). Hetero-cisnormativity negates the complexity of gender,
sexual orientation, and sexuality and disregards
diversity and fluidity. This is all the more important since sexual identities, orientations, and practices of transgender and gender diverse (TGD)
people are characterized by an enormous diversity
(Galupo et al., 2016; Jessen et al., 2021; Thurston
\& Allan, 2018; T'Sjoen et al., 2020). Likewise, a
strong cross-cultural tendency toward allonormativity---the assumption that all people experience
sexual attraction or interest in sexual activity---
negates the diverse experiences of TGD people,
especially those who locate themselves on the
asexual spectrum (McInroy et al., 2021; Mollet,
2021; Rothblum et al., 2020).

The World Health Organization (WHO, 2010)
emphasizes sexual health depends on respect for
the sexual rights of all people, including the right
to express diverse sexualities and to be treated
respectfully, safely, and with freedom from discrimination and violence. Sexual health discourses
have focused on agency and body autonomy,
which include consent, sexual pleasure, sexual
satisfaction, partnerships, and family life (Cornwall
\& Jolly, 2006; Lindley et al., 2021). In light of
this, the WHO defines sexual health as ``a state
of physical, emotional, mental, and social
well-being in relation to sexuality and not merely
the absence of disease, dysfunction, or infirmity.
Sexual health requires a positive and respectful
approach to sexuality and sexual relationships as
well as the possibility of having pleasurable and
safe sexual experiences, free of coercion, discrimination, and violence. For sexual health to be
attained and maintained, the sexual rights of all
persons must be respected, protected, and fulfilled'' (WHO, 2006, p.~5). This includes individuals on the asexual spectrum, who may not
experience sexual attraction to others but may
still choose to be sexual at times (e.g., via
self-stimulation) and/or experience interest in
forming and building romantic relationships (de
Oliveira et al., 2021).

Scientific attention to the sexual experiences
and behaviors of TGD people has grown in recent
years (Gieles et al., 2022; Holmberg et al., 2019;
Klein \& Gorzalka, 2009; Kloer et al., 2021;
Mattawanon et al., 2021; Stephenson et al., 2017;
Tirapegui et al., 2020; Thurston \& Allan, 2018).
This expansion within the literature reflects a
sex-positive framework (Harden, 2014), a framework that recognizes both the positive aspects
such as sexual pleasure (Laan et al., 2021) and
potential risks associated with sexuality
(Goldhammer et al., 2022; Mujugira et al., 2021).
Studies of TGD people's sexuality, however, often
lack validated measures, an appropriate control
group, or a prospective design (Holmberg et al.,
2019). Additionally, most focus exclusively on
sexual functioning (Kennis et al., 2022), and thus
neglecting sexual satisfaction and broader operationalizations of sexual pleasure beyond functioning. The effects of current TGD-related
medical treatments on sexuality are heterogeneous
(Özer et al., 2022; T'Sjoen et al., 2020), and there
has been little research on the sexuality of TGD
adolescents (Bungener et al., 2017; Maheux et al.,
2021; Ristori et al., 2021; Stübler \& Becker-Hebly,
2019; Warwick et al., 2022). While sex-positive
approaches to counseling and treatment for sexual
difficulties experienced by TGD individuals have
been proposed (Fielding, 2021; Jacobson et al.,
2019; Richards, 2021), to date there is insufficient
research on the effectiveness of such interventions. Focusing on the promotion of sexual
health, the World Association for Sexual Health
(WAS) asserts the importance of sexual pleasure
and considers self-determination, consent, safety,
privacy, confidence, and the ability to communicate and negotiate sexual relations as major facilitators (Kismödi et al., 2017). WAS asserts sexual
pleasure is integral to sexual rights and human
rights (Kismödi et al., 2017). To contribute to
the sexual health of TGD people, health care
professionals (HCPs) need both transgender-related
expertise and sensitivity (Nieder, Güldenring
et al., 2020). With the goal of improving sexual
health care for TGD people to an ethically-sound,
evidence-based and high-quality level, HCPs must
provide their health services with the same care
(i.e., with transgender-related expertise), respect
(i.e., with transgender-related sensitivity), and
investment in sexual pleasure and sexual satisfaction as they provide for cisgender people
(Holmberg et al., 2019).

In many societies, nonconforming gender
expressions can elicit strong (emotional) reactions, including in HCPs. Thus, when initiating
a health-related contact or establishing a therapeutic relationship, a nonjudgmental, open and
welcoming manner is most likely ensured when
HCPs reflect on their emotional, cognitive, and
interactional reactions to the person (Nieder,
Güldenring et al., 2020). In addition,
transgender-related expertise refers to identifying
the impact the TGD person's intersectional identities and experiences of marginalization and
stigma may have had on their whole self (Rider,
Vencill et al., 2019). To adequately address the
specific physical, psychological, and social conditions of TGD people, HCPs must be aware
these conditions are generally overlooked due to
hetero-cis-normativity, lack of knowledge, and
lack of skills (Rees et al., 2021). It is also important to consider cultural norms in relation to sexuality. For example, in some African cultures, the
idea of sex as taboo restricts the number of
acceptable terms to be used when taking a sexual
history (Netshandama et al., 2017). Culturally
respectful language can facilitate talking openly
about one's sexual history and reduce ambiguity
or shame (Duby et al., 2016). In addition, HCPs
must be sensitive to the history of (mis)use of
sexual identity and orientation as a gatekeeping
function to exclude transgender people from
gender-affirming health care (Nieder \&
Richter-Appelt, 2011; Richards et al., 2014). The
following recommendations aim to improve sexual health care for TGD people.

All the statements in this chapter have been
recommended based on a thorough review of
evidence, an assessment of the benefits and
harms, values and preferences of providers and
patients, and resource use and feasibility. In some
cases, we recognize evidence is limited and/or
services may not be accessible or desirable.

\hypertarget{statement-17.1-we-recommend-health-care-professionals-who-provide-care-to-transgender-and-gender-diverse-people-acquire-the-knowledge-and-skills-to-address-sexual-health-issues-relevant-to-their-care-provision.}{%
\section*{Statement 17.1: We recommend health care professionals who provide care to transgender and gender diverse people acquire the knowledge and skills to address sexual health issues (relevant to their care provision).}\label{statement-17.1-we-recommend-health-care-professionals-who-provide-care-to-transgender-and-gender-diverse-people-acquire-the-knowledge-and-skills-to-address-sexual-health-issues-relevant-to-their-care-provision.}}
\addcontentsline{toc}{section}{Statement 17.1: We recommend health care professionals who provide care to transgender and gender diverse people acquire the knowledge and skills to address sexual health issues (relevant to their care provision).}

It is important HCPs addressing the sexual
health of TGD people be familiar with commonly
used terminology (see Chapter 1---Terminology)
and invite those seeking care to explain terms
with which the provider may not be familiar. In
this context, it is also important HCPs (are
prepared to) take a sexual history and offer treatment (according to their competencies) in a
gender-affirming way with a sex-positive approach
(Centers for Disease Control, 2020; Tomson et al.,
2021). However, HCP's should apply greater
importance to the terminology that the TGD
person uses for their own body over more traditionally accepted or used medical terminology
(Wesp, 2016). When talking about sexual practices, it is advisable to focus on body parts (e.g.,
``Do you have sex with people with a penis, people with a vagina, or both?''; ACON, 2022) and
what role they play in their sexuality (e.g.,
``During Sex, do any parts of your body enter
your partners body, such as their genitals, anus,
or mouth?''; ACON, 2022).

\hypertarget{statement-17.2-we-recommend-health-care-professionals-who-provide-care-to-transgender-and-gender-diverse-people-discuss-the-impact-of-gender-affirming-treatments-on-sexual-function-pleasure-and-satisfaction.}{%
\section*{Statement 17.2: We recommend health care professionals who provide care to transgender and gender diverse people discuss the impact of gender-affirming treatments on sexual function, pleasure, and satisfaction.}\label{statement-17.2-we-recommend-health-care-professionals-who-provide-care-to-transgender-and-gender-diverse-people-discuss-the-impact-of-gender-affirming-treatments-on-sexual-function-pleasure-and-satisfaction.}}
\addcontentsline{toc}{section}{Statement 17.2: We recommend health care professionals who provide care to transgender and gender diverse people discuss the impact of gender-affirming treatments on sexual function, pleasure, and satisfaction.}

To achieve gender-affirming care, it is crucial
HCPs providing transition-related medical interventions be sufficiently informed about the possible effects on sexual function, pleasure, and
satisfaction (T'Sjoen et al., 2020). Since clinical
data indicate that TGD people score significantly
lower in sexual pleasure compared to cisgender
individuals, this is even more important (Gieles
et al., 2022). If the HCP cannot provide information about the effects of their treatment on
sexual function, pleasure, and satisfaction, they
are at least expected to refer the individual to
someone qualified to do so. If the sexuality-related
effects of their treatment are unknown, HCPs
should inform their patients accordingly. As
introduced above, the sexuality of TGD people
often challenges heteronormative views.
Nevertheless, there is a large amount of literature
(e.g., Bauer, 2018; Laube et al., 2020; Hamm \&
Nieder, 2021; Stephenson et al., 2017) highlighting the spectrum character of sexuality that does
not fit into expectations of what male and female
sexuality entails (neither cis- nor transgender),
let alone that of gender diverse people (e.g., nonbinary, agender, genderqueer). Thus, these aspects
should be carefully considered by HCPs as
cisnormativity, heteronormativity, and
transition-related medical interventions, all have
a strong impact on sexual health.

Sexual pleasure has been well documented as
a factor in improving sexual, mental, and physical
health outcomes (Anderson, 2013). Next to sexual
function, HCPs providing sexual health care must
address sexual pleasure and satisfaction as a key
factor within sexual health. Historically sexual
health care has been disease focused, and this is
particularly true for research and clinical practice
in working with TGD patients. Although competent sexual health care regarding HIV and STIs
is necessary, integration of valuing sexual pleasure
of TGD patients is also necessary. Calls for integrating sexual pleasure as a focal point in STI
prevention education and interventions rest on
the understanding that pleasure is a motivator of
behavior (Philpott et al., 2006). TGD people are
concerned about their sexual pleasure and need
HCPs who are knowledgeable about the diversity
of sexual practices and anatomical functioning
particular to TGD health care.

\hypertarget{statement-17.3-we-recommend-health-care-professionals-who-provide-care-to-transgender-and-gender-diverse-people-offer-the-possibility-of-including-the-partners-in-sexuality-related-care-if-appropriate.}{%
\section*{Statement 17.3: We recommend health care professionals who provide care to transgender and gender diverse people offer the possibility of including the partner(s) in sexuality-related care, if appropriate.}\label{statement-17.3-we-recommend-health-care-professionals-who-provide-care-to-transgender-and-gender-diverse-people-offer-the-possibility-of-including-the-partners-in-sexuality-related-care-if-appropriate.}}
\addcontentsline{toc}{section}{Statement 17.3: We recommend health care professionals who provide care to transgender and gender diverse people offer the possibility of including the partner(s) in sexuality-related care, if appropriate.}

When appropriate and relevant to clinical concerns, inclusion of a sexual and/or romantic partner(s) in sexual health care decision-making can
increase TGD patients' sexual well-being and
satisfaction outcomes (Kleinplatz, 2012). TGD
people may choose a range of transition-related
medical interventions, and these interventions
may have mixed results in shifting experiences
of anatomical dysphoria (Bauer \& Hammond,
2015). When discussing the impact of medical
interventions on sexual functioning, pleasure, and
satisfaction, inclusion of partner(s) can increase
knowledge of potential changes and encourage
communication between partners (Dierckx et al.,
2019). Because the process of transitioning is
often not a completely solitary endeavor, the
inclusion of sexual and/or romantic partners in
transition-related health care can facilitate the
process of ``co-transitioning'' (Lindley et al., 2020;
Siboni et al., 2022; Theron \& Collier, 2013) and
can also support sexual growth and adjustment
both in the individual as well as in the relationship. Social and psychological barriers to sexual
functioning and pleasure, including experiences
of gender dysphoria, stigmatization, lack of sexual
and relationship role models, and limited skills,
can have negative impacts on overall sexual
health (Kerckhof et al., 2019). Supportive,
gender-affirming sexual communication between
partners improves sexual satisfaction outcomes
for TGD people (Stephenson et al., 2017; Wierckx,
Elaut et al., 2011).

Inclusion of sexual and/or romantic partners
offers an additional opportunity to set realistic
expectations, disseminate helpful and accurate
information, and facilitate gender-affirming positive
communication related to sexual health. Ultimately,
however, it is important to recognize individual
choices related to gender health and transition are
the patients to make, not a partner's decision. It
is important the inclusion of partners in sexual
health-related care occur only when appropriate
and as desired by patients. Contraindications might
include interpersonal dynamics that are abusive or
violent, in which case patient safety overrides partner involvement. Finally, it is critical HCPs treat
all people in an affirming and inclusive manner,
including sexual and romantic partners. This
means, for example, monitoring and addressing
assumptions and potential biases about the gender
or sexual orientation of a patient's partner(s) or a
patient's relationship structure.

\hypertarget{statement-17.4-we-recommend-health-care-professionals-counsel-transgender-and-gender-diverse-people-about-the-potential-impact-of-stigma-and-trauma-on-sexual-risk-behavior-sexual-avoidance-and-sexual-functioning.}{%
\section*{Statement 17.4: We recommend health care professionals counsel transgender and gender diverse people about the potential impact of stigma and trauma on sexual risk behavior, sexual avoidance, and sexual functioning.}\label{statement-17.4-we-recommend-health-care-professionals-counsel-transgender-and-gender-diverse-people-about-the-potential-impact-of-stigma-and-trauma-on-sexual-risk-behavior-sexual-avoidance-and-sexual-functioning.}}
\addcontentsline{toc}{section}{Statement 17.4: We recommend health care professionals counsel transgender and gender diverse people about the potential impact of stigma and trauma on sexual risk behavior, sexual avoidance, and sexual functioning.}

The TGD community is disproportionately
impacted by stigma, discrimination, and violence
(de Vries et al., 2020; European Union Agency
for Fundamental Rights, 2020; McLachlan, 2019).
These experiences are often traumatic in nature
(Burnes et al., 2016; Mizock \& Lewis, 2008) and
can create barriers to sexual health, functioning,
and pleasure (Bauer \& Hammond, 2015). For
example, stigmatizing narratives about
transgender sexualities can increase dysphoria
and sexual shame, increasing potential avoidance
of the sexual communication needed for safety
and optimizing pleasure (Stephenson et al.,
2017). Research demonstrates stigma, a history
of sexual violence, and body image concerns can
negatively impact sexual self-esteem and agency,
for example the ability to assert what is pleasurable or to negotiate condom use (Clements-Nolle
et al., 2008; Dharma et al., 2019). Additionally,
gender dysphoria can be exacerbated by past
trauma experiences and ongoing trauma-related
symptoms (Giovanardi et al., 2018). It may be
difficult for some TGD individuals to engage
sexually using the genitals with which they were
born, and they may choose to avoid such stimulation altogether, disrupting arousal and/or
orgasmic processes (Anzani et al., 2021; Bauer
\& Hammond, 2015; Iantaffi \& Bockting, 2011)
or result in complex feelings about orgasm
(Chadwick et al., 2019). HCPs providing
gender-affirming counseling and interventions
must be knowledgeable about the spectrum of
sexual orientations and identities (including asexual identities and practices) to avoid assumptions
based in heteronormative, cisnormative, allonormative modes of behavior or satisfaction while
also affirming the potential impacts of stigma
and trauma on sexual health and pleasure
(Nieder, Güldenring et al., 2020). Some level of
disconnect or dissociation may at times be present, particularly in the case of acute trauma
symptoms (Colizzi et al., 2015). It is important
HCPs be aware of these potential impacts on
sexual health, functioning, pleasure, and satisfaction, so they may refer patients as needed to
trauma-informed sexual counselors, mental
health providers, or both, who may be of further
assistance and may also normalize and validate
TGD patients exploring multiple diverse pathways of healing and accessing sexual pleasure.

\hypertarget{statement-17.5-we-recommend-any-health-care-professional-who-offers-care-that-may-impact-sexual-health-provide-information-ask-about-the-expectation-of-the-transgender-and-gender-diverse-individual-and-assess-their-level-of-understanding-of-possible-changes.}{%
\section*{Statement 17.5: We recommend any health care professional who offers care that may impact sexual health provide information, ask about the expectation of the transgender and gender diverse individual, and assess their level of understanding of possible changes.}\label{statement-17.5-we-recommend-any-health-care-professional-who-offers-care-that-may-impact-sexual-health-provide-information-ask-about-the-expectation-of-the-transgender-and-gender-diverse-individual-and-assess-their-level-of-understanding-of-possible-changes.}}
\addcontentsline{toc}{section}{Statement 17.5: We recommend any health care professional who offers care that may impact sexual health provide information, ask about the expectation of the transgender and gender diverse individual, and assess their level of understanding of possible changes.}

Transition-related care can affect sexual function, pleasure, and satisfaction, both in positive
and negative ways (Holmberg et al., 2018;
Kerckhof et al., 2019; Thurston \& Allan, 2018;
Tirapegui et al., 2020). On the positive side,
gender-affirming care can help TGD people
improve their sexual functioning and increase
their sexual pleasure and satisfaction (Kloer et al.,
2021; Özer et al., 2022; T'Sjoen et al., 2020). On
the negative side, however, data indicate problematic sexual health outcomes due to hormonal
and surgical treatments (Holmberg et al., 2018;
Kerckhof et al., 2019, Stephenson et al., 2017;
Weyers et al., 2009). Transition-related hormones
may affect mood, sexual desire, the ability to
have an erection and ejaculation, and genital tissue health, which in turn can impact sexual function, pleasure and sexual self-expression
(Defreyne, Elaut et al., 2020; Garcia \& Zaliznyak,
2020; Kerckhof et al., 2019; Klein \& Gorzalka,
2009; Wierckx, Elaut et al., 2014). TGD people
who wish to use their original genital anatomy
for penetrative sex may benefit from medications
that address sexual health side effects of hormone
therapy, such as erectile dysfunction, medications
for TGD persons taking estrogen or antiandrogens, and topical estrogen and/or moisturizers
for TGD persons experiencing vaginal atrophy or
dryness due to testosterone therapy.

Sexual desire, arousal, and function may also
be affected by the use of psychotropic drugs
(Montejo et al., 2015). As some TGD people are
prescribed medication to treat depression
(Heylens, Elaut et al., 2014), anxiety (Millet et al.,
2017) or other mental health concerns (Dhejne
et al., 2016), their potential side effects on sexual
health should be considered.

Many gender-affirming surgeries can have significant effects on erogenous sensation, sexual
desire and arousal as well as sexual function and
pleasure. The impact of these changes for patients
may be mixed (Holmberg et al., 2018). Chest
surgeries (breast reduction, mastectomy, and
breast augmentation) and body contouring surgeries, for example, may offer desired changes in
form and appearance thereby reducing psychological distress that can disrupt sexual functioning but may adversely affect erogenous sensation
(Bekeny et al., 2020; Claes et al., 2018; Rochlin
et al., 2020). Genital surgeries in particular can
potentially affect sexual function and pleasure in
adverse ways, although they are likely to be experienced positively as the patient's body becomes
more aligned with their gender, potentially opening new avenues for sexual pleasure and satisfaction (Hess et al., 2018; Holmberg et al., 2018;
Kerckhof et al., 2019).

There are numerous examples of this in the extant literature:

\begin{itemize}
\tightlist
\item
  Surgery may result in a decrease, a total loss, or a possible increase in erogenous stimulation and/or experienced sensation compared with the patient's presurgery anatomy (Garcia, 2018; Sigurjónsson et al., 2017).
\item
  A particular surgical option may be associated with specific limitations to sexual function that may manifest immediately, in the future, or at both timepoints, and which patients should consider before finalizing their choice when considering different surgical options (Frey et al., 2016; Garcia, 2018; Isaacson et al., 2017).
\item
  Postsurgical complications can adversely affect sexual function by either decreasing the quality of sexual function (e.g., discomfort or pain with sexual activity) or by precluding satisfactory intercourse (Kerckhof et al., 2019; Schardein et al., 2019).
\end{itemize}

In general, satisfaction with any medical treatment is heavily influenced by the patient's
expectations (Padilla et al., 2019). Furthermore,
when patients have unrealistic expectations
before treatment, they are much more likely to
be dissatisfied with the outcome, their care, and
with their HCP (Padilla et al., 2019). Therefore,
it is important to both provide patients with
adequate information about their treatment
options and to understand and consider what
is important to the patient with regard to outcomes (Garcia, 2021). Finally, it is important
the HCP ensure patients understand the potential adverse effects of a treatment on their sexual
function and pleasure so that a well-informed
decision can be made. This is relevant for both
meeting the standard of informed consent (i.e.,
discussion and understanding) and for providing
an opportunity to offer further clarification to
patients and, if desired, to their partners (Glaser
et al., 2020).

\hypertarget{statement-17.6-we-recommend-health-care-professionals-who-provide-care-to-transgender-and-gender-diverse-people-counsel-adolescents-and-adults-regarding-prevention-of-sexually-transmitted-infections.}{%
\section*{Statement 17.6: We recommend health care professionals who provide care to transgender and gender diverse people counsel adolescents and adults regarding prevention of sexually transmitted infections.}\label{statement-17.6-we-recommend-health-care-professionals-who-provide-care-to-transgender-and-gender-diverse-people-counsel-adolescents-and-adults-regarding-prevention-of-sexually-transmitted-infections.}}
\addcontentsline{toc}{section}{Statement 17.6: We recommend health care professionals who provide care to transgender and gender diverse people counsel adolescents and adults regarding prevention of sexually transmitted infections.}

The WHO (2015) recommends HCPs implement brief sexuality-related communication in
primary care for all adolescents and adults.
Therefore, TGD persons who are sexually active
or considering sexual activity may benefit from
sexuality-related communication or counseling
for the purpose of HIV/STI prevention. These
conversations are particularly important as TGD
persons are disproportionately impacted by
human immunodeficiency virus (HIV) and other
sexually transmitted infections (STIs) relative to
cisgender persons (Baral et al., 2013; Becasen
et al., 2018; Poteat et al., 2016). However, few
data are available for non-HIV STIs, such as chlamydia, gonorrhea, syphilis, viral hepatitis, and
herpes simplex virus (Tomson et al., 2021). The
United Nations Joint Programme on HIV/AIDS
estimates transgender women are 12 times more
likely than other adults to be living with HIV
(UNAIDS, 2019). A meta-analysis estimated a
pooled global HIV prevalence of 19\% among
transgender women who have sex with men
(Baral et al., 2013). HIV/STI risk is concentrated
among TGD subgroups at the confluence of multiple biological, psychological, interpersonal, and
structural vulnerabilities. In particular, transfeminine persons who have sex with cisgender men,
belong to minoritized racial/ethnic groups, live
in poverty, and engage in survival sex work are
at elevated HIV/STI risk (Becasen et al., 2018;
Poteat et al., 2015; Poteat et al., 2016). Less is
known about HIV/STI risk among transgender
men or gender diverse persons AFAB. Small studies in high-income countries indicate a
laboratory-confirmed HIV prevalence of 0-4\%
among transmasculine people (Becasen et al.,
2018; Reisner \& Murchison, 2016). Almost no
research has been conducted with transmasculine
people who have sex with cisgender men in
high-HIV-prevalence countries. Despite limited
epidemiologic data, transmasculine persons who
have sex with cisgender men frequently report
HIV/STI risk related to receptive vaginal and/or
anal sex (Golub et al., 2019; Reisner et al., 2019;
Scheim et al., 2017) and may be more susceptible
to HIV acquisition from vaginal intercourse than
(pre-menopausal) cisgender women due to
hormone-related vaginal atrophy.

HCPs will need to supplement general guidelines by developing the knowledge and skills
needed for discussing sexual health issues with
TGD people, such as the use of gender-affirming
language (see Statement 17.1 in this chapter). It
is critical HCPs avoid assumptions about HIV/
STI risk based solely on a patient's gender identity or anatomy. For example, many transgender
people are not sexually active, and TGD persons
may use prosthetics or toys for sex. To provide
appropriate prevention counseling, HCPs should
inquire about the specific sexual activities TGD
people engage in, and the body parts (or prosthetics) involved in those activities (ACON,
2022). Well-prepared HCPs (including, but not
limited to mental health providers) may also
engage in in-depth counseling with their patients
to address the underlying drivers of HIV/STI risk
(see Statement 17.3 in this chapter).

In all cases, HCPs should be sensitive to the
collective and individual histories of TGD people
(e.g., stereotypes and stigma about trans sexualities and gender dysphoria) and should explain to
patients the reasons for sexuality-related inquiries
and the voluntary nature of such inquiries. In
discussing HIV/STI prevention, HCPs should refer
to the full range of prevention options including
barrier methods, post-exposure prophylaxis,
pre-exposure prophylaxis, and HIV treatment to
prevent onwards transmission (WHO, 2021).
Trans-specific considerations for pre-exposure prophylaxis are addressed in Statement 17.8.

\hypertarget{statement-17.7-we-recommend-health-care-professionals-who-provide-care-to-transgender-and-gender-diverse-people-follow-local-and-world-health-organization-guidelines-for-human-immunodeficiency-virussexual-transmitted-infections-hiv-stis-screening-prevention-and-treatment.}{%
\section*{Statement 17.7: We recommend health care professionals who provide care to transgender and gender diverse people follow local and World Health Organization guidelines for human immunodeficiency virus/sexual transmitted infections (HIV/ STIs) screening, prevention, and treatment.}\label{statement-17.7-we-recommend-health-care-professionals-who-provide-care-to-transgender-and-gender-diverse-people-follow-local-and-world-health-organization-guidelines-for-human-immunodeficiency-virussexual-transmitted-infections-hiv-stis-screening-prevention-and-treatment.}}
\addcontentsline{toc}{section}{Statement 17.7: We recommend health care professionals who provide care to transgender and gender diverse people follow local and World Health Organization guidelines for human immunodeficiency virus/sexual transmitted infections (HIV/ STIs) screening, prevention, and treatment.}

Like cisgender patients, TGD adolescents and
adults should be offered screening for HIV/STIs
in accordance with existing guidelines and based
on their individual risk of HIV/STI acquisition,
considering anatomy and behavior rather than
gender identity alone. Where local or national
guidelines are unavailable, WHO (2019a) offers
global recommendations; more frequent screening
is recommended for transgender people who have
sex with cisgender men as a key population
affected by HIV.

Gender-affirming genital surgeries and surgical
techniques have implications for STI risks and
screening needs, as outlined in recent guidelines
from the US Centers for Disease Control
(Workowski et al., 2021). For instance, transfeminine persons who have had penile inversion
vaginoplasty using only penile and scrotal skin
to line the vaginal canal are likely at lower risk
of urogenital Chlamydia trachomatis (C. trachomatis) and Neisseria gonorrhoeae (N. gonorrhoeae), but newer surgical techniques that
employ buccal or urethral mucosa or peritoneum
flaps could in theory increase susceptibility to
bacterial STIs relative to the use of penile/scrotal
skin alone (Van Gerwen et al., 2021). Routine
STI screening of the neovagina (if exposed) is
recommended for all transfeminine persons who
have had vaginoplasty (Workowski et al., 2021).
For transmasculine persons who have had metoidioplasty with urethral lengthening, but not vaginectomy, testing for bacterial urogenital STIs
should include a cervical swab because infections
may not be detected in urine (Workowski
et al., 2021).

Further, it is important for HCPs to offer testing at multiple anatomical sites as STIs in transgender patients are often extragenital
(Hiransuthikul et al., 2019; Pitasi et al., 2019).
Consistent with WHO (2020) recommendations,
self-collection of samples for STI testing should
be offered as an option, particularly if patients
are uncomfortable or unwilling to undergo
provider-collected sampling due to gender dysphoria, trauma histories, or both. Where relevant,
integration of HIV/STI testing with regular serology used to monitor hormone therapy may better
facilitate access to care (Reisner, Radix et al.,
2016; Scheim \& Travers, 2017).

\hypertarget{statement-17.8-we-recommend-health-care-professionals-who-provide-care-to-transgender-and-gender-diverse-people-address-concerns-about-potential-interactions-between-antiretroviral-medications-and-hormones.}{%
\section*{Statement 17.8: We recommend health care professionals who provide care to transgender and gender diverse people address concerns about potential interactions between antiretroviral medications and hormones.}\label{statement-17.8-we-recommend-health-care-professionals-who-provide-care-to-transgender-and-gender-diverse-people-address-concerns-about-potential-interactions-between-antiretroviral-medications-and-hormones.}}
\addcontentsline{toc}{section}{Statement 17.8: We recommend health care professionals who provide care to transgender and gender diverse people address concerns about potential interactions between antiretroviral medications and hormones.}

For TGD adolescents and adults at substantial
risk of HIV infection (generally defined as an
ongoing serodiscordant relationship or condomless sex outside of a mutually monogamous relationship with a known HIV-negative partner;
WHO, 2017), pre-exposure prophylaxis (PrEP) is
an important HIV prevention option (Golub
et al., 2019; Sevelius et al., 2016; WHO, 2021).
To encourage uptake of PrEP, in 2021 the US
Centers for Disease Control recommended all
sexually active adolescents and adults be informed
about PrEP and offered it if requested (CDC,
2021). For treatment among people living with
HIV, transgender-specific guidelines are available
in some settings (e.g., Panel on Antiretroviral
Guidelines for Adults and Adolescents, 2019).

For both HIV prevention and treatment, there
are antiretroviral dosing and administration considerations specific to TGD persons. For oral PrEP,
only daily dosing is currently recommended for
TGD persons as studies demonstrating the effectiveness of event-driven PrEP with emtricitabine/
tenofovir disoproxil fumarate (TDF) have been
limited to cisgender men (WHO, 2019c). In addition, while emtricitabine/tenofovir alafenamide
(TAF) is a new oral PrEP option, as of early 2022
it is not recommended for people at risk of HIV
acquisition through receptive vaginal sex due to a
lack of evidence (CDC, 2021). Finally, long-acting
injectable formulations of both PrEP and HIV
treatment are increasingly available (e.g., cabotegravir for PrEP), and while they are recommended
for all patients who might benefit from injectable
options, indicated injection sites (i.e., the gluteal
muscle) may be unsuitable for individuals who
have used soft tissue fillers (Rael et al., 2020).

There is little evidence supporting the occurrence of drug-drug interactions between
gender-affirming hormones and PrEP medications. A few small studies, primarily relying on
self-reported PrEP use, have shown reduced
PrEP drug concentrations in transgender women
undergoing hormone therapy, although
concentrations remained in the protective range
(Yager \& Anderson, 2020). A subsequent
drug-drug interaction study using directly
observed PrEP therapy failed to detect an impact
of hormone therapy on PrEP drug concentrations in transgender women and found transgender women and men taking hormone therapy
achieved high levels of protection against HIV
infection (Grant et al., 2020). Most importantly,
for many TGD people, no impact of PrEP on
hormone concentrations has been detected. With
regard to HIV treatment, specific antiretroviral
medications may impact hormone concentrations; however, these can be managed by selecting alternative agents, monitoring and adjusting
hormone dosing, or both (Cirrincione et al.,
2020) as detailed in guidelines from the US
Department of Health and Human Services
(Panel on Antiretroviral Guidelines for Adults
and Adolescents, 2019). Nevertheless, concerns
about drug-drug interactions, particularly interactions that may limit hormone concentrations,
represent a barrier to the implementation and
adherence to antiretroviral therapy for HIV prevention or treatment (Radix et al., 2020; Sevelius
et al., 2016). Therefore, it is advisable for HCPs
to proactively address such concerns with those
who are candidates for PrEP or HIV treatment.
Integration of PrEP or HIV treatment with hormone therapy may further reduce barriers to
implementation and adherence (Reisner, Radix
et al., 2016). Integration may be achieved
through colocation or through coordination
with an HIV specialist if the primary care provider does not have the necessary expertise.
Some TGD persons may benefit from standalone
PrEP or sexual health services that provide
greater privacy and flexibility, and thus differentiated service delivery models are needed
(Wilson et al., 2021).

\hypertarget{mental-health-1}{%
\chapter{Mental Health}\label{mental-health-1}}

This chapter is intended to provide guidance to
health care professionals (HCPs) and mental health
professionals (MHPs) who offer mental health care
to transgender and gender diverse (TGD) adults. It
is not meant to be a substitute for chapters on the
assessment or evaluation of people for hormonal or
surgical interventions. Many TGD people will not
require therapy or other forms of mental health care
as part of their transition, while others may benefit
from the support of mental health providers and
systems (Dhejne et al., 2016).

Some studies have shown a higher prevalence of
depression (Witcomb et al., 2018), anxiety (Bouman
et al., 2017), and suicidality (Arcelus et al., 2016;
Bränström \& Pachankis, 2022; Davey et al., 2016;
Dhejne, 2011; Herman et al., 2019) among TGD
people (Jones et al., 2019; Thorne, Witcomb et al.,
2019) than in the general population, particularly
in those requiring medically necessary
gender-affirming medical treatment (see medically
necessary statement in Chapter 2---Global
Applicability, Statement 2.1). However, transgender
identity is not a mental illness, and these elevated
rates have been linked to complex trauma, societal
stigma, violence, and discrimination (Nuttbrock
et al., 2014; Peterson et al., 2021). In addition, psychiatric symptoms lessen with appropriate
gender-affirming medical and surgical care (Aldridge
et al., 2020; Almazan and Keuroghlian; 2021; Bauer
et al., 2015; Grannis et al., 2021) and with interventions that lessen discrimination and minority
stress (Bauer et al., 2015; Heylens, Verroken et al.,
2014; McDowell et al., 2020).

Mental health treatment needs to be provided by
staff and implemented through the use of systems
that respect patient autonomy and recognize gender
diversity. MHPs working with transgender people
should use active listening as a method to encourage
exploration in individuals who are uncertain about
their gender identity. Rather than impose their own
narratives or preconceptions, MHPs should assist
their clients in determining their own paths. While
many transgender people require medical or surgical
interventions or seek mental health care, others do
not (Margulies et al., 2021). Therefore, findings from
research involving clinical populations should not
be extrapolated to the entire transgender population.

Addressing mental illness and substance use
disorders is important but should not be a barrier
to transition-related care. Rather, these interventions to address mental health and substance use
disorders can facilitate successful outcomes from
transition-related care, which can improve quality
of life (Nobili et al., 2018).

All the statements in this chapter have been
recommended based on a thorough review of
evidence, an assessment of the benefits and
harms, values and preferences of providers and
patients, and resource use and feasibility. In some
cases, we recognize evidence is limited and/or
services may not be accessible or desirable.

\hypertarget{statement-18.1-we-recommend-mental-health-professionals-address-mental-health-symptoms-that-interfere-with-a-persons-capacity-to-consent-to-genderaffirming-treatment-before-gender-affirming-treatment-is-initiated.}{%
\section*{Statement 18.1: We recommend mental health professionals address mental health symptoms that interfere with a person's capacity to consent to genderaffirming treatment before gender-affirming treatment is initiated.}\label{statement-18.1-we-recommend-mental-health-professionals-address-mental-health-symptoms-that-interfere-with-a-persons-capacity-to-consent-to-genderaffirming-treatment-before-gender-affirming-treatment-is-initiated.}}
\addcontentsline{toc}{section}{Statement 18.1: We recommend mental health professionals address mental health symptoms that interfere with a person's capacity to consent to genderaffirming treatment before gender-affirming treatment is initiated.}

Because patients generally are assumed to be
capable of providing consent for care, whether
the presence of cognitive impairment, psychosis,
or other mental illness impairs the ability to give
informed consent is subject to individual examination (Applebaum, 2007). Informed consent is
central to the provision of health care. The health
care provider must educate the patient about the
risks, benefits, and alternatives to any care that
is offered so the patient can make an informed,
voluntary choice (Berg et al., 2001). Both the
primary care provider or endocrinologist prescribing hormones and the surgeon performing
surgery must obtain informed consent. Similarly,
MHPs obtain informed consent for mental health
treatment and may consult on a patient's capacity
to give informed consent when this is in question. Psychiatric illness and substance use disorders, in particular cognitive impairment and
psychosis, may impair an individual's ability to
understand the risks and benefits of the treatment
(Hostiuc et al., 2018). Conversely, a patient may
also have significant mental illness, yet still be
able to understand the risks and benefits of a
particular treatment (Carpenter et al., 2000).
Multidisciplinary communication is important in
challenging cases, and expert consultation should
be utilized as needed (Karasic \& Fraser, 2018).
For many patients, difficulty understanding the
risks and benefits of a particular treatment can
be overcome with time and careful explanation.
For some patients, treatment of the underlying
condition that is interfering with the capacity to
give informed consent---for example treating an
underlying psychosis---will allow the patient to
gain the capacity to consent to the required treatment. However, mental health symptoms such as
anxiety or depressive symptoms that do not affect
the capacity to give consent should not be a barrier for gender-affirming medical treatment, particularly as this treatment has been found to
reduce mental health symptomatology (Aldridge
et al., 2020).

\hypertarget{statement-18.2-we-recommend-mental-health-professionals-offer-care-and-support-to-transgender-and-gender-diverse-people-to-address-mental-health-symptoms-that-interfere-with-a-persons-capacity-to-participate-in-essential-perioperative-care-before-gender-affirmation-surgery.}{%
\section*{Statement 18.2: We recommend mental health professionals offer care and support to transgender and gender diverse people to address mental health symptoms that interfere with a person's capacity to participate in essential perioperative care before gender-affirmation surgery.}\label{statement-18.2-we-recommend-mental-health-professionals-offer-care-and-support-to-transgender-and-gender-diverse-people-to-address-mental-health-symptoms-that-interfere-with-a-persons-capacity-to-participate-in-essential-perioperative-care-before-gender-affirmation-surgery.}}
\addcontentsline{toc}{section}{Statement 18.2: We recommend mental health professionals offer care and support to transgender and gender diverse people to address mental health symptoms that interfere with a person's capacity to participate in essential perioperative care before gender-affirmation surgery.}

The inability to adequately participate in
perioperative care due to mental illness or substance use should not be viewed as an obstacle
to needed transition care, but should be seen as
an indication mental health care and social support be provided (Karasic, 2020). Mental illness
and substance use disorders may impair the ability of the patient to participate in perioperative
care (Barnhill, 2014). Visits to health care providers, wound care, and other aftercare procedures (e.g., dilation after vaginoplasty) may be
necessary for a good outcome. A patient with a
substance use disorder might have difficulty
keeping necessary appointments to the primary
care provider and the surgeon. A patient with
psychosis or severe depression might neglect
their wound or not be attentive to infection or
signs of dehiscence (Lee, Marsh et al., 2016).
Active mental illness is associated with a greater
need for further acute medical and surgical care
after the initial surgery (Wimalawansa
et al., 2014).

In these cases, treatment of the mental illness
or substance use disorder may assist in achieving
successful outcomes. Arranging more support for
the patient from family and friends or a home
health care worker may help the patient participate sufficiently in perioperative care for surgery
to proceed. The benefits of mental health treatments that may delay surgery should be weighed
against the risks of delaying surgery and should
include an assessment of the impact on the
patients' mental health delays may cause in
addressing gender dysphoria (Byne et al., 2018).

\hypertarget{statement-18.3-we-recommend-when-significant-mental-health-symptoms-or-substance-abuse-exists-mental-health-professionals-assess-the-potential-negative-impact-mental-health-symptoms-may-have-on-outcomes-based-on-the-nature-of-the-specific-gender-affirming-surgical-procedure.}{%
\section*{Statement 18.3: We recommend when significant mental health symptoms or substance abuse exists, mental health professionals assess the potential negative impact mental health symptoms may have on outcomes based on the nature of the specific gender-affirming surgical procedure.}\label{statement-18.3-we-recommend-when-significant-mental-health-symptoms-or-substance-abuse-exists-mental-health-professionals-assess-the-potential-negative-impact-mental-health-symptoms-may-have-on-outcomes-based-on-the-nature-of-the-specific-gender-affirming-surgical-procedure.}}
\addcontentsline{toc}{section}{Statement 18.3: We recommend when significant mental health symptoms or substance abuse exists, mental health professionals assess the potential negative impact mental health symptoms may have on outcomes based on the nature of the specific gender-affirming surgical procedure.}

Gender-affirming surgical procedures vary in
terms of their impact on the patient. Some procedures require a greater ability to follow preoperative planning as well as engage in peri- and
postoperative care to achieve the best outcomes
(Tollinche et al., 2018). Mental health symptoms
can influence a patient's ability to participate in
the planning and perioperative care necessary for
any surgical procedure (Paredes et al., 2020). The
mental health assessment can provide an opportunity to develop strategies to address the potential negative impact mental health symptoms may
have on outcomes and to plan support for the
patient's ability to participate in the planning and
care. Gender-affirming surgical procedures have
been shown to relieve symptoms of gender dysphoria and improve mental health (Owen-Smith
et al., 2018; van de Grift, Elaut et al., 2017).
These benefits are weighed against the risks of
each procedure when the patient and provider
are deciding whether to proceed with the treatment. HCPs can assist TGD people in reviewing
preplanning and perioperative care instructions
for each surgical procedure (Karasic, 2020).
Provider and patient can collaboratively determine the necessary support or resources needed
to assist with keeping appointments for perioperative care, obtaining necessary supplies, addressing financial issues, and handling other
preoperative coordination and planning. In addition, issues surrounding appearance-related and
functional expectations, including the impact of
these various factors on gender dysphoria, can
be explored.

\hypertarget{statement-18.4-we-recommend-health-care-professionals-assess-the-need-for-psychosocial-and-practical-support-of-transgender-and-gender-diverse-people-in-the-perioperative-period-surrounding-gender-affirmation-surgery.}{%
\section*{Statement 18.4: We recommend health care professionals assess the need for psychosocial and practical support of transgender and gender diverse people in the perioperative period surrounding gender-affirmation surgery.}\label{statement-18.4-we-recommend-health-care-professionals-assess-the-need-for-psychosocial-and-practical-support-of-transgender-and-gender-diverse-people-in-the-perioperative-period-surrounding-gender-affirmation-surgery.}}
\addcontentsline{toc}{section}{Statement 18.4: We recommend health care professionals assess the need for psychosocial and practical support of transgender and gender diverse people in the perioperative period surrounding gender-affirmation surgery.}

Regardless of specialty, all HCPs have a responsibility to support patients in accessing medically
necessary care. When HCPs are working with
TGD people as they prepare for gender-affirming
surgical procedures, they should assess the levels
of psychosocial and practical support required
(Deutsch, 2016b). Assessment is the first step in
recognizing where additional support may be
needed and enhancing the ability to work collaboratively with the individual to successfully
navigate the pre-, peri-, and postsurgical periods
(Tollinche et al., 2018). In the perioperative
period, it is important to help patients optimize
functioning, secure stable housing, when possible,
build social and family supports by assessing their
unique situation, plan ways of responding to
medical complications, navigate the potential
impact on work/income, and overcome additional
hurdles some patients may encounter, such as
coping with electrolysis and tobacco cessation
(Berli et al., 2017). In a complex medical system,
not all patients will be able to independently
navigate the procedures required to obtain care,
and HCPs and peer navigators can support
patients through this process (Deutsch, 2016a).

\hypertarget{statement-18.5-we-recommend-health-care-professionals-counsel-and-assist-transgender-and-gender-diverse-people-in-becoming-abstinent-from-tobacco-nicotine-prior-to-gender-affirmation-surgery.}{%
\section*{Statement 18.5: We recommend health care professionals counsel and assist transgender and gender diverse people in becoming abstinent from tobacco/ nicotine prior to gender-affirmation surgery.}\label{statement-18.5-we-recommend-health-care-professionals-counsel-and-assist-transgender-and-gender-diverse-people-in-becoming-abstinent-from-tobacco-nicotine-prior-to-gender-affirmation-surgery.}}
\addcontentsline{toc}{section}{Statement 18.5: We recommend health care professionals counsel and assist transgender and gender diverse people in becoming abstinent from tobacco/ nicotine prior to gender-affirmation surgery.}

Transgender populations have higher rates of
tobacco and nicotine use (Kidd et al., 2018).
However, many are unaware of the
well-documented smoking-associated health risks
(Bryant et al., 2014). Tobacco consumption
increases the risk of developing health problems
(e.g., thrombosis) in individuals receiving
gender-affirming hormone treatment, particularly
estrogens (Chipkin \& Kim, 2017).

Tobacco use has been associated with worse outcomes in plastic surgery, including overall complications, tissue necrosis, and the need for surgical
revision (Coon et al., 2013). Smoking also increases
the risk for postoperative infection (Kaoutzanis
et al., 2019). Tobacco use has been shown to affect
the healing process following any surgery, including
gender-related surgeries (e.g., chest reconstructive
surgery, genital surgery) (Pluvy, Garrido et al.,
2015). Tobacco users have a higher risk of cutaneous necrosis, delayed wound healing, and scarring disorders due to hypoxia and tissue ischemia
(Pluvy, Panouilleres et al., 2015). In view of this,
surgeons recommend stopping the use of tobacco/
nicotine prior to gender-affirmation surgery and
abstaining from smoking up to several weeks postoperatively until the wound has completely healed
(Matei \& Danino, 2015). Despite the risks, cessation may be difficult. Tobacco smoking and nicotine use is addictive and is also used as a coping
mechanism (Matei et al., 2015). HCPs who see
patients longitudinally before surgery, including
mental health and primary care providers, should
address the use of tobacco/nicotine with individuals
in their care, and either assist TGD people in
accessing smoking cessation programs or provide
treatment directly (e.g., varenicline or bupropion).

\hypertarget{statement-18.6-we-recommend-health-care-professionals-maintain-existing-hormone-treatment-if-a-transgender-and-gender-diverse-individual-requires-admission-to-a-psychiatric-or-medical-inpatient-unit-unless-contraindicated.}{%
\section*{Statement 18.6: We recommend health care professionals maintain existing hormone treatment if a transgender and gender diverse individual requires admission to a psychiatric or medical inpatient unit, unless contraindicated.}\label{statement-18.6-we-recommend-health-care-professionals-maintain-existing-hormone-treatment-if-a-transgender-and-gender-diverse-individual-requires-admission-to-a-psychiatric-or-medical-inpatient-unit-unless-contraindicated.}}
\addcontentsline{toc}{section}{Statement 18.6: We recommend health care professionals maintain existing hormone treatment if a transgender and gender diverse individual requires admission to a psychiatric or medical inpatient unit, unless contraindicated.}

TGD people entering inpatient psychiatric, substance use treatment, or medical units should be
maintained on their current hormone regimens.
There is an absence of evidence supporting routine cessation of hormones prior to medical or
psychiatric admissions. Rarely, a newly admitted
patient may be diagnosed with a medical complication necessitating suspension of hormone
treatment, for example an acute venous thromboembolism (Deutsch, 2016a). There is no strong
evidence for routinely stopping hormone treatment prior to surgery, and the risks and benefits
for each individual patient should be assessed
before doing so (Boskey et al., 2018).

Hormone treatment has been shown to improve
quality of life and to decrease depression and
anxiety (Aldridge et al., 2020; Nguyen et al., 2018;
Nobili et al., 2018; Owen-Smith et al., 2018,
Rowniak et al., 2019). Access to gender-affirming
medical treatment is associated with a substantial
reduction in the risk of suicide attempt (Bauer
et al., 2015). Halting a patient's regularly prescribed hormones denies the patient of these
salutary effects, and therefore may be counter to
the goals of hospitalization.

Some providers may be unaware of the low risk
of harm and the high potential benefit of continuing transition-related treatment in the inpatient
setting. A study of US and Canadian medical
schools revealed that students received an average
of 5 hours of LGBT-related course content over
their entire four years of education (Obedin-Maliver
et al., 2011). According to a survey of Emergency
Medicine physicians, who are often responsible for
making quick decisions about medications as
patients are being admitted, while 88\% reported
caring for transgender patients, only 17.5\% had
received any formal training about this population
(Chisolm-Straker et al., 2018). As education about
transgender topics increases, more providers will
become aware of the importance of maintaining
transgender patients on their hormone regimens
during hospitalization.

\hypertarget{statement-18.7-we-recommend-health-care-professionals-ensure-if-transgender-and-gender-diverse-people-need-inpatient-or-residential-mental-health-substance-abuse-or-medical-care-all-staff-use-the-correct-name-and-pronouns-as-provided-by-the-patient-as-well-as-provide-access-to-bathroom-and-sleeping-arrangements-that-are-aligned-with-the-persons-gender-identity.}{%
\section*{Statement 18.7: We recommend health care professionals ensure if transgender and gender diverse people need inpatient or residential mental health, substance abuse, or medical care, all staff use the correct name and pronouns (as provided by the patient), as well as provide access to bathroom and sleeping arrangements that are aligned with the person's gender identity.}\label{statement-18.7-we-recommend-health-care-professionals-ensure-if-transgender-and-gender-diverse-people-need-inpatient-or-residential-mental-health-substance-abuse-or-medical-care-all-staff-use-the-correct-name-and-pronouns-as-provided-by-the-patient-as-well-as-provide-access-to-bathroom-and-sleeping-arrangements-that-are-aligned-with-the-persons-gender-identity.}}
\addcontentsline{toc}{section}{Statement 18.7: We recommend health care professionals ensure if transgender and gender diverse people need inpatient or residential mental health, substance abuse, or medical care, all staff use the correct name and pronouns (as provided by the patient), as well as provide access to bathroom and sleeping arrangements that are aligned with the person's gender identity.}

Many TGD patients encounter discrimination
in a wide range of health settings, including hospitals, mental health treatment settings, and drug
treatment programs (Grant et al., 2011). When
health systems fail to accommodate TGD individuals, they reinforce the longstanding societal
exclusion many have experienced (Karasic, 2016).
Experiences of discrimination in health settings
lead to avoidance of needed health care due to
anticipated discrimination (Kcomt et al., 2020).

The experience of discrimination experienced
by TGD individuals is predictive of suicidal ideation (Rood et al., 2015; Williams et al., 2021).
Gender minority stress associated with rejection
and nonaffirmation has also been associated with
suicidality (Testa et al., 2017). Denial of access
to gender appropriate bathrooms has been
associated with increased suicidality (Seelman,
2016). However, the use of chosen names for
TGD people has been associated with lower
depression and suicidality (Russell et al., 2018).
Structural as well as internalized transphobia
must be addressed to reduce the incidence of
suicide attempts in TGD people (Brumer et al.,
2015). To successfully provide care, health settings must minimize the harm done to patients
because of transphobia by respecting and accommodating TGD identities.

\hypertarget{statement-18.8-we-recommend-mental-health-professionals-encourage-support-and-empower-transgender-and-gender-diverse-people-to-develop-and-maintain-social-support-systems-including-peers-friends-and-families.}{%
\section*{Statement 18.8: We recommend mental health professionals encourage, support, and empower transgender and gender diverse people to develop and maintain social support systems, including peers, friends, and families.}\label{statement-18.8-we-recommend-mental-health-professionals-encourage-support-and-empower-transgender-and-gender-diverse-people-to-develop-and-maintain-social-support-systems-including-peers-friends-and-families.}}
\addcontentsline{toc}{section}{Statement 18.8: We recommend mental health professionals encourage, support, and empower transgender and gender diverse people to develop and maintain social support systems, including peers, friends, and families.}

While minority stress and the direct effects of
discriminatory societal discrimination can be
harmful to the mental health of TGD people,
strong social support can help lessen this harm
(Trujillo et al., 2017). TGD children often internalize rejection from family and peers as well as
the transphobia that surrounds them (Amodeo
et al., 2015). Furthermore, exposure to transphobic
abuse may be impactful across a person's lifespan
and may be particularly acute during the adolescent years (Nuttbrock et al., 2010).

The development of affirming social support
is protective of mental health. Social support can
act as a buffer against the adverse mental health
consequences of violence, stigma, and discrimination (Bockting et al., 2013), can assist in navigating health systems (Jackson Levin et al.,
2020), and can contribute to psychological resilience in TGD people (Bariola et al., 2015; Başar
and Öz, 2016). Diverse sources of social support,
especially LGBTQ + peers and family, have been
found to be associated with better mental health
outcomes, well-being, and quality of life (Bariola
et al., 2015; Başar et al., 2016; Kuper, Adams
et al., 2018; Puckett et al., 2019). Social support
has been proposed to facilitate the development
of coping mechanisms and lead to positive emotional experiences throughout the transition process (Budge et al., 2013).

HCPs can support patients in developing social
support systems that allow them to be recognized
and accepted as their authentic identity and help
them cope with symptoms of gender dysphoria.
Interpersonal problems and lack of social support
have been associated with a greater incidence of
mental health difficulties in TGD people (Bouman,
Davey et al., 2016; Davey et al., 2015) and have
been shown to be an outcome predictor of
gender-affirming medical treatment (Aldridge
et al., 2020). Therefore, HCPs should encourage,
support, and empower TGD people to develop
and maintain social support systems. These experiences can foster the development of interpersonal skills and help with coping with societal
discrimination, potentially reducing suicidality
and improving mental health (Pflum et al., 2015).

\hypertarget{statement-18.9-we-recommend-health-care-professionals-should-not-make-it-mandatory-for-transgender-and-gender-diverse-people-to-undergo-psychotherapy-prior-to-the-initiation-of-gender-affirming-treatment-while-acknowledging-psychotherapy-may-be-helpful-for-some-transgender-and-gender-diverse-people.}{%
\section*{Statement 18.9: We recommend health care professionals should not make it mandatory for transgender and gender diverse people to undergo psychotherapy prior to the initiation of gender-affirming treatment, while acknowledging psychotherapy may be helpful for some transgender and gender diverse people.}\label{statement-18.9-we-recommend-health-care-professionals-should-not-make-it-mandatory-for-transgender-and-gender-diverse-people-to-undergo-psychotherapy-prior-to-the-initiation-of-gender-affirming-treatment-while-acknowledging-psychotherapy-may-be-helpful-for-some-transgender-and-gender-diverse-people.}}
\addcontentsline{toc}{section}{Statement 18.9: We recommend health care professionals should not make it mandatory for transgender and gender diverse people to undergo psychotherapy prior to the initiation of gender-affirming treatment, while acknowledging psychotherapy may be helpful for some transgender and gender diverse people.}

Psychotherapy has a long history of being used
in clinical work with TGD people (Fraser, 2009b).
The aims, requirements, methods and principles
of psychotherapy have been an evolving component of the Standards of Care from the initial
versions (Fraser, 2009a). At present, psychotherapeutic assistance and counseling with adult TGD
people may be sought to address common psychological concerns related to coping with gender
dysphoria and may also help some individuals with
the coming-out process (Hunt, 2014). Psychological
interventions, including psychotherapy, offer effective tools and provide context for the individual,
such as exploring gender identity and its expression, enhancing self-acceptance and hope, and
improving resilience in hostile and disabling environments (Matsuno and Israel, 2018). Psychotherapy
is an established alternative therapeutic approach
for addressing mental health symptoms that may
be revealed during the initial assessment or later
during the follow-up for gender-affirming medical
interventions. Recent research shows, although
mental health symptoms are reduced following
gender-affirming medical treatment, levels of anxiety remain high (Aldridge et al., 2020) suggesting
psychological therapy can play a role in helping
individuals suffering from anxiety symptoms following gender-affirming treatment.

In recent years, the uses and potential benefits
of specific psychotherapeutic modalities have
been reported (Austin et al., 2017; Budge, 2013;
Budge et al., 2021; Embaye, 2006; Fraser, 2009b;
Heck et al., 2015). Specific models of psychotherapy have been proposed for adult transgender
and nonbinary individuals (Matsuno \& Israel,
2018). However, more empiric data is needed on
the comparative benefits of different psychotherapeutic models (Catelan et al., 2017).
Psychotherapy can be experienced by transgender
persons as a fearful as well as a beneficial experience (Applegarth \& Nuttall, 2016) and presents
challenges to the therapist and to alliance formation when it is associated with gatekeeping for
medical interventions (Budge, 2015).

Experience suggests many transgender and nonbinary individuals decide to undergo genderaffirming medical treatment with little or no use
of psychotherapy (Spanos et al., 2021). Although
various modalities of psychotherapy may be beneficial for different reasons before, during, and after
gender-affirming medical treatments and varying
rates of desire for psychotherapy have been reported
during different stages of transition (Mayer et al.,
2019), a requirement for psychotherapy for initiating gender-affirming medical procedures has not
been shown to be beneficial and may be a harmful
barrier to care for those who do not need this type
of treatment or who lack access to it.

\hypertarget{statement-18.10-we-recommend-reparative-and-conversion-therapy-aimed-at-trying-to-change-a-persons-gender-identity-and-lived-gender-expression-to-become-more-congruent-with-the-sex-assigned-at-birth-should-not-be-offered.}{%
\section*{Statement 18.10: We recommend ``reparative'' and ``conversion'' therapy aimed at trying to change a person's gender identity and lived gender expression to become more congruent with the sex assigned at birth should not be offered.}\label{statement-18.10-we-recommend-reparative-and-conversion-therapy-aimed-at-trying-to-change-a-persons-gender-identity-and-lived-gender-expression-to-become-more-congruent-with-the-sex-assigned-at-birth-should-not-be-offered.}}
\addcontentsline{toc}{section}{Statement 18.10: We recommend ``reparative'' and ``conversion'' therapy aimed at trying to change a person's gender identity and lived gender expression to become more congruent with the sex assigned at birth should not be offered.}

The use of ``reparative'' or ``conversion'' therapy
or gender identity ``change'' efforts is opposed
by many major medical and mental health organizations across the world, including the World
Psychiatric Association, Pan American Health
Organization, American Psychiatric and American
Psychological Associations, Royal College of
Psychiatrists, and British Psychological Society.
Many states in the US have instituted bans on
practicing conversion therapy with minors.
Gender identity change efforts refers to interventions by MHPs or others that attempt to
change gender identity or expression to be more
in line with those typically associated with the
person's sex assigned at birth (American
Psychological Association, 2021).

Advocates of ``conversion therapy'' have suggested it could potentially allow a person to fit
better into their social world. They also point
out some clients specifically ask for help changing
their gender identities or expressions and therapists should be allowed to help clients achieve
their goals. However, ``conversion therapy'' has
not been shown to be effective (APA, 2009;
Przeworski et al., 2020). In addition, there are
numerous potential harms. In retrospective studies, a history of having undergone conversion
therapy is linked to increased levels of depression,
substance abuse, suicidal thoughts, and suicide
attempts, as well as lower educational attainment
and less weekly income (Ryan et al., 2020; Salway
et al., 2020; Turban, Beckwith et al., 2020). In
2021, the American Psychological Association
resolutions states that ``scientific evidence and
clinical experience indicate that GICEs {[}gender
identity change efforts{]} put individuals at significant risk of harm'' (APA, 2021).

While there are barriers to ending gender identity ``change'' efforts, education about the lack of
benefit and the potential harm of these practices
may lead to fewer providers offering ``conversion
therapy'' and fewer individuals and families
choosing this option.

  \bibliography{book.bib,packages.bib}

\end{document}
